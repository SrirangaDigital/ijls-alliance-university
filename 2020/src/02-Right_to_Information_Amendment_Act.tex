\setcounter{figure}{0}
\setcounter{table}{0}
\setcounter{footnote}{0}

\articletitle{Right to Information (Amendment) Act, 2019- A Mockery on The Autonomy of Administration }
\articleauthor{Ankit Anand\footnote{Assistant Professor, School of Law, SRM Institute of Science \& Technology, Chennai}}
\lhead[\textit{\textsf{Ankit Anand}}]{}
\rhead[]{\textit{\textsf{Right to Information (Amendment) Act, 2019-...}}}

\begin{multicols}{2}

\heading{Introduction}

\noi
Baron Acton’s thought: \textit{“Power corrupts and absolute power corrupts absolutely”} is suitable
in the context of the RTI Amendment Act, 2019 because of the absolute discretionary power
conferred to the CG.

\noi
There are four basic tenets of good governance- “accountability, transparency, public
participation, and predictability” associated with the government. We can imagine public
participation only when information is easily available to the public concerning the affairs of
the government. RTI Act encourages responsibility and transparency in the affairs of all
organs of government which makes the government more answerable. Prior to the RTI Act,
2005 the administrative authorities had the discretionary authority under the “Official Secrets
Act, 1923” not to provide adequate information to the public which ultimately resulted in
maladministration, bribery, and misuse of power.

\noi
The RTI Act came into existence in the year 2005 and it is one of the most historical
legislations which created a milestone in the legal history of India. The objective of the Act
was to give a sense of empowerment to the citizens: bring “transparency and accountability”
to the affairs of government. It was a major move to make the citizens well informed about
the affairs of the government.

\vspace{-.15cm}

\heading{Significance of The RTI Act, 2005}

\vspace{-.15cm}

\noi
Before discussing the relevant attributes of the Amendment Act, 2019, it is important to
summarise the pertinent provisions of the RTI Act, 2005 which are:

\noi
Firstly, the key objective of this Act was to bring transparency and empower the citizens of
this country. The former president of the USA, Barak Hussain Obama, in his speech to the
administration made a similar point on the importance of information:

\noi
\begin{quoting}
\textit{“In our democracy, the Freedom of Information Act (FOIA), which encourages
accountability through transparency is the most prominent expression of a profound
national commitment to ensuring an open government. At the heart of that
commitment is the idea that accountability is in the interest of the government.”}
\end{quoting}

\noi
The RTI Act authorises citizens of India to seek information from the government and at the
same time, the government has to furnish the relevant information to the applicant in a
specified time-bound manner. The application can be filled in both online and offline mode.
The Information comprises “any material in any form, including records, documents, memos,
e-mails, opinions, advice, press releases, circulars, orders, logbooks, contracts, reports,
papers, samples, models, data material as well as any kind of information such as tapes,
cassettes, videos, diskettes, etc.”\footnote{Right to Information Act, § 2(f), 2005 (India).} 
 There are certain kinds of information that is not subject to
disclosure at any cost.\footnote{Right to Information Act, § 8, 2005 (India).}

\noi
Secondly, as per the statutory provisions, to have any piece of information, an application
should be filed before the “Public Information Officer (PIO)”. In case, the PIO didn’t provide
the required information within the period of thirty days,\footnote{Right to Information Act, § 7(1), 2005 (India).} 
 the applicant can raise the issue before the First Appellate Authority (FAA). The FAA is bound to furnish the same
information within forty-five days.\footnote{Right to Information Act, § 19(6), 2005 (India).} Even if the FAA didn’t provide the same information
within that period, the applicant can go for the second appeal to the “Central Information
Commission or the State Information Commission” which are the quasi-judicial body which
has the power to conduct inquiries and impose penalties for the enforcement of the right of
the applicant.

\noi
Thirdly, it is related to the fees. The applicant has to pay a minimal fee of Rs. 10 along with
the application\footnote{Right to Information Act, § 6, 2005 (India).} and “Below Poverty Line” category is exempted to pay the requisite fee.\footnote{Right to Information Act, § 7, 2005 (India).} As
far as the format of application is concerned, The Act does not prescribe any format of the
RTI application, but the application should contain the full disclosure which includes name,
address, signature, and questionnaire along with the name and position of the PIO.

\heading{Key Highlights of The RTI Amendment Act, 2019}

\noi
First and foremost, The Amendment Act, 2019 conferred discretionarily arbitrary authority to
the Central Government to decide the “term of office of Chief Information Commissioner
(CIC) and Information Commissioners (ICs) at the Centre and the State level”. However,
prior to the amendment, the tenure of the CIC and ICs at the Centre and the State level was of
five years or sixty-five years whichever comes earlier.

\noi
Secondly, The RTI Amendment Act, 2019 gave authority to the Central Government to
determine the “salaries, allowances, and the other terms \& conditions of service of the CIC
and ICs at the Centre and the State level.” But prior to the Amendment Act, “salaries,
allowances, and the other terms and conditions of service of the CIC and ICs at Centre level”
was equivalent to those of the “Chief Election Commissioner and the Election
Commissioners” respectively. However, “salaries, allowances and the other terms
\& conditions of service of the CIC and the ICs at the State level” was identical to the “salary
of the Election Commissioner and the Chief Secretary of the state” respectively.

\noi
Thirdly, The Amendment Act, 2019 also abolished the provision of the other benefits
including the “Pension or Retirement benefits conferred to the CIC and ICs at the Centre and
the State level” from the previous government service. However, prior to the Amendment
Act, at the time of the appointment, if the CIC and ICs at the Centre \& the State level were
already getting “Pension or Retirement benefits” for his/her prior government service, their
salaries will be reduced by an amount equal to the “Pension or Retirement benefits”.
\end{multicols}

\heading{Comparative Analysis of The RTI Act, 2005 and RTI (Amendment) Act, 2019\footnote{Right to 
Information Act (2005); Right to Information (Amendment) Bill (2019); PRS }}

{\setlength\tabcolsep{3.5pt}
\renewcommand{\arraystretch}{1.5}
{\fontsize{11}{13}\selectfont
\begin{longtable}{|p{1cm}|p{2.5cm}|p{5.9cm}|p{5.9cm}|}
\hline
\multicolumn{1}{|m{1cm}|}{\centering SI. No} &
\multicolumn{1}{m{2.5cm}|}{\centering Statuory Provision} & 
\multicolumn{1}{m{5.9cm}|}{\centering RTI Act, 2005} & 
\multicolumn{1}{m{5.9cm}|}{\centering RTI (Amendment) Act, 2019}\\
\endfirsthead
\hline
\endhead
\hline
\endfoot
\hline
 1.& Term of Office (Section 13) & Section 13 stipulates the
“term of the CIC and ICs”
was of five years or sixty five years whichever
comes earlier. & The Amendment Act
abolished this
particular provision and confers the
absolute power in the
hand of the CG to
decide the term of the
CIC and ICs. \\\hline
2. &Term of Office (Section 16) &Section 16 stipulates the
“term of the State CIC and
ICs” was of five years or
sixty-five years whichever
comes earlier & The Amendment Act
abolished this
provision and confers
the absolute power in
the hand of the CG to
decide the term of
State CIC and ICs.\\\hline
 3.& Salary (Section 13)  & Section 13 stipulates that
the “salaries, allowances,
and other terms
\&conditions of service” of
the CIC will be equal to the
“salaries, allowances and
other terms \& conditions of
service” of the Chief
Election Commissioner.   & The Amendment Act
abolished this
particular provision
and confers the
absolute power in the
hand of the CG to
decide the “salaries,
allowances, and other
terms \&conditions of
service” of the
Central CIC and ICs.\\  
 & & Section 13 further
stipulates that the “salaries,
allowances, and other
terms \&conditions of
service” of the ICs will be
equal to the “salaries,
allowances, and other
terms \&conditions of
service” of the Election
Commissioner. & \\\hline
 4.& Salary (Section 16)  & Section 16 stipulates that the “salaries, allowances,
and other terms
\&conditions of service” of
the State CIC will be
similar to the “salaries,
allowances, and other
terms \&conditions of
service” the State Election
Commissioner. & The Amendment Act abolished this
particular provision
and confers the
absolute power in the
hand of the CG to
decide the “salaries,
allowances, and other
terms \&conditions of
service” of the State
CIC and ICs.\\
& &Section 16 further
stipulates that the “salaries,
allowances, and other
terms \&conditions of
service” of the State ICs
will be similar to that of
the “salaries, allowances
and other terms
\&conditions of service” of
the Chief Secretary to the
State Government. & \\\hline
 5.&Deductions  & Section 27 specifically
stipulates about deduction.
It stated if the CIC and the
ICs at the Centre and the
State level at the time of
appointment are already
getting the “Pension or
Retirement benefits” for
his/her prior government
service, their salaries will
be deduced by an amount equal to the “Pension or
Retirement benefits”. & The Amendment Act,
2019 abolished these
statutory provisions.
\end{longtable}
}}


\vspace{3pt}

\begin{multicols}{2}

\heading{The Justification of The Government}

\noi
The justification of the government in introducing these amendments to the Act is the
following:

\vspace{-.3cm}

\begin{enumerate}
\itemsep=0pt
\item Firstly, the Central Government claimed that there is a lacuna in the RTI Act itself for
making the “Election Commission of India” and the “Central \& State Information
Commission” in the parlance with each other. The “Election Commission of India”
derives its authority from the Constitution of India under Article 324 whereas
“Central \& State Information Commission” derives its authority from the RTI Act,
2005. The government further claimed that we can’t equate a constitutional body and
a statutory body; through this amendment, we would rationalize their status. The
government also claimed that it would bring accountability and transparency.

\noi
The government in their defence contended that the “Election Commission” being a
constitutional body is totally different from the “Information Commission” which is a
statutory body and due to which both the bodies can’t be equated. But, we have the
reference of Central Vigilance Commission Act, 2003 which is contrary to the excuse
of the government. Under the Act the “salaries, allowances, and other terms
\& conditions of service” of the Central Vigilance Commissioner will be equal to the
“salaries, allowances and other terms \& conditions of service” of the Chairperson of
Union Public Service Commission (UPSC) and the “salaries, allowances, and other
terms \& conditions of service” of the Vigilance Commissioner will be equal to the
“salaries, allowances and other terms \& conditions of service” of the member of
UPSC.\footnote{The Central Vigilance Commission Act, § 5(7), 2003 (India).} 
If the Central Vigilance Commission, being a statutory body can be equated
with a constitutional body like the UPSC, then, why will the Information Commission
be the only exception?
\end{enumerate}

\vspace{-.3cm}

\noi
Apart from that an interesting fact associated with the RTI Act, 2005 is that the original RTI
bill recommended the salaries, and the allowances of the CIC and ICs were on parlance with 
the secretaries and the additional secretaries respectively. But the parliamentary committee
consists of the then BJP MP Ram Nath Kovind (now the President), Balavant Apte, and other
leaders recommended to change the same and increase it to the same level as the chief
election commissioner and other election commissioners for the CIC and ICs respectively
which is ironical in itself.\footnote{RTI Act amendment: Former information commissioners, activists criticise government move, (February 15,2020, 16:45 PM), \url{https://www.apnlive.com/rti-act-amendment-former-information-commissioners-activistscriticise
-government-move/}}

\vspace{-.3cm}

\begin{enumerate}
\addtocounter{enumi}{1}
\item Secondly, there is a contradictory provision in the RTI Act itself. Under the RTI Act,
the decision of the CIC is in parlance with the decision of the Supreme Court’s Judge.
However, the decision of the CIC can be challenged before the High Court which is
contradictory. The Central Government claimed that the Amendment Act would
strengthen the whole structure of the RTI Act.

However, there are other statutory bodies that are considered to be on par with the
judges of the Supreme Court as the members of “National Green Tribunal (NGT) and
the National Human Rights Commission (NHRC)” but the Central Government did
not address this issue to have uniformity.
\end{enumerate}

\vspace{-.3cm}

\heading{Objection Against RTI Amendment Act, 2019}

\vspace{-.2cm}

\begin{enumerate}
\itemsep=0pt
\item It is not in consonance with the pre-legislative consultation policy, 2014

\item Crumbling the federalism of the RTI Act.

\item Menace to the autonomy of Impartiality and the Independence of Information Commission.

\item Delegation of Excessive powers to the Central Government. 
\end{enumerate}

\vspace{-.3cm}

\heading{Pre-Legislative Consultation Policy}

\noi
The Central Government came up with the pre-legislative consultation policy in the year
2014, which stipulates that any ministry/department which is going to make new laws or
amend the existing provision of any law, in that case, it is the duty on the part of the
government to put such draft bill before the public domain for discussion. The point of view
of a person who is going to be affected by such a law should be taken into consideration by
the government.

\noi
It is not the first time the government tried to amend the RTI Act, 2005. They already tried in
the year 2012 and 2017 and the draft proposal was put before the public domain for healthy
discussion and recommendation. However, ironically the new Amendment Act was passed by
the parliament without putting it before the public domain. The Government has not acted as
per the guidelines stated in the pre-legislative consultation policy, 2014 by not placing the
amended draft proposals in the public domain. Apart from that, there was no consultation
with the ICs before passing the RTI (Amendment) Bill, 2019. The government on the other
hand put an excuse that the public is not at all involved in the process because the amendment
is going to affect only the RTI officers and government.

\heading{Crumbling The Federalism of The RTI Act, 2005}

\noi
Prior to the Amendment Act, 2019, there was a crystal clear bifurcation of separation of
power between the centre and the state legislature concerning the “term, salaries and
allowances” of the ICs at the centre and state level which exemplify the “federal structure” of
the RTI Act. The prominent reasons for providing federal structure are to give functional and
financial independence from the government.

\noi
Under Amendment Act, 2019, the Central Government has the unreasonable discretionary
authority to determine the “term, salaries and allowances of the CIC \& ICs at the Central
level as well as the State level”. The Act stipulated that the salaries and allowances of the ICs
of the centre shall be paid out of the “consolidated fund of India” and of state ICs shall be
paid out of the “consolidated fund of state.” As per the constitutional provision, neither the
Central Government nor the Parliament has any authority over the “consolidated fund of
state” except in case of “President’s rule in the state under Article 356 of the Constitution of
India.” It is hard to digest as to under what authority parliament through its law-making
power delegate the centre to decide the “salaries and allowances” of the State ICs.\footnote{\textit{The Right to Information Is Dead. Here Is its Obituary,} (February 19, 2020, 13:32 PM),
\url{https://thewire.in/government/the-right-to-information-is-dead-here-is-its-obituary}}

\vspace{2cm}

\heading{Menace to The Autonomy of Impartiality \& Independence of Information Commission}

\noi
One of the prominent reasons why the Parliamentary Standing Committee equates the CIC
with the Chief Election Commission rather than a civil servant because of the aspect of independence of the commission. The purpose of the RTI Act is to safeguard the interest of
the public by providing the details without any distress of the Centre and State Government.

\noi
The Amendment Act curtailed the impartiality as well as independence of the ICs by
lowering its rank and by authorizing the Central Government to decide the “salaries,
allowances, and tenure” of these authorities. This directly makes the whole institution as a
bureaucrat under the authority of the Central Government which won’t allow the Commission
to work in an impartial way.\footnote{\textit{Id.}}

\heading{Delegation of Excessive Powers to The Central Government}

\noi
It was an absolute legislative function of the statutory authority under the RTI Act, 2005 to
fix the “tenure, salaries, allowances and other conditions of services of the CIC and ICs at the
Centre and State level”. But conferring the absolute authority in the hand of the Central
Government tends to excessive delegation.

\noi
In the \textit{Re Delhi laws case}, The Hon’ble SC held that “The legislature does not have authority
to delegate its essential law-making functions into the hands of the executive.”\footnote{AIR SC 332 (1951)} “The
legislature should keep the work of essential legislative functions to itself such as
determining the legislative policy and laying down standards which are enacted into a rule of
law and it can leave the task of subordinate legislation which by its very nature is ancillary to
the statue to the subordinate bodies.”\footnote{Municipal Corporation of Delhi v. Birala Cotton, Spinning and Weaving Mills and others, AIR SC 1232 (1968)}

\newpage

\noi
In the case of \textit{A. N. Parasuraman vs. Tamil Nadu,}\footnote{4 SCC 683 (1989)} the Hon’ble Supreme Court held that “the
provisions of the Tamil Nadu Private Educational Institutions (Regulation) Act 1966 is not in
consonance, both on the ground of excessive delegation as well as the violation of the Article
14 of the Constitution of India as it did not contain adequate guidelines to the executive for
the exercise of the delegated legislative power.”

\vspace{-.1cm}

\noi
If we analytically examine the RTI Amendment Act, 2019, it is evident that the parliament
gratuitously conferred excessive legislative authority to the Central Government for deciding
the “tenure, salaries, allowances and other conditions of services of the CIC \& ICs at the 
Centre and State level” and did not lay down any standard rule-making guidelines for the
Central Government relating to “tenure, salaries, allowances and other conditions of
services.”

\heading{The Drawbacks of The Amendment Act, 2019}

\noi
According to the researcher, the loopholes associated with the Amendment Act, 2019 are as
follows:

\vspace{-.4cm}

\begin{enumerate}
\itemsep=0pt
\item The Amendment Act empowered the Central Government absolutely which directly
deteriorates the basic notion and structure of the RTI Act.

\item The Amendment Act gave arbitrary authority to the Central Government. Therefore,
the aspect of neutrality of the ICs mutilated and it made the ICs more trustworthy to
the government.

\item The Amendment Act made the Central and State Information Commissioner
dependent on the Central Government. Therefore, the autonomy of the Independence
of Information Commission becomes weak.

\item The Amendment Act moderates the status of the Information Commissioner at the
Centre \& State level from those of the Judges of the SC which resulted in lack of
authority to issue directives to the officers.

\item One of the major drawbacks of the Amendment Act is that it was passed by the
Parliament without taking into consideration the view of the general public.

\item The RTI Act brought accountability and transparency in the governance but the
Amendment Act certainly took away the transparency because of the arbitrary power
which is being conferred to the Central Government.
\end{enumerate}

\vspace{-.3cm}

\heading{Conclusion}

\noi
The RTI Act was enacted by the government was to bring “transparency, culpability and
accountability” to the administration in all the organs of the government. The act was enacted
to empower the citizen of India to access all kinds of information and at the same time
safeguard the independence of the ICs so that they could accomplish their duties freely
without any undue influence.

\noi
The RTI Act had brought a revolutionary change in the whole administrative system and
gradually transformed India where any citizen of this country has the statutory right to access
every kind of information associated with any organ of the government. The RTI served as a
weapon in the hands of citizens of India to deter the “misuse of power” and authorize them to
ask genuine questions from the bureaucracy and the governments. As Justice A. P. Shah,
(former Chief Justice, Delhi and Madras High Courts) described the importance of
information as:

\noi
\begin{quoting}
\textit{“Information is the currency that every citizen requires to participate in the life and
governance of society”}
\end{quoting}

\noi
It is the basic right of every citizen of India to seek information because without information
we cannot expect a vibrant democracy. We have witnessed that the citizens of this country
utilized this statutory right and unearthed some of the famous scams like Adarsh Housing
Society Scam, Commonwealth Games Scam, 2G spectrum scam etc.,\footnote{\textit{10 Years of RTI: 5 Stories Of Ordinary People Using The Act For Positive Change,} ( January 17, 2012, 15:50
PM), \url{https://www.huffpost.com/archive/in/entry/2015/10/13/rti-10-years_n_8277290.html}} However, the RTI Act
had a two-fold feature. On one hand, it provided the statutory rights to citizens of this country
and on the other hand, it uses to hold all organs of government accountable at the same time. 

\noi
The Amendment Act abridges the autonomy, independence, and impartiality of the
Information Commission by diminishing their status and lowering their rank. It conferred
vital legislative functions to the Central Executive which is irrational and capricious and it
tends to be excessive delegation. Therefore, the New Amendment Act altered the
fundamental notion of the RTI Act. It is right to connote that the new amendment captivates
the impartiality, autonomy, authority, and independence of ICs and makes them mere puppets
in the hand of the Central Government. It conferred absolute authority to the Central
Government which is a direct threat to the freedom of speech and expression of citizens and
gradually it might lead to the dictatorship.

\end{multicols}
