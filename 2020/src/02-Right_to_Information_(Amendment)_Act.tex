\setcounter{figure}{0}
\setcounter{table}{0}

\articletitle{Right to Information (Amendment) Act, 2019- A Mockery on The Autonomy of Administration }
\articleauthor{Ankit Anand\footnote{Assistant Professor, School of Law, SRM Institute of Science \& Technology, Chennai}}
\lhead[\textit{\textsf{Ankit Anand}}]{}
\rhead[]{\textit{\textsf{Right to Information (Amendment) Act, 2019-...}}}

\begin{multicols}{2}

\heading{Introduction}

\noi
Baron Acton’s thought: \textit{“Power corrupts and absolute power corrupts absolutely”} is suitable
in the context of the RTI Amendment Act, 2019 because of the absolute discretionary power
conferred to the CG.

\noi
There are four basic tenets of good governance- “accountability, transparency, public
participation, and predictability” associated with the government. We can imagine public
participation only when information is easily available to the public concerning the affairs of
the government. RTI Act encourages responsibility and transparency in the affairs of all
organs of government which makes the government more answerable. Prior to the RTI Act,
2005 the administrative authorities had the discretionary authority under the “Official Secrets
Act, 1923” not to provide adequate information to the public which ultimately resulted in
maladministration, bribery, and misuse of power.

\noi
The RTI Act came into existence in the year 2005 and it is one of the most historical
legislations which created a milestone in the legal history of India. The objective of the Act
was to give a sense of empowerment to the citizens: bring “transparency and accountability”
to the affairs of government. It was a major move to make the citizens well informed about
the affairs of the government.

\heading{Significance of The RTI Act, 2005}

\noi
Before discussing the relevant attributes of the Amendment Act, 2019, it is important to
summarise the pertinent provisions of the RTI Act, 2005 which are:

\noi
Firstly, the key objective of this Act was to bring transparency and empower the citizens of
this country. The former president of the USA, Barak Hussain Obama, in his speech to the
administration made a similar point on the importance of information:

\textit{“In our democracy, the Freedom of Information Act (FOIA), which encourages
accountability through transparency is the most prominent expression of a profound
national commitment to ensuring an open government. At the heart of that
commitment is the idea that accountability is in the interest of the government.”}

\noi
The RTI Act authorises citizens of India to seek information from the government and at the
same time, the government has to furnish the relevant information to the applicant in a
specified time-bound manner. The application can be filled in both online and offline mode.
The Information comprises “any material in any form, including records, documents, memos,
e-mails, opinions, advice, press releases, circulars, orders, logbooks, contracts, reports,
papers, samples, models, data material as well as any kind of information such as tapes,
cassettes, videos, diskettes, etc.”2  There are certain kinds of information that is not subject to
disclosure at any cost.3

\noi
Secondly, as per the statutory provisions, to have any piece of information, an application
should be filed before the “Public Information Officer (PIO)”. In case, the PIO didn’t provide
the required information within the period of thirty days,4\footnote{}  the applicant can raise the issue
before the First Appellate Authority (FAA). The FAA is bound to furnish the same
information within forty-five days.5\footnote{} Even if the FAA didn’t provide the same information
within that period, the applicant can go for the second appeal to the “Central Information
Commission or the State Information Commission” which are the quasi-judicial body which
has the power to conduct inquiries and impose penalties for the enforcement of the right of
the applicant. 


\end{multicols}
