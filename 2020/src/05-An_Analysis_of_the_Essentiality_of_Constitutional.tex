\setcounter{figure}{0}
\setcounter{table}{0}
\setcounter{footnote}{0}


\articletitle{An Analysis of the Essentiality of Constitutional Morality in Contemporary India}\label{2020-art5}
\articleauthor{Ishika B Prabhakar\footnote{BA LLB (3$^{\rm rd}$ year), Christ (Deemed to be University), Bangalore.}}
\lhead[\textit{\textsf{Ishika B Prabhakar}}]{}
\rhead[]{\textit{\textsf{An Analysis of the Essentiality of Constitutional...}}}

\begin{multicols}{2}

\heading{Introduction}

\noi
Constitutional Morality has not been defined in the Constitution but inferences to its meaning
have been made through judgements. One such judgement would be the Sabarimala case,\footnote{Indian Young Lawyers Association v. State of Kerala, 2018 (8) SCJ 609}
wherein the Court stated the definition of constitutional morality in the verdict. It stated that,

\noi
\textit{“We must remember that when there is a violation of the fundamental rights, the term
‘morality’ naturally implies constitutional morality and any view that is ultimately taken by
the Constitutional Courts must be in conformity with the principles and basic tenets of the
concept of this constitutional morality….”}\footnote{Ananya Chakravarti, \textit{Constitutional Morality in the context of the Indian Legal System,} 3 INTERNATIONAL JOURNAL OF LAW MANAGEMENT \& HUMANITIES 64, 66-67 (2020)}
. The aforementioned statement was made with reference to Article 25(1) of the Indian Constitution. The doctrine of constitutional morality
was exercised as opposed to the doctrine of essentiality, the latter doctrine that was espoused
in the \textbf{\textit{Shirur Mutt}} case\footnote{The Commissioner, Hindu Religious Endowments, Madras v. Shri Lakshmindar Thirtha Swamiyar of Shirur Mutt, 1954 SCR 1005}  by a seven-judge Bench of the Supreme Court wherein the Court
took upon the task of deciding and bifurcating between the essential and non-essential
practices of religion. The Judiciary went against the autonomy of customary religious
practices and hence, the judgement was highly criticized due to the deemed usage of
constitutional morality as judicial overreach. Instances of constitutional morality is
imperative despite the popular opinions of society as the Judiciary is the only body capable of
ensuring an individual’s Fundamental Rights when the State fails to do so through the laws it
enacts.

\vspace{1.2cm}

\heading{(I) An Analysis of Several Cases and\\ Legislations in Relation
to the Juxtaposition of Judicial Activism, Judicial Restraint and\\
Constitutional Morality}

\noi
Fundamental Rights have been conceived in a liberal spirit and seek to draw reasonable
balance between individual freedom and social control.\footnote{M. P Jain, \textit{Indian Constitutional Law,} 878 (8$^{\rm th}$ ed. 2018)}
 The essence of the Constitution lies in the spirit with which the Constituent Assembly conceived Fundamental Rights. Judicial
Activism in itself is a constant battle between granting that individual freedom or withholding
it for the purpose of social control. 

\noi
Judicial activism can do the society a great service and disservice, the same can be analysed
below.

\noi
In the case of \textbf{\textit{AK Gopalan v. State of Madras}},\footnote{AK Gopalan v. State of Madras, AIR 1950 SC 27}  a very narrow view was taken by the
Majority of the Bench. The case that had an opportunity to change the course of how the
system viewed a person’s Fundamental Rights and liberty, fell short to an insurmountable
degree at the time. The pressure of the current Indira Gandhi’s Government influenced and
weighed heavily on this decision. Justice Chandrachud and Justice Bhagwati have stated and
addressed their regrets over the decision of the case.\footnote{Maneesh Chhibber, \textit{35 years later, the former Chief Justice of India pleads guilty,} THE INDIAN EXPRESS, (Sep 16th, 2011), \url{http://archive.indianexpress.com/news/35-yrs-later-a-former-chief-justice-of-india-pleadsguilty/847392/}}  What would have been viewed as Judicial restraint being exercised by the Judges that made the decision to maintain and keep
the law at status quo at that time, can now be understood to be a false notion. Judicial
activism is when a Judge uses his/her knowledge to prevent foreboding circumstances that
negatively affect the society as a whole. In the aforementioned case, however, considering the
political pressure on the Judiciary, the decision can be said to have been made through the
exercise of Judicial activism that did not effectively help India progress as a society, it only
helped the Judiciary maintain a harmonious relationship with the incumbent’s Government at
the time. This case is an example of the disservice that Judicial activism does to the essence
of our Constitution.

\noi
In \textbf{\textit{Maneka Gandhi v. Union of India}},\footnote{Maneka Gandhi v. Union of India, (1978) 1 SCC 248}  Justice Chandrachud and Justice P.N Bhagwati made
amends for their decision in the A K Gopalan case that adversely affected the plight of the
people. Article 21 which was in its dormant state at this point in time for nearly three decades
sprung back to life by the decision made in this case. Justice Iyer J stated how formative and seminal the drafting of Article 21 is by comparing it to the magna carta of protecting the life
and liberty of people.

\noi
Justice Bhagwati also stated and regarded Article 21 of the Constitution in a similar way as
he evaluated it to be an important part of a democratic society. This case was a key factor that
galvanized the transformative Judicial stance on the importance of the core Constitutional
principles, slowly refurbishing the existing laws on freedom, liberty and equality. The
emphasis on constitutional values that led the judges to make this historic and progressive
decision can be argued to be constitutional morality. Judicial activism can go either way, that
is, it can adversely affect the society or bode well for the society as a whole. Constitutional
Morality on the other hand, is akin to having a premonition about foreboding circumstances
that serve as an indication to the dangers of a current norm or law and its adverse effect on
future societal predicaments. The Judges of the Courts are revered and put on a pedestal to
make decisions with foresight and wisdom. Viewing Constitutional Morality as arbitrary is
akin to viewing the whole Judicial System as inadequate and arbitrary as Constitutional
Morality is nothing more than the Constitution’s looming presence over the sensibilities of
the Judges at all times. Judges are humans after all, subjectivity does somehow seep through
their ironclad objective dealings with Judicial matters but that is known as Judicial Activism.
A decision fuelled by a Judge’s views cannot invalidate a decision or deem it to be arbitrary
if the decision is made with the Constitution acting as the conductor of the same. The purpose
of the doctrine of Constitutional Morality is the sole adherence to the principles of the
Constitution and the goal of the same is to uphold and preserve those said democratic
principles for the welfare of the society.

\noi
In \textbf{\textit{Balaji v. State of Mysore}},\footnote{Balaji v. State of Mysore, 1963 AIR 649} the Court espoused that Caste cannot be the only determinant of what constitutes as backwardness and that the right to equality and equal protection of law must still be maintained whilst addressing protective discrimination. This is an example of
how Judicial activism developed in India to have a positive effect on society.

\noi
Judicial Review in itself is a form of Judicial activism as its priorities always lies with the
betterment of society and the correction of current laws to strengthen that agenda. Public Interest Litigations is also a form of Judicial activism as it purports the aforementioned ideals
of what constitutes as Judicial activism. 

\noi
In \textbf{\textit{Common Cause v. Union of India}},\footnote{Common Cause v. Union of India, (2018) 5 SCC)}  the Apex Court took cognizance of passive
euthanasia through living wills which can be communicated by the patients in case of
terminally ill diseases. It recognized the personal autonomy of people with chronic diseases
and their right to a dignified death and autonomy, which is an augmentation of the right to
life as given under Article 21, which is also inclusive of personal liberty. This is what entails
Constitutional Morality, the ability to understand and show empathy for the people of this
country. 

\newpage

\noi
Constitutional Morality on the other hand, is usually in a latent state. It has been mentioned in
the judgement of \textbf{\textit{Kesavananda Bharati}},\footnote{Kesavananda Bharati v. UOI, AIR 1997 SC 1461 } wherein the Judges laid down that the Preamble is
a part of the Constitution. Constitutional Morality from a broad perspective is essentially
what the Preamble constitutes. The aforementioned decision accentuates the importance of
the doctrine of Constitutional Morality as the act of using the same to discern the issues of a
dispute in order to deliver verdicts that ensures Judicial justice, is nothing but delivering
verdicts that are in consonance with the sacred pledge of the Constitution, that is, the
Preamble. Constitutional morality can be stated as a way in which the Court enacts laws that
is applicable beyond it’s time, in the interest of the public. 

\noi
In the case of \textbf{\textit{Justice K. S. Puttuswamy v. Union of India and others}},\footnote{Justice K S Puttuswamy v. Union of India and Ors, (2017) 10 SCC 1} the Apex Court
affirmed the Right to Privacy of individuals to be protected as a Fundamental Right as under
Articles 14, 19, and 21 of the Constitution.

\noi
This is still considered to be Judicial overreach by several people of the society. The aim of
Constitutional morality is to ensure social justice and Constitutional evolvement. In most
cases, the social morality of a particular society will not be in consonance with that of the
need of Constitutional morality, however, Constitutional morality takes precedence over the
majority opinions at the time as it is the duty of the Court to be forthcoming and prospective
in nature.

\heading{(II) Jurisprudential Analysis of Constitutional Morality}

\noi
Constitutional Morality can be understood to be closely related to Jeremy Bentham’s concept
of Utilitarianism and John Stuart Mill’s concept of the harm principle. Constitutional
Morality is built and derived from public morality. The prevailing public morality of a
society largely influences the Court’s decisions whilst they exercise Constitutional morality.
The concept of Utilitarianism determines the ethical nature of an action on the basis of the
same producing the greatest good for the greatest number. Homosexuality was decriminalised
in India because the majority of society thought it to be necessary, that is, the greatest number
of people considered it to be overdue, hence, decriminalising the same was considered to be
the greatest good. The harm principle is determined on the basis that a person’s liberty can
only be restrained if the exercise of the same causes harm to another person. According to the
harm principle, an individual’s freedom and liberty cannot be curtailed based on moralistic
and paternalistic principles. In the case of Constitutional morality and the exercise of the
same to decriminalize homosexuality, it was decided so by the Court because homosexuality
does not cause harm to another human being. Questions of moralistic principles does not
factor in whilst determining what actions are harmful, hence, the hurting of people’s
sentiments cannot be a contributor in the consideration of homosexuality being an offence
punishable under law.

\heading{(III) The Need of Constitutional Morality and the Application of the Same}

\noi
Constitutional Morality has to understood to be more than a utopian ideal. The inalienable
rights given under the Constitution to the citizens of India is proof of the agenda that
Constitutional Morality is an attainable goal, and the exercise of the same will bring about
Judicial justice and the same has been espoused in detail in Part III, IV and V of the
Constitution.

\noi
The \textbf{\textit{Sabrimala}} case is an apt example of the exercise of Constitutional morality in opposition
to the views of society’s majority. The society as a whole considered this to be Judicial
overreach as they went beyond their authority to decide something that infringed upon
people’s customs and beliefs. The Court is anything but an ecclesiastical authority type figure
and hence, this did seem out of touch from their jurisdiction. However, in pursuance of
Constitutional Morality that seeks to be progressive even in the face of opposition, the 
decision did not fall short of the aforementioned ideals. This decision has opened doors to
decriminalize and amend other discriminative religious customs in the future.\footnote{ Apurva Vishwanath, \textit{Sabarimala majority ruling: Review pending, scope widened,} THE INDIAN EXPRESS,
(Nov 15th, 2019, 9:55AM),  \url{https://indianexpress.com/article/explained/simply-put-review-pending-scopewidened-in-sabarimala-verdict-6120277/}} Constitutional Morality has to be understood to be more than a utopian ideal.\footnote{Chakravarti, \textit{Supra note} 4, at 5.} Any inalienable right given under the Constitution to the citizens of India is proof of the agenda that Constitutional
Morality is an attainable goal. The tussle between social morality and that of Constitutional
morality makes for productive and good governance.

\noi
In the case of \textbf{\textit{Navtej Singh Johar}},\footnote{Navtaj Singh Johar \& Ors v. Union of India, (2016) 7 SCC 485} the verdict of this case is the embodiment of
Constitutional principles, the Court analysed why this law was discriminatory and fell back
on the core principles of the Constitution to right a wrong law, this is essentially what
constitutes as the use of Constitutional Morality to make a decision.

\vspace{-.1cm}

\noi
In the case of \textbf{\textit{Joseph Shine v. Union of India}},\footnote{Joseph Shine v. UOI, 2018 SC 1676} the old-fashioned and obtuse law under
Section 497 of the Indian Penal Code, which espoused that it is an offence for a man to have
sexual relations or intercourse with the wife of another man without the consent of the same,
was held to be unconstitutional on the ground that this law only existed to recognize and
consider the wants and needs of the husband at the cost of disallowing the wife to be in
control of her own sexuality, and by making it a norm to have women ask their husbands for
agency rather than recognising that women give themselves sexual agency and don’t require
permission from their husbands for the same. The norm of women being the property of their
husbands was debunked and held to be against the constitutional ideals of dignity and
equality. The basic tenets of Constitutional Morality are exercised to deliver that verdict.

\vspace{-.1cm}

\noi
In the case of \textbf{\textit{Shreya Singhal v. Union of India,}} the Judges stated that dedication and
adherence to the Constitution is very much a tenet of Constitutional morality\footnote{Shreya Singhal v. UOI, AIR 2015 SC 1523} as they struck down Section 66 A of the Information Technology Act, 2000,\footnote{Information Technology Act, 2000, § 66 A} deeming it to be
unconstitutional.

\vspace{-.1cm}

\noi
In the case of \textbf{\textit{Shayara Bano v. Union of India}},\footnote{Shayara Bano v. UOI, 2017 (9) SCC 1} the Apex Court took cognizance of the
predicament that the practice of giving a divorce through talaq-e-biddat cannot be considered
as an essential religious practice as under Article 25 of the Constitution. The Court rightfully
criminalised this draconian religious practice, this decision is the testament to the fact that the
application of Constitutional Morality has shown great promise in reaffirming the
Fundamental Rights of citizens.

\vspace{-.1cm}

\noi
In the current scenario, \textbf{the Transgender protective legislation}\footnote{The Transgender Persons (Protection of Rights) Act, 2019, § 2(k), § 18} passed by the Government
has several discriminatory provisions. This legislation is extremely regressive, for example, it
demands proof of a gender reassignment surgery for a person to be considered as transgender.
It ironically provides barely any protective measures for trans genders and intersex citizens in
spheres such as education, employment and healthcare. Several provisions of the Act have
been contended to be discriminatory and the legislation has been challenged before the
Supreme Court. This decision can be anticipated to yield two kinds of results: to either surge
with the current, prevailing social morality of the public or it can and will do justice to this
minority through the exercise of Constitutional morality.

\vspace{-.1cm}

\noi
\textbf{The Unlawful Activities (Prevention) Amendment Act, 2019,} Section 35 of the Act was
amended to and give the Central Government the power to label an individual as a ‘terrorist’
under Schedule IV of the Act. Prior to the Amendment, only organizations could have been
designated in this manner. This is the result of an extremely despotic incumbent Government
grappling at all means necessary to wield its power to make arbitrary decisions so as to alter
the society and silence the dissenters into submission. In this amendment, there is no
provision for the accused to be heard before a Court of law, it criminalizes a person without
due process. Should these provisions be challenged in a Court of Law, the Court must view
this matter in the purview of Constitutional Morality so as to impugn its validity.

\vspace{-.1cm}

\noi
Another instance of the Courts exercising Constitutional Morality over the years are in the
matters related to marriage dissolution and live-in relationships according to the \textbf{Hindu Marriage Act, 1955},\footnote{Hindu Marriage Act, 1955, § 10, § 13, § 13 B (2)} in predicaments where a cooling period of six months, being Judicial
Separation, is usually granted for spouses who could use some time apart in order to rethink
ending their marriage. This period, however, in cases when the spouses are beyond a
reparable relationship, the Court forgoes the same and grants a divorce. This action goes
against the social morality as India is extremely conservative in terms of the sacredness of
marriage, and how spouses must make every effort to stay together.\footnote{Dr. Ashutosh Hajela, \textit{Legal Realism Via Constitutional Morality in India: A Critical Analysis,} SOCIAL SCIENCE RESEARCH NETWORK 39, 49-52 (2019) } The Judiciary has also endorsed the view that not all relationships need to come out of holy matrimony and are
becoming more agreeable to live-in relations through the several decisions made over the past
few years.\footnote{Indra Sarma v. V. K. V Sarma 15 SCC 755 (SCC:2013); D Velusamy v. D. Patchaiammal, 10 SCC 469
(SC:2010)} In the pursuit of Constitutional Morality, however, these archaic norms are
undone in order to free the people shackled by those very ideals.

\noi
In light of recent turbulent times that India has faced with the Citizenship Amendment Act,
2019 and the farm bills recently passed in the Parliamentary sessions of 2020, it is apparent
that this country is in desperate need of an objective, impartial, and strong voice to reassure
the people of the Country that the Constitution and rule of law still remain the sovereign
voice and influence over all matters. Every issue in this country is stemming from a deeprooted polarisation on the lines of religion and caste. The CAA is already a law that is
published in the Gazette of India, the Court needs to adjudicate on challenges against the
same with the ideals of the Constitution in mind. Several petitions will be filed to challenge
the said bills if it becomes a law without taking into consideration the grievances of the
farmers. If the matter does reach that level, it is up to the Judiciary to right the wrongs of the
Government. A strong, unwavering voice that silences despotism and adjudicates matters
within the purview of Constitutional Morality is the need of the moment.

\heading{(IV) The Naysayers of Constitutional Morality : Their Point of View}

\noi
The opposing view of the essentiality of Constitutional Morality is that too much of the same
will lead to arbitrary and unlimited power bestowed upon one organ of the Government, that
is, the Judiciary. The Attorney General of India, K.K Venugopal, has stated that if the use of Constitutional Morality continues, the destination will be hard to ascertain. He also stated that
it’s a dangerous weapon that can be wielded by the Judiciary.\footnote{Apoorva Mandhani, \textit{“Constitutional Morality: A Dangerous Weapon, It Will Die with Its Birth: K. K Venugopal”,} LIVE LAW, (9$^{\rm th}$ Dec, 2018, 7:14PM), \url{https://www.livelaw.in/constitutional-morality-a-dangerousweapon-it-will-die-with-its-birth-kk-venugopal/?infinitescroll=1}} The use of Constitutional
Morality can only lead to one destination and that is a more tolerant society. There can never
be too much of Constitutional Morality as the exercise of the same is done under
circumstances wherein the Judiciary is trying to reinstate the essential character and
principles of the Constitution into current norms and laws that have been led astray from
those essential ideals.

\heading{Conclusion}

\noi
After an analysis of what constitutes as Constitutional morality and what constitutes Judicial
overreach, several times, they are synonymous in nature. In order to get ahead with the times
of new age, several old customs and practices have to be forgone, and that in itself will be
considered as Judicial Overreach. It should not be considered as assimilation or the loss of the
Indian character, tradition and identity, it should, instead, be considered as a revision to the
Indian character and essence. In a world wherein artificial intelligence taking over humans in
the near future has become a known and accepted reality, it is a crime for the Courts to deny
the efficacy of the doctrine of Constitutional Morality. Keeping that in mind, the ill-defined
doctrine of Constitutional morality and the lack of Jurisprudential backing deems it to be
difficult to defend against naysayers. India is in dire need of a jurisprudential backing of
Constitutional morality so as to solidify the basis for the same without the smokescreens and
ambiguity. However, when decisions are made and judgements are given through the prism
of Constitutional Morality, the current archaic norms and practices are revamped through a
new lens, which then helps reduce over criminalization and betters policy-making.
Constitutional morality is largely based on public morality, but when the prevailing public
morality is against Constitutional principles, the Courts exercise Constitutional morality to
protect the minorities from the prevailing, vast majority opinion. Sedition in India is wielded
as a powerful weapon against anyone with a difference of opinion from that of the
Government in power. The over criminalization of sedition offences can only be rectified by
the Courts through the adjudication of the same through the prism of Constitutional morality.
The only change that can be brought about to laws that are largely misused and that advertently leave the people without any power and protection, is through the Courts that
remain impartial and change the dialogue and unfair use of legislations through the exercise
of Constitutional morality by the highest authority, that is, the Apex Court. If the Courts,
however, are biased, no real change can be brought about. Former Chief Justice, Ranjan
Gogoi, appointing Judges himself for an in-house inquiry into the allegations of sexual
harassment against him and then disregarding the complainant using his platform is an
example of how bias can permeate even through the Apex Court. Nothing about the whole
ordeal was fair and impartial. Instances as such, ruin the sanctity and significance of the
Supreme Court and it makes people lose faith in the Judicial system and its power to deliver
justice. Hence, the Government in power can influence and sway the Courts to feed their
agenda, but the Courts must remain impartial. The people deserve more than the Courts
providing old solutions to new problems. The precedents aforementioned have guided India
into becoming more open and accepting and a lot less obtuse and despotic, but the battle has
not been won yet. India is considered to be partly free and an electoral autocracy as of 2021
by the released USA and Swedish reports respectively. Even if the relevance and authenticity
of the same is questioned by various naysayers, the important aspect is that the world is
taking notice of the grievances of people that our own Government is tone deaf and blind to.
We as a country have a long and perpetual trudge ahead of us. The transparency and
insightfulness of the Judiciary will be the focal point in affirming and assuring us that the
necessary change required for India to blossom into what our forefathers envisioned for us as
a tolerant society, will and can be achieved. The transformative effect of the Maneka Gandhi
case in the 20$^{\rm th}$ century set the tone for the way in which matters were adjudicated in a Court
of Law after the same, a revamping of that effect is gravely essential by the polarized and
divided India of the 21$^{\rm st}$ century.

\end{multicols}
\label{end2020-art5}
