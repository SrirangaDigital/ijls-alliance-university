\setcounter{figure}{0}
\setcounter{table}{0}
\setcounter{footnote}{0}


\articletitle{Globalisation and its Implication on Law Relating to Women and Their Work-Life Balance}
\articleauthor{Dr. N. Dasharath\footnote{Professor and Chairman, Dean \& Faculty of Law, University Law College and Department of Studies in Law, Bangalore University} and  Sujatha V. Durgekar\footnote{Research Scholar, University Law College and Department of Studies in Law, Bangalore University}}
\lhead[\textit{\textsf{Dr. N. Dasharath and Sujatha V. Durgekar}}]{}
\rhead[]{\textit{\textsf{Globalisation and its Implication on Law...}}}

\begin{multicols}{2}

\heading{Introduction}

\noi
Indian women have been revered in the Indian society giving them the form of Goddesses. In
all the Hindu scriptures, she is revered and worshiped as an all-empowering force. The
Ramayana and the Mahabharata, both the stories were woven around the women only. But
from time immemorial, she has been confined to domestic work. Work and life are
interrelated. It is impossible to live a life without work. Every human being has to work to
earn his bread. Indian society has been a patriarchal society where man is the bread earner of
the family. In this patriarchal society a woman has always been subordinate to men. Even the
term “Women” is not independent of the word “Men”. Ever since ages, women were
considered to be the weaker section of society and her role was limited to domestic work at
home.

\noi
However, the modern civilization and globalization has set in considerable changes in the
society especially with regard to the entry of women in work force by providing them with
ample opportunities. This has also resulted in violation of the international as well as national
laws prescribing eight hours of work, which has been discussed below, in order to achieve the
global target thereby conflicting work and family life. The main causes for conflict in worklife balance being gender discrimination, work area ethics and various policies and family
culture, despite the ‘Equal Remuneration Act’, various provisions of Constitution, Factories
Act etc. This has been substantiated by the International Labour Organisation’s Wage Report
and the World Economic Forum’s Global Gender Gap Reports. “The Strategy for New
India\@75” document has also suggested inflexibility in hours of work and absence of parttime works as causes for imbalance in work and family life.

\heading{The International Laws Empowering Women}

\noi
The movement to realise the women electoral rights was started in 1848 and gradually, it
became a mass movement. In the late nineteenth century, women in the United States sought
equal rights, opportunities and liberty for women. The Brandeis\footnote{Muller v. Oregon, 1908 US SC, \url{www.encyclopedia.com/politics/encyclopedias-almanacs-transcripts-andmaps/brandeis-brief}}  case was the first eye opener for the international community with respect to the hours of work of Women. The famous
case of Brandies brief became a landmark case which took place in the United States legal
history. The Supreme Court, in this case, upheld the law of the state of Oregon which
restricted the working hours of women in factories and laundries.

\noi
Women in India realised the importance of the right to vote in the early nineteenth century
which was granted subsequently. In the year 1945, to maintain peace and security, friendly
relations and promote human rights and equal rights internationally, the United Nations
Organisation was established. The year 1975, the International year of women, the first
‘World Conference on women’ was held in Mexico by the United Nations to focus only on
women’s issues. Subsequently, in the very next year, there were discussions on policies and
issues that impact women such as equality in pay, promotion of equal rights and opportunities
for women around the world. The United Nation assented to the ‘Convention on the
Elimination of All Forms of Discrimination Against Women’ in 1979, which ensured
abolition of all practices of discrimination against women. Article 1 of the Treaty defines
discrimination against women as “Any distinction, exclusion or restriction made on the basis
of sex which has the effect or purpose of impairing or nullifying the recognition, enjoyment
or exercise by women, irrespective of their marital status, on a basis of equality of men and
women, of human rights and fundamental freedoms in the political, economic, social,
cultural, civil or any other field”.\footnote{Sylvanna M. Falcon, \textit{Convention on the Elimination of All Forms of Discrimination Against Women,} (December 12, 2020),  \url{https://www.britannica.com/event/Convention-on-the-Elimination-of-All-Forms-ofDiscrimination Against Women.}} Then again in 1995, during the fourth world conference on
women in Beijing, “The Beijing Declaration and the Platform for Action”, was adopted by
189 countries for ‘women empowerment and gender equality’\footnote{\textit{World Conferences on Women,} UNWOMEN, (Dec. 10, 2020), \url{http://www.unwomen.org/en/how-wework/intergovernmental-support/world-conferences-on-women.}}.

\noi
In addition, there are various conventions of International Labour Organisation (ILO) for
women providing gender equality. One of the ILO Conventions, Convention No. 100 of
1951, provides for equal pay for men and women workers for equal work. This is to make
sure that there is no discrimination in the society on the grounds of gender with respect to the
determination of basic or minimum wage and other emoluments in cash or kind by employer
to worker in employment. The Convention No. 111 of 1958, for Discrimination (Employment
and Occupation) excludes discrimination on the basis of race, colour, religion, gender, etc.
which deprives equality of opportunity in occupation and employment and behaviour in
employment. In the Convention no. 156, 1981 of workers with family responsibilities
includes men and women workers. It also includes workers with children, elders who need
their care and such responsibilities restrict them from participating in economic activity. The
problems of workers with additional responsibilities of family and others are wider issues and
in the modern context necessary to be considered in the national policies for implementation.
Convention No. 183 of 2000 for maternity protection ensures safe work for pregnant ladies,
maternity leave of 14 weeks, to protect the health of child and mother, six weeks compulsory
leave after child birth. Prenatal leave can be extended and woman can return to same position
after maternity leave. There can be no termination from employment on the grounds of
pregnancy.

\noi
Convention No. 175 of 1994 on Part-time work, defines part time work as hours of work less
than normal hours of work. It also promises same protection for the part time workers similar
to that of full-time workers such as right to organise and bargain collectively, to form unions,
safety and health of workers, no discrimination in employment and occupations, social
security, paid leave, maternity protection etc. This helps women to continue their work
despite family responsibilities. Home Work Convention No. 177, 1996, also gives an
opportunity to work from home for remuneration with equal treatment like other workers.
Home workers also have the right to join unions, safety, health, social security, training etc.
Convention No. 89 of 1948, Night Work for women (Revised) is not applicable to manager
and health and welfare services but only to those engaged in manual work. There is no
provision for night work for women in public or private industrial undertakings except in
family undertaking. Convention No. 171 of 1990, the Night Work Convention, 1990,
provides for night work which is not less than 7 consecutive hours. This includes the work
done from the midnight to early morning of 5 a.m. This limit on night work for workers
needs to be determined by competent authority after consulting the representative 
organisations of employers and employees and workers or by collective agreements. It is
necessary to take specific measures for night workers to protect their health, assist them to
meet their family and social responsibilities, maternity protection, free health assessment,
alternate work for prenatal and postnatal workers.\footnote{ILO,NORMLEX,\url{https://www.ilo.org/dyn/normlex/en/f\&p=NORMLEX/PUB:12100:0::NO::P12100_INSTRU
MENT_ID:312316} (Dec. 12, 2020).}  In India the night work for women has
not been followed except for the women in IT and ITES sectors. However, “The
Occupational Safety, Health and Working Conditions Code, 2020, provides for night shifts
for women. Women can be employed in all establishments for work between 7 p.m. and 6
a.m. with their consent. The employer has to follow safety measures, working hours and
working conditions as prescribed by the appropriate government while employing women on
nigh shifts”.\footnote{The Occupational Safety, Health and Working Conditions Code, 2020, S. 43, No. 37, Act of Parliament, 2020 (India).} The government notification for IT and ITES companies exempting these
employees from the application of Industrial Employment (standing Orders) Act 1946\footnote{Government of Karnataka, Labour Department, LD 194 LET 2016 (Notified on May 25, 2019).}
which was again extended for another five years from the date of publication of the
notification in the official gazette, also mandates formation of grievance and internal
committees to prevent, prohibit and redress sexual harassment to ensure adequate measures
for their protection. 

\heading{Empowerment of Women at the National Level}

\noi
After the Constitution of India came into force, equal rights were granted to both men and
women under various provisions such as Article 14 dealing with equality before law and all
are equal in the eyes of law. Under Article 15(3) of the Indian Constitution there can be
special provisions for only women and children. Under 73 and 74$^{\rm th}$ Constitutional
Amendment Act, 1992, constitutional status was granted to Panchayati Raj Institutions and
Urban local bodies. And also, as per article 243D (2) of the Constitution of India, there are
one third number of total seats reserved for SC\&ST women in the Panchayats. The
Constitution 108$^{\rm th}$ amendment bill, 2008, prescribes one-third reservation of all seats for
women in the Lok Sabha and State Legislative Assemblies. Under the Directive Principles of
State Policy, Article 39(d) of the Constitution of India provides for equal pay for equal work.
In pursuance of Convention No. 100, 1951, the Equal Remuneration Act, 1976 was brought
into force. The Equal Remuneration Act, 1976 proposes for equal pay for equal work for both 
men and women. It prohibits discrimination against women workers at the time of
recruitment, promotion, training etc. Article 42 of the Constitution of India, stipulates for just
and humane conditions of work and maternity relief. The Maternity Benefit Act, 1961
provides for full paid absence from work. It also provides for a twelve weeks leave in which
not more than six weeks of maternity benefit before delivery can be availed. Maternity
Benefit Amendment Act, 2017 increased paid maternity leave to 26 weeks from 12 weeks out
of which not more than eight weeks maternity leave before delivery can be availed and work
from home options in applicable cases. Even for women adopting baby a below 3 months,
maternity leave of 12 weeks is provided. Article 43 of the Indian Constitution provides for a
living wage and conditions of work ensuring decent standard of life. Various labour
legislations such as Factories Act, 1948, Mines Act, 1952 etc. provide for specific provisions
for safety and welfare of women and also restrict working hours to 8 hours or 9 hours in a
day.

\heading{Globalisation and Women}

\noi
The whole of mankind, irrespective of religion. race, creed or gender, do have the right to
participate and pursue in material well-being and their spiritual development with their
fundamental freedom and dignity, to achieve economic security and equal opportunity as
provided under the ILO Declaration of Philadelphia, 1944.\footnote{ILO, \textit{Gender Equality and Non-Discrimination,} \url{www.ilo.org/legacy/english/inwork/cb-policyguide/declarationofPhiladelphia/944.pdf} (Dec. 12, 2020).} The International Labour
Organisation’s first Convention No. 1, 1919, on working hours limits the working hours to
eight hours in a day. Subsequently, many other conventions related to working time came
into force. However, globalisation induced competitive scenario led to long working hours to
meet the global market requirements. Because of the extended responsibility, the quality and
life balance also were affected. A focus on improvement in working hours can facilitate
compatibility between work life and family life. Long working hours, night shift, lead to
conflicts in happy family life. Women prefer part-time work as they can devote more time to
family responsibilities. The International Labour Organisation’s ‘workers with family
responsibilities’ Convention No. 156, 1981, provides for equal opportunity and equal
treatment for both men and women and it aims to enable persons with family responsibilities
in employment to exercise their right without being discriminated and without any conflict 
between their employment and responsibilities and also to cater to the needs of workers
having family responsibilities by providing child care, family services and facilities.

\noi
Globalisation opened up the trade barriers between various countries. It brought in the global
production and market culture and practice but seemingly left behind the advanced work
culture with the focus on quality work-life balance. It increased competition and the entry of
women in work force. The production and sales target gained more importance than the
health and the well-being of the workers. This also resulted in change in working conditions,
especially, increased working hours to meet production targets for the day or for the year.
There were drastic changes in working hours of workers. The standard 9-5 jobs were
compromised by different working schedules. Cost reduction, long working hours, shift work
and night work are a few important aspects of globalization. Though, as discussed above,
there are umpteen number of legislations in favour of women for their empowerment, we find
women struggling to manage home and office with multifarious responsibilities. As women
are burdened with more responsibilities, it has taken a toll on their career. Therefore, there is
a gap in their salary when compared to that of men. Equal Remuneration Act, 1976, was
enacted with an objective to pay equally for the same kind of work without any
discrimination. Even though this Act has been passed, the economic participation,
opportunities, educational attainment, health and survival and political empowerment remain
a distant dream. According to the wage report of the International Labour Organisation, the
gender remuneration gap is very high in India by International Standards, though it has
decreased from 48\% in 1993-94 to 34\% in 2011-12. The gender wage gap is there among
casual, regular, urban and rural workers. The average remuneration of casual rural female
workers is the lowest in India. Low pay and wage inequality remain a serious challenge to
India’s path to achieving decent working conditions and inclusive growth for women.\footnote{ILO, \textit{India Wage Report: Wage Policies for decent work \& Inclusive growth,} \url{https://www.ilo.org/
wcmsp5/groups/public/asiaa/ro-bangkok/sro-new delhi/ documents/ publication/ wcms_638305.pdf.}} As
per the World Economic Forum, Global Gender Gap Report 2018, India has occupied 142$^{\rm nd}$
position out of 149 countries.\footnote{\textit{India ranks 108$^{\rm th}$ in WEF gender gap index 2018, The Economic Times,}  \url{https://economictimes.indiatimes.com/news/economy/indicators/india_ranks_108th-in-wef-gendergap-index2018/articleshow/67145220.cms/}}

\heading{Women and Work-Life Balance}

\noi
The balance between time allocated to work and other aspects of life i.e., work-life balance
includes family and social life of the worker. The various factors affecting work-life balance
are excessive work load which forces them to work long hours without rest. As a result, they
are unable to manage the responsibilities at home. Because of which their mental well-being
as well as their physical well-being may be affected by stress and anxiety which will hamper
their performance both at home and office leading to work-life imbalance. In France, 35-hour
week legislation was introduced under the reduction of weekly hours of work
recommendation, 1962, (No. 116) with an objective to improve work-life balance did prove
to be effective. Even part time work promotes better work- life balance. As per a Japanese
study in 2010, there has been job stress, shift work and long working hours which has
resulted in a poor work-life conflict and physical as well as mental well-being.

\noi
There are various kinds of work arrangements which can bring in work-life balance in the life
of women. Flexible work schedule can bring in harmonious family environment at home.
Flexible working and part-time working arrangements such as extended lunch breaks to
enable care of elderly relatives, variable hours to enable staff to complete school pick up and
a gradual change in hours to facilitate the return to full time working for parents of young
children.\footnote{Andukari Raj Shravanthi et.al., \textit{Work-Life Balance of women in India,} International Journal of Research in
Management Sciences, 1, (2013) , \url{http://www.researchgate.net/publication/ 266374097_
Work_Life_Balance_of_Women_in_India.}} Workers’ friendly working hours can result in quality of production and increased
output as there will be work-life balance in their lives. Work sharing among workers can
reduce the load on them and thereby reduce stress. However, there are cases where there is a
need, during emergency, for production of essentials due to high demand resulting in increase
in working hours. This can be achieved with increased number of shifts to meet the
requirements. The case mentioned below showcases it.

\noi
In the case of Pfizer (P) Ltd. Bombay v, The Workmen,\footnote{AIR 1963 SC 1103} the appellant company was
manufacturing lifesaving drugs at its factory. The factory had multiple shifts with different
timings and the machinery installed in the factory was not completely used. There was
insufficient production. As a result, the appellant could not meet the demand for its products.
Therefore, it decided to have three shifts to increase production and quality of a particular 
drug P.A.S. The appellant issued notice to the respondents to introduce three shifts to
increase production.

\noi
Conciliation efforts failed and the matter was referred to Industrial Tribunal. The Tribunal
gave its award against the appellant. The Tribunal held that the three shifts was inconvenient
to the workers. The workers will be compelled to work at night and better quality of products
will not be produced. The Tribunal also held that production of the drug known as P.A.S. did
not require continuous working in three shifts. But the Tribunal reduced the number of
holidays from 27 to 10.

\noi
The Indian Supreme Court held that the appellant be allowed to introduce three shifts in the
factory. The process of manufacture of the drug P.A.S. was continuous and as it took 20
hours, three shifts were inevitable. In order to improve the quality and avoid rejection of the
products, it was important that the shifts system should have three shifts. By introducing three
shifts, both quality and quantity will improve. Three shifts were also allowed for
Pharmaceutical Departments which produced ointment, injections, other pharmaceutical
products, packing, filling, washing, tablet and capsules three shifts were allowed. The
objection of the workers that three shifts would involve work at night and hence was not
desirable was rejected. Another objection that the introduction of three shifts would involve
the beginning of the work at 7-20 a.m. which was an unduly early hour for work, was also
rejected. The honourable court rejected the contention of the appellant that the standing
orders stipulated more than one shift, it was entirely in the discretion of the management to
carry out changes without the due diligence by industrial adjudication. While allowing the
introduction of three shifts, the court considered the importance and necessity of more
production of the drug and the court was influenced by the existence of emergency in the
country. The Supreme Court increased the number of paid holidays annually from 10 to 16
and reduced the number of public holidays to 16 every year. Both appeals were allowed.

\noi
Another consequence of long working hour is suicides of workers. Suicides due to pressure
of work are increasing all over the world. Recent studies in the United States, Australia,
Japan, South Korea, China, India and Taiwan have shown an increase in suicides due to
deterioration in working conditions. A lot of changes have taken place due to globalisation
which has changed their working patterns. Earlier, there were strong trade unions in
industries to demand their rights and working conditions. But now, there exists the job
insecurity, heavy work load, etc. In Foxconn Technology Group in China, eighteen migrant 
workers who were of the age between 17 and 25 attempted suicide at one of Foxconn’s main
factories in 2010. Fourteen of them died. They were working on assembly lines
manufacturing electronic gadgets for multinational companies such as Dell, Sony, Apple etc.
One of the women suicide survivors of Foxconn, a seventeen-year-old girl, Tian Yu, said that
she was forced to work 12 hour shifts without meals to work overtime and had only one day
off every second week. Subsequently, Apple published a set of standards for treatment of
workers. In a documentary ‘Apple’s Broken Promises’ by BBC, it was shown that exhausted
workers were asleep on 12 hour shifts and workers pressurized by managers at new supplier,
Pegatron Shanghai, latest iPhone assemblers. Apple states that it monitors its supplier’s
practices with its annual supplier responsibility reports. However, the labour rights activists
and researchers continue to allege that the workers in Apple’s supply chains are abused.\footnote{Sarah Waters and Jenny Chan, \textit{Workplace Suicides are rising and globalisation is to blame,} NEWSWEEK
(Dec 14,2020), \url{https://www.newsweek.com/workplace=suicides-are-rising-and-ceos-are-blame-490941.}}

\heading{Judiciary on Gender Equality}

\heading{Case 1}

\noi
Vishaka \& Others v. State of Rajasthan,\footnote{(1997) 6 SCC 241} is a landmark case on prevention of sexual
harassment at work place. A social worker was gang raped in a village in Rajasthan. The writ
petition was filed by certain social activists and NGOs as a class action for securing the true
principle of gender equality and to protect women from sexual harassment in all work places
through a defined process. Also, for ensuring the protection of the fundamental rights of
working women under articles 14, 19 and 21 of the Indian Constitution.

\noi
The Supreme Court issued certain guidelines to govern the behaviour of employers and
others at work place. Gender equality includes protection from sexual harassment which is a
universally recognised basic human right. In the absence of specific law to protect women
from sexual harassment at workplace, the Supreme Court issued the following guidelines:

\begin{enumerate}
\item It is the duty of the employer or other responsible persons to prevent acts of sexual
harassment and provide procedures to resolve, prosecute such activities.

\item Express prohibition of sexual harassment at workplace by notification, circular.

\item Rules of public sector include prohibitions and penalties against the offenders.

\item Standing orders under Industrial Employment Standing Orders Act, 1946 to include
prohibition of sexual harassment at workplace.

\item Proper working conditions, work leisure, health and hygiene to be provided and there
should not be any hostile environment to women.

\item Complaints for such acts under Indian Penal Code.

\item Complaint mechanism to be provided.\footnote{\url{https//indiankanoon.org/doc/1031794.}}
\end{enumerate}

\noi
Subsequently, the Sexual Harassment of Women at Workplace (Prevention, Prohibition and
Redressal) Act, 2013 came into force.

\heading{Case 2}

\noi
Mackinnon Mackenzie \& Co. Ltd. v. Audrey D’Costa \& Another\footnote{(1987) 2 SCC 469} On an application by the
respondent under section 7 of the Equal Remuneration Act, 1976 before the authority on the
ground that the total wages of confidential lady stenographers were less than the male
stenographers in the general pool who were performing the same duties and it amounted to
gender discrimination under the Act. The authority held that the respondent was not entitled
to the relief as she was not paid at rates less favourable than those paid to male stenographers.
On appeal, the appellate authority held that there was discrimination between male and
female stenographers and the respondent was entitled to reliefs. The petitioner petitioned
under article 226 of the Constitution of India challenging the order of the Appellate
Authority. The Supreme Court upheld the order of the Appellate Authority and the matter
was remitted back to the Appellate Authority, the Deputy Commissioner of Labour
(Enforcement), under the Equal Remuneration Act, 1976 for computing the amount payable
to the respondent.

\heading{Factors Affecting Balance of Work and Family}

\noi
The India Strategy\@75 report of Niti Aayog\footnote{NITI AAYOG, \textit{Strategy for New India}\@75, \url{http://niti.gov.in/sites/default/files/2019- 01/Strategy_for_New_India_0.pdf}} indicates that the female Labour Force
Participation Rate (LFPR) has reduced from 23.7 (26.7\% in rural areas and 16.2\% in urban areas) more so in the rural areas where it has gone down from 49\% to 26.7\%. Because of which she needs to balance between her professional commitment and the domestic
responsibility. At times, it is challenging and most of them opt out of the professional role to
manage the domestic chores. This fact has weakened the cause of women independence in
India. To manage both the roles, an effective balance is essential between the professional
and personal domestic responsibilities. Balancing to have an effective equilibrium between
the personal and the professional commitment is what is dealt as the subject of ‘work- life
balance’. There are three factors which generally affect the work- life balance. One is the
gender discrimination, work area ethics and policies and the family culture. When these three
different areas are not complementing with each other in its performance, there is a conflict to
disturb the work-life balance leading to high stress. To march ahead in this highly
competitive world and more so the Indian demographic scene, it is needed to demonstrate the
reduction of excessive work hours for women and increasing the time spent at home with
their family.

\noi
In the present shrinking world, the work stress due to the lack of balance has led to the high
levels of stress among women. A few Indian Corporates have identified this challenge and
have undertaken many measures to balance the work-life but many remain blind to this fact
in the race to achieve higher and higher targets for profit maximization. The globalization did
bring the employment opportunities but it has taken away the happiness from the dining hall
of our families. In the present-day, women is employed in many Corporates, more so in the
Information Technology and Information Technology Enabled Services. In both these
sectors, the working hours framework has been a challenge for the job seeking employees.
They have to keep working in silence because if they raise issues they may be sacked by their
bosses. Though there are many safeguards for the women working in Factories the same has
been compromised in the Information Technology and Information Technology Enabled
Services Sector. These companies have been exempted from the purview of the labour laws
vide the Karnataka Government notification LD 53 LET 2013 dated January 25, 2014.

\noi
In an empirical analysis of call centre employees carried out to examine the intricate
relationship between the structure and the employees, to withstand long hours of work, it was
observed to undergo certain transformations in the characteristics. The study also revealed the 
cultural transformation of urban Indian Labour into a global proletariat. In the participant
observation conducted at International Tech Park, Bangalore, it was found that 53 per cent.
workers worked in nine hours’ shift while 29 per cent worked in eight hours’ shift and 3 per
cent were working in 10-12 hours shifts with a break of one hour. The schedule is same for
sweatshop labourers who worked for MNC ancillary factories on the outskirts of Bangalore
city. Sometimes, they had to work for 15 hours due to the shortage of staff. It was also
observed that when call flow is high, group leaders do not let them take breaks.\footnote{Divya C. McMillin, \textit{Outsourcing Identities: Call Centres and Cultural Transformation in India,} 41, Economic and Political Weekly, 235-241, (2006).} The shift
working hours of the women working for Information Technology Enabled Services has
seriously affected their work-life balance. And also, the project related works in the
Information Technology organisation has also life-threatening challenges for women to work
alone during holidays and off -work time. This has exposed women to mental stress and also
physical threat also. Even though some companies have taken precautions, overall, the issues
remain still a challenge without any positive commitment from the management of many
companies. The law also has been compromised by giving exemption by the policy makers in
the pretext of employment generation.

\noi
Research shows that flexible work arrangements allow individuals to integrate work and
family responsibilities in time and space and are instrumental in achieving a healthy work and
family balance. The presence of large number of women in their workforce and their drive for
careers have resulted in increasing attention to work and family balance issue.\footnote{Reimara Valk \& Vasanthi Srinivasan, \textit{Work-Family Balance of Indian Women in Software Profession: A
Qualitative Study,} (Dec 12), \url{www.sciencedirect.com/science/article/S0970389610000832.}}

\heading{Government’sovernment’s Initiative to Empower Women}

\noi
The NITI Aayog (National Institution for Transforming India), is the Indian Government’s
think tank to provide advice on specialized subjects for innovation and entrepreneurial
support to the Government of India, It has drafted a document for the purpose titled, ‘The
Strategy for New India\@75”\footnote{NITI AAYOG, \url{https://niti.gov.in/writereaddata/files/Strategy_for_New_India (Dec. 12, 2020).}} which states that the women employment needs to be
increased. It also advised on the increased Maternity Benefit Leave under the new Maternity
Benefit (Amendment) Act, 2017, and the Sexual Harassment of Women at Workplace
(Prevention, Prohibition and redressal) Act, 2013, which have empowered and ensured the
balance of the work stress at the working place. It redresses inequalities on gender bias. The document also states that in the provision of the ‘Swacch Bharat’ mission compulsory toilets
for women at work place and educational institutions needs to be initiated and their working
conditions needs to be improved. It also covers the gender issues in the workplace, by
committing to create and strengthen the institutional and structural barriers to enhance the
female working environment and participation. The document further concludes by
identifying the constraints as the inflexibility in working hours, lack of availability of
creches, safety etc. The absence of opportunities for part-time work has also been identified
as a factor affecting the work- life balance.\footnote{\textit{Id..}}

\heading{Conclusion and Suggestions}

\noi
Every worker gets pressurized by long working hours without rest and breaks. It takes a toll
on their health. Especially, women have more stress and anxiety due to dual role played by
them. Worker friendly non-standard work arrangement can reduce conflicts between work
and family and reduce stress leading to a harmonious family life.

\noi
Three broad types of work life balance strategies have been created to help employees
balance their work and non-work lives. Flexible work options, specialised leave policies and
dependent care benefits. The flexible work options have become once again a topic of
importance in the COVID-19 situation where companies and the employees have taken up
work from home policies as the thing of the present and future. Flexitime, Job Sharing, Home
telecommuting, Work-at-Home Programs, Part-time Work, Shorter work days for parents,
Bereavement of Leave, Paid Maternity Leave, Paid leave to care for sick family members,
Paternity Leave.\footnote{Supra note 11.} should be looked into as the options for balance in the changing
environment of work culture more so for the working women.

\noi
Flexibility in work schedule options help women perform better and can ensure good health.
There should be appropriate laws to protect women from working excessively long hours and
unsocial working hours. There should be gap between the actual hours of work and preferred
hours of work for women. The exemption of the Industrial Employment (Standing Orders)
Act, 1946, for Information Technology and Information Technology Enabled Services should
be reviewed or modified to suit the conditions and ensure the compliance with applicable
laws on working time and adhere to the standards to stop exploitation of women. All the industries need to provide the dependent care assistance, child care facilities, laundry facilities and canteen facilities. 

\noi
Women have to set their priorities in their professional as well as family lives to strike a
balance between their work and family. They need support at work as well as at home.
Policies and practices at work place should be supportive of women to carry out their dual
responsibilities without any conflict and lead a peaceful life. The governments also need to
relook into these above concepts of work from home and the flexible working time in the
present era of technology and the Pandemic. Women being the torchbearers of society need
all the help from the national and international law makers to ensure balanced work and
quality life for a better future generation.

\end{multicols}
