\setcounter{figure}{0}
\setcounter{table}{0}
\setcounter{footnote}{0}


\articletitle{Globalisation and its Implication on Law Relating to Women and Their Work-Life Balance}
\articleauthor{Dr. N. Dasharath\footnote{Professor and Chairman, Dean \& Faculty of Law, University Law College and Department of Studies in Law, Bangalore University} and  Sujatha V. Durgekar\footnote{Research Scholar, University Law College and Department of Studies in Law, Bangalore University}}
\lhead[\textit{\textsf{Dr. N. Dasharath and Sujatha V. Durgekar}}]{}
\rhead[]{\textit{\textsf{Globalisation and its Implication on Law...}}}

\begin{multicols}{2}

\heading{Introduction}

\noi
Indian women have been revered in the Indian society giving them the form of Goddesses. In
all the Hindu scriptures, she is revered and worshiped as an all-empowering force. The
Ramayana and the Mahabharata, both the stories were woven around the women only. But
from time immemorial, she has been confined to domestic work. Work and life are
interrelated. It is impossible to live a life without work. Every human being has to work to
earn his bread. Indian society has been a patriarchal society where man is the bread earner of
the family. In this patriarchal society a woman has always been subordinate to men. Even the
term “Women” is not independent of the word “Men”. Ever since ages, women were
considered to be the weaker section of society and her role was limited to domestic work at
home.

\noi
However, the modern civilization and globalization has set in considerable changes in the
society especially with regard to the entry of women in work force by providing them with
ample opportunities. This has also resulted in violation of the international as well as national
laws prescribing eight hours of work, which has been discussed below, in order to achieve the
global target thereby conflicting work and family life. The main causes for conflict in worklife balance being gender discrimination, work area ethics and various policies and family
culture, despite the ‘Equal Remuneration Act’, various provisions of Constitution, Factories
Act etc. This has been substantiated by the International Labour Organisation’s Wage Report
and the World Economic Forum’s Global Gender Gap Reports. “The Strategy for New
India\@75” document has also suggested inflexibility in hours of work and absence of parttime works as causes for imbalance in work and family life.

\heading{The International Laws Empowering Women}

\noi
The movement to realise the women electoral rights was started in 1848 and gradually, it
became a mass movement. In the late nineteenth century, women in the United States sought
equal rights, opportunities and liberty for women. The Brandeis\footnote{Muller v. Oregon, 1908 US SC, \url{www.encyclopedia.com/politics/encyclopedias-almanacs-transcripts-andmaps/brandeis-brief}}  case was the first eye opener for the international community with respect to the hours of work of Women. The famous
case of Brandies brief became a landmark case which took place in the United States legal
history. The Supreme Court, in this case, upheld the law of the state of Oregon which
restricted the working hours of women in factories and laundries.

\noi
Women in India realised the importance of the right to vote in the early nineteenth century
which was granted subsequently. In the year 1945, to maintain peace and security, friendly
relations and promote human rights and equal rights internationally, the United Nations
Organisation was established. The year 1975, the International year of women, the first
‘World Conference on women’ was held in Mexico by the United Nations to focus only on
women’s issues. Subsequently, in the very next year, there were discussions on policies and
issues that impact women such as equality in pay, promotion of equal rights and opportunities
for women around the world. The United Nation assented to the ‘Convention on the
Elimination of All Forms of Discrimination Against Women’ in 1979, which ensured
abolition of all practices of discrimination against women. Article 1 of the Treaty defines
discrimination against women as “Any distinction, exclusion or restriction made on the basis
of sex which has the effect or purpose of impairing or nullifying the recognition, enjoyment
or exercise by women, irrespective of their marital status, on a basis of equality of men and
women, of human rights and fundamental freedoms in the political, economic, social,
cultural, civil or any other field”.\footnote{Sylvanna M. Falcon, \textit{Convention on the Elimination of All Forms of Discrimination Against Women,} (December 12, 2020),  \url{https://www.britannica.com/event/Convention-on-the-Elimination-of-All-Forms-ofDiscrimination Against Women.}} Then again in 1995, during the fourth world conference on
women in Beijing, “The Beijing Declaration and the Platform for Action”, was adopted by
189 countries for ‘women empowerment and gender equality’\footnote{\textit{World Conferences on Women,} UNWOMEN, (Dec. 10, 2020), \url{http://www.unwomen.org/en/how-wework/intergovernmental-support/world-conferences-on-women.}}.

\noi
In addition, there are various conventions of International Labour Organisation (ILO) for
women providing gender equality. One of the ILO Conventions, Convention No. 100 of
1951, provides for equal pay for men and women workers for equal work. This is to make
sure that there is no discrimination in the society on the grounds of gender with respect to the
determination of basic or minimum wage and other emoluments in cash or kind by employer
to worker in employment. The Convention No. 111 of 1958, for Discrimination (Employment
and Occupation) excludes discrimination on the basis of race, colour, religion, gender, etc.
which deprives equality of opportunity in occupation and employment and behaviour in
employment. In the Convention no. 156, 1981 of workers with family responsibilities
includes men and women workers. It also includes workers with children, elders who need
their care and such responsibilities restrict them from participating in economic activity. The
problems of workers with additional responsibilities of family and others are wider issues and
in the modern context necessary to be considered in the national policies for implementation.
Convention No. 183 of 2000 for maternity protection ensures safe work for pregnant ladies,
maternity leave of 14 weeks, to protect the health of child and mother, six weeks compulsory
leave after child birth. Prenatal leave can be extended and woman can return to same position
after maternity leave. There can be no termination from employment on the grounds of
pregnancy.

\noi
Convention No. 175 of 1994 on Part-time work, defines part time work as hours of work less
than normal hours of work. It also promises same protection for the part time workers similar
to that of full-time workers such as right to organise and bargain collectively, to form unions,
safety and health of workers, no discrimination in employment and occupations, social
security, paid leave, maternity protection etc. This helps women to continue their work
despite family responsibilities. Home Work Convention No. 177, 1996, also gives an
opportunity to work from home for remuneration with equal treatment like other workers.
Home workers also have the right to join unions, safety, health, social security, training etc.
Convention No. 89 of 1948, Night Work for women (Revised) is not applicable to manager
and health and welfare services but only to those engaged in manual work. There is no
provision for night work for women in public or private industrial undertakings except in
family undertaking. Convention No. 171 of 1990, the Night Work Convention, 1990,
provides for night work which is not less than 7 consecutive hours. This includes the work
done from the midnight to early morning of 5 a.m. This limit on night work for workers
needs to be determined by competent authority after consulting the representative 
organisations of employers and employees and workers or by collective agreements. It is
necessary to take specific measures for night workers to protect their health, assist them to
meet their family and social responsibilities, maternity protection, free health assessment,
alternate work for prenatal and postnatal workers.\footnote{ILO,NORMLEX,\url{https://www.ilo.org/dyn/normlex/en/f\&p=NORMLEX/PUB:12100:0::NO::P12100_INSTRU
MENT_ID:312316} (Dec. 12, 2020).}  In India the night work for women has
not been followed except for the women in IT and ITES sectors. However, “The
Occupational Safety, Health and Working Conditions Code, 2020, provides for night shifts
for women. Women can be employed in all establishments for work between 7 p.m. and 6
a.m. with their consent. The employer has to follow safety measures, working hours and
working conditions as prescribed by the appropriate government while employing women on
nigh shifts”.\footnote{The Occupational Safety, Health and Working Conditions Code, 2020, S. 43, No. 37, Act of Parliament, 2020 (India).} The government notification for IT and ITES companies exempting these
employees from the application of Industrial Employment (standing Orders) Act 1946\footnote{Government of Karnataka, Labour Department, LD 194 LET 2016 (Notified on May 25, 2019).}
which was again extended for another five years from the date of publication of the
notification in the official gazette, also mandates formation of grievance and internal
committees to prevent, prohibit and redress sexual harassment to ensure adequate measures
for their protection. 

\heading{Empowerment of Women at the National Level}

\noi
After the Constitution of India came into force, equal rights were granted to both men and
women under various provisions such as Article 14 dealing with equality before law and all
are equal in the eyes of law. Under Article 15(3) of the Indian Constitution there can be
special provisions for only women and children. Under 73 and 74$^{\rm th}$ Constitutional
Amendment Act, 1992, constitutional status was granted to Panchayati Raj Institutions and
Urban local bodies. And also, as per article 243D (2) of the Constitution of India, there are
one third number of total seats reserved for SC\&ST women in the Panchayats. The
Constitution 108$^{\rm th}$ amendment bill, 2008, prescribes one-third reservation of all seats for
women in the Lok Sabha and State Legislative Assemblies. Under the Directive Principles of
State Policy, Article 39(d) of the Constitution of India provides for equal pay for equal work.
In pursuance of Convention No. 100, 1951, the Equal Remuneration Act, 1976 was brought
into force. The Equal Remuneration Act, 1976 proposes for equal pay for equal work for both 
men and women. It prohibits discrimination against women workers at the time of
recruitment, promotion, training etc. Article 42 of the Constitution of India, stipulates for just
and humane conditions of work and maternity relief. The Maternity Benefit Act, 1961
provides for full paid absence from work. It also provides for a twelve weeks leave in which
not more than six weeks of maternity benefit before delivery can be availed. Maternity
Benefit Amendment Act, 2017 increased paid maternity leave to 26 weeks from 12 weeks out
of which not more than eight weeks maternity leave before delivery can be availed and work
from home options in applicable cases. Even for women adopting baby a below 3 months,
maternity leave of 12 weeks is provided. Article 43 of the Indian Constitution provides for a
living wage and conditions of work ensuring decent standard of life. Various labour
legislations such as Factories Act, 1948, Mines Act, 1952 etc. provide for specific provisions
for safety and welfare of women and also restrict working hours to 8 hours or 9 hours in a
day.

\heading{Globalisation and Women}

\noi
The whole of mankind, irrespective of religion. race, creed or gender, do have the right to
participate and pursue in material well-being and their spiritual development with their
fundamental freedom and dignity, to achieve economic security and equal opportunity as
provided under the ILO Declaration of Philadelphia, 1944.\footnote{ILO, \textit{Gender Equality and Non-Discrimination,} \url{www.ilo.org/legacy/english/inwork/cb-policyguide/declarationofPhiladelphia/944.pdf} (Dec. 12, 2020).} The International Labour
Organisation’s first Convention No. 1, 1919, on working hours limits the working hours to
eight hours in a day. Subsequently, many other conventions related to working time came
into force. However, globalisation induced competitive scenario led to long working hours to
meet the global market requirements. Because of the extended responsibility, the quality and
life balance also were affected. A focus on improvement in working hours can facilitate
compatibility between work life and family life. Long working hours, night shift, lead to
conflicts in happy family life. Women prefer part-time work as they can devote more time to
family responsibilities. The International Labour Organisation’s ‘workers with family
responsibilities’ Convention No. 156, 1981, provides for equal opportunity and equal
treatment for both men and women and it aims to enable persons with family responsibilities
in employment to exercise their right without being discriminated and without any conflict 
between their employment and responsibilities and also to cater to the needs of workers
having family responsibilities by providing child care, family services and facilities.

\noi
Globalisation opened up the trade barriers between various countries. It brought in the global
production and market culture and practice but seemingly left behind the advanced work
culture with the focus on quality work-life balance. It increased competition and the entry of
women in work force. The production and sales target gained more importance than the
health and the well-being of the workers. This also resulted in change in working conditions,
especially, increased working hours to meet production targets for the day or for the year.
There were drastic changes in working hours of workers. The standard 9-5 jobs were
compromised by different working schedules. Cost reduction, long working hours, shift work
and night work are a few important aspects of globalization. Though, as discussed above,
there are umpteen number of legislations in favour of women for their empowerment, we find
women struggling to manage home and office with multifarious responsibilities. As women
are burdened with more responsibilities, it has taken a toll on their career. Therefore, there is
a gap in their salary when compared to that of men. Equal Remuneration Act, 1976, was
enacted with an objective to pay equally for the same kind of work without any
discrimination. Even though this Act has been passed, the economic participation,
opportunities, educational attainment, health and survival and political empowerment remain
a distant dream. According to the wage report of the International Labour Organisation, the
gender remuneration gap is very high in India by International Standards, though it has
decreased from 48\% in 1993-94 to 34\% in 2011-12. The gender wage gap is there among
casual, regular, urban and rural workers. The average remuneration of casual rural female
workers is the lowest in India. Low pay and wage inequality remain a serious challenge to
India’s path to achieving decent working conditions and inclusive growth for women.\footnote{ILO, \textit{India Wage Report: Wage Policies for decent work \& Inclusive growth,} \url{https://www.ilo.org/
wcmsp5/groups/public/asiaa/ro-bangkok/sro-new delhi/ documents/ publication/ wcms_638305.pdf.}} As
per the World Economic Forum, Global Gender Gap Report 2018, India has occupied 142$^{\rm nd}$
position out of 149 countries.\footnote{\textit{India ranks 108$^{\rm th}$ in WEF gender gap index 2018, The Economic Times,}  \url{https://economictimes.indiatimes.com/news/economy/indicators/india_ranks_108th-in-wef-gendergap-index2018/articleshow/67145220.cms/}}


\end{multicols}
