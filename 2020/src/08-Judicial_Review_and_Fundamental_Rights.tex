\setcounter{figure}{0}
\setcounter{table}{0}
\setcounter{footnote}{0}


\articletitle{Judicial Review and Fundamental Rights : Key Features of Constitutionalism}
\articleauthor{Priyanka Choudhary\footnote{Assistant Professor, School of Law, University of Petroleum and Energy Studies, Dehradun} and Kush Kalra\footnote{Research Scholar, Sharda University, Greater Noida}}
\lhead[\textit{\textsf{Priyanka Choudhary and Kush Kalra}}]{}
\rhead[]{\textit{\textsf{Judicial Review and Fundamental Rights...}}}

\begin{multicols}{2}

\heading{Introduction}

\noi
While being perceived as a composed record\footnote{The U.K. continues to operate without a written constitution. Similarly, the Hungarian Kingdom of the AustroHungarian Monarchy (and until 1945) was without a written constitution, and yet it qualified as a constitutional
state in its time, with a number of important statutory documents, charters, and treaties. Today Israel and New
Zealand have written bits and pieces of ordinary laws which deal with constitutional issues but without
entrenchment.}  that recommends the plan of government, the
Constitution is considered less as a sanction of the relations among social entertainers. The
legitimate Constitution subordinates legislative issues to law. Likewise, political
arrangements are to be found and defended inside legitimate edges, and arrangements are
reached through lawful cycles (for example in protected mediation). Yet, the lawful
Constitution doesnot end the private connection among Constitution and constitutionalism:
from a material perspective, a constitution is a constitution decisively on the grounds that it
fulfils the rudimentary assumptions for constitutionalism. The custom that constitutes formal
authoritative records and constitutionalist assumptions are interrelated.

\noi
The lawful idea of the Constitution implies that it turns out to be essential for the overall set
of laws and needs to fulfil the proper states of present-day law\footnote{According to the advocates of the unwritten constitution, a charter is too rigid, while the constitution that
manifests itself in traditions enables a more flexible approach. That the judges have nothing to apply is more of
an advantage, because it upholds the separation of the branches of power, inasmuch as it excludes the possibility
of government by judges at the same time}. As a composed authoritative
report, it is fit to legal legitimate application. In law-focused present-day states, constitutions
accept commonness in the legitimate circle.

\noi
Constitutions are legitimately restricting, yet they are more adaptable than a standard rule
with restricted ability to figure out what will occur in its name. In numerous regards, they are
just casings. Furthermore, it isn't just that the casing is regularly loaded up with sudden
substance, yet additionally that the very edge may change its shape. 'The Constitution is just
to give a chance through which a framework may create.

\heading{Concept of Constitution and\\ Constitutionalism}

\noi
The term ‘constitution,’ or its equivalent in other languages, existed long before modern
constitutions emerged. But it designated a different object. Originally, it used to describe the
state of the human body, it was soon applied to the body politic, yet not in a normative sense
but as a description of the situation of a country as determined by a number of factors such as
its geography, its climate, its population, its laws etc. In the eighteenth century, the meaning
was often narrowed to the state of a country as determined by its basic legal structure. But
still the notion ‘constitution’ was not identified with those laws. Rather, the term continued to
describe the state of a country insofar as it was shaped by its basic laws. However, the basic
laws themselves were not the ‘constitution’ of the country. ‘Constitution’ remained a
descriptive, not a prescriptive, term.\footnote{D. Grimm, Types of Constitutions, 98, in M. Rosenfeld and A. Sajó, eds. The Oxford Handbook of Comparative Constitutional Law (Oxford University Press, 2012) 100}

\noi
A constitution is a “charter of government deriving its whole authority from the governed.\footnote{Black’s Law Dictionary.} The constitution sets out the form of the government. It specifies the purpose of the
government, the power of each department of the government, the state society relationship,
the relationship between various governmental institutions, and the limits of the
government.” Today a constitution is easily identified with a legal document of the same
name, arranging public institutions of government.

\noi
Constitutionalism stands for a set of interrelated concepts, principles, and practices of
organizing and thereby limiting government power in order to prevent despotism. It suggests
that power may be limited by techniques of separation of powers, checks and balances, and
the protection of fundamental rights along a pre-commitment. It seeks to provide adequate
institutional design to cool passions without forfeiting government efficiency. By formalizing
these solutions in a legally binding instrument (the constitution), constitutionalism provides
the necessary limitations of government (sovereign) power and affirms the legitimate
exercise thereof.

\noi
Constitutionalism is often described as a liberal\footnote{‘Liberal’ in this book is used in its nineteenth-century European sense (‘classic liberalism’), meaning emphasis on individual liberty and the free market as an extension of this freedom and designing the defence of liberty against successive threats. Liberalism can be a political philosophy; as a political movement it animated
constitution writing and it was a nationalist movement in many nineteenth-century societies. Liberalism is
intimately related to constitutionalism. Liberal in U.S. political usage is close to ‘progressive’, social
democratic, or welfarist in the European sense.} political philosophy that is concerned with
limiting government. Consequently, it is attacked for weakening the government when the state needs to be strong. Limiting what government can do, however, does not necessarily
result in a weaker state. A community may need a government that is strong enough to
defend it from its enemies. Beyond this point ‘strength’ is of little assistance. At first glance
a government seems weak where the streets are not safe. But the U.S. is a country with a high
incidence of violent crime: is the U.S. a weak state? In certain dictatorships there are
policemen around every corner and the crime rate is low, so one would say that these are
strong states. Yet, such strength and security are of dubious value where the police use their
position to induce fear or extract bribes from the population. In short, strength is not an
analytically helpful category for the study of constitutions and governments. Efficiency is a
completely different matter.

\vspace{.5cm}

\noi
Viable constitutions are pragmatic. Even revolutionary constitutions reflect concessions and
actual compromises that enable the peaceful co-existence of different groups, including
minorities and losers. A critical revisionist would say that constitutions are either victors’
justice or—more often—dirty deals to protect the interests of elites which feel that they are
losing their privileged position or face uncertain political outcomes.\footnote{Note that, in contrast to this criticism, many of the contemporary social values which were granted constitutional status and priority are not directly elitist: social rights and anti-poverty and equality programmes in the constitution may be intended by elites to deceive the public, but technically these are not about privileges of the elite of the day}  Rights and strong remedies to cure the violations of rights are granted to all, not for the sake of
constitutionalism’s liberty, but simply in order to protect these elites from being called to
account and loss of status in the future. Constitutions may be deals that consolidate the
political power of elites. And yet, the resulting constitution may still serve the community as
a whole (although often at the expense of certain groups living in that community).

\heading{Principle of Constitutionalism}

\noi
Constitutionalism often is regarded as a doctrine of political legitimacy. Constitutionalism
prima facie requires justification of state actions against a higher law. At its core, this higher
law is meant to structure the political process. Yet, as a concept, constitutionalism involves
more than mere legality; it aims to posit a wider and deeper criterion of good governance as
well as political conventions and norms to be attained in the collective life of a nation. The  central principle in constitutionalism is the “respect for human worth and dignity. It is by no
means a static principle and the core elements identified are bound to change as better ways
are devised to limit government and protect citizens, it is the institutionalization of these core
elements that matter. Nevertheless, constitutionalism needs to be distinguished from both
democracy and the rule of Law.\footnote{\textit{Supra} Note 6}

\vspace{.5cm}

\heading{Fundamental Rights and Constitutionalism}


\vspace{-.3cm}

\noi
\begin{quoting}
“Fundamental rights should be such that they should not be liable to
reservation and to changes by Acts of legislature” -Begum AizazRasul\footnote{Constitutional Asembly Debates, 264 Vol. VII, 1948.}
\end{quoting}

\vspace{-.3cm}

\noi
What kind of rights would one need in order to ensure freedom in a political system? What
follows, if a right is claimed as ‘fundamental’? Who is bound by it? The government, or the
citizens, too? And what does ‘being bound’ mean: to honour the claim, or non-interference,
or the unconstrained activity of the holder of the right? Or the protection and promotion of
the right by the government? Could individuals or the authorities prevent anyone from
obstructing an action that is based on a right? Shall the government call to account the
violators of such rights? These are some of the questions that a constitution-maker and
constitutional practice have to answer regarding fundamental rights.

\noi
The constitutional recognition of fundamental rights reflects a presumption in favour of the
primacy of liberty. It expresses a social agreement and promises that the government will
operate for the sake of free individuals. Fundamental rights are constitutionalized to counter
majoritarian and statist bias. Sadly, the value and primacy of freedoms is far from selfevident, especially when it comes to the freedom of others, especially different others (be
they intellectuals, sexual or ethnic minorities, or believers of another religion). To stand up
for the freedoms of these other’s is hardly ‘natural’. Freedoms are vulnerable, especially
where the resulting behaviour is unusual and repellent to traditional feelings. Liberty is not a
matter of popularity, modesty, or courtesy. There are important moral reasons to respect
freedom and the capacity of humans to choose the good life they like. 

\noi
While a human or fundamental right claim indicates priority, the basis for the claim remains
contested. At the time when fundamental rights were incorporated into the U.S. Constitution
or the 1789 French Declaration, they may have had a narrow scope, but they were considered
a matter of unconditional respect of the individual stemming from the nature of man (human being) or the nature of things (natural law). In a modern and also a much earlier (medieval)
approach, these rights emanate from the equal dignity of humans that is to be unconditionally
respected in the political community. Or, in a different perspective, all human beings have
human rights simply by virtue of their existence as equal moral beings.\footnote{‘The source of human rights is man’s moral nature . . .’ J. Donnelly, Universal Human Rights in Theory and
Practice, 3rd ed. (Cornell University Press, 2013) 15.}

\noi
The moral reading of fundamental rights blames the alternative consequentialist
understanding as undermining the primacy of the individual who shall be the only measure of
humans. Interestingly, there are certain justifications in international human rights law which
refer to an instrumental concept of human rights, granting rights a status that is nevertheless
hard to undermine on standard consequentialist grounds. For example, the Preamble to the
Universal Declaration of Human Rights construes human rights as indispensable against
barbarism: ‘disregard and contempt for human rights have resulted in barbarous acts which
have outraged the conscience of mankind.’ Likewise, the French Declaration stated already in
1789 ‘that the ignorance, neglect, or contempt of the rights of man is the sole cause of public
calamities and of the corruption of governments’. These arguments indicate how human
(fundamental) rights fit into the programme of constitutionalism as anti-despotism.

\heading{What do the Fundamental Rights Imply?}

\noi
What follows from the constitutional requirement that freedom is the rule, and its limitation is
the exception? As a minimum, it means respect of the maxim: “That Which Is Not Forbidden
Is Permitted”. Legislation must respect liberty. The government must have good, valid, even
compelling reasons, if it wishes to prohibit a conduct. It can regulate, restrict, or prohibit
what in itself does not harm anyone only if it is specifically authorized. The constitutional
recognition of rights changes the nature of the political discourse and legitimate action.
Certain arguments which are disrespectful of the fundamental rights are difficult to make, and
become easy prey to the argumentum ad Hitlerum: who praises censorship, denies the
importance of independent courts, or praises racial discrimination will be compared to Hitler,
a parallel which should have (or at least used to have in principle) annihilating effects for the
targeted position. Human rights operate as conversation stoppers, representing the ultimate
incontestable common values of the political community. Even censors have to stand up for
freedom of speech and introduce restrictions only in the name of facilitating a better exchange of ideas. The contemporary attempts to dethrone human rights are intended to
change the prominent cultural power of the fundamental rights.

\noi
The fundamental rights bind the State, but what does this bond mean? To a certain extent the
government has a duty to guarantee the enforcement of the rights connected with liberty.
Where a public actor hampers the exercise of a liberty, the government shall remedy this by
giving effect to liberty and (perhaps) eliminating the causes of the curtailment by calling to
account those who violated the fundamental right. But, the contours of the obligation are not
at all clear. Does the individual have a right to compensation, if her constitutional rights are
violated, but no further law specified these rights? Is there further compensation, if these
rights were violated by an entity or individual acting in the name of the government? And
what if they are infringed by a private actor? It took a long time (and legislative enactment)
for the Constitution to become the legal basis for damages for constitutional torts, even in the
U.S. where ordinary judges read the Constitution with perseverance.

\noi
Rights are rights, but sovereignty is sovereignty, since the days when the king could do no
wrong. The binding force of constitutional rights means also that the government shall follow
it in its own actions. The state’s duty to respect rights does not necessarily entail legal
responsibility for the disregard of a right even if it seems to be a logical necessity.
Constitutional pragmatism does always follow logic, especially where tradition supports
immunity.

\heading{Limiting Fundamental Rights?}

\noi
In some early constitutions rights were worded as if they were absolutes. However, the 1789
French Declaration clearly admits the possibility of limitations. Article 2 declares liberty,
property, security, and resistance to oppression as imprescriptible and natural human rights.
To be ‘imprescriptible and natural’, however, does not mean to be ‘exempt of restriction’.
The rights of man were to be determined by law. But the 1789 Declaration goes further. It
names the grounds for restriction: not to harm others, be compatible with the rights of others,
no abuse.\footnote{Because of political resistance at this stage, when it comes to religion the ‘established Law and Order’ is the limit.} These limitations are accepted as compatible with the imprescriptible character of the natural rights as the right to liberty, property, security,\footnote{Security (sûreté) as personal freedom means that no one can be arbitrarily arrested and convicted.} and resistance to oppression
(Article 2).\footnote{‘Imprescriptible’ or ‘unalienable’ does not mean that the rights cannot be limited; it means that people cannot resign from these rights. For example, a man cannot become a slave of his own accord}

\noi
That the details of fundamental rights protection are defined by legislation is a source of
constitutional problems. By its very nature, a legal definition means delimitation. Definitions
include some and exclude others, therefore, it is important to know who sets the definitions,
as this is the same person who decides on the exclusions. The legislative branch which
is entrusted with setting out the details on the protection of fundamental rights (or of
governmental obligations associated with rights) is also endowed with the duty to express and
protect the common good or public interest. Views regarding the relation between individual
rights and other constitutional interests often collide and people are trained to believe that
public interest is above the private, although this maxim is missing from constitutions and for
good reasons.

\noi
When it comes to fundamental and human rights, constitutions speak of rights and not
interests. To claim that the public interest shall prevail against the private interest does not
answer the dilemma of restricting fundamental rights: here an actual fundamental right
protecting the freedom of an individual is curtailed by a putative public interest.

\vspace{.5cm}

\noi
A right can be formulated as absolute: arguably in the U.S., as formulated by the
‘First Amendment’, free speech can be understood as absolute. Dignity is understood as
inviolable in this sense, for example, in Germany, but it remains difficult to apply, as it offers
little judicially applicable guidance.\footnote{Ch. McCrudden, Human Dignity and Judicial Interpretation of Human Rights, 19 European Journal of
International Law (2008) 655. The inviolable and supreme dignity of the person as a right is practically never
used directly by the German Constitutional Court for deciding cases. } The German Basic Law (and many other constitutions)
define several distinct reasons for the restriction of fundamental rights. The scope (and hence
the limits) of many rights are subject to definition by law (but subject to proportionality).
Moreover, specific restrictions may apply to the military, and laws regarding defence may
restrict freedom of movement and the inviolability of the home. Finally, the fundamental
rights of those who abused specific fundamental rights can be forfeited by the
Constitutional Court. Sometimes the restriction of rights has no separately attached condition.

\vspace{-.1cm}

\noi
For example, when people assemble in public places, it has to be without arms and
peaceful.\footnote{The need for the protection of public order led to the introduction of such measures in the Belgian Constitution as early as 1831.}

\vspace{-.1cm}

\heading{Judicial Review and Constitutionalism}

\vspace{-.1cm}

\noi
The literal meaning of the terminology judicial review refers to the revision of the decree or
sentence of an inferior court by a superior court. It has a more specialized importance in
pubic law, especially in nations having a composed constitution which are established on the
idea of restricted government. The tenet of legal audit has been begun and created by the
American Supreme Court, despite the fact that there is no express arrangement in the
American Constitution for the legal survey. In Marbury v. Madison,\footnote{(1803) 1 Cranch 137.} the Supreme Court
clarified that it had the force of legal survey. Justice George Marshall said, "Absolutely each
one of the individuals who have outlined the composed Constitution examine them as
framing the basic and foremost law of the countries, and thus, the hypothesis of each such
Government should be that a demonstration of the assembly, offensive to the Constitution is
void"

\vspace{-.1cm}

\noi
The fair-minded organization of equity (the 'ability to pass judgment') requests the protection
of the legal branch from the political branches (the assembly and the leader).\footnote{Ch.-L. Montesquieu, The Spirit of the Laws [1748], A. M. Cohler, B. C. Miller and H. S. Stone, trans. and eds. (Cambridge University Press, 1992) 157. The power to judge is not equal to the two other powers.} This was a
long way from unimportant in the eighteenth century: in prior occasions in the European
governments judging and law-production both served an undifferentiated equity. To
exacerbate the situation, courts were frequently the apparatuses of illustrious absolutism and
a wellspring of join. Despite this practice, the legal executive has gotten generally
acknowledged as the third part of force in America. Yet to be determined sought after by
means of detachment of forces the legal executive most importantly fills in as a beware of
different branches. As it is less political than different branches, and it doesn't order its own
assets, it is the 'most un-risky one.

\vspace{-.1cm}

\noi
When set up, the legal executive can be and will be more free in its activities of the other two
branches than those can at any point be of one another. Until it goes to the requirement of
legal choices, the legal executive is best off being left alone by different branches, given that
its states of activity are managed and its accounts are accommodated. Constitutionalism
attempts to restrict the potential for invasion with moderate achievement, however where 
lawmakers are adequately partitioned they will depend on sacred statutes and even implement
those standards.\footnote{Federalist No. 78 (Hamilton), 464, in A. Hamilton, J. Madison and J. Jay, The Federalist Papers [1787–8] (Mentor, 1961) 465. Least dangerous—‘to the political branches’}

\noi
The institutional plan of legal arrangements and association has become more intricate as of
late with decent requests for the responsibility of the legal executive. Legal responsibility
sounds contradictory to legal freedom and fair-mindedness from the start. However, when the
established assurance of legal freedom prevails with regards to protecting the legal executive
from different branches, an arrangement for life is hard to shield notwithstanding wild
negligence for proficient guidelines or broad defilement on the seat. 

\noi
While the legal executive is intended to keep out of the political space, the goal of capability
clashes and political race questions, legal survey of managerial activity, and protected
arbitration fill in as minds the forces and desires of the political branches.

\heading{United State of America}

\noi
Judicial Review in the United States alludes to the force of a court to survey the defendability
of a resolution or settlement, or to survey a regulatory guideline for consistency with a rule, a
deal, or the actual constitution. Article III of the U.S. Constitution expresses, "the legal force
of the United States, will be vested in Supreme Court, and in such second rate courts as the
Congress may every now and then appoint and build up… the legal force will stretch out to
all cases, in law and value, emerging under this Constitution, the laws of the United States,
and deals made, or which will be made, under their position… In all cases influencing
representatives, other public pastors and representatives, and those where a state will be
party, the Supreme Court will have unique locale. In the wide range of various cases under
the watchful eye of referenced, the Supreme Court will have re-appraising ward, both as to
law and actuality, with such special cases, and under such guidelines as the Congress will
make."

\noi
Along these lines, legal audit as perceived in the U.S.A., lays on a basic establishment. The
Constitution is the incomparable law, which was appointed by individuals, a definitive
wellspring of all political power. It presents restricted forces on the public authority. In the
event that the public authority intentionally or unwittingly violates these limits, there should
be some authority capable to hold it in charge, to upset its unlawful endeavour, and consequently to vindicate and protect intact the desire of individuals as communicated in the
Constitution, courts practice this force.

\noi
In Marlbury v. Madison\footnote{1803 \underline{U.S. LEXIS} 352 } Justice Marshall made legal survey not just the main foundation of the established superstructure, and yet the most critical of the American commitment to the
craft of the public authority. Also, this precept was the brainchild of Justice Marshall who
stated that judges are coordinated by the actual constitution, made vow to help the
constitution, which comprises of the foremost rule that everyone must follow. It is an
obligation put upon judges to survey any law which is repulsive to the constitution. The
Supreme Court affirmed this force of legal inspecting over both government and the State
laws in Fletherv. Peck and in this manner got for itself the part of boss mediator and authority
of constitution.

\noi
Judicial review in the United States refers to the power of a court to review the
constitutionality of a statute or treaty, or to review an administrative regulation for
consistency with a statute, a treaty, or the constitution itself. Article III of the U.S.
Constitution states, \textit{“the judicial power of the United States, shall be vested in Supreme
Court, and in such inferior courts as the Congress may from time to time ordain and
establish…the judicial power shall extend to all cases, in law and equity, arising under this
Constitution, the laws of the United States, and treaties made, or which shall be made, under
their authority…In all cases affecting ambassadors, other public ministers and consuls, and
those in which a state shall be party, the Supreme Court shall have original jurisdiction. In
all the other cases before mentioned, the Supreme Court shall have appellate jurisdiction,
both as to law and fact, with such exceptions, and under such regulations as the Congress
shall make.”}

\heading{India}

\noi
Under the constitution of India, powers are restricted in the two different ways. First and
foremost, there is the division of power between the Union and the States. Parliament is
capable to pass laws just as for those subjects which are ensured to the residents against each
type of administrative infringement. Furthermore, the Supreme Court remains in a special
position wherein it is able to practice the force of evaluating authoritative establishments both
of parliament and the state governing bodies.

\noi
Legal audit is an extraordinary weapon in the possession of judges. It involves the force of a
court to hold illegal and unenforceable any law or request dependent on such a law or some
other activity by a public power which is conflicting or in struggle with the essential tradition
that must be adhered to. Truth be told, the investigation of sacred law might be portrayed as
an investigation of the teaching of legal survey in real life. The courts have ability to strike
down any law, in the event that they trust it to be illegal.

\noi
"Article 372 (1) builds up the legal survey of the pre-protected enactment comparatively.
Article 13 explicitly pronounces that any law, which repudiates any of the arrangement of the
part III of the Constitution of India for example, the principal rights will be void. The
equivalent has additionally been seen by our Supreme Court. The Supreme and High courts
are comprised of the defender and underwriter of Fundamental Rights under Articles 32 and
226. Articles 251 and 254 say that if there should arise an occurrence of in steadiness among
association and State laws, the State law will be void."

\noi
In any case, in a few cases, it has held that the Supreme Court can go about as the caretaker,
protector of privileges of individuals and popularity based arrangement of government just
through the legal audit. In Keshwanandbharti v. State of Kerala,\footnote{AIR 1973 SC 1461.} it was held that the
"judicial review is an 'essential component' of the constitution and can't be altered. The extent
of legal review is adequate in India, to make the Supreme court an incredible organization to
control the movement of the executive and legislature."

\noi
Under Indian Constitution, legal survey can advantageously be ordered under three heads:\footnote{Justice Syed Shah Mohammed Quadri, “Judicial Review of Adminstrative Action”, 6 SCC (Jour) 1 (2001).}

\vspace{-.3cm}

\noi
\begin{enumerate}[label=\roman*)]
\itemsep=0pt
\item "Judicial review of Constitutional Amendments.- This has been the topic of thought in
different cases by the Supreme Court; of them worth referencing are: Shankari Prasad case,\footnote{\textit{Shankari Prasad Singh Deo v. Union of India,} AIR 1951 SC 458.}
Sajjan Singh case,\footnote{\textit{Sajjan Singh v. State of Rajasthan,} AIR 1965 SC 845.} Golak Nath case,\footnote{\textit{Golak Nathv. State of Punjab,} AIR 1967 SC 1643.} KesavanandaBharati case, Minerva Mills case,\footnote{\textit{Minerva Mills v. Union of India,} AIR 1980 SC 1789.} Sanjeev Coke case\footnote{\textit{Sanjeev Coke Mfg. Co. v. Bharat Coking Coal Ltd.,} (1983) 1 SCC 147.} and Indira Gandhi case.\footnote{\textit{Indira Nehru Gandhi v. Raj Narain,} 1975 Supp SCC 1.} The trial of legitimacy of Constitutional
corrections is adjusting to the essential highlights of the Constitution".

\item "Judicial review of Legislation of Parliament, State Legislatures just as Subordinate
Legislation. - Judicial survey in this classification is in regard of authoritative capability and
infringement of central rights or some other Constitutional or administrative restrictions"; 

\item "Judicial review of Administrative Action of the Union of India just as the State
Governments and specialists falling inside the importance of State. It is important to
recognize legal survey and legal control. The term legal audit has a prohibitive meaning when
contrasted with the term legal control. Legal survey is administrative, as opposed to
restorative in nature".\footnote{M.P. Jain and S.N. Jain, \textit{Principles of Administrative Law: An Exhaustive Commentary on the Subject  Containing Case-law Reference} (Indian \& Foreign) 1779 (Wadhwa and Company Nagpur, New Delhi, 6th  edn 2007).}
\end{enumerate}

\vspace{-.2cm}

\heading{Comparative Study of Judicial Review\\ Between India and U.S.A.}

\vspace{-.1cm}

\noi
The extent of Judicial Review in India is to some degree surrounded when contrasted with
that in the U.S.A. In India the principal rights are not so comprehensively corded as in the
U.S.A. furthermore, limits there on have been expressed in the actual Constitution and this
assignment has not been left to the courts. The constitution creators "received this
methodology as they felt that the courts may think that it is hard to work act the constraints
on the crucial rights and the equivalent should be set down in the actual constitution". The
constitution producers additionally felt that the Judiciary ought not be raised at the degree of
'Super assembly', whatever the support for the strategies logy received by the constitution
creators, the unavoidable aftereffect of this has been to limit the scope of legal survey in
India.

\vspace{-.1cm}

\noi
It must, in any case, be yielded that the American Supreme Court has burned-through its
ability to decipher the constitution generously and has made so exhaustive a utilization of the
"fair treatment of law condition that it has gotten in excess of a more translator of law. Be
that as it may, took great consideration not to epitomize the fair treatment of law proviso in
the constitution". Actually, the composers of the Indian constitution chose to exemplify the
term 'methodology set up by law'. It can refute laws in the event that they disregard arrangements of the constitution however not on the ground that they are terrible laws. As
such the Indian Judiciary including the Supreme Court is anything but a Third Chamber
asserting the ability to sit in judgment on the strategy encapsulated in the enactment passed
by the lawmaking body.

\vspace{-.1cm}

\noi
In this way, in the expressions of Justice Cardozo, the central worth of legal survey rather lies
"in making vocal and perceptible thoughts that may somehow be hushed, in giving them
congruity of life and of articulation, in controlling and coordinating the decision inside the
cut-off points where decision officers."\footnote{B.N. Cardozo, \textit{The Nature of Judicial Process,} 94 (Universal Law publishing Co. Pvt. Ltd., 2004).}

\vspace{-.1cm}

\heading{Conclusion}

\vspace{-.1cm}

\noi
Constitutionalism is a matter of taste and manners. There can be an invitation in the
constitution that ‘the conduct of government be transparent’ (Ethiopia, Article 12(1)), but
such words make little difference, if the rulers believe that they can do anything without any
explanation. Contemporary constitutions exist on the foundations of a set of beliefs and
commitments. Constitutional expectations are to be shared by the power holders and their
constituency. As a result, a long-term perspective, applicable to future governments, emerges
that is not limited to drafting technicians and politicians, but is deeply connected with public
politics, with such problems and political conflict involving the people that require lasting
institutional solution.

\noi
Constitutionalism is supposed to answer the question: how do we ‘construct enduring forms
of political order? The fate of revolutionary power sharing will depend on many things
besides constitutional creativity; culture, economics, and geopolitics will make a tremendous
difference. Nonetheless, the creative role of constitutionalism is easy to underestimate . . .’\footnote{B. Ackerman, The Future of Liberal Revolution (Yale University Press, 1992) 3 (emphasis added).}
Constitutionalism, written into law, does not replace the cement of society, but it is an
important active ingredient of the cementing compound. Government may have a leading role
in integrating society; and in such cases additives become particularly important.
British constitutionalism survives without a written constitution.\footnote{For a less enthusiastic home-grown appraisal see K. D. Ewing, Bonfire of the Liberties. New Labour, Human Rights, and the Rule of Law (Oxford University Press, 2010).} There, so the canonical contemporary doctrine insists, judges cannot review the constitutionality of statutes, the majority of civil liberties and fundamental rights are not guaranteed by entrenched protective laws, and—at least in theory—Parliament can reshape the political system whenever it
desires. Without idealizing the political system that seems to prevail in the United Kingdom,
one can assume with near certainty that the withdrawal of constitutional freedom is out of the
question in that country.

\noi
For constitutional provisions to be meaningfully and effectively operative there must be
institutional and cultural machinery, which is partially created by the constitution itself, to
implement, enforce and safeguard the constitution. Judicial Review is one of the key
components in implementing and safeguarding the spirit of Constitutionalism. An
independent judiciary, independent constitutional review, and the notion of the supremacy of
law all work together to ensure that the letter and spirit of the constitution are complied with
in the working of a constitutional government. Constitutionalism is the philosophy of the
constitution, which imposes limitation upon the exercise of power. So, the overall view can
be concluded in the words of Frankfurter,J. that Judicial review, itself a limitation on the
popular government, is a fundamental part of our constitution.

\vspace{.5cm}

\heading{Suggestions:}

\noi
\begin{enumerate}
\item “The first thing that needs to be done is to codify the law on the subject of Judicial
Review.

\item The trend at present is to vest jurisdiction with new institutions of administrative
nature but it is not clear what will happen to the concept of Judicial Review and how
the independence of the administrative institutions will be protected.

\item The concept of Judicial Review at times has assumed political overtones; the
amendments so often made to the Constitution have raised challenges before the
Judiciary as to what it should do when they are challenged before them”.
\end{enumerate}

\end{multicols}
