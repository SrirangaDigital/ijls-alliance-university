\setcounter{figure}{0}
\setcounter{table}{0}
\setcounter{footnote}{0}


\articletitle{Repressed Coastal Population - An Analysis of Rights of Coastal Communities and Ineffective Intricacies of Coastal Law Regime in India}\label{2020-art3}
\articleauthor{Asha.J\footnote{Ph.D. Scholar, Alliance School of Law, Alliance University, Bangalore.}}
\lhead[\textit{\textsf{Asha.J}}]{}
\rhead[]{\textit{\textsf{Repressed Coastal Population’ -...}}}

\begin{multicols}{2}

\noi
\textit{“The earth has enough for everyone’s need but not for everyone’s greed.”}
 
\vspace{-.3cm} 
 
{\hfill\textit{-Mahatma Gandhi}}

\vspace{-.3cm}  
 
\heading{Introduction}

\noi
With a 138-crore population, India faces several fatalities as a consequence of natural
disasters.~Climate change, geo climatic conditions and high socio-economic vulnerability are
the main reasons for coastal issues in India. Furthermore, catastrophic incidents and different
types of complex environmental conundrums are increasing due to the industrial and
economic activities in coastal zones, wherein half of the Indian population lives. Enormous
economic renovations and urbanisation has ensued in deprivation as well as ecological
damage to coastal zones in India. Even after three decades of Coastal law, India is still
combating coastal issues; sustaining the livelihood security of fishing communities,
preserving the health of coastal eco systems and bio diversities which are crucial to overall
sustainability of coastal regions.\footnote{Hemant Kumar. A. Chouhan et al, \textit{Coastal ecology and Fishing Community in Mumbai,} 51 EPW 39 (Sep.24,
2016) } But while striving to accomplish the economic goals of our
country, our coastal policies have failed to incorporate the powerful concept that survival of
human beings depends on their harmonization with nature. There is a close collaboration
between man and nature. But at the same time, development is as important as environmental
protection. Coasts offer crucial components for social and economic development to the
world. However, the unregulated development in coastal zones have resulted in livelihood
challenges to coastal communities and this article mainly addresses the tripartite issue of
development, environmental protection, and livelihood challenges of coastal communities in
India. Therefore, the objective of this paper is to study the legal and environmental
advancements for the sustainable development of our coasts as well as to analyse the issues
faced by coastal inhabitants.

\heading{Coastal Law Regime in India}

\noi
Coastal Regulations in India tracked its concepts from UN Conference on Human
Environment, held in Stockholm in 1972. Based on that ‘The Environment Protection Act
(EPA) 1986’\footnote{Environmental Protection Act, 1986, Act No.29, Acts of Parliament,1986 (India).} was enacted to execute India’s commitments as a party to the conference. The
Coastal Regulation Zone (CRZ) Notification of 1991 was made under the provisions of the
Environmental Protection Act 1986, to protect coastal environments and also the social
security and livelihood securities of fishing communities in India. For the last three decades,
this subordinate legislation is the only trail in managing coastal zones of India. Coastal
Regulation Zone Notification is applicable to the entire Indian Coast including the Andaman
\& Nicobar Islands and the Lakshadweep Islands. It generally covers the coastal stretches of
seas, bays, estuaries, creeks, rivers, and backwaters influenced by tidal action up to the
defined distance into the land from High Tide Line (HTL).

\noi
The CRZ Notification was introduced with the following three main purposes:

\vspace{-.3cm}

\begin{enumerate}[label=\roman*)]
\itemsep=0pt
\item To arrive at a balance between development needs and protection of natural resource.

\item To prohibit and/or regulate the activities which are harmful for both coastal
communities and environment.

\item To plan for a sustainable management, so that the livelihoods of millions of people
are protected, and the coastal environment is preserved for the future generation. 
\end{enumerate}

\vspace{-.3cm}

\noi
The Coastal Regulation Zone Notification (CRZ) in India was initially saluted as a
progressive law by fish worker activists and environmental groups as they already recognised
that coastal areas needed some reforms from unregulated development. CRZ Notifications
were also intended to govern human and industrial activity close to the coastline, to protect
the fragile ecosystems near the sea. CRZ Notification was enacted to restrict certain kinds of
activities like large constructions, setting up of new industries, storage or disposal of
hazardous material, mining, or reclamation and bunding, within a certain distance from the
coastline. The real issue is that the delicate areas next to the sea are home to many marines
and aquatic life forms and are also endangered by climate change. Therefore, they need to be
protected against unregulated development. However, India’s fiscal transformations had
influence on the functioning of CRZ notification and it damagingly touched the objectives of 
CRZ. It is critiqued that considerations of economy overweighed ecology. The clauses of the
notification prohibiting and restricting activities along the coast remained unimplemented,
and the mandatory Coastal Zone Management Plans were also not implemented. Further,
there was no proper institutional mechanism to ensure execution of CRZ Notification. So,
The Honourable Supreme Court in \textbf{S. Jagannath v. Union of India},\footnote{S. Jagannath v. Union of India, AIR 1997 SC 811.} declared that sea
beaches and sea coasts are gifts of nature and any activity which pollutes these natural
resources, or the gift of nature cannot be permitted to function . In this case, a shrimp farming
culture industry by modern method was harming the eco system, polluting potable ground
water and exhausting plantations. All of these activities were held to be violative of
constitutional provisions and non-implementation of Coastal Regulation Zone Notification
was debated expansively by the Supreme Court of India. The court further held that before
the induction of any such industry in a fragile coastal area should necessarily pass the strict
ecological test. After this judgment, there have been several amendments to the CRZ
Notification, 1991, based on the recommendations of various committees which are
consistent with the basic objective of the notification. But there were continued difficulties
posed by the CRZ Notification, to its effective implementation, from the beginning. In 2019
the CRZ was restructured to regulate the activities of coastal zones in India. However, its
main criticism was the disregard for livelihood perspectives of coastal communities in the
wake of continuous disasters which are happening in our country.

\heading{Genesis of CRZ}

\noi
In 1981, the then Prime Minister Mrs. Indira Gandhi, through a letter to Chief Ministers of all
coastal states of India, insisted to start initial measures for protection of coastal areas to
enhance their conditions. In 1982, The Ministry of Environment and Forests set up working
groups to prepare environmental guidelines for the development of beaches and coastal areas
and it resulted in the formation of Working Group on Beach Development Guidelines, 1983.
It was the guidelines for the development of Beaches, Tourism, Industrial Development,
Urban and Rural Development, special areas such as mangroves, scenic areas, corals, oceanic
islands etc. In 1991, the CRZ Notification came into force based on Environment Protection
Act of 1986. During these years various committees were appointed, and based on the
recommendations of Dr. Swaminathan committee, MoEF revised the earlier notification and the CRZ Notification, 2011 revision came into existence. But both Notifications faced severe
criticisms and later in 2019, CRZ Notification 2011 was revised and the CRZ 2019 came into
force. However, the CRZ 2019 notification faced huge disagreements from its stakeholders
for various reasons\footnote{Amisha Aggarwal, \textit{Climate Change and coastal Zone Regulation: Regulation of Coastal Protection, an analysis of CRZ Notification}, 2018, 13 IOSR-JESTFT 8,49-56(Aug. 2019).}

\heading{Constitutional Mandate for Coastal Protection}

\noi
Right to pollution free environment is a fundamental right under Art.21 of our constitution.
Protection of natural resources and environment is encompassed in our fundamental duties
also. India is known for enacting necessary environmental Acts from time to time in
accordance with various international conventions including the Stockholm and Rio
declarations. More than this,our Constitution guarantees to its citizens certain fundamental
rights such as Article 21\footnote{INDIA CONSTI. art.21.}  which guarantees to every person ‘the right to life’. Consequent to
judicial interpretation, now includes the right to a clean environment and access to natural
resources.\footnote{Abdul Jabbar Haque, \textit{The Right to live in healthy and pollution free environment: An analysis in Constitutional Perspective,} 3 NJEL 1, 30-38(2020).} It is relevant to note that India has given a constitutional status to environment
protection by imposing a duty on the State to protect and improve the environment under
Article 48A and Art. 51 A (g). It also imposes a fundamental duty on every citizen to protect
and improve the environment. Art.47 puts a duty on the state to raise the standard of living
and to improve public health which depends on the quality of the environment. In \textbf{Subash
Kumar V. State of Bihar},\footnote{Subash Kumar v. State of Bihar, AIR 1991 SC 420} it was held that the state is under a Constitutional Obligation to
protect the right to environment and citizens have a right to the wholesome environment. In
\textbf{M.C. Mehta v. Union of India}\footnote{M.C. Mehta v. Union of India, 2002 (2)SCR 963} also, the court held that Art. 39(e), 47 and 48A cast a duty
on the state to secure public health and environmental protection. Art. 51 A(g)\footnote{INDIA CONSTI. art. 51 A § g} places a
fundamental duty on the state to protect public health and environment. Through many
judicial decisions\footnote{Indian Council for Enviro Legal Action v. Union of India,(1996)5 SCC 281, Kinkeri Devi v. State Of HP (AIR 1988 HP 4) , S. Jagannath v. Union of India (1997 SC 811)} it is reaffirmed that Coastal regions and water bodies are an integral part
of our environment, and the states are obligated to protect the coasts of India.

\heading{Coastal Community and Coastal Regulations in India}

\noi
The coasts of India are generally facing environmental damage, displacement of coastal
communities and causing hurt to the livelihoods of millions who depend on the sea for their
survival. Coastal communities argue that traditionally the coastal land belongs to them and
their rights in such coastal areas should be respected. But other than a brief mention in the
preamble of the CRZ Notification as well as in the interpretation available in the 1996
judgment of the Supreme Court, there is very little in the CRZ Notification vis-à-vis fisher
rights. \textbf{S. Jagannath v. Union of India}\footnote{\textit{Id}} , identified the adverse impacts of coastal
pollution caused by non-traditional and unregulated prawn farming. It held that \textit{the
intention of the CRZ Notification is to guard the ecologically fragile coastal areas and
to maintain the aesthetic qualities of the seacoast. The setting up of modern shrimp
aquaculture farms near the seacoast is perilous and is degrading the marine ecology,
coastal environment and the aesthetic uses of the sea coast.} The Court concluded that
prawn farming industries should be banned in the coastal Regulation zones under the
CRZ Notification 1991 because their functioning was in violation of numerous
environmental laws. However, the Court allowed the traditional systems of aquaculture
to continue by taking into consideration the traditional coastal communities in that area. 

\noi
The objective of the CRZ Notification was to control ecological damage to coastal areas
caused by pollution, maintain coastal livelihood security, uphold the traditional rights of
fishermen and maintain the aesthetic value of the coast. However, there were no concrete
provisions and measures that explicitly defined the rights of fishers. At present, coastal
people are facing the issues of landlessness, unemployment, and homelessness. An evaluation
of the CRZ Rules should link larger issues of livelihood and environmental sustainability of
coastal regions. Insights to small scale fishery- based livelihoods and environmental
sustainability should be taken into consideration while framing coastal policies. Coastal
livelihood issues and the developmental activities in coasts and its repercussions on the lives
and livelihood of fishing community is of serious debate. According to CRZ 2019, the
country’s coastline which is currently protected will be thrown up for development and this
move will lead to the development of resorts, hotels, and mega housing projects, ultimately
leading to the uprooting of fishermen. It does not define activities which are to be prohibited in the coastal zones. Instead, it allows state governments to identify economically significant
areas and allow industries to grow. Also, the notification is silent on the management of these
zones and suffers from so many loopholes as it fails to consider the biological diversity ,
demographic patterns, and distribution of natural resources in the coastal zones even if the
area is ecologically fragile.\footnote{Raghav Parthasarathy \& Vikas Gahlot, \textit{Coastal Regulation Zone: A Journey From 1991 Till 2019,} NLS ENV.LAW,( Jun.9, 2020) \url{http://nlse.org/coastal-regulation-zone-a-journey-from 1991- till-2019/.}} Furthermore, the Provision for the development of new ports
which might be disastrous for India’s ecological balance. In \textbf{M. Wilfred v. Ministry of
Environment and Forests}\footnote{M. Wilfred v. Ministry of Environment and Forests, 2016 SCC Online 426} , the applicants have claimed that the site of the proposed port
project, in its immediate vicinity, is inhabited by small scale fishermen who depend on
Coastal and offshore water for fishing as a part of their livelihood. In the above case the
applicants, seek to protect and safeguard coastal areas of outstanding natural beauty and areas
likely to be inundated due to rise in sea level consequent upon global warming. The areas
were declared by the Central Government or the concerned authorities at the State/ Union
territory level from time as CRZ under CRZ Notification of 2011. In this case the court
ordered Central Government to go for an Environment Impact Assessment (EIA) and CRZ
clearances before implementing the projects. Recently in \textbf{Worly Koliwada Naksha Matsya
Vyavasai Sahakari Society v. Municipal Corporation of Navi Mumbai},\footnote{Worly Koliwada Naksha Matsya Vyavasai Sahakari Society v. Municipal Corporation of Navi Mumbai,
WP(L) No. 560 of 2019 dated 16. 7.2019} The Supreme
Court stopped the coastal road project in Mumbai on the basis that it had not obtained
environmental clearance from the authorities, and it had adverse impact on the coastal
community. So, the judicial pronouncements arising out of coastal issues initiated by
environmental activists are quite often enough to understand the law. But it is criticised that,
through the new notification in 2019, the policy makers have regularised the violations. It
will protect structures built on the seaward side of the existing roads and structures built
contrary to CRZ in the name of development facilities for temporary tourism infrastructure. It
will also have a negative impact on the fisheries as it will restrict the movement of fishermen
in the inter habitation segments. Indian fishermen have been using the fishing waters and the
land to process their catch, repair their nets, or sell their products as common property
resources. If these areas are provided for tourism infrastructure development, the means of
livelihood of local inhabitants will be in distress. Additionally, the new notification may lead
to them being treated as encroachers and may lead to their displacement without any compensation. In \textbf{Ramdas Janardhan Kohli and others v. Secretary MOEFCC and
others},\footnote{Ramdas Janardhan Kohli and others v. Secretary MOEFCC and others, MANU/GT/0056/2015 (India).} the traditional fisherman sought compensation from City and Industrial
Development Corporation (CIDCO) as well as the Oil and Natural Gas Corporation (ONGC).
Fisherman residing in coastal areas around Mumbai had objected to infrastructure activities in
the region, citing potential loss of their means of livelihood. They argued that urbanization
has caused environmental damages to fishing areas and had a negative impact on more than
1600 families. They had been damaged by a project launched by CIDCO. The fisherman
complained that they used to catch fish varieties near the shore but now that area had been
destroyed by CIDCO. National Green Tribunal awarded compensation worth INR 950
million to be split between 1630 affected families and held them liable for damaging
environment and affecting livelihood of fisherman community in that area. In \textbf{Alexio Arnolfo
Perera v. State of Goa},\footnote{Alexio Arnolfo Perera v. State of Goa (2014) SCC Online NGT 6655.} The court ordered against Goa Government’s temporary shack
policy for tourism development as it was against CRZ Notification. Therefore, our judiciary
has been proactively interfering in conservative as well as livelihood issues of coastal zones.

\vspace{-.2cm}

\heading{Coastal Population Expecting Environmental Displacements in
Future}

\vspace{-.2cm}

\noi
The essential characteristic of coastal populations is that they are primitive traits and stay in
peculiar geographical location. They are economically backward with unique cultural identity
and are usually isolated from the mainstream community. This weaker section of society
which got separated over several parameters was always retained out of the mainstream
society and have thus become ignorant towards their rights and means to redress their
problems. They are also prone to social, economic and environmental challenges. The main
encounters confronted by the coastal community are as follows.

\vspace{-.2cm}

\begin{enumerate}
\itemsep=0pt
\item Vulnerability to natural calamities and climate change

\item Threats to coastal population and infrastructure 

\item Livelihood securities of coastal people

\item Non-identification of the special needs of coastal people in the ecologically sensitive areas.

\item Rapidly increasing pollution and associated urbanisation and commercialisation resulting in detrimental fishing methods.

\item Legal uncertainty related to land rights and other rights.

\item Competition over limited coastal spaces and resources.

\item Environmental displacements in future.
\end{enumerate}

\vspace{-.2cm}

\noi
The Coastal community fears that CRZ, 2019 will be a shaded period for coastal
communities like fisherman, toddy tappers and farmers. It is complained that they will be
displaced as the Non development Zone is reduced from 500 meters to 50 meters. National
Fish Worker Forum (NFF) have expressed their fear that all Sagarmala packages, plans and
projects will uproot the livelihoods of traditional fish workers. These projects will only
benefit corporates and are against the interests of coastal community of India.

\noi
Lack of concern for disasters and climate change are also a concern for the community.
Okhy, Gaja, Fani Cyclones and devastative floods have created huge losses to fishing
communities and it is the need of hour to formulate policies to compensate their losses due to
natural calamities. NFF says that CRZ, 2019 is a strategy which impacts the livelihood of
small-scale fish workers and it is a move to privatise the coasts and hand it over to
corporates.\footnote{National Fish workers Forum (NFF), Press Release Dated 26-5-2019.} While executing The Coastal law in 1991, the livelihood aspects of coastal
community were given significance. Dr. Swaminathan committee set up as an aftermath of
tsunami, went as far as suggesting a Land Right Recognition Law and suggested that specific
protection should be provided to traditional communities who subsist on coastal areas only on
the basis of their customary rights. But the recommendation was never executed, and the
concept of customary ownership itself is grabbed by way of tourism and other developmental
purposes. Over the last three decades, the regulation has been amended thrice and revised
around 34 times. The coastal community views it as a lack of community and environmentoriented policy. National Fish Workers Forum says that the policy should include a well
demarcated hazard line and should factor the effects of climate change. Further, they say that
CRZ 2019 will pave a way to future disasters and the coastlines will be more exposed in the
upcoming years. The CRZ, 2019 is merely giving more access to the corporate /tourism, land mafia for development, and the coastal community’s livelihood as well as the environment
are being ignored.\footnote{\textit{Fishermen’s body rejects CRZ 2019, Demands rollback,} TIMES OF INDIA,(Feb.25,2019 12.45PM) \url{www.times of india.com.}}

\noi
Coastal communities are vulnerable to unforeseen events such as Tsunami, a regional
flood/cyclones. They are not resilient to normally recurring hazards. The deprivation of
coastal environment is primarily due to human induced actions which jeopardizes food
security, livelihood, fiscal development, and existence of coastal communities. They are not
naturally resilient to coastal hazards.\footnote{Sushama Guleria et al., \textit{Coastal community resilience: Analysis of resilient elements in 3 districts of Tamil Nadu State, India,}16 JNL. OF COASTAL CONSERVATION 1,101-110(Mar. 2012).} Many internal assessments of post-tsunami relief and
rehabilitation, undertaken mostly by international non-governmental organisation and local
NGOs, highlight the significant gaps that exist between goals and achievements as well as
recognise that the felt needs of local people have been inadequately addressed. While many,
including fishers, are arguing that initial relief was quite effective, though restricted to the
villages near the main roads, rehabilitation has been haphazard with no clear goals both for
the rehabilitator and the rehabilitated.\footnote{Ajith Menon et al, \textit{Reconfiguring the Coast,} 43 EPW 16 ,35-38 (Apr. 19-25,2008).} Many NGOs entered the rehabilitation arena
completely ignorant about the socio-economic issues relevant to coastal communities, and
consequently blamed the shortcomings of delivery on poor implementation and local political
and social dynamics.\footnote{Senthil Babu, \textit{Coastal accumulation in Tamilnadu,}46 EPW 48, 12-13(Nov. 26,2011) } But such explanations are unfinished. To understand developments in
fishing villages and issues of coastal community, it is necessary to delve into the uneven
antiquity of coastal management in the context of shifting urgencies along the coast. This
description will also highpoint the challenges gaining for integrated coastal zone
management. Besides the physical damage, the tsunami left an indelible stamp on people's
minds that a fear intensified by requirement of NGOs and the governmental intervention.
Post-tsunami relief and rehabilitation exertions have not agonised from a lack of funds but
due to lack of governance and legal uncertainty. The coastal areas which have been
customarily inhabited by traditional fishing communities are also antagonized with largescale industrial growth and development. The coastal hazards are aggravated by rapid
urbanization and unplanned human settlements, poorly engineered construction, lack of
adequate infrastructure, poverty, and inadequate environmental practices such as
deforestation, mangrove destruction, and land degradation etc. Thus, the coastal policies 
should corroborate the need for proper risk assessment as this would aid the coastal
community in planning and responding to coastal hazards, making the coastal population
safer from the risk of disasters. The 1992 Earth Summit in Rio de Janeiro, contributed to new
perspectives about coastal management to include the role of education in engaging people to
work towards a more sustainable future for the world's coastal areas. In response to these
challenges and international trends, governments at all levels and non-governmental
organizations should develop policies, strategies, and programs to support more integrated
and effective coastal disaster reduction.

\noi
Since the inception of the CRZ notification, fishing communities of several states have been
trying to negotiate with the MoEF for the protection of their customary rights and
representation in the decision-making process. To maintain social stability and promote
distributional justice, local coastal communities should be allotted clearly drafted, specific
use and property rights about specific areas. Fishing collectives and environmental groups
objected to the latest CRZ 2019 notification which opened India’s coastline for enhanced
commercial activities, primarily on the following grounds.

\vspace{-.35cm}

\begin{enumerate}[label=$\bullet$]
\itemsep=0pt
\item No prior consultation was held with coastal communities, especially the fisher folks.

\item The lifting of development restrictions would be disastrous for coastal environment and
traditional communities living there. 

\item Interfering with ecologically sensitive coastal areas would leave them more vulnerable
to natural hazards.
\end{enumerate}

\vspace{-.3cm}

\noi
So, the coastal community has articulated their displeasure towards CRZ, 2019 and has
demanded a comprehensive CRZ Act which ensures their rights. They also claim for an allinclusive study on assessment of vulnerability and inclusive community participation which
can afford an important guide to coastal planning and resource allocation at various levels. It
can benefit to raise public awareness about the risks, and such initiatives must envision
prevention of catastrophic disasters and sustainable recovery in the aftermath of a disaster. It
can also reduce the coastal community’s vulnerability to natural disasters.

\vspace{-.1cm}

\noi
The CRZ notification is critical to the lives and livelihood of communities comprising of
170 million people or 14\% of Indian population living across 70 coastal districts, 66 main 
lands and four island territories.\footnote{Meenakshi Kapoor, \textit{Ignoring objections ,India finalises New Coastal Laws,} INDIA SPEND(Mar.21,2020)
\url{www.indiaspend.org, art.14.com/post/ignoring-own-experts-fisherfolks-experts-india-finalises-new-coastal-law-investigation.}} Their future especially that of the marginalised
communities, is directly linked to the health and disaster preparedness of coasts.

\vspace{-.1cm}

\noi
The 2019 CRZ notification violates the balance between ecosystem and development.\footnote{Vinod. K. Dhargalkar et al., \textit{CRZ Notifications 2018- Disastrous to eco system functioning,} 6 INT.JNL. OF
ECOLOGY AND ECO. SOLUTION 1, 10-15(Mar.2019).}
The property rights and economic development in coastal zones are severely hampered
with several unrealistic and unachievable restrictions when applied with a common
yardstick throughout the country. The mandatory 50-meters buffer zone for mangrove
forest in private land with an expanse of more than 1,000 sq. m has been taken by the
present notification. This will affect the coastal ecosystem. The notification has given
relaxation in Coastal Regulatory Zone and this will be helpful for people with small land
holdings but the disastrous impact of it on ecology will be against the coastal community.

\vspace{-.1cm}

\noi
Environment Scientists and green activists have expressed their concerns regarding
unbridled construction activities on the coastal areas and its negative impact which are to
be addressed. They have also warned the government against gifting coastal areas to the
tourism sector in the name of fishermen. As huge populations live in the coastal areas, their
need for economic development and subsistence activities infringe on the quality of the
environment in that region.\footnote{A. Ramachandran et al. \textit{Coastal Regulation Zone Rules in Coastal Panchayats (Villages) of Kerala, India vis-à-vis socio economic impacts from recently introduced people’s participatory programme for local selfgovernance and sustainable development,} 48 OCEAN AND COASTAL MGMT. 632- 635(2005), \url{www.Elsevier.com/locate/ocecoman.}}

\vspace{-.1cm}

\noi
According to National Disaster Management Authority, up to 36 million Indians are likely
to encounter coastal floods due to rising sea levels by 2050.\footnote{Study by Climate Center, National Disaster Management Authority, October 2018.} So, ensuring protection to
coastal populations and structures from risk of inundation from extreme weather and
geological events is the need of the hour. It should guarantee that the livelihoods of coastal
populations are not unduly hampered by these frequent amendments.

\noi
To maintain social stability and promote distributional justice, local coastal communities
should be assigned with clearly drafted, specific use property rights about specific coastal
areas. The coastal communities often have established experienced and practices to
manage local ecosystems sustainably. It is proved in many cases that assigning exclusive 
rights to local communities can also help to protect coastal eco-systems. Accordingly, the
rights and obligations laid down in CRZ notification should have a clear and increasingly
comprehensive content which will be enforceable in the Courts.\footnote{Abhishek Das, \textit{Coastal Regulation Zone: Governance and Conservation,} LAWYERS CLUB INDIA\\(OCT 9,2019.12.18PM) \url{http://www.lawyersclunindia.com/articles}}

\heading{Rights of Coastal Population- Unidentified and Repressed?}

\noi
{\large\bfseries Right to Life and Livelihood:} Strengthening the livelihoods of fishing
communities and maintaining coastal ecologies and biodiversity are vital for the
sustainability of coastal regions of India. The rising environmental vulnerabilities expand
deprivation of coastal ecosystems and livelihood security of coastal communities. Traditional
and customary rights in relation to fisheries and living space, as well as historic rights of
coastal fishing communities are not recognised in the Coastal Regulation Zone Notifications.
Ensuring traditional coastal community rights is of great significance and possibly to ensure
social justice for traditional fishing communities is to designate a zone to protect their right
through which only we can sustain their fundamental right to life and livelihood.

\noi
{\large\bfseries Right to Pollution Free Environment}: This right is included indirectly as a
part of Art.21 by various judicial interpretations. Environment deterioration can eventually
endanger life of present and future generations. It includes right to survive as species, quality
of life, the right to live with dignity, right to good environment and right to livelihood. All
these rights are implicitly recognised as Constitutional rights. In \textbf{Subash Kumar V. State of
Bihar},\footnote{Subash Kumar V. State of Bihar, (1991) 1 SCC 598} it was held that the right to life includes right to enjoyment of pollution free
environment and if anything endangers or impairs that quality of life in derogation of laws a
citizen has recourse to Art.32 for removing the pollution which is detrimental to his life.
Further in a series of cases like \textbf{M.C. Mehta v. Kamal Nath},\footnote{M.C. Mehta v. Kamal Nath, (1997) 1 SCC 388} \textbf{Enviro-Legal Action V.
Union of India,}\footnote{Enviro- Legal Action V. Union of India, (1996) 3 SCC 212} reiterated the same opinion that the right to pollution free environment is a
part of Art.21.

\noi
The poor and the under privileged classes of coastal people and other indigenous classes of
people are usually suffering the burden of environmental glitches. Ironically, the crisis is due
to unsustainable and destructive models of development. Anyway, right to pollution free 
environment as a part of Art.21 through the decisions of Supreme Court have become the bed
rock of environmental jurisprudence. So, the destruction and depletion of coastal ecosystem
and its people depending on the natural/ coastal resources of their own locality to meet their
basic needs will be violative of their fundamental right. They also have the right to enjoy life,
livelihood, cultural sustenance, aesthetics of natural surroundings. The violations of these
rights may lead to other violations such as displacement and sustainable common property
management, loss of access to productive land, destruction to life support system etc. So, a
better understanding of diverse coastal system should be there to assure the coastal
communities right to pollution free environment as their fundamental as well as a human
right.

\noi
{\large\bfseries Right to Development:} Prof. Upendra Baxi said that, development is a
participatory process of implementing all rights for all people. Right to development is a
holistic concept and development vis-a -vis environment has been a placard of all concerned
stakeholders. In Coastal issues also, the most discussed area is whether we should give
priority to environment or to development. But, in \textbf{Vellore citizens Forum v. Union of
India,}\footnote{Vellore citizens Forum v. Union of India, (1996) 5 SCC 647} the court already settled that development and ecology are no longer opposed to
each other and sustainable development has to be accepted as a viable concept to eradicate
poverty and improve quality of human life while supporting the surrounding ecosystems. So
coastal community also have the right to development along with sustainable development of
Coasts. It should be kept in mind that development encompasses much more than economic
wellbeing and includes the whole spectrum of civil, cultural, economic, political, and social
process for the improvement of people’s wellbeing and realisation of their full potential.\footnote{N.D. Jayal v. Union of India, (2004) 9 SCC 362.}
Therefore, while implementing new policies for coastal development, the above-mentioned
concepts should be considered for the protection of coastal community. Present coastal law
regime in India mainly focuses on development of coastal areas only and unfortunately does
not include the development of coastal population. So, experts argue that both should go hand
in hand to achieve the expected sustainable development in coastal regions of India. Further
in \textbf{Nature Lovers Movement v. State of Kerala},\footnote{Nature Lovers Movement v. State of Kerala, AIR 2000 Ker.31.} case it was held that there should be an
adjustment and reconciliation in between preservation of environment and development of 
economy. Therefore, while implementing guidelines of an appropriate developmental policy
to coasts it should also analyse the obstacles and implementation deficits in sustainable
coastal management.

\noi
{\large\bfseries Right to Participation in Coastal Management:} Public participation is
recognised as crucial in making environmental governance more robust. 'Participatory'
mechanisms in environmental governance are advocated for a variety of reasons, including an
implied emphasis on participation as furthering justice and equity, ambitions to make
participative or deliberative measures as supplements or alternatives to representative
democracy and enhancement of legitimacy of controversial environmental decisions.\footnote{Naveen Thayyil, \textit{Public Participation in Environmental Clearances in India: Prospects for democratic
decision making,} 56 JILI 4, 463-492(Oct- Dec. 2014)} In
connection with the notion of sustainable development, the Rio Declaration stated that
environmental concerns are to be solved with the participation of all concerned people at the
relevant level. Despite of the agreement on the importance of public participation in
environmental decision-making there is a clear lack of consensus on what public participation
is supposed to mean and more importantly on what it is supposed to accomplish.\footnote{Nichola Tilche, \textit{In what ways is the emphasis on public participation a positive development in Environmental law? An analysis of Aarhus Convention and its impact on EU Environmental law and policy,} 1 EPLR 1-
23(2011).} The
Preamble of Aarhus convention\footnote{UNECE Convention on Access to Information ,Public Participation in Decision making and Access to Justice
in Environment matters. Signed on 25 June 1998.} says that involvement of stakeholders and public at large
will improve the substantive quality and outcomes of Environmental decisions. It is said that
consultation with public and interest groups may unquestionably increase the knowledge and
help to make more technical and holistic environmental decisions.

\noi
The marginalised coastal community in India is discriminated and always kept away from
coastal policy making and it is also evident in the latest CRZ Notification, 2019. The coastal
community remained silent spectators of development, but the obligation of environmental
degradation usually affects them. These marginalised sections of people are away from
material benefits and from environmental decision making. Even the coastal zone
management plans were not available to them and were against the basic concept of
environmental democracy in coastal planning. In \textbf{Kaloor Joseph v. State of Kerala},\footnote{OP NO. 20278 of 1997, dated 2nd June1998(unreported)} The
court observed that the state cannot refuse the right of citizen’s access to Coastal  
Management Plan. Right to information of Citizen is the right protected by our Constitution
of India under Art.21. State or Local authorities cannot refuse it and they have a duty to
publish CZMP. The court also directed the government to give sufficient publicity to
management Plan prepared by the State so as to enable the public to know whether there is
any deficiency in that plan and to ensure that CZMP was properly implemented and not
violated by anyone and also to ensure that it’s not violative of rights of coastal people. It is
said that economic activities in the ecological sensitive areas should be able to assure social
justice in distributing environmental resources also.\footnote{Pavan Vinayak et al. \textit{The role of Principles of environmental justice in securing equality in distribution of environmental resources- A critique,} 6 IN LAW MAGAZINE NLSIU(Nov.2020)} So, implementation of the concepts of
environmental justice and environmental democracy is the need of the hour otherwise the
environmental/ coastal resources will unequally be distributed between the capitalist and
fishermen. The fishermen will become the scape goats of the hazards of coastal degradation.
Hence, the right to participate in coastal policy making and awareness and informed consent
for coastal development should be made available to the coastal community and all other
stakeholders.

\heading{Securing and Enforcing Property Rights/Land Rights}

\noi
The Coastal community is oblivious of their land rights. The individual and collective rights
of fishing communities over coastal lands became highly debated in recent years after the
frequent amendments of CRZ Notifications in India, as communities do not even have titles
and deeds for their houses and settlements even after 60 years of Independence. In the past
few decades, fisher folk have prepared and planned to request land rights from the State
because of the rapid development in these areas. There have been tourism/development
projects, and interest groups who have been waiting to take over the coastal land and its
abundant resources. However, the coastal community’s traditional claims over coastal lands
have precluded the scope of privatisation in the coastal areas to some extent. When CRZ
2019 came into force the land mafia threatened the coastal community and purchased the
coastal properties by saying that their properties will be displaced by the provisions of recent
notification.\footnote{Kerala Coastal Zone Management v. Maradu Municipality, 2019 (2) KLJ 944.} So, proper alertness should be given to them by unambiguously defining
property rights and assurance should be given to them that the policy changes will not
adversely affect them. Then only they can be capably safeguarded against meddlers. The land 
ownership issues in coastal areas are gaining eminence with use-based entitlements.
Construction of Special Economic Zones, land mafias/land acquisitions, non - recognition of
coastal rights have changed the concept of land ownership in coastal zones.\footnote{Sudarshan.S. Rodreguez, \textit{Coastal land rights of Fishing communities- Claims for survival,} Dakshin
Foundations, Bangalore 42(2010)} They tend to
provide more handler-based rights in coastal Zones, where right is delegated over land for a
demarcated commercial or development activity. Furthermore, under the Coastal Regulation
Zone Notification,2019 the coastal dweller’s right to use coastal lands and concept of coastal
commons etc. are not recognized.

\noi
The fishing communities and coastal lands rapport is very intricate in nature. Most of the
studies on fishing communities mention the absence of official legal ownership rights for
fishers over coastal lands.\footnote{Sebastian Mathew, \textit{Coastal Management Zone- Implications of Fishing Communities,} 43 EPW 25, 17-21 (Jun.21-27,2008)} One of the dominant rationales for non-identification of their right is due to the poor condition of life and the inadequate conditions of the marine fishing populations in India. The gathering of the unabridged community on a narrow strip of land
along the length of coastline also increase their exposures and it is a consequence of the
highly spread nature of the fishery resource and the dangerous of fishing operations by them.
Every fisherman chooses to live on the seafront to spot the sea and to land his craft safely.
This also have implications on whether their lands were considered ‘authorised constructions’
under the CRZ Notification or not. In general, the traditional occupation of fishing seems to
need a closer association with the sea than with the land and it is decisive for signifying the
interests of fishing communities. The fishing community people have been the traditional
inhabitants of the coast. Their livelihood adjoining the sea is deep-rooted in their connotation
with the sea. They never identified the need to prove their right over coastal lands through
land titles. The triviality is that fishing community is also conferred to land-based
regulations, but it is not recognised by them due to their social, economic, educational
backwardness. The recognition of land rights of local inhabitants of coastal area is a necessity
and it is claimed that the local inhabitants should be avoided from the strict application of
CRZ notification. So, while considering the social preferences and beliefs regarding the land
rights of coastal people, a liberal approach to eco system people and their traditional land
accessibility is highly required as it has become a sensitive issue now a day.

\heading{Right to Coastal Commons}

\noi
Coastal common means the coastal resources that come under the realm of common property
from which various communities and stakeholders derive economical, and ecological
benefits.\footnote{Niskula Jamir et al., \textit{Coastal Commons and Conservation Cascades: Indian saga at Protection attempts under CRZ Notifications, 1} ENV.LAW AND SOCIETY JNL,NUALS,45-66 (2013).} Indian coastal regulations are quiet on fisherfolk’s traditional rights to coastal and
ocean spaces. High courts and Supreme Court of India upheld the right to commons in case
of fishing communities and other inland communities. Indian courts established that
foreshore between the high tideline and low tideline belong to the government and fisherfolks
can work their trade without unsettled to the land. In \textbf{Abas v. Andi Chettiyar},\footnote{Abas v. Andi Chettiyar AIR 1963 Mad.74.} a single
judge in Madras High Court denied the plea of a property owner to possess and fence certain
forelands and upheld the customary rights of fisherfolks to dry fish and park boats on the land
. Thus, public access to the commons has been ferociously guarded by Indian Courts. In
\textbf{Jagpal Singh v. State of Punjab},\footnote{Jagpal Singh v. State of Punjab AIR 2011 SC 396} The Supreme Court held that enclosure of a village pond
by real estate developers is illegal, and the court ordered to evict all illegal /unauthorised
occupants of village and restored it for common use of villagers. This is a watershed
judgment of The Supreme Court which ensures the rights of coastal commons\footnote{Mukul Kumar et al, \textit{Mapping the coastal commons : Fisher Folks and the politics of Coastal Urbanisation in Chennai,} 29 EPW 48, 46-53(Nov.29,2014)}

\noi
The CRZ 1991, although is a delegated legislation, cited the traditional rights and customary
uses of coastal areas. But it was restricted to the housing purposes and other built-up spaces
for their works. It did not admit the customary livelihood, culture, and other uses of the
common fisherfolks. CRZ 2011offered certain concessions to the fisherfolks and instructed
maritime states to prepare detailed Coastal Zone Management Plans, land -use plans and
other long- term plans for CRZ area. However, the important clauses in CRZ 2011 remain
unimplemented and CRZ 2019 is silent about all these issues. The frequent amendments
weakened the law and because of its poor enforcement leads to the spread of privatisation,
encroachments, urbanisation, and as a result withdrawal of coastal people from coastal spaces
have increased in an unprecedented manner. For example, Kohli community in Mumbai are
losing their traditional space. It is connected to real estate booming in coastal areas for the
purpose of tourism and other developmental and business activities. So, the need of
protecting the right to coastal commons is a necessary evil to promote the community values, aesthetics, culture, and tradition of coastal people and thereby to promote their rights and
needs in a significant method.

\heading{Suggestions}

\noi
The traditional coastal communities are integral to the coastal areas of India. The State must
recognise the rights of the fishing community’s livelihood issues as well as the access and use
of beach space. The CRZ Notification initially contained certain provisions that protected the
interests of fisher communities whose livelihood depended on a healthy coast. If coastal
communities are involved in the implementation of the notification and its monitoring, it
could ensure the protection of coastal habitats as well as relieve the state government from
the responsibility of being involved in the day-to-day implementation of the notification.
Therefore, it is important for the notification to emphasise the necessity of capacity building
and active participation of local governments and citizens in rural and urban coastal areas. In
its contemporary form, the above concepts are absent in the notification. Several traditional
fisherfolk and coastal community institutions bargain themselves against jointing in CRZ
2019 because the notification does not recognise their presence or their potential role in
coastal conservation. It would be useful to study how these community institutions can play a
role in the implementation and monitoring of coastal zone management laws.

\noi
Further state should regulate the entry of external actors in eco-sensitive areas and to clear the
demands to strengthen the CRZ Notification while recognising and guaranteeing rights of
fisher communities to resources and its management which must be obligatory in present day
scenario. There has been a consistent demand to recognise the rights of fishing communities
to access and use the coastal space. There was also a very clear mandate for the unambiguous
role and responsibility of fishing communities in the management and protection of the
coasts. Serious flaws and omissions in the enforcement and implementation design should be
avoided. Vagueness and arbitrariness of the new categorisation, deficiency of clear
procedure, definitions, objectives, lack of transparency, non-involvement of stakeholders are
reasons for the failure of present CRZ Notification.\footnote{ Ayush Verma, \textit{A Sail through the management of coastal zones in India,} BLOGIPLEADERS (JUL. 16, 2020)
Blog ipleaders.in/Sail-Management-Coastal-Zone -India.} To avoid all these issues, it can suggest
the following recommendations.

\heading{Recommendations}

\vspace{-.4cm}

\begin{enumerate}
\itemsep=0pt
\item Use of the coastal area only for direct resource-based livelihoods and infrastructure
needs of the country.

\item Sustainable development of the coast and its resources in future.

\item Participation of stakeholders in the decisions related to the use and management of the
coast and its resources.

\item Evolve and strengthen government policies along with community-based governance
structures.

\item Special provisions for the control, rights and access of coastal resources and coastal
lands to ensure the life and livelihood of coastal people and for coastal protection.

\item Ensure coastal land rights of fishing communities as defined by socio-cultural
traditional boundaries to make a balance between development planning, coastal
management, and fisher livelihood security.

\item Inclusion of a list of livelihood activities which are permissible in CRZ region and it
should be specially designed for the protection of coastal community.

\item Relaxing the No Development Zone (NDZ) should not be detrimental to the interests
of traditional coastal communities.

\item Allowing dwelling units for fisherfolk

\item Permitting economic activities like tourism under the ownership of fisher folk.
\end{enumerate}

\vspace{-.2cm}

\noi
The above suggestions cannot guarantee the livelihood security of fishing communities
absolutely. While the government has established policies and legislations for conserving
marine resources, the state governments and respective forest and marine departments are
facing issues in implementing these regulations. But we can understand that the present
patchwork Notification does not solve the issue of conflicts and issue of access and rights
over the coastal space which is inherently allied to the livelihood security of fisherfolk. It is
crucial that rights of traditional fisher communities to coastal spaces be legally recognised
and expanded in prevalence over infrastructural development in such places. Merely
permitting a set of fishing activities and facilities within the CRZ or any coastal law falls
short of the fisherfolk’s demand for control over coastal governance. Ignoring this element
will only lead to further conflict between fisherfolk and non-fisher interest groups like large
tourism/ industrial mafia and government authorities, as being witnessed in the State of Goa
and Kerala where large developmental activities by corporates are causing problems to the coastal community. Hence it is vehemently recommended that the conservation methods
should strike a balance between the economy and ecology at the same time.\footnote{Radha Krishan, \textit{Environmental Protection: Legal and Human Right Perspective,} 2 NJEL 1,30-34(2019)
\url{www.stmjournals.com}} Most
importantly, the model of community-based conservation that found success in other
countries should be adopted in Indian context also.

\noi
Further, enunciating human rights with the environment generates a rights-based approach to
environmental protection as it will be obligated to develop new approaches and policies to
safeguard the rights of individuals and communities reliant on natural resources. It will also
help in providing remedies to people harmed by environmental degradation. Taking a human
rights approach to coastal protection is beneficial that it reinforces the concept of mutual
goals and the serious outcome each may have on the others. The understanding in terms of an
incompatible burden between development perquisites, environmental protection and human
rights of local inhabitants must be abridged in coastal zones, both at international and
national levels.

\heading{Conclusion}

\noi
The coastal law regime in India faces a lot of turbulence in the present scenario and the CRZ
notification leaves so many issues unaddressed and unclear. It is high time to identify the
paths for improvement in Indian environmental law especially the coastal regulations because
the present CRZ is incompatible with its own objectives due to its deceptive mechanism,
deficient procedures and lack of transparency. The issues of coastal encroachments
throughout the country are continuing and it is witnessing the lacunae in existing legislation.
The concept of development has historically taken precedence over coastal protection, and it
is not in conjunction with stakeholders as well. The Grant of coastal clearances is a key step
in the statutory framework to balance ecological concerns of the coastal environment, and
concerns of coastal communities about the quality of their immediate surroundings, including
issues of sustainable access of poor and marginal communities to common property resource
for their everyday subsistence. There is a need for having public consultation and
participation in coastal management, and a provision for such participation and consultation
should be included in the Indian CRZ Regime. The paper describes and analyses the
extremely limited space for public participation in the existing CRZ regime. It argues for a
broad-based coastal law that has public consultation at multiple stages, right from screening, 
scoping, and appraisal during EIAs, to post-clearance monitoring as well as compliance of
clearance conditions. It is possible that a comprehensive, dedicated national legislation for
coastal zones which even while adopting uniform approach to multifarious issues faced by
Indian coastal areas will give States enough room to address the local issues of coastal
community. In the current pace of economic growth, a battle for coastal land grab to fuel
development projects, tourism and so on is the truth of coastal spaces. This suggests that a
sense of urgency must accompany for a new legislation. Until then, each individual fishing
settlement will have to fight its own encounter against each local developmental menace and
bear in an unjust political space. The present CRZ Framework is neither comprehensive nor
exhaustive, so a Pan National Law is necessary to ensure environmental sustainability as well
as livelihood security of the coastal community. The Shailesh Nayak committee in 2014
described the threat of displacement of fisher communities as coasts were being opened
up for development. CRZ 2019 expands such issues for those who live in various
informal settlements along the coast. Therefore, while drafting coastal policies we
should focus on a central theme of \textit{equitable, sustainable, pro-poor, pro-fisher housing
and development models for our coasts.}

\end{multicols}
\label{end2020-art3}
