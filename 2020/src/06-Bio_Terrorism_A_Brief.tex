\setcounter{figure}{0}
\setcounter{table}{0}
\setcounter{footnote}{0}


\articletitle{Bio-Terrorism – A Brief Legislative Scrutiny}
\articleauthor{Smt. Jayamol P.S\footnote{Assistant Professor, Vaikunta Baliga College of Law, Udupi} and Dr. Rangaswamy.D\footnote{Assistant Professor, Karnataka State Law University, Hubballi}}
\lhead[\textit{\textsf{Smt. Jayamol P.S and Dr. Rangaswamy.D}}]{}
\rhead[]{\textit{\textsf{An Analysis of the Essentiality of Constitutional...}}}

\begin{multicols}{2}

\noi
\textit{“Wars can no longer contain the population, so biological terrorism will become the
weapon of choice.” -- David Icke}

\heading{Introduction}

\noi
Terrorism, by all means, challenges the stability of societies and the peace of mind of the
people living in those societies. It is an age-old phenomenon that has evolved into an
international network and threatens international peace, democracy, and development. In the
modern era, the impact of terrorism is not limited to a particular region. With the advent of
technology, access to resources and information, the beginning of the 21$^{\rm st}$ century can be
treated as an era of globalized terrorism.\footnote{Gus Martin, Terrorism and Homeland Security3 (2011).}
 The new terrorism of today is characterized by the
threat of weapons of mass destruction. That is evident in the causalities and destruction that
happened in the terrorist attacks all across the globe in the past few years. The advent of
technology, biotechnology, microbiology, molecular chemistry, and genetic engineering has
opened new vistas for mankind. But, on the other side, it has some adverse effects in the form
of manufacture and proliferation of biological and chemical weapons. Modern technology,
innovative sources of funding, and world network connections have given terrorists
extraordinary capabilities that were demonstrated in different ways.\footnote{Shashi Shukla, Emerging New Trends of Terrorism: Challenges Before the United Nations The Indian Journal
of Political Science September 1,2020,9.30PM. \url{http://www.jstor.com/stable/41856202}}
 Consequently, countries
are forced to spend the kind of intellectual, physical, and other resources in monitoring and
assessing the activities of the various terrorist organizations around the world. The harm
caused by the international terrorist movement has been described as 'transnational harm' that
poses a serious challenge to national and international security.\footnote{\textit{Id.}}
Terrorist groups have
generally sought to achieve their objectives with small arms and conventional explosives.
This tendency may be changing, however, with the emergence of more deadly forms of
terrorist activities. 

\noi
Biological terrorism is rampant than before and more threatening than any other explosives or
chemicals. Preventing or countering bio-terrorism will be extremely difficult. The process for
creating biological weapons is now available on the internet and anyone with modest finances
and basic training in biology and engineering could develop an effective weapon at little
cost.\footnote{L.R.Reddy, Bio-Terrorism As A Public Health Threat,23,( 2002).}  The terrorist groups vowed to destruct, might deliberately produce and disseminate
disease agents that are contagious in humans, such as pneumonic plague bacteria or various
types of haemorrhagic fever viruses, to trigger widespread epidemics that would undermine
social structures.\footnote{Jonathan B.Tucker, Chemical/Biological Terrorism: Coping with a New Threat, Politics and the Life Sciences, Sep.1,2020, 9.30 PM \url{http://www.jstor.com/stable/4236227.}}  The outbreak of corona virus in 2020 has caused unprecedented
consequences all across the globe. Still, discussions and deliberations are going on relating to
the source and nature of viruses leading to the suspicion that it is a bio weapon developed by
China.

\heading{Meaning and Definition of Bio-Terrorism}

\noi
The term ‘terrorism’ doesn’t have a widely accepted definition. However, what is commonly
accepted is its efficacy to penetrate terror and thus to pressurize the people. Bio weapons and
chemical weapons are often used together. But the Chemical warfare agents are manmade
poisons, whereas biological warfare agents are microorganisms and naturally occurring
toxins that cause illness or death in people, livestock, and crops.\footnote{\textit{Id.}}  Biological weapons are so
deadly that it can destruct all living organisms.\footnote{\url{https://www.unog.ch/80256EE600585943/(httpPages)/29B727532FECBE96C12571860035A6DB?OpenDocu
ment}}  Biological warfare agents are the type of
organism or toxin used in a weapons system, which is dangerous to humans, plants and
animals. Normally, there are five different categories of biological agents that could be used
in warfare or terrorism. These bio-agents include\footnote{Bacteria-single-cell organisms that cause diseases such as anthrax, brucellosis, tularemia, and plague.} bacteria, rickets, viruses, fungi, toxins
etc. 

\noi
Bioterrorism is the dissemination of biological agents\footnote{In 1969, the U.N. General Assembly} into the population by an individual
or group intended to cause severe illness injury or death. As mentioned above, the diseasecausing organisms like bacteria, viruses, fungi, or rickettsia or poisons or toxins can be used
in biological weapons. These bio agents can be modified from their original form to make 
them more adaptable for using as weapons.\footnote{\url{https://www.unog.ch/80256EE600585943/(httpPages)/29B727532FECBE96C12571860035A6DB? Open Doc
ument.}} The United States, Biological Weapons AntiTerrorism Act of 1989, defines biological agents. These microorganisms are varied in nature
and number. It includes bacteria, viruses, fungi, rickettsia or protozoa or infectious substance,
or any naturally occurring, bioengineered, or synthesized component of any such
microorganism or infectious substance. It is capable of causing death, disease, or other
biological malfunction in a human, an animal, a plant, or another living organism.\footnote{Durward Johnson and James Kraska, Some synthetic Biology may not be covered by the Biological Weapons
Convention,(July24..2020, 9.00 AM).} It is
evident, that the agents can be used as bio-weapon by the extremists and terrorists. India's
Public Health (Prevention, Control, and Management of Epidemics, Bioterrorism, and
Disasters) Bill, 2017 defines bio-terrorism.\footnote{Sec. 2(1)(b) Public Health (Prevention, Control, and Management of Epidemics, Bioterrorism, and
Disasters)Bill, 2017.} According to which, bio-terrorism is the
purposeful adaption of biological agents to cause disease or death of human beings or any
animal or plant through the dissemination of microorganisms or toxins through any medium.
The slow and steady destructive capacity of bio-weapons have got wider acceptance and it is
acknowledged by the world community. That is the main threat involved with bio-weapons. 

\heading{Background of Bio-Weapons}

\noi
The long and un chequered history of bio-weapons show that throughout the ages, all across
the globe, there have always been efforts to use germs and disease as weapons. The
indigenous South Americans deliberately used plant-derived arrow poisons such as curare
and also toxin from poison. This is used mainly for hunting.\footnote{\textit{Supra n.5.}} The first bio-weapon ever used
in the history was small pox which was adopted by the British army during the French and
Indian War. The British gave to the Native Americans clothes that are used by the British
people who had infected with small poxes.\footnote{Weapons of mass destruction,
\url{https://www.globalsecurity.org/wmd/intro/bio_smallpox.htm\#:~:text=The\%20outbreak\%20of\%20smallpox\%20 in\%}} The result was that the widespread outbreak of
small pox and British army’s victory in the battle. The development and applying of bioweapons were common during the world wars where it was indiscriminately used by all
nations. During World War I, Germany used horses succumbed with anthrax disease. Later,
this method was adopted by other nations also. There were instances of using arthropods and 
vector-borne pathogens as weapons in wars.\footnote{Manas Sarkar, Bio-terrorism On Six Legs: Insect Vectors Are The Major Threat To Global Health Security, 2 Public Health (September3,2020,9.45PM) \url{http://www.webmedcentral.com/article_view/1282}} Likewise, many instances of the manufacture,
use, and proliferation of bio-weapons were present in history. As a result, to prevent the
indiscriminate use of bio-weapons the Biological Weapon Convention (BWC) has been
adopted. The present outbreak of the Corona virus points the finger towards the Chinese labs.
It can be expected that the coming years will reveal the exact fact behind the present
pandemic.

\heading{Impact of Bio-Weapons}

\noi
The intentional use of pathogens or other biological agents for terrorism has proved highly
effective and cause damage on a larger scale than “traditional” terrorist attacks. Keeping the
societies under severe threat, it could accelerate fear and showcase distrust far beyond those
communities immediately affected. Ongoing biological research programs for both defensive
and offensive purposes have attained highly advanced stages in many countries like Russia,
United States, China, Britain, Iran, Iraq, Canada, etc.\footnote{Randall D.Kats, Friendly Fire: The Mandatory Military Anthrax Vaccination Program,1836,50 Duke Law
Journal(2000).} The impact of bioterrorism would be
more when compared with the traditional forms of terrorism.\footnote{Because of the ability of pathogenic microorganisms to multiply rapidly within the host, small quantities of a biological agent if widely disseminated through the air as a respirable aerosol can inflict casualties over a large
area}  It affects the psychology of
people in large numbers. The source cannot be identified and the impact cannot be assessed
easily which puts the laymen to great trauma and mental stress. Another important obstacle is
the manufacture, storage, sale, and use of these bio-weapons. It is not so easy to locate who
all are engaged in the manufacture or uses this weapon.

\noi
It is a general notion that a chemical release or a major explosion is far more manageable
than the biological challenges posed by smallpox or anthrax. In a biological attack, the
consequences could be more devastating. For eg., an anthrax attack might produce casualties
numbering hundreds or thousands because of their stability and infectious nature.20\footnote{Leonard A.Cole,Countering Chem-Bio Terrorism: Limited Possibilities,15politics and the Life Sciences,
(September 2,2020,8.00PM) \url{https://www.jstor.org/stable/4236233.}} After an
explosion or a chemical attack, the worst effects of the incident can be easily overcome. The
dimensions of the catastrophe can be defined, the toll of injuries and deaths can be
ascertained. But in bio-weapons each day new cases can be expected and in new areas which 
highlights the destructive capacity of the bio-weapons.\footnote{Supra n at.4.} The ascertainment of catastrophe is
the most difficult part of the bio-weapon. The impact would be slow and steady, but it
contaminates wider areas and a large number of people across the country and later across the
globe. The bio-agents are difficult to find out as they are virtually undetectable and can be
handled with relative ease by properly trained persons. They are highly contagious with a
short and predictable incubation period and infective in low doses. The well-planned
perpetrators have all means to protect or treat their forces and population against these
infectious agents or the toxins.\footnote{Piyali Sengupta \& Ayushi Agrawal, Emerging Threat of Bio-Terrorism: An International Perspective,3 Journal of Politics \& Governance,82( 2014).}

\heading{Bio-weapons and public health}

\noi
Almost all of the materials and items of equipment used to cultivate Bio agents have
commercial applications and are easily available in the market. It is being used in the
manufacturing of food products, animal feed supplements, drinks, bio pesticides, vaccines,
and pharmaceuticals. Seed cultures of pathogenic bacteria such as anthrax can be purchased
from commercial vendors by sending a request letter on the letterhead of a university or
research institute.\footnote{Jonathan B.Tucker, Chemical/Biological Terrorism: Coping with a New Threat,15 Politics and the Life Sciences,183(August 12, 2020,(.00 PM) \url{http://www.jstor.com/stable/4236227.}} This easy availability makes it the main reason behind public health
issues. One of the main features of the bio-agents is that it is invisible, tasteless, and carries
no smell of its own. Therefore, no reliable biological detection and warning systems are
currently available. Apart from that, the incubation period for the bio-agents to attack the
body after infection may extend depending on the bio-agent. Slowly the number of infected
persons may increase from hundreds to thousands.\footnote{\textit{Id.}} The release of these agents could go
undetected and unnoticed for many days and weeks. The release of bio-agents would then be
followed by mass illnesses, necessitating the first line of response by the public health
community.\footnote{Ronald M. Atlas, Combating the Threat of Biowarfare and Bioterrorism: Defending against biological weapons is critical to global security, Bio Science, Volume 49, Issue 6, June 1999, Pages 465–477,(available at \url{https://academic.oup.com/bioscience/article/49/6/465/229529#94371792} (last accessed on 1.8.2020). }  The application and use of these bio-weapons are manifold. It has been widely
applied by the military forces and other non-state entities for political assassinations, it can
cause economic imbalance through the adverse effect in livestock, environmental degradation, and can also be a reason for the different kinds of diseases, fear, and friction
among the masses.\footnote{\url{https://www.unog.ch/80256EE600585943/(httpPages)/29B727532FECBE96C12571860035A6DB?OpenDoc
ument )last accessed on 1.8.2020.)}} In a nutshell, this invisible, undetected agent can be the root cause
behind the widespread hardships and sufferings of the infected people. Biological warfare
agents would likely to cause significant impacts on the medical care system. Special
medications or vaccines not generally available in standard pharmaceutical stocks would be
required.\footnote{\textit{Supra n} at.18}

\heading{Leal Frame Work Against Bio-Terrorism}

\noi
The terrorists try to attain legitimacy through the threat or act of large-scale violence, and
thereby achieve the ability to impose their values upon other countries.\footnote{Scott Carry, The Tipping Point: Biological Terrorism, 3 journal of Strategic Security,13 (September 3, 2020,
9.00 PM) \url{http://scholarcommons.usf.edu/jss/vol2/iss3/2.}} Biological weapons
and bio-agents are prevalent in societies worldwide. International Organisations have
successfully identified and addressed the problems well in advance and tried to restrict the
use of bio-agents.

\heading{International Conventions}

\noi
International Conventions reflect the general notions of the state parties. Since long way back
itself bio-terrorism which was prevalent worldwide since time immemorial had been taken
note of by the world community. It was culminated in the form of Geneva Protocol

\heading{Geneva Protocol 1925}

\noi
Geneva Protocol\footnote{It was registered in \textit{League of Nations Treaty Series} on 7 September
1929. \url{https://www.un.org/disarmament/wmd/bio/1925-geneva.}} declares that Bio-weapons are prevalent throughout history among nations
and there was a dire need to control it. It prohibits the use of biological weapons. But it has
not restricted the possession and development of biological or chemical weapons which was
the main drawback of the protocol.

\heading{Biological Weapons Convention (BWC), 1972}

\noi
The ‘Convention on the prohibition of the development, production, and stockpiling of
bacteriological (biological) and toxin weapons and on their destruction 1972’, (BWC)\footnote{Signed at London, Moscow, and Washington on 10 April 1972. Entered into force on 26 March 1975.} is the 
first and the only convention prevailing against the prohibition of biological weapons. The
main objective as enshrined in the preamble of the BWC is to achieve progress towards
complete disarmament by prohibiting all types of weapons of mass destruction.\footnote{\textit{See the preamble of the BWC.}} The
convention intends to achieve the following objectives: -

\noi
\textit{Complete prohibition of bio-weapons:} The convention is very specific that it intends to
eradicate biological weapons or such type of bacteriological weapons being used against
people. The parties to the convention affirmed that development and production of bio agents
only for the prevention of diseases, protective or other peaceful purposes. They undertake not
to develop, produce or stockpile microbial or other biological agents for other objectives. It
explicitly prohibits the use of weapons, equipment using such bio agents or toxins for
commercial or armed purposes.\footnote{Article 1.of \textit{BWC.}}

\noi
\textit{Destruction of Biological agents -} The parties to the Convention undertake to destroy or to use
all biological agents in their custody, jurisdiction, or control immediately or within 9 months
of entry into force of the convention, for peaceful purposes.\footnote{\textit{Id..} Article II.} 

\noi
\textit{Non-transfer of biological weapons} - The parties to the convention are prohibited from
transferring biological agents, toxins, weapons, or equipment\footnote{\textit{Id..} Article III. } It also prohibits the
development, production and stockpiling, acquisition or retention of the biological agents.\footnote{\textit{Id..} Article IV. }

\noi
\textit{Cooperation among the Members -} It also prescribes cooperation and consultation in solving
the problems about the application of the convention. It has to be undertaken within the
framework of the United Nations.\footnote{\textit{Id.} Article V.} Breach of obligations of the convention can complain
before the Security Council and for that matter; cooperation has to be extended by the state
parties. The results of the investigation have to be informed to the state parties by the
Security Council.\footnote{\textit{Id.} Article VI (1).} It also encourages the development and application of scientific
discoveries for peaceful purposes. For this, the members can cooperate alone or in connection 
with other organizations. While implementing the Convention it shall not hamper the
economic or technological development of state parties to the convention.\footnote{\textit{Id.} Article X (1)}

\noi
\textit{Amendment procedure and review :} Amendments can be done with the acceptance of majority
of state parties.\footnote{\textit{Id.} Article XI.} For the review of the procedures of the convention, after five years of the
entry into force of this convention, a meeting of state parties is recommended in the
convention in Geneva to see that the preamble and other provisions of the conventions are
realized.\footnote{\textit{Id.} Article XII.} Since the Convention shall be of unlimited duration, the state parties to the
convention can go out of the convention, if it jeopardized the supreme interests of its
country.\footnote{\textit{Id.} Article XIII (1).}

\noi
A review conference was held in Geneva for complying with the provisions enshrined in the
BWC. Moreover, the state parties recognized that, parties to the Convention should ensure
mutual co-operation for peaceful use of bio-weapons which would help to reduce the
complications. Such an objective made them to agree on the annual submission of
confidence-building measures regarding the areas of research centres and laboratories, and
national biological defence research and development programmers, outbreaks of infectious
diseases and similar occurrences caused by toxins; encouragement of publication of results,
and promotion of the use of knowledge; active promotion of contacts, legislation, regulations
and other measures.\footnote{\textit{Id..}}

\heading{Loopholes in BWC}

\noi
Even though the BWC had some novel ideas for the proliferation of bio-weapons and total
disarmament, there are some areas to be pointed out. The BWC does not contain any
provision for monitoring the members about the compliance of the provisions. Still, in many
parts of the world, among the member countries themselves, experimentations in bio-agents
are going on with experts in these areas. As declared by former President of Russia, Boris
Yeltsin, even after a signatory of BWC, the Soviet Union engaged in a secret biological 
weapons program.\footnote{Barry R. Schneider, Biological Weapons Convention –International agreement, (Septtember2,2020,9.30PM) \url{https://www.britannica.com/event/Biological-Weapons-Convention.}} A biological weapons program doesn't require huge plants or a large
number of personnel. That makes it difficult to find out the violators of the BWC among state
parties even after closed monitoring. The state parties are still ignorant or less concerned with
the modus operandi of bio terrorists. The outbreak of corona virus has given a renaissance to
the BWC provisions and its impact among world countries.

\heading{Legislative history in India}

\noi
India, being a democratic country, is most vulnerable to terrorism. There are internal and
external forces playing behind to attack our democracy. In the backdrop of an increased
number of terrorist attacks, India is armed with much legislation to combat terrorism in any
form. The National Investigation Act, 2008\footnote{The Preamble of the Act is to constitute an investigation agency.} is the pioneer law in this regard and the Act has also prescribed for the establishment of a National Investigation Agency for bringing more efficiency in the investigation process in terrorist activities. The amendments were brought
out in the Unlawful Activities (Prevention) Act,1967,\footnote{\textit{See} the Preamble of the Act which is to prevent certain unlawful activities of individuals and associations.} a Code of Criminal Procedure,1973 to
strengthen the fight against terrorism. The changes in the Criminal Procedure Code, bound to
have a bearing on not only the accused of terrorist acts but also on the victims thereof. The
new laws have doubled the length of time as the suspected militants are allowed to be
detained without charge. The tougher portion of UAPA, 2008 will be executed by the
National Investigation Agency. Still, all these enactments lack a specific attention towards the
proliferation of bio agents or threat of bio weapons.

\heading{Public Health (Prevention, Control, and Management of Epidemics, Bioterrorism, and
Disasters) Bill, 2017}

\noi
In India, the first legislative attempt to address the epidemic and bio terrorism commences
with this Bill. The Bill as its objective says that it aims to prevent and manage epidemics,
public health consequences of disasters, acts of bioterrorism, or likelihood of threats.\footnote{\textit{See} the Preamble of the Bill.} It has prescribed provisions for the powers of central government, state and union territories, and
district or local authorities in case of public emergencies. It has also prescribed provisions for observation, quarantine, and for isolating a person or class of persons if the situation
warrants.\footnote{Section 3 of the Bill.}  The provisions included in the Bill are a dire need of the time to combat bioterrorism in Indian soil.

\heading{Findings and Conclusion}

\noi
Terrorism, in any form, is a threat to mankind and it has to be eliminated from the world. The
states have to pool their resources to counter international terrorism including bioterrorism.
New ventures like bio defence systems and strict public health monitoring are the need of the
time to prevent bio-terror attacks. The measures to prevent it at any time is the primary
requirement to respond to its outbreak. Early detection of outbreak and capability to assess
the impact is an essential tool in the case of biological weapons. The time taken to detect a
bio-terror attack is very crucial, as faster the health department can respond to prevent its
exposure and to begin treatment of those who have been exposed. Additional vaccines and
new therapies are needed, and some countries have already developed vaccines. Active
immunization will probably be the best way to protect military forces against a wide variety
of biological threats. Apart from that, the proper and timely identification of the infectious
agent is very important for the protection of the common man and even to the health workers.
The medical team and the health workers have to be equipped with proper Personal
Protective Equipment, masks, gloves, and other protective measures to guard themselves
against contamination, and the antidotes and antibiotics should be available against the bioagents.

\noi
On a close perusal of global and national legal framework it is evident that, the studies and
research conducted to assess the actual efficiency of counter bioterrorism measures are
insufficient. International attempts and regional laws to combat bioterrorism are inadequate
as technology is developing day by day. Both at the national and international level, bio
agents and bio-weapons education and awareness should be given to health professionals and
even to voluntary groups. However, the presence of a convention like BWC will serve the
purpose of a watchdog even among non-signatory members from dispensing with biological
weapon programs. But it is too insufficient because it doesn’t have any strict compliance
mechanism which has to be rectified. It is necessary to strengthen preventive bioterrorism
measures using competent institutions that can better cooperate with the international community in their fight against terrorism and especially bioterrorism. Moreover, global community has to awaken from sleep and a global moral consensus among the states condemning bio terrorism is required.

\end{multicols}
