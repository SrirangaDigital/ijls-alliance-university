\setcounter{figure}{0}
\setcounter{table}{0}
\setcounter{footnote}{0}


\articletitle{Tussle between the US and Iran in the High Seas: Is it a Sign of an Upcoming War?}
\articleauthor{Manu Sharma\footnote{Student, Symbiosis Law School, Pune.}}
\lhead[\textit{\textsf{Manu Sharma}}]{}
\rhead[]{\textit{\textsf{Tussle Between the us and Iran...}}}

\begin{multicols}{2}

\heading{Introduction}

\noi
Since 1980’s, there has been a constant strain in the diplomatic relations between the two
powerful nations of the world, the USA and the Islamic state of Iran. Whether we look at
the Iran- Iraq war of 1980’s or the recent killing of Iran’s superior military commander
General Qasem Soleimani, the constant tussle between the two nations have gathered the
worldwide attention of the experts who are analyzing the possibility of an “armed conflict”
or a “war” in the coming time. This is the recent event which took place in the International
water of the Persian Gulf where the Iranian Coast Guard gunboats traversed the American
naval ships from an enclosed range. The US claims such perusal as an act of harassment and
danger within the limits of high seas.

\noi
The legal dimensions of this particular event on one hand includes the legality of perusal by
the Iranian coast guards and on other hand the use of armed force in the pretext of selfdefense by the US Navy ships. Under the international legal system and principles, the right
to defend self has always been a topic of controversy and debate due to its vagueness and
over- broadness. However, The US used its mechanisms in consonance to both international
as well as its municipal laws. This paper attempts to critically analyze the legal aspect of
this Iranian gunboat harassment with special focus on the engagement rules of the US
Military laws, humanitarian principles and loopholes in the current settings.

\noi
The issue came into headlines on April 15, 2020 when eleven Iranian gunboats repeatedly
traversed six US navy vessels, from an extremely close range during its joint integration
operation with US army Apache helicopters in the international water of the Persian Gulf.
The US Navy claimed this approach of IRCG’s gunboats ‘intentional, dangerous and
harassing.’ The matter escalated pretty quickly when the US president Donald Trump
instructed the Navy to use an aggressive approach including shoot down and destroy the
gunboats harassing them if required. This seems to be the most covert threat of using an
armed action against Iran since authorizing the targeted attack at the Baghdad International
 airport killing the former IRCG’s commander General Soleimani earlier this year. The
author of this paper intends to examine whether the actions of both the nations stand on the
justifiable grounds under the international legal system. Furthermore, particular reference
has been given to the standing rules of US military engagement as they are the result of the
national enactment of the principles of international law pertaining to armed conflicts,
providing an operational and structural framework for any defensive action by US forces.

\heading{Research Questions}

\vspace{-.2cm}

\begin{enumerate}[label=$\alph*)$]
\itemsep=0pt
\item What are the legal dimensions involved in this tussle between the US and the
Republic of Iran in the High seas?

\item  What were the justifications provided by both the countries pertaining to this incident?

\item What limitations does this incident highlight in pretext to the principles of
international law and its customary legal provisions? 
\end{enumerate}

\vspace{-.2cm}

\heading{Research Methodology and Objectives}

\noi
The objective of the research is firstly to analyze, the horizons of international law related to
armed conflict at high seas with this particular incident of Iranian gunboat harassment with
special emphasis to US’s domestic legal and national policy framework to better understand
as to how international law is actually practiced in its real application. Secondly, to analyze
the consequences of this incident on the relation between both the nations which were
already heavily tensed especially after the killing of General Soleimani. Thirdly, to
understand the horizons of compliance and limitations of the international law with the help
of this incident. Lastly, the challenge that comes in upholding international law with a
country like Iran which works against the directives of the US in the region of Middle East
and has openly threatened the Americans.

\noi
The research methodology is ‘doctrinal research’. The author focuses on determining the
position of existing international legal order law, its limitations and possible scope of
improvement with the help of an incident that happened between two powerful nations of
the world, the US and Iran in Persian High seas recently, for which various research papers,
reports, articles, policy have been used as a source. The limitations of the research are that it
is highly theoretical and formalistic.

\heading{Critical Analysis}

\noi
\textbf{\textit{A. Legality of Iranian gunboats perusal of US naval ships}}

\noi
As the facts of the matter objectively represent that the Iranian gunboats repeatedly crossed
the US warships, some from the distance as close as 10 yards, the author believes that the
preliminary question which needs to be taken up for consideration is whether this intentional
and repeated crossover is violative of the provisions of the international law of sea. The
United Nations Convention on the Law of Sea (UNCLOS), talks about the concept of ‘hot
pursuit’\footnote{Stanly Johny, \textit{Analysis: What is next in Iran- US Conflict?,} The Hindu, (January 8, 2020).
\url{https://www.thehindu.com/news/international/analysis-what-is-next-in-iran-us-conflict/article30510865.ece}}  which provides that a coastal state can pursues a foreign ship if they have a
genuine reason to believe that the ship is violating or has violated the laws and regulations
of that particular state within the limits of their territorial waters. Moreover, it also provides
this right to the states if any foreign ship is within the range of a contiguous zone and there
has been an infringement of certain established exceptional rights for that zone like
immigration, fiscal, customs, sanitary laws, and piracy. In the present case there is no
denying to the fact that the US ships were in the contiguous zone but only performing their
military operations and there is no instance of them violating any law pertaining to the
contiguous zone. Hence the legitimacy of the actions of the IRGC’s gunboats pursuing them
in this context seems questionable.

\vspace{-.1cm}

\noi
\textbf{\textit{Use of armed forces in the pretext of self-defense: International Perspective}}

\vspace{-.1cm}

\noi
Any instance of use of force between two or more states in the pretext of defending self
needs to fulfill the standards of the law on the use of force i.e. jus ad bellum to be lawful.
The United Nations (UN) Charter under its chapter VII\footnote{UN General Assembly, \textit{Convention on the Law of the Sea,} 10 December 1982, 1833 U.N.T.S. 397’, art.111. Available at: \url{https://www.refworld.org/docid/3dd8fd1b4.html [accessed October 14, 2020].}}  enshrines Article 51 that authorizes
a state to initiate an armed action against the belligerent state as a matter of its inherent right
to defend itself individually as well as collectively, until the Security Council (UNSC) takes
requisite measures to maintain international peace and security. Hence, self-defense is a
‘circumstance precluding wrongfulness’ of a state’s use of force that would otherwise be
violative of the prohibition stated in Article 2 (4) of the Charter\footnote{United Nations, \textit{Charter of the United Nations,} 24 October 1945, 1 UNTS XVI, art.51. Available at: \url{https://www.refworld.org/docid/3ae6b3930.html [accessed October 14, 2020}} and its customary
international law counterpart. Now in this case, there has been no resolution passed by the UNSC and hence US government must base its future actions of engaging IRCG’s gunboats
on Article 51.

\vspace{-.1cm}

\noi
\textbf{\textit{B. Interpretation of the Event: USA’s viewpoint}}

\vspace{-.1cm}

\noi
Pertaining to this particular incident, the US government has put forward two contentions in
expounding the application of Article 51 and other customary laws. The authorities in their
first contention rejected the current prevailing and accepted view propounded by the
International Court of Justice (ICJ) in its Nicaragua judgment where it opined that an armed
attack must only be resorted exclusively in situations of the gravest form of use of force.\footnote{ International Court of Justice (ICJ) in its Nicaragua judgment.} Rather, the US backs for the automatic implication of the right of self-defense against any illegal use of force, for which it relied on the ICJ’s judgment in Oil Platforms case11 wherein
the court upheld that the mining of a military vessel sufficiently enough to bring into play
the inherent right of self-defense. This indicates that the US views it lawfully justifiable to
use armed force if the actions of Iranian gunboats subjectively satisfy their parameters of the
use of force. However, the author disagrees with this contention and believes that the right
to use self-defense must be used in exceptional circumstances where all other possible
measures are exhausted. Secondly, the US has Islamic Republic of Iran v. USA, ICGJ 74
(ICJ 2003), contended the anticipatory application of Article 51 in the response to an
imminent attack, an accepted viewpoint under international setting also leaves a grey area as
to what exactly counts as ‘imminent’. The latest articulated interpretation of ‘imminent’ has
been provided in the White House’s 2016 and Policy Frameworks report.\footnote{\textit{Report on the Legal and Policy Frameworks Guiding the United States’ Use Of Military Force and Related National Security Operations,}
 \url{https://obamawhitehouse.archives.gov/sites/whitehouse.gov/files/documents/Legal_Policy_Report.pdf}}

\noi
There are a variety of factors that the US military authorities take into consideration on a
national level in determining the imminency of an armed attack including nature, immediacy,
probability of an attack, injury damage/loss likely to be caused, whether the anticipated
attack is part of a concerted pattern of an ongoing armed attack, alternative measures of selfdefense etc. However, the author believes that it is important to look into it from the
perspective of a military commander who is actually executing it at the unit level when time
is of the essence, and hence a more manageable test should be whether the defensive
measure is opted during the last possible window of opportunity in the fact of an attack that was almost certainly going to occur. Furthermore, finding out that whether an armed attack
is ongoing or imminent is just not sufficient enough to deem the justifiable use of force in
self-defense.\footnote{\textit{Chairman o f the Joint Chiefs of Staff Instr. 3121.01b, ‘Standing Rules of Engagement (Sroe)/Standing Rules For The Use Of Force (Sruf) For U.S. Forces,} \url{https://www.loc.gov/rr/frd/Military_Law/pdf/OLH_2015_Ch5.pdf}}

\heading{\textit{Operationalizing Law of Self- Defense: US’s Standing Rules of Engagement}}

\noi
The Standing Rules of Engagement (SROE) promulgated in 2005, are the most recent
version of directives providing an operational framework to use the right of self-defense by
the US armed forces. They are issued by a competent military authority that outlines the
circumstances as well as the limitations which the US forces will initiate or proceed with
combat engagement with other encountered forces. These standing rules remain in force
and must be abided by the armed personnel during both territorial and extra- territorial
military operations and eventualities unless directed otherwise. The terms used under SROE
for describing an ongoing and imminent attack are ‘hostile act\footnote{\textit{Id.,} Part E, rule 2, cl. (c).}
’ and ‘hostile intent’.\footnote{\textit{Id.,} rule 2, cl. (d).} “A hostile act” refers to ‘an attack (direct/ indirect) or use of force’ against the state of the USA, its armed forces, subjects or its property. On other hand ‘hostile intent’ signifies the
SROE’s operational version of anticipatory self- defense. It can be defined as a situation
carrying an impression of ‘threat of imminent use of force’ against the US, its citizens,
property or forces. To define what actually constitutes imminent for a state is a difficult
question and hence it is always contextual which means based on a subjective assessment of
all circumstantial facts known at that particular point of time, even for the US forces.
Importantly, the SROE in an attempt to bring some clarity in this regard added that
‘imminent’ does not necessarily mean immediate or on- the spot’. This was done to counter
the so called ‘Bush doctrine’ set forth under the National Security Strategy, 2002,\footnote{\textit{The National Security Strategy of the United States of America,} (September 2002). \url{https://2009-
2017state.gov/documents/organization/63562.pdf.}} which
is a general description of an aspect of the US foreign policy post 9/11 attack dealing
exclusively with the strategic horizons of preemptive attack as a means of self-defense.
By addition, it now simply acknowledges that the commanding officer need not to wait for
an actual attack to happen, though in no way it suggests the authorization of so called preventive self-defense.

\noi
The principle of proportionality and necessity are also incorporated in SROE. The military
personnel are trained ‘to not take the first hit’ before defending. The author believes that if
we read this statement in a broader sense that, the SROE caution the use of self-defense only
while the belligerent state exhibits hostile intent or continue to commit hostile acts.
Moreover the ‘Law of War Manual’,\footnote{\url{https://dod.defense.gov/Portals/1/Documents/pubs/DoD}\%20Law\%20of\%20War\%20Manual\%20-}
alternative than using force if there is a demonstration of hostile intent. This is a pure
reflection of the classic understanding of principle of necessity. With respect to
proportionality, SROE contains precise description of what proportionate response shall
be;\footnote{Supra note 13, Enclosure A, Part 4, cl. (3).}  which is use of force that is sufficient enough to conclusively counter to hostile acts or
demonstrations of hostile intent. It acknowledges within its justiciable domain the excessive
use of means and intensity but not the nature, duration and scope of the force from what is
actually needed. Even the Law Manual takes a very similar approach where it says that a
response to an armed attack is proportionate to an extent of repealing the belligerent forces
and restoring the peace and security of the disturbed area. Looking at both these principles
the U.S’s national policies with regards to armed attacks fulfills the obligations of
International humanitarian law as it disallows the use of force when other alternatives are
available, attack is non- imminent and non- continuous.

\noi
The current mission specific rules in SROE must be formulated in a more tailored fashion to
facilitate the accomplishment of a particular operation including during actual hostilities,13\footnote{Department of Peacekeeping Operations Military Division, \textit{Guidelines for the Development of Rules of
Engagement (ROE) for United Nations Peacekeeping Operations,} MD/FGS/0220.0001 (May 2002).
 \url{https://www.aaptc.asia/images/resourcess/9_Rules_of_Engagements/120_Roe_Guidelines.pdf}}
which currently are largely classified because they might reveal US’s forced tactics,
techniques etc. to the belligerent state, giving an example declaring a particular organization
as ‘hostile’ will permit the forces to engage with its members on the basis of their status of
being the member of that designated group. But, since this is only allowed during an armed
conflict, an action to this effect by US forces on IRCG Navy gunboat in high seas or any other
Iranian forces would be unlawful unless there starts an armed conflict.

\heading{SROE’s \textit{Categories of Self- Defense}}

\noi
There are three categories of self-defense recognized by SROE which are- individual, unit
and national.\footnote{ Supra note 13, Part E, rule 2.} The very difference in this hierarchy lies in the level of authority and
responsibilities in which the right to self- defense is exercised. As already mentioned above
the defending of the US, its forces and under certain circumstances its persons and property
constitutes the requirements of national self-defense. It has to be noted here that the
commanders of the unit are also authorized to exercise national self-defense. The category
of Unit self-defense by contrast provides an inherent right to self-defense and also put
certain obligations in return upon the commanding officers of the unit based on any warship
or aircraft or any other place of their operation.

\noi
Though, the SROE specifically creates a legal distinction between the category of national
and unit self-defense, there is no real or qualitative difference between a unit responding or
the entire military structure because the legal basis is the same for both under the
international law i.e. self-defense in face of an armed attack. Apart from responding in
group, the SROE also authorizes the members to exercise self-defense in their individual
capacity if required. When acting as part of the unit, such individual self-defense is treated
as sub- set of unit defense and hence stands valid on the legal basis of international
principles. In the present case the author is going to specifically deal with the unit selfdefense because only a unit of the US’s air force was operating in the Persian high seas at
the time they were pursued by the IRCG’s gunboats.

%~ \vspace{.5cm}

\heading{\textit{Unit Self Defense}}

\vspace{.1cm}

\noi
At the unit level, the SROE’s de- escalation principle is designed in a way to satisfy the
principles of necessity\footnote{Supra note 13, Enclosure A, Part 4, cl. (2)} and proportionality.\footnote{Supra note 20.}  It states that the belligerent actor must be
warned and provided with an opportunity to withdraw and cease its threatening actions.
Here the author believes that it is a classic situation of ‘easier said than done’, which seems
to be a challenging decision for a commander to determine whether there is a demonstration
of hostile intent and whether de- escalation should be attempted or not. For example,
during the Iran- Iraq War, 1987, Iraq Air Force attacked the US’s Stark ship in high seas causing the death of 37 US Navy personnel. The Iraqi jets were identified from a distance
and warnings were given to it, however it still launched an armed attack on US ships.
Afterwards, an investigation led by the House of Representatives Armed Services
Committee\footnote{\textit{Report of The Staff Investigation into the Iraqi Attack on the USS Stark of the Committee on Armed Services House of Representatives,} One- Hundredth Congress, 1$^{\rm st}$ Session (June 1987).
\url{https://babel.hathitrust.org/cgi/pt?id=uc1.31210014708372&view=1up&seq=8}}  found out that the pre required condition of de- escalation principles prior to
self- defense in factored into the commander’s inaction to defend the ship.

%~ \vspace{-.2cm}

\noi
While on the other hand, in 1988, a US cruiser was shot down by an Iranian passenger
aircraft Air Flight 655, killing off around 297 on board, that took off from Bandar
Abbas International Airport which was mistakenly identified as combat aircraft operating
in an attacking zone.\footnote{Brad Lendon, In 1988, A US \textit{Navy warship shot down an Iranian passenger plane in the heat of battle,’} CNN World, (January 20, 2020)  \url{https://edition.cnn.com/2020/01/10/middleeast/iran-air-flight-655-us-militaryintl- hnk/index.html}} In present year, Iran also mistakenly took down a Ukrainian 
International Airlines killing approximately 176 people\footnote{Matthew S. Schwartz. \textit{Iranian Report Details Chain of Mistakes In Shooting Down Ukrainian Passenger
Plane,} NPR, (July 20, 2020).  \url{https://www.npr.org/2020/07/12/890194877/iranian-report-details-chain-ofmistakes-in- shooting-down-ukrainian-passenger-pl}} during a tussle between the US
and Iran after the targeted killing of General Soleimani earlier this year. The aim of the
author in giving these two contrasting examples is just to underscore the complex nature of
these decisions, especially in the presence of little moments of deliberation.

%~ \vspace{-.1cm}

\heading{\textit{Margin of Appreciation}}

\noi
Doctrine of Margin of appreciation is an invention of the European Court of Human
rights\footnote{Council of Europe, \textit{The Margin of Appreciation,}
 \url{https://www.coe.int/t/dghl/cooperation/lisbonnetwork/themis/echr/paper2_en.asp}} wherein, they defer to the will of member states, in specific circumstances. For ex- in
case of disturbance to public tranquility or a threat to national security, the courts may
justify State’s restriction on freedom of speech or assembly etc. If such restrictions are in
accordance with the law and necessary considering the facts and circumstances of that
particular situation. The author aims to analyze whether this doctrine can be applicable in
such scenario, particularly in this case. As stated already that under its requirement of
proportionality principle, SROE allows a fair margin of appreciation in the use of means and
intensity of force but not with respect to its nature, duration and scope. There is no debate to the fact how peculiar is to measure the precise degree of force necessary to do so and
therefore the author believes that a fair margin appreciation must be given to the states in
this regard. With respect to this particular case, considering the facts of the situation that the
distance between the IRCG’s navy boats repeatedly traversed the US warships from as close
to a distance of 10 yards, if the US warships have had attacked IRCG’s gunboats, then, the
author feels it could have been defended under the doctrine of margin of appreciation if it
was under a recognized doctrine under the principles of international law. On the other side,
under no circumstances the states indulging in armed attack should use excessive means
and intensity of force from what is actually required. Hence a balanced approach is what a
State must strive for in such situations.

%~ \vspace{.8cm}

\heading{\textit{The Actual Response}}

\noi
The US warships did not use any kind of force on the IRCG’s gunboats even after their
repeated crossing, which shows that their actions were in complete consonance with the
provisions of Standing Rules of Engagement and as well as international legal norms. On
the other side, the gunboats were neither attacking the US warship nor the helicopters which
shows that they were not demonstrating any kind of ‘hostile intent’. However the author
feels that their actions are questionable under ‘Hot pursuit’\footnote{supra note 7} mentioned under UNCLOS. To
appreciate the calculated and sound decision taken by the US unit commander and to
characterize the demonstration of hostile intent by Iranian gunboats, it is to be considered
that the navy boats were armed and the relations between both the countries are highly
strained and tensed. 

\noi
Turning to what the US’s president directed to US warships to shoot and destroy the IRCG’s
navy boats,\footnote{\textit{Trump says US will destroy Iranian gunboats harassing US ships,} Alja Zeera, (April 22, 2020).\\
‘\url{https://www.aljazeera.com/news/2020/4/22/trump-says-us-will-destroy-iranian-gunboats-harassing-us-ships}} the author believes that if this happens, the justifiability will totally depend
upon the circumstances of that time. It must be concluded on the precise facts that firstly a hostile
act has occurred, or demonstration of hostile intent has been done. Secondly, there is no
other alternative left than to employ force against the gun boats to defeat or disable the
imminent attack and lastly the quantum of force used is within the limits of what was
actually needed. Short of any reasons mentioned in SROE for a unit self- defense, the only reason left for the US to engage them would be national self-defense which can only be
brought into play when there is an armed attack by from Iran’s side. Even in that, it has to
be determined that though the Iranian gunboats were not participating momentarily but
there is a possibility that they would be used in future ongoing armed attack against the US
and that it is not the situation in the Persian Gulf today.

\noi
Lastly, talking about the claims of the U.S. Navy, and the approach that the Iranian
gunboats were dangerous and harassing\footnote{Tucker Higgins \& Amanda Macias, \textit{‘Trump says US will ‘destroy’ Iranian gunboats that harass American
ships,} CNBC, (April 22, 2020) \url{https://www.cnbc.com/2020/04/22/trump-says-us-will-destroy-iraniangunboats-that- harass-american-ships.html}} could become the sole reason for using armed
force. Here the author would like to remind that the provisions of SROE are contextual in
application, but the author believes that in most situations practically such activities would
not rise to the level of an imminent armed attack and would not qualify as demonstration of
hostile intent without an indication that gunboats were about to actually use of their
weapon.

\heading{Conclusion and Suggestions}

\noi
From the above discussion it can be concluded that sabre - rattling is acceptable to an extent
where it is not leading to an unlawful threat of use of force because of their unlawful nature
under the provisions of U.N Charter. It is clear from looking at the domestic legal order of
the US for dealing with the dimensions of armed attack that they are clearly defined in the
Standing rules of engagement with other instruments like the War Manual which is in
consonance with the norms of international law. The President’s comment on the whole
issue can be considered a warning to the Iranian government that the US navy units will
avail their right to self-defense which can be considered lawful. The author is of the opinion
that such comments are not appreciable as they might constitute a threat of armed attack and
hence unlawful. Moreover, harassment which does not give an impression of imminent risk
to life or property does not open the gateways to right of self-defense. The actions of IRCG’s
gunboats of repeatedly traversing the US warship in the high seas can be questioned under
the provision of United Nations Convention on Law of Seas. Lastly, looking at the history
of relations between the US and Iran it can be said that lack of sound and reasoned decisions
in situations of harassment can lead to an armed conflict and hence it is important that these 
countries should engage in peace talks or mediation can be initiated by other international
organizations as it is historically evident that small events of prolonged tussle can turn into
full-fledged situation of armed conflict, the recent example of which is Armenia- Azerbaijan
conflict. The author is of the opinion that the present norms of international laws are
outdated and seem to fail in achieving their object as the provisions are very vague and
overbroad which gives opportunity to the powerful states to abuse these loopholes. For
example, the definition of self- defense which is too overbroad that even an unlawful use of
force can be justified under it in the pretext of anticipatory self-defense. Even the principle
of necessity and proportionality are too outdated and undefined thus prone to be misused.
There is an urgent need to change the present structure and form more precise definitions.
Secondly, every member nation must formulate a domestic law which enshrines a more
accurate scope and definition of international principles for e.g. - like the US have their
own. Thirdly, the role of the international organizations like the U.N, which has now turned
completely redundant, must be brought to an urgent institutional change to ensure the
effective implementation of the Charter. Fourthly, some real powers to impose strict penal
provisions against individuals as well as the States violating the provisions of international
law must be there. Lastly, the most important of all is to mend the relations between the
nations who are at tussle with each other, as friendly and co-operative relation between the
States is the key to a peaceful world.

\end{multicols}
