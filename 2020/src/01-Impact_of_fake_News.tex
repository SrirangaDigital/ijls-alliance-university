\setcounter{figure}{0}
\setcounter{table}{0}

\articletitle{Impact of Fake News: An Indian Electoral Perspective}\label{2020-art1}
\articleauthor{Abhishek Dua\footnote{Research Scholar, Alliance School of Law, Alliance University, Bangalore}}
\lhead[\textit{\textsf{Abhishek Dua}}]{}
\rhead[]{\textit{\textsf{Impact of Fake News: an Indian...}}}

\begin{multicols}{2}

\heading{Introduction}

\noi
Fake news, misinformation, disinformation, is not a new phenomenon. It has co-existed with
human civilization, to which history is evidence. However, with the advent and proliferation
of the internet and social media coming of age, the extent of fake news in public sphere has
magnified manifold. Considering India being a mobile first market, where majority of people
identify smart phones as their main device for accessing online news and a considerable
number using it only for accessing online news, more specifically through various kinds of
social media,\footnote{The Reuters Institute, India Digital News Report, 2019.} the crisis of fake news in India becomes more rampant.

\noi
This paradigm shift in the way people in India have begun to access news, placing more
credibility to news on social media platforms than on the traditional sources and the fact that
public opinion is of utmost importance, to determine who rules the country, has inspired
political parties, politicians and their supporters to resort to and optimally use social media
platforms for election campaigning and communication. The 2019 General Elections are a
testimony to the fact, where social media platforms were favourably used not only to
disseminate political agendas, ideologies, manifestos, publicity campaigns, so as to garner a
favourable public opinion, but also undermine the position of the opposition.

\noi
If British Broadcasting Corporation (BBC) reports hold any veracity, WhatsApp in India has
become a vehicle for misinformation and propaganda. For the 2019 general elections, the
Bharatiya Janata Party and the opposition, the Congress party, both took to spread false news
or misinformation in an attempt to mislead the electorate and gain a favourable
advantage.\footnote{Kevin Ponniah, \textit{WhatsApp: The ‘Black Hole’ of Fake News in India’s Election}, BBC News (Apr. 6,
2019), \url{https://perma.cc/5YJS-ZH2A}}
For instance, there was a message in circulation on WhatsApp, about the suicide
bombing against the Indian security forces in Kashmir in 2019, where it was claimed that the
leader of the Congress party had promised a huge amount of money to the
perpetrator/accused and his family and in addition promised to release him and other
terrorists languishing in prison. The people of the state voted for the Congress in the 
approaching elections.\footnote{SnigdhaPoonam \& Samarth Bansal, \textit{Misinformation Is Endangering India’s Election}, The Atlantic (Apr. 1,
2019), \url{https://perma.cc/Y39M-KGWU}} Another message doing rounds was how the Bharatiya Janata Party
was “indulging in war mongering for electoral gains”.\footnote{Anjana Pasricha, \textit{Fake News Inundates India Social Media Ahead of Election}, Voice of America (Apr. 3,2019), \url{https://perma.cc/798C-YMWE}} Yet another interesting piece of
misinformation was regarding a purported circular from the Election Commission of
India,(ECI) which stated that Non-Resident Indian (NRI)s who held an Indian Passport could
vote online.\footnote{Bharat Kancharla, \textit{During the 2019 Lok Sabha elections, only about 150 cases of Fake News reported to Social
Media Platforms by ECI}, FACTLY, (August, 2019) \url{https://factly.in/during-the-2019-lok-sabha-elections-onlyabout-150-cases-of-fake-news-reported-to-social-media-platforms-by-eci/}} Innumerable such instances of fake news on social media, inciting national,
religious, or any other sentiment, during elections, came to light, with the intent to gain
traction and to undermine the image of the opposition, in order to gain a favourable
advantage in elections.

\heading{Impact, Perils \& Definition of Fake News}

\noi
The big question then is, how does fake news affect or make an impact on elections? The
answer to the corollary lies in the basic concepts of motivated reasoning and confirmation
bias, apart from the obvious facts, that social media has easy access and outreach, as it is
inclusive, which abridges the gap between the political parties and the electorate and
moreover bypasses the regulatory regime and critics, which otherwise traditional media is
subjected too.\footnote{Anuradha Rao, How did Social Media Impact India’s 2019 General Election?, ENGAGE, (December, 2019)
\url{https://www.epw.in/engage/article/how-did-social-media-impact-india-2019-general-election}}

\noi
The concept of motivated reasoning describes our affinity to believe in what we want to
believe in more willingly, than what we don’t want to believe in, while confirmation bias is
an affinity to find, interpret and even remember information, that underpins our beliefs and
then accept such information, in the exclusion of other information.\footnote{Jonathan Maloney, \textit{Confirmation Bias \& Motivated Reasoning}, INTELLIGENT SPECULATION, (April,
2019) \url{https://www.intelligentspeculation.com/blog/confirmation-bias-amp-motivated-reasoning}} Therefore, any news in
public domain, to which people have an affinity towards or which people want to believe in,
shall be or is accepted easily, even if it is false/ fake. That is just how the menace of fake
news spreads its fangs and influences the voting choices of people.

\noi
Even with the adulation and popularity that fake news has harvested, no Indian statute or
regulatory guidelines have yet defined fake news or laid down the standards for defining it.
Therefore, before any stance or legal action can be undertaken against occurrences of fake news, 
it would be expedient to first amend the existing legal and regulatory framework, by
inserting an appropriate definition of the term.

\noi
Interestingly, it has been experienced in other countries that a plain definition of the term fake
news, as consisting of falsehood, may lead to an ambiguous, overreaching and an
inefficacious definition, if applied to India, as in the case of Malaysia’s Anti-Fake News Act,
2018, which defines fake news as \textit{“news, information, data, and reports, which is or are}
wholly or partly false or in any other form capable of suggesting words or idea.\footnote{Sohini Chaterjee, 
Akshat Agarwal, \textit{Regulating fake news in India is tough}, MINT, (July, 2018) \url{https://www.livemint.com/Opinion/ECLC6qtR9ij8nDz33xaCEN/Regulating-fake-news-in-India-is-tough.html}}”

\noi
The preceding definition would be a failure and would fall on its face, if applied in a
democracy like India, where citizens are guaranteed freedom of speech and expression\footnote{INDIA CONST. art. 19 c. 1(a)} as a
fundamental right, under the Indian Constitution, subject to reasonable restrictions,\footnote{INDIA CONST. art. 19 c. 2} viz, i) in
the interest of the sovereignty and integrity of India, ii) the security of the State, iii) friendly
relations with foreign States, iv) public order, v) decency or morality, vi) or in relation to
contempt of court, and v) defamation or incitement to an offence.

\noi
Speech in India then, may not only be restricted owing to its consequences, but also because
of its substance, as the state patently, has an overarching interest in restricting speech, the
direct consequence of which is violence. Interestingly, in case of obscenity or defamation,
though violence is not a consequence, speech is restricted because of the value system of the
state, as it is believed that it would erode public morality. Therefore, any restriction on speech
must have an immediate correlation with a specific head set out in Article 19(2). The
government otherwise cannot restrict speech merely in the ‘public interest’, or because it is
‘false’, neither of which are heads under Article 19(2). Therefore, appropriately defining the
term fake news becomes of paramount importance, in order to evaluate its
repercussions/consequences and to take cognizance of it.

\noi
In the case of France, the law against information manipulation or fake news, puts down the
following criteria’s to evaluate if a piece of information is fake news, viz, it must be evident,
it should have been deliberately disseminated on a substantial scale, the direct consequence of
which should be, disturbance of the peace and tranquillity or a compromise of the outcome of
an election.\footnote{Government of France, \textit{Against Information Manipulation}, (December 7, 2018)
\url{https://www.gouvernement.fr/en/against-information-manipulation.}}

\noi
Although the definition appears to be robust and encompassing in its criteria, if looked at
from the Indian perspective, only the last two criteria would fit in the Indian scheme of
things. The first criteria would still be ambiguous, as it does not differentiate between
undisruptive misinformation and confirmable misinformation, which may cause social harm
or malign the reputation of an individual.

\noi
The line is slippery and not easy to draw, as the term fake news in itself is unstructured and
shapeless and includes, though not limited to, unconfirmed content, manipulated videos,
hoaxes, and even morphed pictures in the veil of memes.

\noi
However, after travelling the contours of fake news, it is ascertained that the following criteria
should fall within the ambit of the definition of fake news, so as to serve its purpose, in the
Indian scenario, viz

\vspace{-.3cm}

\begin{enumerate}
\itemsep=0pt
\item Any confirmable misinformation or disinformation, 

\item Intentionally disseminated to the public at large, purporting to be true 

\item With the potency to threaten life, public peace and tranquillity or national security or
an outcome of an election.
\end{enumerate}

\vspace{-.3cm}

\heading{Ramifications of Fake News}

\vspace{-.15cm}

\noi
Since ours is a parliamentary democracy, the trite saying ‘democracy is for the people, of the
people and by the people’ aptly applies. The will of the people is paramount which is
expressed through their vote and becomes the basis of the authority of the 
government.\footnote{People's Union of Civil Liberties vs. Union of India \& Anr., (2003) AIR SC 2363} It
then becomes of paramount importance that the people/voter should have access to true,
correct, objective and unbiased information to be able to make a reasoned and rational choice,
which is ultimately expressed through the ballot by means of a vote.

\noi
In context of Indian Democracy, fake news decimates all means to support the voters to make
an informed and rational choice, by completely misinforming and misleading the voter with
false, incorrect, biased and prejudiced information and in consequence retards and
extinguishes the voter’s ability to make an informed and rational choice, thereby infringing
the voter’s freedom of speech and expression and changing the complete complexion of the
political landscape.

\noi
The hazards of fake news do not end there. Fake news further gives an undue advantage to
the candidate disseminating fake news in political communication and propaganda, over the 
candidate not relying on it, hence vitiating an equal contesting ground, which is indeed a
violation of the fundamental right to equality14.\footnote{INDIA CONST. art. 14} 

\noi
To add to its serious effects, dissemination of fake news comes for a price and considering its
nature, the money spent on it, is not revealed as an election expense, which is mandated by
the Conduct of Election Rules, hence violating it.

\vspace{-.15cm}

\heading{Existing Legal Framework}

\vspace{-.15cm}

\noi
There are no specific provisions under Indian law that specifically deal with fake news.
However, there are provisions available, dealing with it, in a piecemeal manner, falling under
the Indian Penal Code, 1860, Criminal procedure Code, 1974 and The Information
Technology Act, 2000, The Indian Telegraph Act, 1885 which is applicable both offline as
well as online. However specific provisions dealing with it, in relation to elections fall under
‘The Representation of the People Act, 1951’ and guidelines issued by The Election
Commission of India, from time to time, which are as follows:

\vspace{-.15cm}
\noi
{\large\bfseries 1. Silent Period:}

\vspace{-.15cm}

\noi
The Representation of the People Act, 1951, which governs the conduct of elections in India,
prohibits advertising and campaigning on TV and other electronic media by candidates and
political parties during the “silent period,” which is during the period of 48 hours ending with
the hour fixed for conclusion of poll in a constituency.\footnote{Representation of the People Act, No. 43 of 1951, § 126, \url{https://perma.cc/L7FX-SFMK}} However, individuals are not
prohibited from expressing their private opinions during the silent period as per The Election
Commission of India (ECI).\footnote{Kanchan Chaudhari, \textit{We Can’t Stop Individuals from Using Social Media 48 Hours before Polls, ECI Tells
Bombay HC,} HINDUSTAN TIMES (Jan. 12, 2019), \url{https://perma.cc/GX7H-EL62}}

\noi
Considering the increasing cases of fake news during elections, a committee was established
by the ECI to review and suggest changes to the provision of silent-period. The committee
submitted its report in January 2019,\footnote{Press Release, Election Commission, Report of the Committee on Section 126 of the Representation of the
People Act, 1951 Submitted to the Commission (Jan. 10, 2019), \url{https://perma.cc/9Q5E-ZMBG}} proposing an extension of the scope of the forty-eighthour 
ban to cover print media and “intermediaries” as defined in section 2(w) of the Information Technology Act.

\noi
{\large\bfseries 2. Model Code of Conduct (MCC):}
The Model Code of Conduct (MCC)\footnote{Election Commission of India, Model Code for the Guidance of Political Parties and
Candidates, \url{https://perma.cc/34BQ-WMTW.}} is a
compilation of guidelines, issued by the ECI prior to the conduct of elections, to political
parties and contesting candidates, to regulate their conduct, in connection with elections and
ensure the elections are conducted in a free and fair manner.”\footnote{Roshni Sinha, \textit{Model Code of Conduct and the 2019 General Elections}, PRS (Mar. 11,
2019), \url{https://perma.cc/63XM-TWSS}.} The MCC comes into force
from the date the “election schedule is announced and remains in force till the election results
are announced.”\footnote{Id.}

\noi
However, even though the MCC are not enforceable by law, some of its provisions may be
enforced through corresponding provisions in other statutes such as the Indian Penal Code,
1860, Code of Criminal Procedure, 1973, and Representation of the People Act, 1951. To
back the non-enforceability nature of MCC, the ECI proposes that the non-enforceable nature
of MCC should be preserved; failing which elections may never be completed with judicial
proceedings typically take a long period to conclude. Concurrently, the Standing Committee
on Personnel, Public Grievances, Law and Justice, recommends making the MCC legally
binding by making it a part of the Representation of Peoples Act, 1951, as already most
provisions of the MCC are enforceable through corresponding provisions in other statutes,
mentioned above.\footnote{Id.}

\noi
Prior to the 2019 general elections, the ECI published and issued the Manual on the Model
Code of Conduct\footnote{Election Commission of India, Manual on Model Code of Conduct (Mar. 2019), \url{https://perma.cc/55KXCW49}.} as guidance for political parties and candidates, including information on
the Model Code, enabling law, instructions, and court decisions.\footnote{Manual on Model Code of Conduct: About This File, Election Commission of India, \url{https://perma.cc/8E2R3NGM}} The Manual makes explicit mention of a Compendium of Instructions on Election Expenditure Monitoring (February,
2019)\footnote{Election Commission of India, Compendium of Instructions on Election Expenditure Monitoring (Feb.
2019), \url{https://perma.cc/Y54E-WNBA}.} and Instructions of the Commission with respect to use of Social Media in Election
Campaigning.\footnote{Letter from ECI to Chief Electoral Officers et al., Instructions of the Commission with respect to Use of
Social Media in Election Campaigning, Letter No. 491/SM/2013/Communication (Oct. 25,} Instructions on social media contain guidelines on “information to be given by candidates about their social media accounts,” precertification of political advertisements,
and “expenditure on campaigning through the internet including social media websites.”\footnote{Amogh Dhar Sharma, \textit{How Far Can Political Parties in India Be Made Accountable for Their Digital Propaganda?}, SCROLL.IN (May 10, 2019), \textit{https://perma.cc/64VU-LDA5}}

\vspace{-.15cm}
\noi
{\large\bfseries A. Social Media Account Information:}

\vspace{-.15cm}
\noi
Rule 4A of the Conduct of Elections Rules, 1961, requires candidates or proposers of
candidates to submit an affidavit (Form 26) at the time of filing their nomination
papers.\footnote{Conduct of Elections Rules, 1961, Rule 4A, \url{https://perma.cc/DA6U-J5T8}} Paragraph 3 of this Form requires the candidate to provide the ECI with his/her “email ID” and a list of any social media accounts.\footnote{Id., Form 26, para. 3, \url{https://perma.cc/23XQ-QVZ4;} Instructions of the Commission with respect to Use of Social Media in Election Campaigning.}

\vspace{-.15cm}
\noi
{\large\bfseries B. Social Media Content Pre-Certification:}

\vspace{-.15cm}
\noi
The ECI requires pre-approval / pre-certification of political advertisements on social media
and have directed for the establishment of a grievance cell and appointment of grievance
officers.”\footnote{Sahana Udupa, Elonnai Hickok \& Edward Anderson, \textit{Can Extreme Speech Online Be Regulated Without
Curbing Free Speech? This Series Finds Out}, SCROLL.IN (May 9, 2019), \url{https://perma.cc/G6TS-8QLF}.}

\noi
Pursuant to the order of the Supreme Court of India\footnote{ Ministry of Information \& Broadcasting Vs M/s Gemini TV and Others (2004) 5 SCC 714}, requiring political parties, candidates,
persons, to pre-certify the use of political advertisements on electronic media, including
television channels and cable operators, the ECI issued detailed guidelines in this regard,\footnote{Letter from Election Commission of India to Chief Secretaries et al., Supreme Court Order Dated 13th April, 2004 for Pre-certification of Political Advertisement on Electronic Media, –Letter No. 491/MCMC/2018/Communication (Sept. 13, 2018) (re: applicability of 2004 directions throughout territory of India at all times), \url{https://perma.cc/U2MM-U862}.} which are as under:

\noi
Every registered/national/state political party and every contesting candidate intending to
issue advertisements on electronic media and/or television channels and/ or on cable network,
will have to apply to the ECI for pre-certification of all political advertisements before their
publication. Subsequently the ECI issued an order, dated 27.08.2012, subsequent to which
Media Certification and Monitoring Committees at district and State levels were given the
responsibilities of pre-certification of such advertisement along with other functions viz
acting against Paid News etc.

\noi
The ECI issued another set of instructions in this regard according to which, as social media
websites were also electronic media by definition, the Commission’s instructions relating to 
pre-certification of advertisements would also apply mutatis mutandis to websites including
social media websites and shall fall under the purview of pre-certification.”\footnote{Id}

\noi
{\large\bfseries 3. Expenditure of Social Media Content to be Divulged}

\noi
According to The Representation of the People Act, 1951, every candidate is “required to
keep a separate and correct account of all expenditure relating to elections, incurred from the
date of nomination, to the date of declaration of result, both dates inclusive.”\footnote{Representation of the People Act, 1951, S.77(1)} Pursuant to
the Supreme Court of India judgement,\footnote{Common Cause v. Union of India, Writ Petition (Civ.) No. 13 of 2003.} the ECI required candidates and political parties to
submit a statement of account of expenditure on elections to the ECI within a period of 75
days of assembly elections and 90 days of Lok Sabha elections. Interestingly in 2013 the ECI
clarified all doubts by issuing another set of instructions,\footnote{Instructions of the Commission with respect to Use of Social Media in Election campaigning.} which clarified that the
expenditure on social media was a part of the expenditure in connection with elections and
that, candidates and political parties shall have to include all expenditure on campaigning,
including expenditure on advertisements on social media, for maintaining a correct account
of expenditure and for submitting the statement of expenditure.

\noi
Recently, apart from self-regulation and the Voluntary Code of Ethics adopted by social
media platforms to tackle the problem of fake news, a new bill\footnote{The Fake News (Prohibition) Bill, 2019.} has been introduced in the
Lok Sabha and awaits accent.

\noi
On the other hand, tech platforms need to ensure the use of sophisticated algorithms to
provide the public with correct, accurate, and truthful information in public domain and
remove/delete/abandon fake information/propaganda on social media platforms, as soon as
they are detected.

\newpage

\noi
Concurrently, lawmakers need to look at appropriate legislative measures, a new law dealing
with fake news or amendment of existing laws to that effect, so that it is made an independent
criminal offense. More significantly, the need of the hour is to device a strong and a proactive
execution mechanism, where cases of fake news are identified and taken cognizance of at the
earliest, within a given time frame. 

\end{multicols}
\label{end2020-art1}
