\setcounter{figure}{0}
\setcounter{table}{0}
\setcounter{footnote}{0}


\articletitle{India’s Dire Need For a New Comprehensive Pandemic Law}
\articleauthor{Romala Menon\footnote{Ph.D Research Scholar, Alliance School of Law, Alliance University, Bangalore.}}
\lhead[\textit{\textsf{Romala Menon}}]{}
\rhead[]{\textit{\textsf{India’s Dire Need For a New...}}}

\begin{multicols}{2}

\vspace{-.2cm}

\begin{quoting}
\textbf{\textit{“This pandemic has magnified every existing inequality in our society” – Melinda Gates}}
\end{quoting}

\vspace{-.2cm}

\heading{Introduction}

\noi
India, just like the remainder of the globe is affected due to COVID-19. On March 11th 2020,
the eruption was declared by the World Health Organization as a public health exigency and
defined as a pandemic. Legislation plays a critical role in the containment of the disease.
Despite there being various legislations to control public health, a significant question arises
as to whether colonial-era laws such as ‘The Epidemic Disease Act 1897’, ‘The Indian Penal
Code 1860’, and ‘The Disaster Management Act 2005’ are sufficient enough to encounter the
challenges posed by a twenty first century pandemic?”\footnote{Ijphrd August 2020, SCRIBD available online at\\ \url{https://www.scribd.com/document/495356157/Ijphrd-August2020 (last visited Feb 13, 2020).}}

\noi
COVID-19 pandemic caused by the novel coronavirus SARS-CoV-2 as stated,\footnote{Pandemics: Risks, Impacts, and Mitigation, NATIONAL CENTRE FOR BIOTECHNOLOGY INFORMATION,\\
 \url{https://pubmed.ncbi.nlm.nih.gov/30212163/}\#::text=Evidence\%20suggests\%20that\%20the\%20likelihood, will\%20continue\%20and\%20will\%20intensify. (last visited Feb 21, 2020).} started at
Wuhan within the Hubei province of China and has unfolded swiftly around the world. A
year has passed, and the virus still remains wreaking mayhem in several nations including
India. Throughout such pandemic times, legislation plays a key role in the containment of the
disease and legal acts have to be implemented to manage and overcome the pandemic.\footnote{Coronavirus Disease (COVID-19) - events as they happen, WORLD HEALTH ORGANIZATION, available online at\\ \url{https://www.who.int/emergencies/diseases/novel-coronavirus-2019/events-as-they-happen}~(last~visited Feb 15, 2020).}
Pandemic and Epidemic are the two terms people exchange frequently to refer to the stages
of the spreading of any infectious disease. An Epidemic is a disease that affects a large
number of people within a community, population, or region. Whereas a Pandemic is an
epidemic that spreads over multiple countries or continents. Pandemics have occurred
throughout human history. Pandemics seem to be increasing in frequency over the last
century on account of zoonotic transmission of diseases, urban development, alterations in
the way lands are used, rise in international travel, and exploitation of natural resources. The 
consequences of the Pandemic are devastating to humanity. Future Pandemics are inevitable
and unpredictable.

\noi
Humanity has been ravaged by plagues and epidemics throughout its existence, typically
ever-changing the course of history and at alternative times obliterating entire civilizations. It
is relevant to trace the timeline of the worst epidemics and pandemics from prehistoric to the
present time to grasp the background and analysis the context of this epidemic. The past
provides an introduction for any discussion that pertains to Covid 19.\footnote{Damir Huremović, BRIEF HISTORY OF PANDEMICS (PANDEMICS THROUGHOUT HISTORY) PSYCHIATRY OF PANDEMICS: A MENTAL HEALTH RESPONSE TO INFECTION OUTBREAK (2019), \url{https://www.ncbi.nlm.nih.gov/pmc/articles/PMC7123574/}(last visited Feb 18, 2020).}

\noi
The Athenian Plague occurred in 430-26 B.C., during the Peloponnesians War which was
fought between the city – states of Athens and Sparta. The historic account of the Athenian
plague is provided by Thucydides, who survived the plague himself and described it in
his \textit{History of the Peloponnesian War.}\footnote{A. Thucydides, HISTORY OF THE PELOPONNESIAN WAR, Book 2, Chapter VII. Pages. 89–100., trans. Crawley R. available online on Digireads.com Publishing; 2017 Sept. ISBN-10: 1420956418.} The Athenian plague originated in Ethiopia, and it spread throughout Egypt and Greece. The Antonine Plague was yet another outbreak that
occurred between the period of 165-180 A.D., which was a couple of centuries after the
Athenian Plague. The Justinian Plague which was popularly known as a pandemic wherein
“No One Was Left to Die”. It originated during the mid of sixth century A.D. which was also
known as the “real plague” pandemic (i.e., caused by Yersinia Pestis) from Ethiopia, moving
through Egypt, or in the Central Asian steppes, where it then travelled along the caravan
trading routes. 

\noi
The pestilence quickly spread throughout the Roman world and beyond from either of these
two locations. The Justinian plague generally followed trading routes, like most of the other
pandemics, providing an “exchange of infections as well as of The Black Death Plague was a
global outbreak of bubonic plague that originated in China in 1334, arrived in Europe in
1347, following the Silk Road. Within 50 years of its reign, by 1400,\footnote{The Editors of Encyclopaedia Britannica. Black death, Encyclopaedia Britannica; 2018 Sept available online in  \url{https://www.britannica.com/event/Black-Death.}}  it almost swept the
global population and reduced it from 450 million to below 350 million, The Great Plague
of 1665 was the last and one of the worst pandemics of the centuries-long
outbreaks, killing 100,000 Londoners in just seven months. All public
entertainment was banned and the victims were forcibly shut into their homes to prevent the spread of the disease. The lethal outbreak took place in Jessore, India. It
spread to Myanmar, Sri Lanka and around 1820 in Thailand, Philippines, Iraq and Europe.
This then spread to other parts of the world, from 1829-1833, Moscow, Finland, Poland,
Canada, US. 

\noi
There were about six waves that resulted in deaths as well. The epidemic that started in India
spread to Europe moving rapidly along with the expansion of trade in the nineteenth century.
Influenza or ‘the flu’ has been categorised under three types, namely, Type A, Type B and
Type C.\footnote{FAQs on zoonotic influence: WHAT ARE THE KNOWN TYPES OF INFLUENZA VIRUS? World Health Organization, Regional Office for South East Asia, 2017}  Type A is responsible for regular outbreaks in humans. A flu is further divided into
three, pandemic, seasonal and zoonotic. Historical data illustrates the danger of transmission
of influenza between animals and humans that can potentially contribute to the emergence of
a pandemic.”\footnote{FAQs on zoonotic influence: WHAT ARE THE KNOWN TYPES OF INFLUENZA VIRUS? World Health Organization, Regional Office for South East Asia, 2017} The Spanish flu pandemic appeared during the first decades of the twentieth
century and got established as the first true global pandemic. It also became the last true
global pandemic with devastating consequences for societies across the globe.\footnote{CDC: Remembering the 1918 influenza pandemic, available online in  \url{https://www.cdc.gov/features/1918-flupandemic/index.html.}} It was caused
by the H1N1 strain of the influenza virus,\footnote{A. Antonovics J, Hood ME, Baker CH Nature, MOLECULAR VIROLOGY: WAS THE 1918 FLU AVIAN IN ORIGIN? 2006 Apr 27; 440} Within months, the deadly H1N1 strain of influenza virus had spread to every corner of the world. This pandemic was also the first one where the long-lingering effects could be observed and quantified. Smallpox was a highly
contagious disease for which Edward Jenner developed the world’s first vaccine in 1798.
Caused by the Variola virus, it was a highly contagious disease with prominent skin eruptions
(pustules) and mortality of about 30\%.

\noi
One of the deadliest diseases because of its mortality, Ebola virus, is transmitted from
animals to humans. The first epidemic occurred in Republic of Congo in 1976. Ebola virus,
endemic to Central and West Africa, with fruit bats serving as a likely reservoir, appeared in
an outbreak in a remote village in Guinea in December 2013. The total number of cases
figured to 28,652 with affected countries include Italy, Mali, Nigeria, Senegal, Spain, United
Kingdom and The United States. HIV/AIDS is a slowly progressing global pandemic
cascading through decades of time, different continents, and different populations, bringing 
new challenges with every new iteration and for every new group it affected. It started in the
early 1980s in the USA, causing significant public concern as HIV at the time inevitably
progressed to AIDS and ultimately, to death. Severe Acute Respiratory Syndrome (SARS)
was the first outbreak in the twenty-first century that managed to get public attention. Caused
by the SARS Corona virus (SARS-CoV), started in China. The outbreak of SARS provided
an opportunity to study the use and impact of public health informatics and population health
technology to detect and fight a global epidemic. A disease that has a high fatality rate, Nipah
Virus infection was first detected in Malaysia and Singapore in 1990s.\footnote{ Epidemics and Pandemics in India throughout History: A Review Article/ Nippah virus} An onslaught of
nature, deforestation that led to the fruit bats (the carriers of the virus), losing their habitat
and coming in close contact with domestic animals and humans resulted in Nipah outbreak.\footnote{What is Nipah Virus? The Indian Express, Express Web Desk, February 9, 2020}
Nipah is a sort of a new-age plague.

\noi
The Nipah outbreak reported in Kozhikode and Malappuram districts of Kerala in May 2018
was the third of Nipah Virus Outbreaks in India, the earlier being in 2001 and 2007, both in
West Bengal. The outbreak was managed by the state government and the central government
agencies acknowledging a success story.

\noi
Therefore, it is evident that unsanitary environments and pollution caused by humans are the
most important reasons for the fast-spreading of pandemics. In additional to the present,
developments in transport and information technology created the world as a global village,
which makes the spreading of contagious diseases inevitable.\footnote{LePan, Nicholas. “Visualizing the History of Pandemics.” Visual capitalist, 14 March 2020, available online at\\ \url{www.visualcapitalist.com /history-of-pandemics-deadliest}} The world currently is in the middle of an ongoing pandemic - Covid 19 conjointly known as SARS-CoV-2 was 1$^{\rm st}$ reported in Wuhan, China early as December 2019 when a bunch of patients affected by the respiratory illness of unknown cause was connected to the live wet markets existing in China. A pandemic outbreak of this size, proportion, and consequences was unique, unparalleled,
and unprecedented. The world was forcibly shut down on account of Covid-19. All spheres,
sectors, and aspects of human life are impacted negatively due to the pandemic.

\heading{The Impact of Covid19 in India}

\noi
The pandemic and the nationwide lockdown imposed on account of COVID 19 has adversely
affected the economy and livelihoods of millions in India.

\heading{Economic Impact:}

\noi
The effect of the (COVID-19) pandemic had not just carried the worldwide economy to a halt
and had also crashed important global stock exchanges which prompted the loss of billions of
dollars wiped out very quickly. Studies done by UBS globally predicts that 20-30\% of
industries will leave China because of the course of Coronavirus and continuous trade wars.
Being the rate of infection \& rate of death (compared to 1 M pop) in India did not appear to
be as high as in different nations, prudent steps received in 2020 managed a serious hit to the
nation's significant business areas - with the service industry bearing the biggest brunt of the
assessed loss. India's growth in the fourth quarter of the fiscal year 2020 went down to 3.1\%
as per the Ministry of Statistics. Every one of the significant business sectors in India
confronted the individual set of difficulties because of the pandemic, MSMEs and
unorganised sectors are said to be the worst hit. On the record of Covid-19, the fundamental
areas involve a significant dip in part of India's GDP: Agriculture, Industry, Service areas
impacts appear in the table beneath.\footnote{ESTIMATED ECONOMIC IMPACT FROM COVID – 19 IN INDIA 2020, by Sector, Published by Statista Research Department, available online at  \url{https://www.statista.com/statistics/1107798/india-estimatedeconomic-impact-of-coronavirus-by-sector/}}

\vspace{-.3cm}

\begin{enumerate}[label=$\blacktriangleright$]
\itemsep=0pt

\item April to June:

The growth output of the Agricultural Sector was 3.40\%

The growth output of the Industrial Sector was -20.60\%

The growth output of the Services Sector was -38.10\%

\item July to September:

The growth output of the Agricultural Sector was 3.40\%

The growth output of the Industrial Sector was -2.10\%
 
The growth output of the Services Sector was -11.40\% 

\item October to December:

The growth output of the Agricultural Sector was 3.90\%

The growth output of the Industrial Sector was 2.70\%

The growth output of the Services Sector was -1.00\%
\end{enumerate}

\vspace{-.3cm}

\noi
India’s unemployment rate has skyrocketed to nearly 23.52\% in Aug 2020, which
forced millions of migrant workers to go back to their native villages and their
livelihood was critically affected.\footnote{6 COVID-!9 IMPACT ON UNEMPLOYMENT RATE IN INDIA 2020 – 2020, Published by Statista Research Department, available online at\\ \url{https://www.statista.com/statistics/1111487/coronavirus-impact-onunemploymentrate/}\\\#::text=COVID\%2D19\%20impact\%20on\%20unemployment\%20rate\%20in\%20India\%202020\%2D2021 \&text=In\%20January\%202021\%2C\%20India\%20saw,rate\%20 of\%20 over\%20 six\%20 percent.\&text=A\%20 damaging\%20impact\%20on\%20an,24\%20percent\%20in\%20April\%2020 20.}

\heading{Political Impact:}

\noi
The COVID-19 pandemic has led to an enormous impact on politics both internationally and
domestically. It has affected the political systems and governments of multiple countries that
indulge suspension of some legislative conferences, international political meets, deaths of
multiple politicians and even the elections were rescheduled because of the concern of virus
spread.\footnote{Van Holm, E., Monaghan, J., Shahar, D. C., Messina, J. P., \& Surprenant, C. (2020). The impact of political ideology on concern and behaviour during COVID-19. Available online at SSRN 3573224.} The pandemic has placed governments within the world to react resolutely and
quickly, however, the political responses are varied across the countries. In the Republic of
India additionally, the political responses between the state and the centre are variable that
has political changes within the country. Politically several super-power nations started
supporting one another and extended their political support to low developed nations. India
extended its support by donating \& supply of the vaccines to 95 countries, though in India,
the percentage of people given the 1$^{\rm st}$ dose of vaccination stood at 14.14\% and the percentage
of completely vaccinated was only at 3.7\% as of May 2021. 

\heading{Legal Impact:}

\noi
The COVID-19 had an impact on the global court functioning and the legal system.
Ultimately, the impact of Covid-19 on the legal market is proven by new regulations enacted
impromptu in many areas, including leases; finance or commercial agreements; employment
regulations concerning temporary or definitive dismissals; tax exemptions, payments or
returns delays; licences or authorisation terms and foreign investments etc. On march 2020,
the Supreme Court of India has declared that, it'll be hearing only the urgent matters. Similar
announcements are made by all High Courts and other courts of various states, that wedged the legal profession in India, particularly the litigating legal professionals. The Supreme
Court plans the judicial proceedings step by step to be shifted towards technological
integration into the Indian legal system which led the Honourable Supreme Court to send
notices and summons through digital platforms. The virtual courts and dispute resolution held
online are now a reality being a hybrid model of physical and virtual court processes and has
its pros and cons between the lawyers, the clients and also the legal systems.

\heading{Social and Cultural Impact:}

\noi
COVID-19 had social effects on the lives of the population. It had an impression on the
psychological well-being of the individuals as their social life was disrupted. The closure of
schools and colleges, offices, hotels, restaurants, malls, and multiplexes has disrupted the
relationships among individuals and their perceptions of fellow feeling towards others due to
social distancing.\footnote{Bhagat, R.B., Reshmi, R.S., Sahoo, H., Roy, A.K. and Govil, D., 2020. \textit{The COVID-19, migration and livelihood in India: challenges and policy issues.} Migration Letters, 17(5), pp.705-718.} The gatherings reduced, and the people were forced to remain at their
home. Anxiety, depression, distress, and insomnia showed higher percentages throughout the
lockdown period resulting in psychoneurotic thoughts. There has been progressive closure of
the people with reduced social relationships moving their social support because of mobility
restrictions. People are considered as social entities engineered on social facts and forceful
distancing from others to tackle the emergency incremented social phobias among the
population.\footnote{Saladino, V., Algeri, D. and Auriemma, V., 2020. \textit{The psychological and social impact of Covid-19: new perspectives of well-being.} Frontiers in psychology, 11, p.2550.} At this juncture, technological devices started getting vital roles in the way of
life.

\heading{Deficiencies in the Existing Indian Legal Framework to Deal With the Pandemic-\\Glaring Drawbacks That Require Redressal}

\noi
India has a plethora of laws to deal with an epidemic or pandemic. India has largely depended
on The Epidemic Disease Act 1897, the Indian Penal Code 1860, and The Disaster
Management Act 2005 to decide its response to the current pandemic - COVID 19.The
Epidemic Disease Act, 1897, Indian Penal Code, 1860 and The Disaster Management Act,
2005 that India has largely depended on to decide its response to the pandemic are archaic,
obsolete, insufficient, and weak to deal with the consequences of a pandemic of this size, 
such as the coronavirus.\footnote{20 missing ???????} There are no substantiative provisions in either of the Acts that
provide solutions to the innumerable issues that have cropped up on account of COVID 19.
The Executive is drafting laws on matters under the Act by issuing Notifications and Orders,
assuming the role of the Legislature. The Government is utilising the Epidemic Diseases Act,
1897 and The Disaster Management Act, 2005 as if the Legislature has granted limitless
powers to the Executive. No such boundless powers are granted to the Executive under the
Acts. Even an Emergency declared under Article 352 requires parliamentary supervision.
Checks and balances have to be put in place appropriately to supervise the powers exercised
by the Executive.

\heading{The Epidemic Disease Act of 1897\footnote{Harleen Kaur, CAN THE INDIAN LEGAL FRAMEWORK DEAL WITH THE COVID-19 PANDEMIC? A REVIEW OF THE EPIDEMIC DISEASES ACT BAR AND BENCH - INDIAN LEGAL NEWS, available online at\\ \url{https://www.barandbench.com/columns/can-the-indian-legal-framework-deal-with-the-covid-19-pandemic-a-review-of-the-epidemics-diseases-act}}}

\noi
Following the bubonic plague of 1896, the Infectious Disease Act of 1897 became obsolete.
It comprises only four parts and has undergone a few amendments. The Antique Epidemics
Disease Act has been revised recently with many reminders, as it does not include any
recommendations to prevent or mitigate epidemics. What constitutes a' dangerous infectious
disease' is not described by the old three-page-and-four-section EDA. It provides the
executive with unfettered powers to enforce orders. The EDA has been relied on by the state
and central governments to deal with the health aspect of this scourge that has invaded our
country.

\noi
The Infectious Disease Act, 1897 leaves the state with the power to determine which steps to
initiate, and it is difficult for them to make the correct response since those leading the states
are not medical experts. EDA has not specified what a harmful or contagious disease is, and
there are still no procedures for determining if such diseases are an epidemic. Another
drawback of the EDA is that there are no provisions on the sequestering and sequencing of
medication and vaccine distribution and how the preventive measures can be implemented.

\noi
\textbf{Act Not Comprehensive -} This Act is short. It only contains four sections and has gone
through a few amendments. There have been several reminders recently to update the antique
Epidemics Disease Act, as it does not contain any guidelines to prevent or alleviate
epidemics.

\vspace{-.1cm}

\noi
\textbf{Unbridled Power Granted to Government -} It gives unfettered powers to the executive to
promulgate notifications and orders. The State and Central Governments have banked on the
EDA to deal with the health aspect of this scourge that has invaded our nation. The state has
been coercive, dictatorial, and arbitrary in its action of handling the pandemic. Regulations
have been crafted one after another to exact public obedience, failure of which invited
imprisonment.

\vspace{-.1cm}

\noi
\textbf{Scope of the Act Exceeded -} The Doctrine of Ultra Vires is likely to be invited as the scope
of the Act has been transcended through the regulations that have been passed under the
Epidemics Diseases Act 1897.\footnote{Cclsnluj, COVID-19 – VII: IS THE INDIAN LEGAL FRAMEWORK CAPABLE OF HANDLING THE CORONAVIRUS PANDEMIC THE CRIMINAL LAW BLOG (2020), available online at\\ \url{https://criminallawstudiesnluj.wordpress.com/2020/04/19/covid-19-vii-is-the-indian-legal-framework-capableof-handling-the-coronavirus-pandemic/}}

\vspace{-.1cm}

\noi
\textbf{Blanket Immunity to Public Servants -} Disobedience to the regulations is made a
punishable offence while providing immunity to public officers for performing functions
under the law. Extensive powers are granted to the government officers - to admit and isolate
a person, powers of surveillance of individuals and private premises.

\noi
\textbf{Lack of Coordination -} It does not establish any coordination mechanism between states and
the union government at the time of a dangerous epidemic outbreak.

\noi
\textbf{Misuse -} The State can misuse the law for profiling, mass quarantine, and targeting of
individuals.

\noi
\textbf{Drugs -} Another limitation of EDA is that it has no provisions concerning sequestering and
sequencing for the dissemination of drugs and vaccines as well as how preventive measures
should be enforced.\footnote{Manish Tewari, INDIA'S FIGHT AGAINST HEALTH EMERGENCIES: IN SEARCH OF A LEGAL ARCHITECTURE ORF (2020), \url{https://www.orfonline.org/research/indias-fight-against-health-emergencies-in-search-of-a-legalarchitecture-63884/}}

\heading{The Disaster Management Act, 2005}

\noi
The DMA has been built to overcome the threats of natural disasters such as tsunamis,
earthquakes, and cyclones, to rapidly shut down and isolate a small portion of the country.
The architects of DMA did not include provisions to handle an epidemic or pandemic
outbreak, where such natural disasters occur only in a certain part of the place, by disrupting
for a few hours or feeding days the usual day-to-day life. During a pandemic, the first and
foremost rule is for human beings to maintain a social distance. The coronavirus outbreak has been announced by NDMA as a confirmed disaster. Lockdowns were enforced by MHAissued circulars and alerts.

\noi
\textbf{Basis of Regulations issued by MHA Questioned -} The regulations given by the MHA
under the DMA lacks a legal basis. The government is not empowered to pass orders under
section 10(2) of the DMA, 2005. Section 10 under which the orders are issued, empowers the
government to give guidelines or directions only to the government machinery and cannot be
imposed on private individuals or establishments. The orders passed can be nullified under
the Doctrine of Substantive Ultra vires.\footnote{Cclsnluj, COVID-19 – VII: IS THE INDIAN LEGAL FRAMEWORK CAPABLE OF HANDLING THE CORONAVIRUS PANDEMIC THE CRIMINAL LAW BLOG (2020), available online at\\  \url{https://criminallawstudiesnluj.wordpress.com/2020/04/19/covid-19-vii-is-the-indian-legal-framework-capableof-handling-the-coronavirus-pandemic/}}

\noi
\textbf{\textit{Punishments for Disobedience -}} The punishments prescribed in the DMA 2005 are for
disaster-centric acts like reconstruction, relief assistance, false warning, raising a false claim
to obtain repair during the period a disaster strikes, and not for the outbreak of a
communicable disease.

\noi
\textbf{\textit{Punishments provided in the Act are Unused and Dormant} –} \textit{A complaint has to be raised to the National} or State Disaster Management Authority by giving a 30 days’ notice under
Section 60 of the DMA for taking cognizance of any offence committed under the act,
thereby hindering quick action which is essential at the time of an epidemic.

\heading{The Indian Penal Code, 1860}

\noi
People are being charged under sections 188, 269, and 270 of IPC for violating lockdown
orders. If the orders passed by any government authority are violated, the same is
prosecutable under Section 188 of IPC. This is a broad provision that deals with the
disobedience of an order passed by any government authority.

\vspace{-.1cm}

\heading{Conclusion}

\vspace{-.1cm}

\noi
The Indian State’s response to the Covid-19 pandemic reveals a stark truth: The central
government, has invoked the Disaster Management Act of 2005 (DMA), whereas many state
governments have invoked the Epidemic Diseases Act of 1897 (EDA), both the existing laws,
one is colonial vintage and the other has wide, vaguely-worded with umbrella clauses that
enable the Central and the State governments to initiate any measure they see necessary to
handle a disaster or epidemic. India’s administrative authorities have used these two different 
and non-related laws to impose a nationwide “lock-down”—the precise legal definition of
that remains unclear as it seals the state borders, suspend transportation services and
individual movement and justify intensive quarantining and social distancing necessities.
It is necessary to notice that neither of those laws are classic “emergency laws”. Neither the
EDA nor the DMA involves politically risky, formal measures such as the proclamation of an
emergency or the deferment of Parliament. Rather, within the context of COVID-19, the
EDA and DMA functioned as sanctioning laws, permitting governments expansive, nearly
unchecked, powers without requiring the politically fraught task of declaring an emergency
and suspending civil liberties. Going forward, therefore – and beyond Covid-19 – the task
seems to be twofold: that specialize in narrowing the scope of umbrella legislation that
effectively authorizes rule by decree without the legal safeguards and political responsibilities
of an emergency declaration, and on articulating a brand new “EPIDEMIC AND
PANDEMIC ACT” and within the same approach contributively to a legal culture aimed
towards restoring a robust judicial review over government action supposedly for the benefit
of the people at large.

\vspace{-.1cm}

\heading{Suggestions}

\vspace{-.1cm}

\noi
A suggestion to revisit, revise, update and amend the existing Epidemics Disease Act 1897 or
to strike it down to pave way for a new exclusive comprehensive law for Pandemics is a need
of the hour. Based on challenges observed from the current laws, it is suggested that there is a
necessity to develop a new comprehensive pandemic law to replace the 1897 Act and to deal
with future pandemics by considering the following recommendations:

\vspace{-.2cm}

\begin{enumerate}[label=$\bullet$]
\itemsep=0pt

\item The Bill drafted by the Union Ministry of Health and Family Welfare – PUBLIC HEALTH PREVENTION, CONTROL, AND MANAGEMENT OF EPIDEMICS, BIOTERRORISM AND DISASTERS BILL 2017 – should replace the 100-year-old Epidemics Diseases Act 1897 to counterstrike any emergency swiftly.

\item Most importantly, the new legislation should lay down clear definitions of epidemic, pandemic and what an infectious disease is the need of the hour.

\item The act must provide for the following stringent Covid appropriate behaviour and Covid protocols during the pandemic – To Wear a Mask, Keep Social Distance and Sanitise.

\item The Act must ban social gatherings like weddings or funeral and mass gatherings for religious congregations, sporting events and election campaigns during a pandemic.

\item The Act must protect doctors and frontline health workers in terms of their health,
salaries, work hours, insurance and safety.

\item Since health is a state function in India, the states must be involved in the
development of this law and their views incorporated.

\item The Act must ensure that India has a proper healthcare infrastructure in place to meet
future pandemics. All the healthcare systems should ensure that people are provided
with equal and qualitative care in both private and public hospitals without any form
of discrimination based on gender, race, religion, or age.

\item The new law must provide procedural guarantees against abuse of State power when
interfered with the Right to Privacy of an individual.

%~ \vspace{.1cm}

\item The new law should have the provisions to impose immediate lockdowns to stop the
spread on a larger level.

%~ \vspace{.1cm}

\item The new law must define the procedures that the central and local government must
undertake during the pandemic, promoting cooperative federalism. The procedures
should be proposed by health and medical experts and strictly followed in case of a
pandemic. Measures such as lockdown should be clearly defined and the times where
they can be applied.

\vspace{.1cm}

\item The new law should ensure that such measures do not comprise the Indian
Constitution, especially on human rights, fundamental rights and the freedom of
people and the same are not trampled upon in the name of national interest.
Separation of powers is enshrined in the Constitution and the Act should ensure that
the Executive does not assume the role of the Legislature.

%~ \vspace{.1cm}

\item The new law should recommend establishing a National Authority that will
coordinate all the responses in case of a pandemic or epidemic making it responsible
for ensuring that resources are allocated evenly in the case of an epidemic.

%~ \vspace{.1cm}

\item The new law requires the government to invest and develop high technology such as
artificial intelligence to facilitate surveillance which shall clearly state when and how
such technology can be used to avoid infringing on people’s privacy. While
conducting surveillance, the individuals being surveyed must be informed, and such
surveillance must not invade the basic rights of the person.

\item The development of the new law should involve public participation in making it an
effective law in force during an outbreak.

\item The new law should have a flat rate policy introduced during a pandemic situation to
curb black marketing and increase in the price of essential commodities, such as,
vaccines, medical equipment’s, medicines etc.,

\item The new law must impose on the Government and the political parties to work under
the concept of “ONE PARTY, ONE NATION”, setting aside their political
differences, ideologies and vendetta, so as to unitedly eradicate the common enemy –
“The Virus”.

\item The new law should be capable to instruct the medical fraternity to work with only a
specific set of expertise doctors in treating the pandemic and not to involve all the
doctors compromising the treatments for other health-oriented diseases.

\item The Act ought to accommodate an NDMA like power or body, having portrayal from
both the Centre and states, answerable for planning and executing, recognizable proof,
contact-tracing, isolation, segregation, testing techniques, and treatment. The NDMA
should also authorise the executive to plan for lockdown strategy, supply of essential
and non-essential services, food and relief support, and all non-health services.

\item The Act must give freedom to States to draft and implement as per their local
assessments, preparing health facilities to face challenges at the respective district and
gram panchayats.

\item The Act must provide for a combination of civil and criminal penalties for violating
the orders issued by government authorities and protect frontline workers like
sanitation staff, nurses, doctors, village-level health workers, and police personnel.

\item The Act must provide for distribution of foreign aid that pours in for India

\item The Act must provide that the executive should be transparent while projecting actual
numbers of live cases /fatalities and not conceal real figures to the public.

\item The Act must protect the freedom of speech and expression of protestors who voice
their concern over incorrect and excessive executive action during the pandemic,
prevent the executive to arrest them and curb resistance. The Act must also protect
social media from being clamped down by the executive who wants to silence
criticism and police online content.

\item The Act must safeguard children and cancel all public examinations during a
pandemic, to protect them from getting infected and spreading the infection. The Act
must also provide for alternative sources of internal assessment to examine them at
home online and get the desired results to move forward.

\item The Act must also protect the rights of patients during a pandemic.

\item The Act must make provisions to ramp up laboratories, testing, ICU beds, PPE Kits,
Supply of Oxygen, Critical Medicines etc.

\item The Act must promote genome sequencing, to gauge the impact of the new mutant
variants.

\item The Act must provide for relief measures to migrant labourers who are registered in a
national database of migrant workers

\item The New law must provide for estimating the country’s requirement and controlling
the price, production, availability, distribution, export, import, licencing, patenting,
efficacy, administration and sequencing of vaccines

\item The Act must give importance to the fundamental principle of human dignity which
not only speaks about the dignity of a living person but also includes the dignity of a
person even after his death by performing their last rites according to their religious
rituals at the time of disposing Covid infected bodies of the deceased.
\end{enumerate}

\vspace{-.5cm}

\noi
One can partially blame the Government’s arbitrary, vacillating, haphazard and dictatorial
reaction to COVID-19 on the lack of a national protocol and law, which is designed to
regulate pandemics. The laws India has depended on were not originally crafted to meet the
threats presented by a modern-day pandemic. India desperately needs an exclusive
comprehensive law to control and manage epidemics and pandemics.

\end{multicols}
