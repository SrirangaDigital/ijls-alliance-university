\setcounter{figure}{0}
\setcounter{table}{0}
\setcounter{footnote}{0}


\articletitle{Domain Name Disputes and the Rising Threat of Cybersquatters}
\articleauthor{Jalaj Agarwal\footnote{BA LLB (Hons.) 5$^{\rm th}$ Year, Symbiosis Law School, Pune} and Gracy Bindra\footnote{BA LLB (Hons.) 5$^{\rm th}$ Year, Symbiosis Law School, Pune}}
\lhead[\textit{\textsf{Jalaj Agarwal and Gracy Bindra}}]{}
\rhead[]{\textit{\textsf{Domain Name Disputes and the Rising...}}}

\begin{multicols}{2}

\vspace{-.2cm}

\heading{Introduction}

\vspace{-.1cm}

\noi
In recent decades, the world has been taken over by the internet, giving rise to digital age
menaces. The Internet is now being used for commercial purposes in recent years; this has led
to the transformation in the businesses. With the changing trends of marketing and a
paradigm shift of physical marketplace to e-commerce, many companies have been
successful in their business and commerce positioning services online.\footnote{Michael D. Scott, \textit{Advertising in Cyberspace: Business and Legal Considerations,} COMPUTER LAW., 1
(1995).}
 
\noi
As cyberspace takes over the businesses, the importance of domain names and trademark law
increases. A domain name refers to a computer address by which a company or an individual
can be located by any other internet users.\footnote{Sally M. Abel, \textit{Trademark Issues in Cyberspace: The Brave New Frontier,} MICH. TELECOMM. \& TECH. L. REV., (1999).}  The main aim to pose domain names is to
distinguish and locate the different computers, users, files, and resources accessible over the
Internet.\footnote{Bonifaz, Monica. \textit{Domain names, Internet and Trademarks, infringements in cyberspace,} PAPER REVIEW, (2015).} In usual practice, the companies are inclined to choose domain names which are easy to memorize, everyday words and well- known trademarks by their consumers. The
problem arises when two or more people claim the same name, which is forbidden under
trademark law. Trademark law has often been referred to as it restricts the use of already
registered trademarks because it tends to confuse a potential customer about the profile and
true seller of the product or service involved. The same has been invoked to resolve disputes
between computer users that obtain Internet domain names and the owners of the registered
trademarks.\footnote{Froomkin, A., \textit{The collision of trademarks, domain names, and due process in cyberspace,} COMMUN. ACM., (2001).}  There are various negative consequences of the transition to cyberspace in the form of trademark issues such as cyber-squatting, typo- squatting, mega tagging, renewal
snatching etc.

\heading{Domain Name System and Trademark}

\vspace{-.1cm}

\noi
Every web page has a unique address which not only represents the branding of the company
but also distinguishes it from the other companies in the market. The domain names aid
internet users to remember, locate and access the sites instead of writing the long IP address
in binary computer language.

\noi
A domain name refers to a unique name which recognizes the website7\footnote{Pope, Michael \& Warkentin, Merrill \& Mutchler, Leigh \& Luo, Robert, \textit{The Domain Name System: Past, Present, and Future,} COMMUNICATIONS OF THE ASSOCIATION FOR INFORMATION SYSTEMS, (2012).}
 and contains three parts to it. The first part known as third level domain which contains- “www". which
represents that the website is connected to the world wide web and discoverable on the
internet search engines. The second part is the most essential part which includes the unique
name of the company, for eg. - “Facebook”, and better known as second level domain name.
The last part is known as top level domain name and could be of various types -generic code,
country code, special top- level domain names or restricted use domain names. If it is a
country code, such as- “.in” for India or “.jp” for Japan; it represents a particular country. In
case, a company chooses generic codes, such as- “.com”, “.org”, “.edu”; it represents
deployment to no particular class of organization and is regulated by the Internet Corporation
for Assigned Names and Numbers popularly known as ICANN. Some of the special top-level
domain names are- “. legal”, “. app” etc. Restricted top- level domain names are not allowed
to be used by everyone as the name suggests, such as “. arpa”, “.biz” etc.

\noi
The process of allotment of such domain names differs on case to case basis. It could either
be a first come first serve basis or if a company with legitimate business claims a domain
name of that company name, would be given preference over others.

\vspace{-.1cm}

\noi
In this changing world of e-commerce, the domain name systems have a great significance
and disputes arising out of it have no bounds. This calls for a need for a specific regulating
authority. Since recognising the source of product is an important role played by the domain
name, there is a need to treat them as equivalently important to trademark as far as the legal
protection and recognition is concerned, as this could lead to trademark infringement.8\footnote{Richard L. Baum and Robert C. Combow, \textit{First Use Test in Internet Domain Name Disputes,} NATL. LJ 30, (1996).}

\vspace{-.1cm}

\noi
Trademarks not only give unique identification to the product but have also become a way of
digital branding for various companies. Businesses use fancy, unique and distinct domain
names by often combining two languages, different font and color schemes in order to attract
more users to their websites, thus it is an important tool for communication in business
transactions.

\vspace{-.1cm}

\noi
In the real world, two people belonging from different countries can have one trademark for
goods and services unlike the domain name which belongs to only one person in the virtual
world and need not be for one good/service but could be for the whole company dealing in a
range of different goods and services.

\vspace{-.1cm}

\heading{Understanding Cybersquatting}

\vspace{-.1cm}

\noi
Domain name abuse and misuse in the form of cyber-squatting has increased in great
numbers with the growth in commercial activities and use of cyberspace. In the 1990s, the
internet had become a growing sensation and grew the menace of cyber-squatting, also
known as brand-jacking.\footnote{Jonathan Anschell \& John J Lucas, \textit{What’s in a Name: Dealing with Cybersquatting,} ENT. \& SPORTS LAW 3$^{\rm rd}$ edn., (2003).} The Delhi High Court interpreted the term Cyber-squatting as “an
act of obtaining fraudulent registration of a domain name with the intent to sell it to the
lawful owner of the name at a premium.”\footnote{Manish Vij v Indra Chugh, [2002] AIR Del 243.}

\vspace{-.1cm}
\noi
The ultimate motive of maximum profit maximization drives the menace of cyber-squatting.
Another reason for indulging in cyber-squatting could be to defame and bring bad name to
the company by using fake identities. Registration of a domain name is a cheap and
economical process; however, once a domain name is registered, it allows the party to earn
profit through various means such as by publishing advertisements on the web page or by
pay-per-click advertisements.\footnote{Jordan A. Arnot, \textit{Navigating Cybersquatting Enforcement in the Expanding Internet,} J. MARSHALL REV. INTELL. PROP. L., 13$^{\rm th}$ edn., (2014).} It can also be used to divert user’s traffic from the original
trademark holder’s business by creating confusion in the minds of the consumers, and thereby
causing losses to them.\footnote{Dara B. Gilwit, \textit{The Latest Cybersquatting Trend: Typosquatters, Their Changing Tactics, and How to Prevent Public Deception and Trademark Infringement,} WASH. U. J. L. \& POL’Y 11$^{\rm th}$ edn., (2003).} Moreover, such individuals often sell a registered domain name at
significantly high prices to the legitimate owner of a trademark whose identity is reflected in the domain name by creating a similar one, creating confusion and deception and using unfair
trade practices such as blackmailing and harassing the original owners to gain revenue.\footnote{RASTOGI ANIRUDH, CYBER LAW, LAW OF INFORMATION TECHNOLOGY AND INTERNET (Lexis Nexis 2014).}

\vspace{-.1cm}
\noi
There are various types of cyber-squatting like\footnote{Sankalp Jain, \textit{Cyber Squatting: Concept, Types and Legal Regimes in India \& USA,} SSRN ELECTRONIC JOURNAL, (2015)}-

\vspace{-.1cm}

\noi
1. \underline{Typo squatting-} This type includes ‘URL hijacking’, ‘a sting site’, and ‘a fake URL’
wherein typo squatters take advantage of the mistakes internet users make while searching
the browser and typing web addresses. Due to the similarities in the visuals, fonts and
hardware, the users are often confused. Typo squatters might go to the extent of creating fake
websites using similar logos and colors to divert and confuse the traffic and create malware.

\vspace{-.1cm}

\noi
2. \underline{Identity Theft –} In case an owner forgets or fails to renew their domain, the cyber squatter
takes undue advantage of the situation by misleading the internet users by posing to be the
legitimate owners. They do so by monitoring and targeting such domain names and purchase
them as soon as renewal gets delayed or fails.

\noi
3. \underline{Name Jacking -} Cyber squatters use celebrities or famous personalities posing to be related
to them and illicitly attracting traffic to their website.

\noi
4. \underline{Reverse Cyber-squatting -} Reverse cyber-squatting refers to a scenario wherein the cyber
squatters attempt to secure legitimate domain names to indicate authenticity and create
confusion and undue benefits.

\noi
There is a need to curtail such practices and therefore countermeasures have been devised by
various organizations’, which will be discussed in the next segment of the paper.

\heading{Countermeasures to Cybersquatting}

\noi
In order to prevent the growing threat of Cyber-squatting it is important to have proper
regulation, policies and authorities to counter and seize any such malicious acts. Globally,
Internet Corporation for Assigned Names and Numbers (ICANN) is the organization that
administers the domain name system. It was established in 1998\footnote{THE HISTORY OF ICANN, \url{https://www.icann.org/history}\#:~:text=ICANN\%20was\%20founded\%20in\%201998,the\%20U.S.\\\%20with\%20gl obal\%20participation (last visited on 14 September 2020).} as an American not for profit private organization which undertakes the task of overseeing and supervising the
distribution of IP addresses and domain names thereby managing and coordinating the
domain name system. However, it is pertinent to note that the actual domain name
registration is done by particular domain name registries located in different countries across
the globe.

\noi
In order to resolve and facilitate the disputes arising in relation to domain names, the Uniform
Domain Name Dispute Resolution Policy (UDRP) was established in the year 1999 by the
ICANN.\footnote{UNIFORM DOMAIN-NAME DISPUTE-RESOLUTION POLICY,
\url{https://www.icann.org/resources/pages/help/dndr/udrp-en} (last visited on 14 September 2020)} Since its establishment UDRP has been successfully implemented in resolving a
large number of domain name disputes over the years.\footnote{Zohaib Hasan Khan, \textit{Cybersquatting and its Effectual Position in India,} Vol. 6 Issue 2, IJ SCIENTIFIC \& ENGINEERING RESEARCH, (2015).} Currently there are six approved
dispute resolution providers to which complaints can be filed as per procedure laid down
under the UDRP; they are: World Intellectual Property Organization (WIPO), Asian Domain
Name Dispute Resolution Centre (ADNCRC), National Arbitration Forum (NAF), Canadian
International Internet Dispute Resolution Centre (CIIDRC), Arab Center for Dispute
Resolution (ACDR) and Czech Arbitration Court (CAC).\footnote{ICANN, LIST OF APPROVED DISPUTE RESOLUTION SERVICE PROVIDERS, \url{https://www.icann.org/resources/pages/providers-6d-2012-02-25-en} (last visited on 14 September 2020).} Among them WIPO has been the most popular domain name dispute resolution platform.

\noi
Paragraph 4(a) of the UDRP provides the necessary elements when a trademark holder can
apply to any ICANN dispute resolution service provider. According to this, the UDRP is
capable of resolving disputes which arise when:\footnote{UNIFORM DOMAIN DISPUTE RESOLUTION POLICY,
\url{https://www.icann.org/resources/pages/policy-2012-02-25-en} (last visited on 14 September 2020).}

\begin{enumerate}[label=$\bullet$]
\item The domain name is alike or confusingly similar to the trademark to which the
complainant has rights.

\item The opposite party has no legitimate right or interest in the domain name.

\item The opposite party has registered the domain name with mala fide intentions.
\end{enumerate}

\noi
Furthermore, Paragraph 4(b) enumerates the factors for determining if there’s a case of
registering or using the domain name in bad faith by the concerned party. Various factors
which are taken into consideration for this purpose are:

\begin{enumerate}[label=$\bullet$]
\item The domain name was registered with the main objective of selling it at a higher price
afterwards

\item The domain name was registered with the primary purpose of causing loss to the
business and brand value of the competitor

\item The domain name was registered so as to prevent the rightful owner of the trademark
from acquiring the domain name for its mark

\item The domain name was registered in order to take undue advantage of the brand value
of the complainant’s trademark and attract users to its website by creating confusion
between the two parties. 
\end{enumerate}

\noi
Once the domain name dispute is resolved, the concerned authorities can either transfer the
domain name to the complainant or cancel the domain name altogether. On the other hand, if
the complaint is found without merit it can be rejected by the service providers. UDRP does
not provide for any remedy in the form of monetary damages or any kind of injunctive relief.
In case the losing party is not satisfied with the decision of the authority it can file a lawsuit
against the opposite party in a court of competent jurisdiction within 10 days of the said
decision.\footnote{THE UNIFORM DOMAIN DISPUTE RESOLUTION POLICY, \url{https://www.icann.org/resources/pages/policy-2012-02-25-en} (last visited on 14 September 2020).}

\noi
\textit{World Wrestling Federation Entertainment, Inc. v Michael Bosman}\footnote{World Wrestling Federation Entertainment, Inc. v Michael Bosman, [2000] WIPO, Case no. D99-0001.} was the first case
decided by WIPO through the UDRP. In this case the respondent had first registered the
domain name “worldwrestlingfederation.com” and thereafter offered to sell the domain name
to WWF at a higher price. WWF filed a complaint alleging that the domain name was
registered with mala fide intention by the respondent and was in violation of WWF’s
trademark. The WIPO panel ascertained that the domain name was identical or confusingly
similar to the WWF’s trademark. It further held that the respondent had no bona fide interest
or right in the said domain name and ordered the transfer of the registration of the said
domain name to the complainant.

\noi
Similarly in Philip Morris Incorporated v r9.net,\footnote{Philip Morris Incorporated v. r9.net, [2003] WIPO Case no. D2003-0004.} the complainant was the owner of the wellknown brand and trademark ‘Marlboro’. However, the respondents registered the domain
name “Marlboro.com” against which a complaint was registered alleging that the said domain
name was confusingly similar to the complainant’s trademark and was registered in bad faith.
The allegations were held to be valid by the WIPO panel and the registered name was
transferred to the complainants.

\noi
Overall, the UDRP as operated by the approved service providers under ICANN is very
efficient and popular in resolving domain name disputes. The UDRP process is much quicker
and cheaper than the court litigation, it has an international jurisdiction and the cases are
resolved by individuals who are experts in trademark law which is not always possible in
normal litigation.

\noi
Over the years some countries such as Canada (Canada’s Domain Name Dispute Resolution
Policy) and the United Kingdom (UK’s Domain Dispute Resolution Service) have even
adopted their own dispute resolution mechanism being unrelated to the UDRP.\footnote{Doug Isenberg, \textit{These Countries have adopted the} UDRP, GIGALAW, (2017)  \url{https://giga.law/blog/2017/5/23/these-countries-have-adopted-the-udrp} (last visited on 13 September 2020). } Although India has formulated its own dispute resolution policy, INDRP 2005, which is in line with the
UDRP and the provisions of the Indian Information Technology Act, 2000 it has not been put
to use much on account of it being a mere guiding policy.\footnote{Jayakar, Krishna \& Patricia, \textit{India’s Domain Name Dispute Resolution Process: An Empirical Investigation,} SSRN E-J, (2012).}

\heading{Situation in India}

\noi
Over the years there have been numerous cases of cyber-squatting in India, but with the
unprecedented growth in digital media and internet, there has been a surge in such cases in
recent years. However, currently, there is no specific legislation in India for resolving domain
name disputes such as cyber-squatting. The Indian Trademarks Act, 1999 does not provide
for any specific provision protecting domain names in pursuance of trademark infringement.
Furthermore, the jurisdiction of the act is not extra-territorial; therefore, it does not provide
for adequate protection in case of infringement happening outside the Indian territory.
Similarly, the provisions of the Information Technology Act, 2000 are not sufficient to resolve the domain name disputes in relation to trademark infringement and to curtail the acts of cyber-squatting. 

\noi
However, the Indian courts have been active in resolving cases relating to cyber-squatting
under the laws relating to passing off. Passing off is a common law tort and has been further
developed by the Hon’ble courts to be applied in such cases of domain name disputes. This
can be inferred from the judgment in \textit{Satyam Info way Ltd. v Sifynet Solutions (P) Ltd.,}\footnote{Satyam Infoway Ltd. v. Sifynet Solutions (P) Ltd., [2004] (3) AWC 2366 SC}
wherein the Supreme Court stated that although there is no specific legislation in India with
respect to resolving disputes pertaining to domain names and the Trademarks Act, 1999 also
do not provide adequate protection as its operation is not extraterritorial, however domain
names in India were protected under the laws relating to passing off to the maximum extent
possible.

\noi
A passing off action inter alia restrains the defendant from using the name or trademark of the
complainant so as to cease the respondent from passing off the goods or services to the
general public as that of the complainant. It is used to safeguard the goodwill of the
complainant and protect the general public from such deceitful activities. The applicability of
this principle can be further understood through various landmark cases adjudicated by the
Indian courts in this matter:

\noi
The first case in India pertaining to Cyber-squatting was \textit{Yahoo! Inc. v Akash Arora \& Anr.}\footnote{Yahoo! Inc. v. Akash Arora \& Anr., [1999] IIAD Delhi 229}
in the year 1999. The plaintiff was the owner of the well-known mark “Yahoo!” and also of
the domain name “Yahoo.com.” The defendants however registered a confusingly similar or
identical domain name “YahooIndia.com” that too with similar format and colour scheme
and provided similar services like that of the plaintiff. The Delhi High Court applying the law
of passing restrained the defendant from using the said domain name. Ruling in favour of the
plaintiff the court reasoned that the defendant’s domain name was deceptively similar to
confuse the general public and more of an effort to take undue advantage of the reputation of
Yahoo Inc. 

\noi
\textit{Rediff Communication v Cyberbooth \& Anr.}\footnote{Rediff Communication v Cyberbooth \& Anr., [2000] AIR Bombay 27} was another early case relating to cybersquatting decided by the Bombay High Court. The respondents had registered a domain name
“radiff.com” being similar to the plaintiff’s domain name “rediff.com”. The court decided in
favour of the plaintiff as the defendant’s domain name could create confusion between the
distinctiveness of the two parties. In this case the court further held that domain names are an
important and highly valued asset of the company and need to be adequately protected.

\noi
Similarly, in the case of \textit{Acqua Minerals Ltd. v Mr. Pramod Borse \& Anr.,}\footnote{Acqua Minerals Ltd. v Mr. Pramod Borse \& Anr., [2001] AIR Delhi 463} the plaintiff was
the owner of the trademark “Bisleri” in India. The defendant subsequently registered the
domain name “bisleri.com” which was found to be an infringement of the trademark of the
plaintiff as it was deceptively similar to the plaintiff’s brand. The Court ordered the defendant
to transfer the domain name to the plaintiff.

\noi
\textit{Satyam Infoway Ltd. v Sifynet Solutions}\footnote{Trademark Act, 1999, § 103.} was the first case relating to cyber-squatting decided by the Supreme Court. The plaintiff was the registered owner of the word “Sifynet” which was developed using the initials of its corporate name Satyam Infoway and had
goodwill and reputation in the public. The respondent had registered domain names
“Siffynet.com” and “Siffynet.net” which were deceptively similar with plaintiff’s domain
name “Sifynet.com”. The Apex Court set aside the judgment of the High Court and gave its
decision in favour of Satyam Infoway. It stated that the respondent had adopted the said
domain name with a dishonest intention as the marks were found to be deceptively similar.
The Supreme Court in this case held that domain names were regulated under the Trademarks
Act, 1999 as they inculcated all the features of a trademark.

\noi
Even though the Indian Courts have been fairly active in dispensing cases relating to cybersquatting and providing adequate reliefs, it has been observed that with the increasing number
of domain name disputes parties have started using alternate dispute resolution mechanisms
such as arbitration and mediation for resolving cases relating to cyber-squatting. Parties
generally prefer resorting to the UDRP process offered by WIPO and other ICANN approved service providers rather than formal litigation mechanism offered by the Indian courts for
various reasons.

\heading{Recent Developments}

\noi
The world has recently been struck by a pandemic and as businesses are facing huge losses,
the only sector that has boomed is e-commerce. Every company is shifting to the virtual
world and claiming domain names and devising new ways to approach their clientele and
attract more consumers. This has however increased cybersquatting disputes by manifold
numbers.

\noi
In recent times, in countries like China, various cyber-squatting suits have been filed wherein
giant companies such as “Pinterest” have been targeted and revenues were made using the
advertisements.\footnote{Dana Kerr, \textit{Pinterest wins \$7.2M in legal battle with cybersquatter,} (CNET, 30 Sept 2013) \url{https://www.cnet.com/news/pinterest-wins-7-2m-in-legal-battle-with-cybersquatter/} ( last visited on 05 September 2020)} The main reason attributed to such a large number of suits have been
recognized to be the absence of stringent laws against cyber-squatting, unlike in countries
like Philippines\footnote{\textit{Philippines: Analysis of the Cybercrime Prevention Act of 2012,} CENTRE FOR LAW AND DEMOCRACY, (Nov. 2012).} and United States\footnote{Anti-cybersquatting Consumer Protection Act (ACPA), 15 U.S.C. § 1125(d).} which have specific laws for cyber-squatting. The
United States passed a legislation to govern such a serious and recurring offence, back in
1999, called the Anti-cybersquatting Consumer Protection Act (ACPA).\footnote{Anti-cybersquatting Consumer Protection Act (ACPA), 15 U.S.C. § 1125(d) (1)} The law allows the
perpetrators to be booked for a civil suit and thereby damages.

\noi
Statistics from the World Intellectual Property Organization (WIPO) demonstrate that in
2017, the number of domain name disputes have shown a growth of 1.3 percent since the
preceding year.\footnote{WIPO, \textit{Cybersquatting Cases Reach New Record in 2017,} Geneva, (Mar. 14, 2018), \url{http://www.wipo.int/pressroom/en/articles/2018/article_0001.html} (last visited on11 September 2020)} In 2018, the maximum disputes were from the United States amounting to a
total of 920.\footnote{\textit{Id.}}

\noi
In India, the Hon’ble Bombay High Court gave an important decision recently in June 2020,
in the case of Hindustan Unilever v Endurance Domain and Ors\footnote{Commercial IP Suit [2019] (L) NO.577.} regarding the responsibilities of domain name registrars (Intermediaries) and their technical capabilities in
such cases of Cyber-squatting. The plaintiff was the registered owner of websites
www.hul.co.in and www.unilever.com. However, the plaintiff observed that some parties had
registered domain names such as “info@hulcare.co.in”, “unilevercare.co.in”,
“unilevercare.org.in” and “unlevercare.co.in” which were deceptively similar to its own
website. The Court ruled in favour of the plaintiff ordering an immediate injunction on the
use of these domain names by the infringing parties.

\noi
HUL in this case has also impeded National Internet Exchange of India (NIXI), Endurance
Domains, GoDaddy and other domain name registrars praying the court for blocking of these
websites and continued suspension of registration of such domain names by these registries.
Justice Gautam Patel discussed at length the role of these intermediaries and relief that can be
granted against them. The court held that such intermediaries cannot be asked to permanently
block domain names and suspend the registration of domain names until they are found to be
infringing the rights of another party and are fraudulent in nature. The court observed that any
such decision of blocking or suspending a domain name cannot be taken by the parties alone
without any judicial finding. It is a significant decision for the future of cyber-squatting cases
in India as it clarified the role and liabilities of the intermediaries in such cases. The said
judgment also puts an end on the growing trend of injunction orders being passed by the
courts against these domain name registrars without considering the technical and legal
liabilities of the intermediaries in providing such relief. The world faced a huge number of
cyber -criminal activities during the unprecedented times of COVID-19 pandemic wherein
the global marketplace shifted online. About 48,000 new cases were seen around the globe by
one service provider, WIPO during the lockdown.\footnote{WIPO, \textit{Cybersquatting Case Filing Surges During COVID-19 Crisis,} [June 2020], \url{https://www.wipo.int/amc/en/news/2020/cybersquatting_covid19.html} ( last visited on 10 September 2020).} Attackers targeted some very famous and
major brands such as Facebook, Apple, Netflix and Amazon wherein the customers were
scammed with the help of cyber-squatting techniques and deceived with reward and re bill
scams.\footnote{Janos Szurdi, Cybersquatting, [September, 2020]  \url{https://unit42.paloaltonetworks.com/cybersquatting/}}

\noi
With the growing intersection between trademark and domain name systems in this digital
age, negative consequences have increased drastically and need to be curtailed. In this segment, authors have attempted to list down some grey areas as well as issues that must be addressed at the earliest.

\vspace{-.2cm}
\begin{enumerate}[label=$\bullet$]
\item In India, currently there is no specific legislation for resolving disputes pertaining to
cyber-squatting. The Trademark Act, 1999 does not include a chapter about “cybersquatting” with stringent damages in case of contravention of the law.

\item One of the main issues that cyberspace faces is domain name grabbing wherein
people buy domain names with the intention to later monetize and sell it at high prices
instead for personal usage. This results in the unauthentic and fraudulent use of the
domain names.

\item Due to the increasing online presence of various establishments, mere registration of
domain names with the Registry is not enough for safeguarding the legitimate rights
of parties, there is a dire need for specific statutory provisions regulating and
penalizing acts like cyber-squatting. A proper procedure of registration would help in
safeguarding as well as tracking the ownership of the domain names which are
already registered as trademarks by another party. These records would also be
beneficial in cases where any domain name is used to commit any cyber-crime such
as phishing or identity theft etc. which is punishable under IT Act, 2000.

\item The main issue that arises is about the jurisdiction since the internet has no boundaries
and the trademark laws of every country are territorial in nature. Cyber-squatting can
be committed from distant places which are outside the purview of the national courts
and it poses the questions as to whether to file the case or where the complainant
resides or the defendant. Secondly, the binding value of the decision of the
registration authority is questionable. This often leads to non-reporting of cases of
defaulters getting away after the crime due to the lack of proper regime of punishing.

\item ICANN is a private organization and the involvement of all the countries is voluntary
making it difficult to regulate such a growing and serious issue. In spite of WIPO
being the service provider agency, there is a need to have a treaty/convention
establishing a holistic international organization making it mandatory to all the UN
members to ratify the same. 
\end{enumerate}

\vspace{-.2cm}

\heading{Conclusion and Suggestions}

\noi
In today’s digital environment, domain names are seen as vital business assets. They play an
important part in building brand value, consumer loyalty and popularity. With the growing
business and commerce activities online, the threat of cyber-attacks has also grown
exponentially. Cyber squatters target and attack the identity of well-known businesses so as
to garner undue rewards by use or sell of these fraudulent websites. Considering the rise in
Cyber-squatting cases in the recent years it is pertinent that strong measures are taken to
counter this global menace as this misrepresentation not only infringes the rights of the
legitimate trademark holders but also creates confusion in the general public.

\noi
In a country like India, where there is a lack of digital literacy there is an urgent need to have
strong and specific Cyber and Intellectual Property laws as the current Trademark Act, 1999
and the Information Technology Act, 2000 are not adequate to prevent the cyber squatters
from causing fraud to the general public and targeted businesses. There are certain
recommendations that the authors would like to suggest in order to curb the situation, which
are as follows:

\begin{enumerate}[label=$\bullet$]
\item In India, Trademark Act, 1999 must include a chapter about “cyber-squatting” with
stringent damages in case of contravention of the law. This would require amendment
in the explanation of Section 2(m) to expressly include “domain name” in the
definition of “mark”.

\item Under Trademark Act, 1999, ambit of penalty in case of infringement of
copyright/trademark law must be widened to include online access to goods and
services as well as public information through a website.\footnote{Trademark Act, 1999, § 103.}

\item It is recommended for India to impose a strict liability with severe penalties in case of
cyber-squatting. This is a great learning from the USA which has established
legislations such as Anti-squatting Consumer Protection Act.\footnote{Anti-cyber squatting Consumer Protection Act (ACPA), 1999}

\item It is advised to resolve jurisdictional issues for better execution of laws and the
binding nature of the decision.

\item To avoid frivolous and wrong claims of domain names, registration must be
cancelled, and such acts done in bad faith must be dealt with utmost strictness.

\item Online mediation and expedited arbitration are a great way to resolve conflicts over
the domain name disputes. This would be governed by rules of the WIPO Arbitration
and Mediation Centre which will have a binding nature on the decision. The same
would be reiterated as part of the application process when the companies/individuals
buy their second level domain name

\item Administrative panels should be set up to regulate the domain name challenges and
administer the allotment of second level domain names which tend to be identical or
closely similar to names which could violate the existing intellectual property rights
and put the legitimate owners at huge losses.\footnote{WIPO, \textit{Cybersquatting Case Filing Surges During COVID-19 Crisis,} [June 2020], \url{https://www.wipo.int/amc/en/news/2020/cybersquatting_covid19.html} (last visited on 10 September 2020).}

\item  All the second level domain names must be published on its registration, much like a
trademark application process. This would ensure transparency and avoid deceit and
disputes.

\item  In case of instances of “identity theft” wherein misuse of famous celebrities and
personalities is done, it is advisable to amend Section 66 of the Information
Technology Act, 2000 and Section 469 of the Indian Penal Code, to execute the
criminal liability created by the act of cyber-squatting. This would facilitate the filing
of FIR against the perpetrators and penalize them for hacking of computer systems or
for unauthorized extraction of data from computer systems as they had forged the
electronic records to harm the reputation of others with a mala fide intent.

\noi
Thus, eliminating the menace of cyber-squatting is the need of the hour not only for
preventing fraudulent acts, but also to promote and protect businesses in this growing age of
digitisation and globalisation.
\end{enumerate}

\end{multicols}
