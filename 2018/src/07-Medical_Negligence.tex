\setcounter{figure}{0}
\setcounter{table}{0}
\setcounter{footnote}{0}

\articletitle{Medical Negligence and Evolving Judicial Activism}
\articleauthor{Laxmish Rai\footnote{Associate Professor, A J Institution of Hospital, Administration Mangalore}}
\lhead[\textit{\textsf{Laxmish Rai}}]{}
\rhead[]{\textit{\textsf{Medical Negligence...}}}

\begin{multicols}{2}

\heading{Introduction}

\noi
Medical negligence has become one of the most serious and debatable issues in the country in the
last few decades. The medical profession is one of the noblest profession, not immune to
negligence which often results in the death of the patient or permanent/partial disablement or any
other unhappiness which has adverse effects on the patient’s health. Out of an estimated 52 lakh
medical injuries in India, 98,000 people lose their lives because of medical negligence every year.
It is a serious issue for the country that 10 people fall victim to medical negligence every minute
and more than 11 people die every hour due to medical error in India. It is not a surprise that even
the smallest error committed by a doctor have a life-altering impact on the patients.

\noi
Medical Jurisprudence deals with legal responsibilities, particularly those arising out of doctorpatient relationships such as Negligence, Rights and Duties of a doctor, Consent, Professional
misconduct, and medical ethics. Medical jurisprudence is applying medical knowledge to the
legal field to provide justice in civil and criminal cases. It provides key legal guidelines which
should be followed by a medical practitioner. As the field grew, it gave immense power to the
medical practitioner as they were now playing a very important role by having an expert opinion
in the cases. The area of medical jurisprudence is very ancient but with the advent of technology
and the reforms being added to the legal system, this branch is always under development.

\noi
The legal system in India follows the common law regime originated in England, which comprises
statutes and precedents, which form part of the law of the land. The Indian judiciary is very much 
active when it comes to sensitive topics like medical malpractice and negligence. Over the years
Indian judiciary has given progressive interpretation to laws on medical negligence and tried to
safeguard the patients along with protecting doctors from vicious claims. Beginning with the
efforts of the judiciary to include medical services within the ambit of the consumer protection
act to providing directions about doctor’s liability and quantum of compensation, the judiciary
has tried to fill all the shortcomings of the legislation. The Indian Judiciary relying on the
constitution of India strives to ensure that every citizen of India gets “complete justice”. Article
142 grants power to the Supreme Court for awarding any decree to do “complete justice”. In the
last few years, Article 142 has become a gigantic part of the Supreme Court which is invoked
several times to decide the case on medical malpractice to do “complete justice”.

\noi
While going through the judgments that have been passed by the Supreme Court under Article
142, one can found that the Court has readily intervened in most of the complex issues related to
the environment, health, and religion where the existing laws were found insufficient for the
current scenario. Sabyasachi Mukharji C. J expressed the view that we must do away with the
‘childish fiction' that law is not made by the judiciary in C. Ravichandran Iyer v. Justice A. M.
Bhattacharjee.\footnote{C. RavichandranIyerv. Justice A.M. Bhattacharjee \& Anr., (1995) 5 SCC 457.} The court further stated that the role of the judge is not only to interpret the law
but also to lay new norms of law and to mould the law to suit the changing socio-economic
scenario to make the ideals enshrined in the Constitution meaningful and a reality. Society expects
active judicial roles which formerly were considered exceptional but now routine. The court also
went onto say that “law does not operate in a vacuum. It is therefore intended to serve a social
purpose and it cannot be interpreted without taking into account the social, economic and political
setting in which it is intended to operate. It is here that the Judge is called upon to perform an
artistic function. He has to inject flesh and blood into the dry skeleton provided by the legislature.

\noi
From the above statement, it is clear that not only constitutional interpretation but also statutes
have to be interpreted with the changing times and it is here that the creative role of the judge
appears, thus the judge clearly contributes to the process of legal development. The courts must
not shy away from discharging their constitutional obligation to protect and enforce the human rights of the citizens and while acting within the bounds of law must always rise to the occasion
as ‘guardians of the constitution’, criticism of judicial activism notwithstanding.

\heading{Doctor – Patient Relationship: An Analysis}

\noi
The relationship between doctors and patients is prima facie a healthy one that does not involve
any frictions because it is normally the patients who select the doctor for their illness based on
the reputation and skills of a doctor. Besides, when treatment is successful, patients are thankful
to the doctor even though they have paid the fee. On the other side, medical professionals too
thankful to their patients for the trust and confidence placed by their patients in them. However,
this relationship has been commercialized during the past few decades and as a result, patients
expect over-the-top treatment for their illnesses from doctors. Patients, being more conscious than
before, now consider any side effects or issues in treatment as negligence on the part of their
doctors. Similarly, doctors, have found alternative sources for their income, do not provide much
attention to their patients and are often accused of showing apathy in the course of treatment.
Both these instances added to the increase in unnecessary medico-legal cases being filed against
doctors. To control such nuisance and discourage litigant mentality, the Supreme Court has laid
down guidelines for the criminal prosecution of medical professionals. These guidelines have
resulted in a downward trend in the filing of false medico-legal cases and reduced the harassment
of doctors.

\noi
Doctor’s prosecution can be initiated for various reasons other than negligence or deficiency of
service. Statutes such as the Transplantation of Human Organs Act \footnote{The Transplantation of Human Organs and Tissues Act, 1994, \$ 18, No. 42, Acts of Parliament, 1994 (India).} provides for liabilities of doctors who carry out illegal transplantation. However, the Supreme Court has clarified that
litigations must not be brought against doctors which are aimed at maligning the reputation of the
doctor. The Court has instructed the Central and State governments to frame essential guidelines
in consultation with the Medical Council of India to safeguard the medical professionals and
prevent malicious litigation. The Court has also held that complaints against medical professionals shall be brought only with some prima facie evidence to support the allegation against them.

\heading{Right to Health: Judicial Approach}

\noi
The right to health of an individual has been cited by the judiciary in various cases. All these
cases have contributed to the development of the medico-legal system in India over the years. In
Parmananda Katara v. Union of India,\footnote{Pt. ParmanandKatara v. Union of India, AIR 1989 SC 2039.} it was held by the Supreme Court of India that medical professionals, whether they are working in the public or private sector, have the obligation to provide medical aid to persons who are sick and injured without insisting on completion of legal formalities or procedure established under the Cr.P.C. It recognized that the State is under an obligation to protect life under the Constitution. The Court noted that this obligation is delegated to those who are responsible to provide treatment for saving lives, which includes medical professionals.

\noi
Further clarity on right to health was given by the Court in Paschim Banga Khet Mazdoor Samity
v. State of W. B, \footnote{Paschim Bang KhetMazdoorsamity v. State of West Bengal, (1996) 4 SCC 37.} wherein the government hospitals cite the non-availability of beds as a reason for not providing treatment, violate Article 21 of the Constitution. In Kirloskar Brothers Limited. v. Employees State Insurance Corporation,\footnote{Kirloskar Brothers Ltd. v. Employees' State Insurance Corporation, (1996) 2 SCC 682.} the Court held that workmen also have the
fundamental right to health. It further expanded the obligation of the state to ensure the right to
health from the State to the employer, making employer responsible to observe the right to health
of their workmen.

\noi
In the area of medico-legal cases, the law, which is not quite developed in comparison to Western
countries, the burden has mostly been on the Courts of the country to lay down guidelines to
regulate the course of such cases. The decisions of the Courts in India play a most important role
in medico-legal cases as they provide better clarity than existing legislation in medical law. The Court’s decisions are based on any existing statute, principles of natural justice and the opinions
of experts appointed by the Courts as amicus curiae.

\heading{Judicial Activism in the Arena of Medical Negligence: A Case Survey}

\heading{Dr. Laxman Balkrishna Joshi vs. Dr. Trimbak Bapu Godbole}

\noi
In this case, \footnote{Dr. LaxmanBalkrishnaJoshi v. Dr. TrimbakBapuGodbole, AIR1969 SC 128.} the respondent son suffered an injury in his left leg. The accused doctor while putting the plaster used manual traction with excessive force with the help of three men, although such traction is never done under morphia alone but done under proper general anaesthesia. This gave a tremendous shock causing the death of the boy. On these facts, the Supreme Court held that the doctor was liable to pay damages to the parents of the boy.

\noi
On appeal filed by the appellant before the Supreme Court, held that when a patient arrives before
such a person for treatment, a duty of care is owed to the patient. The duty of care concerns
deciding whether to undertake the case, what treatment is to be given and how the treatment is to
be administered. The Court held that a breach of even one of the duties could give rise to the
institution of medico-legal proceedings by the patient against the medical practitioner.

\heading{Indian Medical Association vs. V.P. Shantha}

\noi
One of the most important judgments concerning medico-legal cases of India came about in the
year 1995. Indian Medical Association v. V P Shantha\footnote{Indian Medical Association v. V.P. Shantha, AIR 1996 SC 550.} brought the medical profession within the ambit of ‘service’ as defined in the Consumer Protection Act, 1986. It defined the relationship
between patients and medical professionals as contractual. Patients who had sustained injuries
during the course of treatment could now sue doctors in ‘procedure free consumer protection
courts for compensation. The Court held that even though services rendered by medical
practitioners are of a personal nature they cannot be treated as contracts of personal service. They are service contracts, under which a doctor too can be sued in Consumer Protection Courts. A
‘contract for service’ means a contract whereby one party undertakes to render services to another,
in which the service provider is not subjected to a detailed direction and control. A ‘contract of
service’ implies a relationship of master and servant which involves an obligation to obey orders
in the work to be performed as well as the mode and manner of performance. The Consumer
Protection Act will also cover if some people are charged, and some are exempted from charges
because of their inability of affording such services will be treated as a consumer under Section
2 (1) (d) of the Act. The Supreme Court observed that medical practice is a profession than an
occupation and medical professionals provide a service to the patients and thus they are not
immune to the claim from damage on the ground of medical negligence.

\heading{Paschim Bengal Khet Mazdoor Samity \& Ors. vs. State of Bengal}

\noi
The duty of care owed to the patient is not only by the doctor or medical practitioner but also by
the medical institution or hospital where the patient is undergoing treatment, including
Government hospitals.\footnote{Paschim Bang KhetMazdoorSamity v. State of West Bengal.,(1996) 4 SCC 37.} The question addressed by the Court in this case, whether the nonavailability of appropriate facilities for providing treatment to the serious injuries suffered by the
petitioner in various State hospitals would amount to infringement of his fundamental right to
life. The Court held that the right to life upheld by the Constitution under Article 21 imposes
obligations on the State to protect the right to life of all persons. Protection of the right to life
includes the preservation of the life of persons. Therefore, the State must do everything in its
power to provide adequate medical infrastructure to treat patients. In this regard, the Court held
that denial of timely medical treatment amounted to a violation of an individual’s right to life. 

\heading{Suresh Gupta vs. Government of NCT \& Another}

In this case,\footnote{Suresh Gupta v. Government of NCT., (2004) 6 SCC 422.} the appellant a doctor by profession accused under Section 304A of IPC of causing the death of his patient. The surgery performed was for removing the patient nasal deformity. The Magistrate in his order opined that the appellant while conducting the operation for removal of the nasal deformity gave incision in a wrong part and due to that blood seeped into the respiratory passage and because of that the patient collapsed and died. The Supreme Court held that from the
medical opinions adduced by the prosecution the cause of death was stated to be `not introducing
a cuffed endotracheal tube of proper size as to prevent aspiration of blood from the wound in the
respiratory passage.’ The court further held that if this act attributed to the doctor, even if accepted
to be true, can be described as a negligent act as there was a lack of care and precaution. But for
this act of negligence, he was held liable in a civil case and it cannot be described to be so reckless
or grossly negligent as to make him liable in a criminal case. For conviction in a criminal case,
negligence and rashness should be of such a high degree which can be described as totally
apathetic.

\heading{Jacob Mathew VS. State of Punjab \& Anr}

\noi
In Jacob Mathew v State of Punjab \& Anr,\footnote{ Jacob Mathew v. State of Punjab., (2005) 6 SCC 1.} the Supreme Court thoroughly dealt with the law relating to (i) negligence as a tort (ii) negligence as a tort as well as crime, (iii) negligence by
medical professionals, (iv) medical professionals and criminal law, (v) reviewed Indian judicial
precedents on criminal negligence and thereafter reached certain conclusions and framed
guidelines regarding the prosecution of medical professionals. The complainant father who was
admitted to the hospital developed breathing difficulty and called the doctor for a diagnosis. It
took more than 25 minutes for the doctor to arrive. The doctor instructed the provision of oxygen
to the patient through an oxygen mask. The patient, however, continued to experience discomfort
and tried to get up from his bed but was restrained by the staff. It was found that the oxygen
cylinder was empty and before arranging a different oxygen cylinder, the patient died due to his
inability to breathe.

\noi
FIR filed against the doctor accusing of criminal negligence and the doctor approached the High
Court for quashing the FIR but the same was rejected. The appellant then approached the Supreme
Court and argued that his arrest was arbitrary and there was no instance of criminal negligence
on his part in providing treatment to the patient. In its final judgment, the Supreme Court observed
that:

\noi
“A private complaint shall not be entertained unless the complainant has produced prima facie
evidence before the Court in the form of a probable opinion given by another competent doctor
to support the charge of rashness or negligence on the part of the accused doctor. The investigation
officer should obtain an independent and competent medical opinion preferably from a doctor in
government service, who can normally be expected to give an unbiased and impartial opinion.”12\footnote{Id.}

\noi
With the aforesaid observations, the Court ruled that unless the arrest of the medical professional
is necessary to collect evidence or for further investigation or unless the investigating officer
opines that the medical professional will not make himself available for prosecution, arrest of the
medical professional cannot be made. This case demonstrates the procedure need to be followed
in cases where medical professionals are accused of criminal negligence.

\heading{Poonam Verma VS.Ashwin Patel \& ORS.}

\noi
In this case,\footnote{PoonamVermav.Ashwin Patel, (1996) 4 SCC 332.} a registered medical practitioner entitled to practice Homoeopathy only prescribed
an allopathic medicine to the patient. The patient died and the wife of the deceased filed case for
the death of her husband on the ground that the doctor was entitled to practice homoeopathy only.
In an appeal before the Supreme Court, the Court, in its assessment of the facts of the case,
addressed the question of negligence and its manifestations. It observed that “negligence may be
active negligence, collateral negligence, concurrent negligence, continued negligence, gross
negligence, hazardous negligence, criminal negligence, comparative negligence, active and
passive negligence, willful or reckless negligence or Negligence per se.”\footnote{Id. para40.} The Court held that
where a person is guilty of negligence per se, there is no need for any further proof. The judgment
identified that the act of the respondent who was a qualified homoeopathy doctor, practicing and
prescribing allopathic medicine amounts to negligence per se. No further evidence required to be
produced by the appellant to establish the respondent’s negligence.

\heading{V. Kishan Rao VS. Nikhil Super Speciality Hospital \& Another}

\noi
The principle of ‘res ipsa loquitur’ being applied in cases of medical negligence was upheld in V.
KishanRao v. Nikhil Super Speciality Hospital \& Another,\footnote{V. KishanRao v. Nikhil Super Speciality Hospital, (2010) 5 SCC 513.} wherein the appellant got his wife
admitted as she was suffering from fever. When the treatment did not have any effect on the
appellant’s wife, he shifted her to a different hospital, where she died within hours. On appeal
before the Supreme Court, it was observed that the patient was shifted from the respondent
hospital to another hospital in a ‘clinically dead’ condition. The Court made an important note
that no expert evidence was needed to prove medical negligence. The principle of res ipsa loquitur
will operate, which means that the complainant will not have to prove the negligence where the
‘res’ (thing) proves it. Instead, it is for the respondent to prove that he/she had acted reasonably
and taken sufficient care to negate the allegation of negligence.\footnote{Id. para47.}

\heading{Balram Prasad VS. Kunal Saha \& Ors}

\noi
Balram Prasad v. Kunal Saha \& Ors,\footnote{Balram Prasad v. KunalSaha, (2014) 1 SCC 384.} the respondent along with his wife Anuradha Saha, came
from the USA on a visit to their home town. The respondent, a doctor himself, noticed that his
wife had a sore throat and low-grade temperature. Within no time, Anuradha’s condition became
worse and she continued suffering from high fever. On consultation with the opposite party doctor
again, it was found that Anuradha was suffering from Angio-neurotic Oedema with Allergic
Vasculitis. She was administered depomedrol as a treatment for the same. However, Anuradha’s
condition had deteriorated to a point where no treatment could save her, and she died after a few
days.

\noi
The Supreme Court made an important observation that there was an increasing trend of medicolegal cases concerning negligence on the part of doctors, meaning that there was a need for strict
rules in the conduct of doctors and appropriate penalties for negligent treatment. The Court stated
that the compensation, which is the highest amount awarded in a medico-legal case in India,
should act as a “deterrent and a reminder” to those doctors and hospitals who do not take their responsibility towards patients seriously.\footnote{Id. para149.} This is important because it was the first time the Court
awarded compensation as a deterrent to other medical practitioners. The case also saw the first
time when the potential income of the deceased was calculated up to 30 years in deciding the
compensation instead of the normal practice of taking account of 10-18 years. Thus, the Kunal
Saha case continues to be a landmark case in the medico-legal arena as it sets new standards of
determination of compensation for medical negligence.

\heading{Paramanand Katara VS. Union of India \& Ors}

\noi
A report titled ‘Law helps the injured to die’ published by the Hindustan Times told the story of
a hit and run case where the victim was denied treatment by the nearest hospital for the reason
that, they are not authorized to handle medico-legal cases and asked to approach another hospital
situated 20 km away. The petitioner, who came across the article, filed a writ petition before the
Supreme Court. The petition requested the issuance of an order to the Union of India to assure
spontaneous medical aid to those injured in an accident.

\noi
The Supreme Court held that the right to life was predominant and would supersede medical and
legal formalities in the case of medical help during an emergency. There can be no second opinion
that the preservation of human life is of paramount importance. That is so because once life is
lost, the status quo ante cannot be restored as resurrection is beyond the capacity of man. There
are no provisions in the Indian Penal Code, Criminal Procedure Code, Motor Vehicles Act etc.
which prevent doctors from attending seriously injured persons and accident case. Serving
individuals during a medical emergency is the duty of the public as well as the doctors and
legislators. No legislation can block a person’s right to receive medical treatment under Article
21 and no doctor can be subjected to harassment in the name of the protocol. \footnote{Pt. ParmanandKatara v. Union of India, AIR 1989 SC 2039.}

\heading{Samira Kohli VS. Dr. Prabha Manchanda \& Ors}

\noi
In this case,\footnote{Samira Kohli v. Dr.PrabhaManchanda, AIR 2008 SC 138.} the Appellant visited the Respondent clinic as she was suffering from prolonged
menstrual bleeding. Ultrasound was done and thereafter laparoscopy test as directed. Appellant
signatures were taken in all documents including consent for surgery. During the laparoscopy
test, the Appellant fell unconscious. Subsequently, the respondent’s assistant rushed out of the
operation theatre and asked Appellant’s mother to sign the consent form for hysterectomy under
general anaesthesia, and thereby her reproductive organs were removed.

\noi
The Apex Court held that consent given for diagnostic and operative laparoscopy and
“laparotomy if needed” does not amount to consent for a total hysterectomy with bilateral
salpingo-oophorectomy. The appellant was neither a minor nor incapacitated or mentally
challenged. As the patient was a competent adult and of sound mind, there was no question of
someone else giving consent on her behalf. The appellant was temporarily unconscious due to
anaesthesia, and as there was no emergency. The respondent could have waited until the appellant
regained consciousness and gave proper consent. The question of taking the patient's mother's
consent does not arise in the absence of an emergency. Consent given by her mother is not valid
or real consent. The question was not about the correctness of the decision to remove reproductive
organs but failure to obtain consent for removal of the reproductive organs as the surgery was
performed without taking consent amounts to an unauthorized invasion and interference with the
appellant's body. The court believed that it is the duty of the state to safeguard the right to life of
every person. Further, there is no common law in India for consent and Indian courts have to rely
on the Indian Contract Act.

\heading{Analysis of Judgments}

\noi
The medical profession undoubtedly a noble profession and medical professionals occupy
responsible positions in society.\footnote{Meera T, Medicolegal cases: What every doctor should know, 30 J. MED SOC 132, 134 (2016).} However, both civil, as well as criminal legal proceedings, can
be initiated against medical professionals for acting negligently. The above-cited cases have
produced path-breaking judgments and set the standards which doctors, patients, hospitals, 
lawyers and courts must follow during the hearing of medico-legal cases. However, to establish
an appropriate legal regime that addresses medico-legal cases, it is necessary to analyze the
judgments and identify those aspects of the judgments that are truly novel and pioneering. An
analysis of the judgments brings about certain important principles, which are as follows:

\begin{enumerate}
\item All doctors owe a duty of care to their patients. Hence, doctors shall be held liable for
negligence when there is a breach of duty of care.

\item Negligence is a subjective issue and should be assessed on a case to case basis. To prove
negligence, it must be shown that a medical professional, who is expected to be working skillfully
in providing his medical treatment and owes a duty of care to persons who depend on his/her
skills, causes loss and suffering by exercising his skill without reasonable care.

\item Though the medical profession is a skilled profession and involves a great amount of risk, the
standard of care is generally higher and should be taken into consideration in medico-legal cases.

\item Negligence can arise not only from positive acts of providing incorrect treatment to patients
but also from negative acts such as not maintaining the patient’s case file, not informing the
patient about consequences of risky medical procedures and not entertaining the patient’s request
to receive a second opinion.\footnote{Malay Kumar Gangulyv.Sukumar Mukherjee, (2009) 9 SCC 219.}

\item Where a doctor provides free medical treatment to all patients, his/her treatment cannot be
classified as ‘service’ as defined under the Consumer Protection Act. However, where a doctor
provides free service to a certain class of patients but charges other patients, such doctors shall be
classified as ‘service’ as defined under the Act.

\item Misrepresentation by doctors regarding their qualification in a particular field of medicine also
attracts the charge of negligence along with other criminal charges.

\item A medical professional cannot be held liable where he/she has performed his/her duty with
utmost care taking all necessary precaution regardless of the outcome of the treatment.

\item Medical professionals should not be unnecessarily harassed or subjected to unwarranted
treatment and threats of criminal prosecution unless necessary. The opinion must be sought by
the investigating officer from a doctor working in a government hospital and unless there is a
possibility that the accused will not turn up during prosecution, he/she should not be arrested.

\item As per the Latin maxim “qui facit per alium facit per se”, meaning that anyone who acts through
another does the act himself, hospitals, nursing homes and even the State can be held vicariously
liable for the negligent acts of doctors employed by them.

\item Criminal negligence requires a higher standard of negligence on part of medical professionals
in medico-legal cases. The medical professional should have acted with ‘gross negligence or
‘recklessness’ to such an extent that his behaviour can be considered a threat to society.

\item Where the facts of the case demonstrate that there was negligence per se on behalf of the
doctor, no further evidence was required to be produced to prove the medical professional’s
negligence.

\item The burden of proof generally lies on the complainant. However, where the maxim ‘res ipsa
loquitur’ is applicable, the burden of proof shifts on the opposite party to demonstrate that there
was no negligence.

\item Compensation awarded in medico-legal cases can not only be ordinary in nature but also
exemplary to act as a deterrent or reminder to medical professionals to take their profession
seriously.
\end{enumerate}

\noi
The aforesaid principles evolved from the judgments passed by the Supreme Court of India prove
that the medico-legal regime of laws has been developed to a great extent by the judiciary. This
has mostly been due to the lack of attention that the medico-legal regime has received from the
legislature. The Courts have often been forced to establish a law about medico-legal cases.

\heading{Conclusion}

\noi
In conclusion, the decisions provide a detailed account of the method in which the law established
by Courts in medico-legal cases has evolved over the years. The judiciary in India has been the
pioneer in the establishment of medico-legal law and procedure. This is demonstrated by the judgments in the various cases analyzed in this article, all of which address different aspects and
issues which arise in medico-legal cases. The Courts have considered aspects such as ‘res ipsa
loquitur’, criminal negligence, negligence per se, vicarious liability and exemplary damages in
several cases and provided sufficient clarity on their applicability in medico-legal cases. The
judgments of the Courts have not been altered by Acts of Parliament and have continued to be
applicable.

\noi
The analysis of the judgments provides insight into the vital aspects of medical and procedural
law that have been considered by Courts while deciding on the outcome of medico-legal cases.
This is followed by a scrutiny of the judgments which outlines the limitations that persist despite
the revolutionary decisions made by Courts. It is important to note that several issues have not
been resolved by Courts and many others have been created due to inconsistency between
judgments passed by Courts. Therefore, the ideal remedy to establish a uniform medico-legal
regime would be through an Act of Parliament which taken account of the various important
judgments passed by Courts as well as the limitations that are prevalent in the current medicolegal regime.
\end{multicols}
