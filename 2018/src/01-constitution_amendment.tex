\setcounter{figure}{0}
\setcounter{table}{0}

\articletitle{Constitution Amendment – An Analysis of Amendment Process}
\articleauthor{Dr. B.S .Reddy\footnote{Professor and Principal, R.L Law College, Davanagere, Former Registrar, Karnataka State Law University, Hubli}}
\lhead[\textit{\textsf{Dr. B.S .Reddy}}]{}
\rhead[]{\textit{\textsf{Constitution Amendment – An Analysis...}}}

\begin{multicols}{2}

\heading{Abstract}

\noi
Our Constitution is a dynamic document. Although this Constitution is as strong and enduring
as we want it to be, there is no longevity. What we can do today might not be entirely
applicable tomorrow. Government pattern must change, and the constitution must adapt itself
to the economic and social development of the nation. The proposed constitution abolished
complex and daunting processes such as a convention or referendum decision. Amendment
powers are left to the central and provincial legislature. It is the approval of the state
legislatures that are needed for modifications to particular matters and there are very few. The
other clauses of the Constitution are left to the Parliament to amend. The main restriction is
that it is made by a vote of not less than two-thirds of the members present and voting in each
House and by a vote of the overall membership of each house. The world is not static; it goes
on changing. The social, economic and political circumstances of the people go on changing
and the constitutional law of the nation must, therefore, adapt in order to the changing needs,
changing the lives of the people. If no arrangements were made for modification of the
constitution, the people would have recourse to extra-constitutional processes including
insurrection to reform the constitution. The Indian constitution’s framers were keen to create
a text that could evolve with a rising population, adapting itself to a rising people’s shifting
circumstances. The Constitution needs to be updated in every period. No-one may say this is
the finish.

\heading{Introduction}

\noi
The Constitution of a country is the fundamental law of the land— the basis on which all other
laws are made and enforced. It has been described as a “superior or supreme law”\footnote{K.C. Wheare: Modern Constitutions, London, 1951, p. 91; Also see Haward Lee Mc. B. in: The Living Constitution, New York, 1948, pp. 7-10.}
 with “perhaps greater efficiency and authority”, and “higher sanctity”,\footnote{J. Quick and B.R. Garran: The Annotated Constitution of the Australian Commonwealth, Sydney, 1991, p. 316.} and more permanence than
ordinary legislation. Nevertheless, an adequate provision of its amendment is considered
implicit in the very nature of a constitution. A democratic Constitution has to be particularly
responsive to changing conditions, since a government founded on the principle of popular
sovereignty, “must make possible the fresh assertion of the popular will as that will change”\footnote{Encyclopaedia of Social Sciences, New York, 1951, Vol. II, p. 21.}

\heading{Rigid or Flexible Constitution}

\noi
Constitutions are usually classified as ‘flexible’ or ‘rigid’ depending upon the process through
which they can be amended. Prof. A.V. Dicey defines two types of Constitutions—the flexible
as ‘one under which every law of every description can legally be changed with the same ease
and in the same manner by one and the same body’, and the rigid Constitutions as ‘one under
which certain laws generally known as constitutional or fundamental laws, cannot be changed
in the same manner as ordinary laws’.\footnote{A.V. Dicey: Introduction to the Study of the Law of the Constitution, London, 1952, p. 127.}

\noi
The United Kingdom having an unwritten Constitution, is the best example of an extremely
flexible Constitution as there is no distinction between the legislative power and the
constituent power. The British Parliament has the power to change the Constitution by the
ordinary process of legislation. As opposed to the U.K. system, the constitutional amendment
has an important place under a written Constitution like that of the U.S.A. Its importance
increases where the system is Federal. In most of the written Constitutions, the power to
amend the Constitutions is either vested in a body other than the ordinary Legislature or it is
vested in the ordinary Legislature, subject to a special procedure. In a federal system,
additional safeguards like the involvement of Legislatures at the State level, are also provided
for with a view to ensure that the Federal set-up does not get altered only at the will of the
Federal Legislature.

\heading{Need for Flexibility in the Constitution}

\noi
Explaining why it was necessary to introduce an element of flexibility in the Constitution,
Pandit Jawaharlal Nehru observed in the Constituent Assembly:

\noi
While we want this Constitution to be as solid and as permanent a structure as we can make
it, nevertheless there is no permanence in Constitutions. There should be a certain flexibility.
If you make anything rigid and permanent, you stop a nation’s growth, the growth of a living,
vital, organic people. Therefore, it has to be flexible.…

\noi
In any event, we should not make a constitution, such as some other great countries have,
which are so rigid that they do not and cannot be adapted easily to changing conditions. Today
especially, when the world is in turmoil and we are passing through a very swift period of
transition, what we may do today may not be wholly applicable tomorrow. Therefore, while
we make a constitution which is sound and as basic as we can, it should also be flexible….\footnote{C.A. Deb., Vol. VII, 8 November 1948, pp. 322-323.}

\heading{Constituent Assembly and the Constitution Amendment in India}

\noi
The makers of the Indian Constitution were neither in favour of the traditional theory of
Federalism, which entrusts the task of constitutional amendment to a body other than the
Legislature, nor in prescribing a rigid special procedure for such amendments. Similarly, they
never wanted to have an arrangement like the British set-up where the Parliament is supreme
and can do everything that is humanly possible. Adopting the combination of the ‘theory of
fundamental law’, which underlies the written Constitution of the United States with the
‘theory of parliamentary sovereignty’ as existing in the United Kingdom, the Constitution of
India vests constituent power upon the Parliament subject to the special procedure laid down
therein.

\noi
During the discussion in the Constituent Assembly on this aspect, some of the members were
in favour of adopting an easier mode of amending procedure for the initial five to ten years.
Dr. P.S. Deshmukh was of the view that the amendment of the Constitution should be made
easier as there were contradictory provisions in some places which would be more and more
apparent when the provisions are interpreted. If the amendment to the Constitution was not
made easy, the whole administration would suffer. Shri Brajeshwar Prasad was also in favour of a flexible Constitution so as to make it survive the test of time. He was of the opinion that
rigidity tends to check progressive legislation or gradual innovation.

\noi
On the other hand, Shri H.V. Kamath was in favour of providing for procedural safeguards to
avoid the possibility of hasty amendment to the Constitution.\footnote{Ibid., Vol. IX, 17 September 1949, pp. 1644-1667}

\noi
Dr. B.R. Ambedkar, speaking in the Constituent Assembly on 4 November 1948, made certain
observations in connection with the provisions relating to amendment of the Constitution. He
said:

\noi
It is said that the provisions contained in the Draft make amendment difficult. It is proposed
that the Constitution should be amendable by a simple majority at least for some years. The
argument is subtle and ingenious. It is said that this Constituent Assembly is not elected on
adult suffrage while the future Parliament will be elected on adult suffrage and yet the former
has been given the right to pass the Constitution by a simple majority while the latter has been
denied the same right. It is paraded as one of the absurdities of the Draft Constitution. I must
repudiate the charge because it is without foundation. To know how simple the provisions of
the Draft Constitution in respect are of amending the Constitution one has only to study the
provisions for amendment contained in the American and Australian Constitutions. Compared
to them those contained in the Draft Constitution will be found to be the simplest. The Draft
Constitution has eliminated the elaborate and difficult procedures such as a decision by a
convention or a referendum….

\noi
 It is only for amendments of specific matters—and they are only few—that the ratification of
the State Legislatures is required. All other Articles of the Constitution are left to be amended
by Parliament. The only limitation is that it shall be done by a majority of not less than twothirds of the members of each House present and voting and a majority of the total membership
of each House. It is difficult to conceive a simpler method of amending the Constitution.

\noi
What is said to be the absurdity of the amending provisions is founded upon a misconception
of the position of the Constituent Assembly and of the future Parliament elected under the
Constitution. The Constituent Assembly in making a constitution has no partisan motive.
Beyond securing a good and workable Constitution it has no axe to grind. In considering the
Articles of the Constitution it has no eye on getting through a particular measure. The future Parliament if it met as Constituent Assembly, its members will be acting as partisans seeking
to carry amendments to the Constitution to facilitate the passing of party measures which they
have failed to get through Parliament by reason of some Article of the Constitution which has
acted as an obstacle in their way. Parliament will have an axe to grind while the Constituent
Assembly has none. That is the difference between the Constituent Assembly and the future
Parliament. That explains why the Constituent Assembly though elected on limited franchise
can be trusted to pass the Constitution by simple majority and why the Parliament though
elected on adult suffrage cannot be trusted with the same power to amend it.\footnote{Ibid, Vol. VII, 4 November 1948, pp. 43-44}

\heading{Procedure for Constitution Amendment in\\ India}

\noi
The Constitution of India provides for a distinctive amending process as compared to the
leading Constitutions of the world. It may be described as partly flexible and partly rigid. The
Constitution of India provides for a variety in the amending process—a feature which has
been commended by Prof. K.C. Wheare for the reason that uniformity in the amending process
imposes “quite unnecessary restrictions” upon the amendment of parts of a Constitution.\footnote{Prof. Wheare: op. cit., p. 143}

\noi
The Constitution of India provides for three categories of amendments.\footnote{Shankari Prasad vs. Union of India, A.I.R. 1951 S.C. 455.} Firstly, those that can be effected by Parliament by a simple majority such as that required for the passing of any
ordinary law—the amendments contemplated in articles 4,\footnote{Article 4 provides that laws made by Parliament under article 2 (relating to admission or establishment of new States) and article 3 (relating to formation of new States and alteration of areas, boundaries or names of existing States) effecting amendments in the First Schedule or the Fourth Schedule and supplemental, incidental and consequential matters, shall not be deemed to be amendments of the Constitution for the purposes of article 368. Thus, for example, the States Reorganisation Act, 1956, which brought about reorganisation of the States in India,
was passed by Parliament as an ordinary piece of legislation. It has been held that power to reduce the total number
of members of Legislative Assembly below the minimum prescribed under article 170 (1) is implicit in the
authority to make laws under article 4 (Mangal Singh vs. Union of India, A.I.R. 1967 S.C. 944)} 169,\footnote{Article 169 empowers Parliament to provide by law for the abolition or creation of the Legislative Councils in
States and specifies that though such law shall contain such provisions for the amendment of the Constitution as
may be necessary, it shall not be deemed to be an amendment of the Constitution for the purposes of article 368.
The Legislative Councils Act, 1957 is an example of a law passed by Parliament in exercise of its powers under
article 169. The Act provided for the creation of a Legislative Council in Andhra Pradesh and for increasing the
strength of the Legislative Councils in certain other States} para 7(2)\footnote{The Fifth Schedule contains provisions as to the administration and control of the Schedule Areas and Scheduled Tribes. Para 7 of the Schedule vests Parliament with plenary powers to enact laws amending the Schedule and lays down that no such law shall be deemed to be an amendment of the Constitution for the purposes of article 368.} of Schedule V and para 21(2)\footnote{Under Para 21 (Sixth Schedule), Parliament has full power to enact laws amending the Sixth Schedule which contains provisions for the administration of Tribal Areas in the States of Assam, Meghalaya, Tripura and Mizoram. No such law, however, is to be deemed to be an amendment of the Constitution for the purposes of article 368.} of Schedule VI fall within this category and are specifically excluded from the purview of article 368 which is the specific provision in the Constitution dealing with the
power and the procedure for the amendment of the Constitution; Secondly, those amendments
that can be effected by Parliament by a prescribed ‘special majority’; and Thirdly, those that
require, in addition to such ‘special majority’, ratification by at least one half of the State
Legislatures. The last two categories being governed by article 368.

\noi
In this connection, it may also be mentioned that there are, as pointed out by Dr. Ambedkar,
“innumerable articles in the Constitution” which leave the matter subject to law made by
Parliament.\footnote{C.A. Deb., Vol. IX, 17 September 1949, p. 1660} For example, under article 11, Parliament may make any provision relating to
citizenship notwithstanding anything in article 5 to 10.\footnote{Other examples include Part XXI of the Constitution—“Temporary, Transitional and Special Provisions” whereby “Notwithstanding anything in this Constitution” power is given to Parliament to make laws with respect to certain matters included in the State List (article 369); article 370 (1) (d) which empowers the President to modify, by order, provisions of the Constitution in their application to the State of Jammu and Kashmir; provisos to articles 83 (2) and 172 (1) empower Parliament to extend the lives of the House of the People and the Legislative
Assembly of every State beyond a period of five years during the operation of a Proclamation of Emergency; and
articles 83(1) and 172 (2) provide that the Council of States/Legislative Council of a State shall not be subject to
dissolution but as nearly as possible one-third of the members thereof shall retire as soon as may be on the
expiration of every second year in accordance with the provisions made in that behalf by Parliament by law.} Thus, by passing ordinary laws,
Parliament may, in effect, provide, modify or annul the operation of certain provisions of the
Constitution without actually amending them within the meaning of article 368. Since such
laws do not in fact make any change whatsoever in the letter of the Constitution, they cannot
be regarded as amendments of the Constitution nor categorised as such.

\noi
In so far as the constituent power to make formal amendments is concerned, it is article 368
of the Constitution of India which empowers Parliament to amend the Constitution by way of
addition, variation or repeal of any provision according to the procedure laid down therein,
which is different from the procedure for ordinary legislation. Article 368, which has been
amended by the Constitution (Twenty-fourth Amendment), Act, 1971\footnote{Before its amendment by the 24th Amendment Act and 42nd Amendment Act, article 368 stood as follows: Art 368, Procedure for amendment of the Constitution: An amendment of the Constitution may be initiated only by the introduction of a Bill for the purpose in either House of Parliament and when the Bill is passed in each House by a majority of the total membership of that House and by a majority of not less than two thirds of the members of that House present and voting, it shall be presented to the President for his assent and upon such assent being
given to the Bill, the Constitution shall stand amended in accordance with the terms of the Bill: Provided that if
such amendment seeks to make any change in: (a) article 54, article 55, article 73, article 162, or article 241, or
(b) Chapter IV of Part V, Chapter V of Part VI, or Chapter I of the Part XI, or (c) Any of the Lists in the Seventh
Schedule, or (d) The representation of States in Parliament, or (e) The provisions of this article, the amendment shall also require to be ratified by the Legislature of not less than one-half of the States by resolution to that effect
passed by these Legislatures before the Bill making provision for such amendment is presented to the President
for assent.} and the Constitution (Forty-second Amendment) Act, 1976, reads as follows:

\noi
368 : Power of Parliament to amend the Constitution and Procedure therefor:

\begin{enumerate}
\item Notwithstanding anything in this Constitution, Parliament may in exercise of its constituent
power amend by way of addition, variation or repeal any provision of this Constitution in
accordance with the procedure laid down in this article.

\item  An amendment of this Constitution may be initiated only by the introduction of a Bill for
the purpose in either House of Parliament, and when the Bill is passed in each House by a
majority of the total membership of that House and by a majority of not less than two-thirds
of the members of that House present and voting, it shall be presented to the President who
shall give his assent to the Bill and thereupon the Constitution shall stand amended in
accordance with the terms of the Bill:

\noi
Provided that if such amendment seeks to make any change in:

\begin{enumerate}[label=\arabic*.]
\item Article 54, article 55, article 73, article 162 or article 241, or

\item Chapter IV of Part V, Chapter V of Part VI, or Chapter I of Part XI, or

\item Any of the lists in the Seventh Schedule, or

\item The representation of States in Parliament, or

\item The provisions of this article,

the amendment shall also require to be ratified by the Legislatures of not less than one-half of
the States\footnote{The words and letters “specified in Part A and B of the First Schedule” were omitted by the Constitution
(Seventh Amendment) Act, 1956, s. 29 and Schedule.}… by resolutions to that effect passed by those Legislatures before the Bill making
provision for such amendment is presented to the President for assent.
\end{enumerate}

\item Nothing in article 13\footnote{Clause 3 was inserted by the Constitution (Twenty-fourth Amendment) Act, 1971 which also added a new clause (4) in article 13 which reads, “Nothing in this article shall apply to any amendment of this Constitution
made under article 368”} shall apply to amendment made under this article.

\item No amendment of this Constitution (including the provisions of Part III) made or purporting
to have been made under this article [whether before or after the commencement of section
55 of the Constitution (Forty-second Amendment) Act, 1976] shall be called in question in
any court on any ground.

\item For the removal of doubts, it is hereby declared that there shall be no limitation whatever
on the constituent power of Parliament to amend by way of addition, variation or repeal the
provisions of this Constitution under this article.
\end{enumerate}

\vspace{-.2cm}

\noi
An analysis of the procedure prescribed by article 368 for amendment of the Constitution
shows that: 1. an amendment can be initiated only by the introduction of a Bill in either House
of Parliament. 2. The Bill so initiated must be passed in each House by a majority of the total
membership\footnote{Total membership in this context has been defined to mean the total number of members comprising the House irrespective of any vacancies or absentees on any account vide Explanation to Rule 159 of the Rules of Procedure
and Conduct of Business in Lok Sabha.} of that House and by a majority of not less than two-thirds of the members of
that House present and voting.\footnote{“Abstentions” in any voting are not taken into consideration in declaring the result of any question. A member who votes “abstention” either through the electronic vote recorder or on a voting slip or in any manner, does so only to indicate his presence in the House and his intention to abstain from voting; he does not record his vote
within the meaning of the words “present and voting”. The expression, “present and voting” refers to those who
vote for “ayes” and for “noes” Lok Sabha Rules Committee Minutes, dated 8-9 September 1970, (Practice and
Procedure of Parliament, 2001, by M.N. Kaul and S.L. Shakdher, p. 604).} There is no provision for a joint sitting in case of
disagreement between the two Houses; 3. when the Bill is so passed, it must be presented to
the President who shall give his assent to the Bill; 4. where the amendment seeks to make any
change in any of the provisions\footnote{These provisions relate to certain matters concerning the federal structure or of common interest to both the Union and the States viz., (a) the election of the President (articles 54 and 55); (b) extent of the executive power of the Union and the States (articles 73 and 162); (c) High Courts for Union territories (article 241); (d) The Union
Judiciary and the High Courts in the States (Chapter IV of Part V and Chapter V of Part VI); (e) distribution of
legislative powers between the Union and the States (Chapter I of Part XI and Seventh Schedule); (f)
representation of States in Parliament; and (g) the provision for amendment of the Constitution laid down in article
368.} mentioned in the proviso to article 368, it must be ratified\footnote{The Constitution (Third Amendment) Act, 1954; the Constitution (Sixth Amendment) Act, 1956; the
Constitution (Seventh Amendment) Act, 1956; the Constitution (Eighth Amendment) Act, 1960; the Constitution
(Thirteenth Amendment) Act, 1962; the Constitution (Fourteenth Amendment) Act, 1962; the Constitution
(Fifteenth Amendment) Act, 1963; the Constitution (Sixteenth Amendment) Act, 1963; the Constitution (Twentysecond Amendment) Act, 1969; the Constitution (Twenty-third Amendment) Act, 1969; the Constitution
(Twentyfourth Amendment) Act, 1971; the Constitution (Twenty-fifth Amendment) Act, 1971; the Constitution
(Twenty-eighth Amendment) Act, 1972; the Constitution (Thirtieth Amendment) Act, 1972; the Constitution
(Thirty-first Amendment) Act, 1973; the Constitution (Thirty-second Amendment) Act, 1973; the Constitution
(Thirty-fifth Amendment) Act, 1974; the Constitution (Thirty-sixth Amendment) Act, 1975; the Constitution
(Thirty-eighth Amendment) Act, 1975; the Constitution (Thirty-ninth Amendment) Act, 1975; the Constitution
(Fortysecond Amendment) Act, 1976; the Constitution (Forty-third Amendment) Act, 1977; the Constitution
(Forty-fourth Amendment) Act, 1978; the Constitution (Forty-fifth Amendment) Act, 1980; the Constitution
(Forty-sixth Amendment) Act, 1982; the Constitution (Fifty-first Amendment) Act, 1984; the Constitution (Fifty-fourth Amendment) Act, 1986; the Constitution (Sixty-first Amendment) Act, 1988; the Constitution (Sixtysecond Amendment) Act, 1989; the Constitution (Seventieth Amendment) Act, 1992; the Constitution
(Seventythird Amendment) Act, 1992; the Constitution (Seventy-fourth Amendment) Act, 1992; the Constitution
(Seventy-fifth Amendment) Act, 1994; the Constitution (Seventy-ninth Amendment) Act, 1999; the Constitution
(Eightyfourth) Act, 2001; the Constitution (Eighty-eighth Amendment) Act, 2003 were thus all ratified by the
State Legislatures after they were passed by both Houses of Parliament before they were presented to the President
for assent.} by the Legislatures of not less than one-half of the States; 5. such ratification is to be by
resolution passed by the State Legislatures; 6. no specific time limit for the ratification of an
amending Bill by the State Legislatures is laid down; the resolutions ratifying the proposed
amendment should, however, be passed before the amending Bill is presented to the President
for his assent;\footnote{With regard to the corresponding provision in the U.S. Constitution viz. Article V which also does not prescribe any time limit for ratification, the U.S. Supreme Court has held that the ratification must be within a reasonable
time after the proposal (Dilllon vs. Gloss 65, Law Ed. 9945) but that the Court has no power to determine what is
a reasonable time (Coleman vs. Miller, 83, Law Ed. 1385). It has further held that the question of efficacy of
ratifications by State Legislatures, in the light of previous rejection or attempted withdrawal, should be regarded
as a political question pertaining to the political departments, with the ultimate authority in the Congress in the
exercise of its control over the promulgation of the adoption of amendment (Coleman vs. Miller, 83, Law Ed.
1385)} 7. the Constitution can be amended:

\vspace{-.3cm}

\begin{enumerate}[label=(\arabic*)]
\itemsep=0pt
\item only by Parliament; and

\item  in the manner provided.

\end{enumerate}

\vspace{-.3cm}

\noi
Any attempt to amend the Constitution by a Legislature other than Parliament and in a manner
different from that provided for will be void and inoperative.\footnote{Abdul Rahiman Jamaluddin vs. Vithal Arjun, A.I.R. 1958 Bombay, 94.}

\noi
Whether the entire Constitution Amendment is void for want of ratification or only an
amended provision required to be ratified under proviso to clause (2) of article 368, is a very
significant point. In a case decided in 1992, this issue was debated before the Supreme Court
in what is now popularly known as Anti-Defection case,\footnote{Kihota Hollohon vs. Zachilhu and others, (1992) 1 S.C.C. 309.} in which the constitutional validity of the Tenth Schedule of the Constitution inserted by the Constitution (Fifty-second
Amendment) Act, 1985 was challenged. In this case, the decisions of the Speakers/Chairmen
on disqualification, which had been challenged in different High Courts through different
petitions, were heard by a five-member Constitution Bench of the Supreme Court. The
Constitution Bench in its majority judgement upheld the validity of the Tenth Schedule but
declared Paragraph 7 of the Schedule invalid because it was not ratified by the required
number of the Legislatures of the States as it brought about in terms and effect, a change in
articles 136, 226 and 227 of the Constitution. While doing so, the majority treated Paragraph 7 as a severable part from the rest of the Schedule. However, the minority of the Judges held
that the entire Constitution Amendment Act is invalid for want of ratification.

\heading{Legislative Procedure and Constitution Amendment}

\noi
Article 368 is not a “complete code” in respect of the legislative procedure to be followed at
various stages. There are gaps in the procedure as to how and after what notice a Bill is to be
introduced, how it is to be passed by each House and how the President’s assent is to be
obtained.\footnote{Shankari Prasad Singh Deo vs. Union of India, A.I.R. 1951 S.C. 458.} This point was decided by the Supreme Court in the Shankari Prasad’s case. Delivering the judgment of the Court, Patanjali Sastri J. observed:\footnote{Ibid.} Having provided for the constitution of a Parliament and prescribed a certain procedure for the conduct of its ordinary
legislative business to be supplemented by rules made by each House (article 118), the makers
of the Constitution must be taken to have intended Parliament to follow that procedure, so far
as it may be applicable consistently with the express provisions of article 368, when they
entrusted to it power of amending the Constitution.

\noi
Hence, barring the requirements of special majority, ratification by the State Legislatures in
certain cases, and the mandatory assent by the President, a Bill for amending the Constitution
is dealt with the Parliament following the same legislative process as applicable to an ordinary
piece of legislation.

\noi
In Lok Sabha, the Rules of Procedure and Conduct of Business make certain specific
provisions with regard to Bills for amendment of the Constitution. They relate to:

\vspace{-.25cm}

\begin{enumerate}[label=$(\alph*)$]
\itemsep=0pt
\item the voting procedure in the House at various stages of such Bills, in the light of the
requirements of article 368; and

\item the procedure before introduction in the case of such Bills, if sponsored by Private
Members.
\end{enumerate}

\vspace{-.25cm}

\noi
Although the ‘special majority’, insisted upon the article 368 is prima facie applicable only to
the voting at the final stage, the Lok Sabha Rules prescribed adherence to this constitutional
requirement at all the effective stages of the Bill, i.e., for adoption of the motion that the Bill
be taken into consideration; that the Bill as reported by the Select/Joint Committee be taken 
into consideration, in case a Bill has been referred to a Committee; for adoption of each clause
or schedule or clause or schedule as amended, of a Bill; or that the Bill or the Bill as amended,
as the case may be, be passed.\footnote{Rules 155 and 157, Rules of Procedure and Conduct of Business in Lok Sabha (Eleventh Ed.) Lok Sabha Secretariat, New Delhi, 2004, p. 67.} This provision, which represents the position arrived at after
consultation with the Attorney-General and detailed discussions in the Rules Committee, is
evidently ex-abundanti cautela. It not only ensures, by a strict adherence to article 368, the
validity of the procedure adopted, but also guards against the possibility of violation of the
spirit and scheme of that article\footnote{Second Report of the Rules Committee, April 1956, Lok Sabha Secretariat, New Delhi} by the consideration of a Bill seeking to amend the Constitution including its consideration clause by clause being concluded in the House with only the bare quorum present. Voting at all the above stages is by division.\footnote{Rule 158, Rules of Procedure, op.cit.} The Speaker
may, however, with the concurrence of the House, put any group of clauses or schedules
together to the vote of the House, provided that if any member requests that any of the clauses
or schedules be put separately, the Speaker shall comply to do so.\footnote{1 Rule 155, Ibid.} The Short Title, Enacting
Formula and the Long Title may be adopted by a simple majority.\footnote{Ibid} For the adoption of
amendments to clauses or schedules of the Bill, a majority of members present and voting in
the same manner as in the case of any other Bill, will suffice.\footnote{Rule 156, op. cit.}

\noi
A Bill for amendment of the Constitution by a Private Member is governed by the rules
applicable to Private Members’ Bills in general. So, the period of one month’s notice applies
to such a Bill also. In addition, in Lok Sabha, such a Bill has to be examined and recommended
by the ‘Committee on Private Members’ Bills before it is included in the List of Business.\footnote{Rule 294, op. cit.}
The Committee has laid down the following principles as guiding criteria in making their
recommendations in regard to these Bills:\footnote{First Report of the Committee on Private Members’ Bills, December 1953, Lok Sabha Secretariat, New Delhi.}

\begin{enumerate}
\itemsep=0pt
\item The Constitution should be considered as a sacred document— a document which should
not be lightly interfered with, and it should be amended only when it is found absolutely
necessary to do so. Such amendments may generally be brought forward when it is found that
the interpretation of the various articles and provisions of the Constitution has not been in
accordance with the intention behind such provisions and cases of lacunae or glaring inconsistencies have come to light. Such amendments should, however, normally be brought
by the Government after considering the matter in all its aspects and consulting experts and
taking such other advice as they may deem fit.

\item Sometime should elapse before a proper assessment of the working of the Constitution and
its general effect is made so that any amendments that may be necessary are suggested because
of sufficient experience.

\item Generally speaking, notice of Bills from Private Members should be examined in the
background of the proposal or measures which the Government may be considering at the
time so that consolidated proposals are brought forward before the House by the Government
after collecting sufficient material and taking expert advice.

\item Whenever a Private Member’s Bill raises issues of far-reaching importance and public
interest, the Bill might be allowed to be introduced so that public opinion is ascertained and
gauged to enable the House to consider the matter further. In determining whether a matter is
of sufficient public importance, it should be examined whether the particular provisions in the
Constitution are adequate to satisfy the current ideas and public demand at the time. In other
words, the Constitution should be adapted to the current needs and demands of the progressive
society and any rigidity which may impede progress should be avoided.
\end{enumerate}

\vspace{-.3cm}

\noi
In Rajya Sabha, the Rules of the House do not contain special provisions with regard to Bills
for amendment of the Constitution and the Rules relating to ordinary Bills apply, subject of
course, to the requirements of article 368.

\heading{Scope of Parliament’s Power to Amend the Constitution}

\noi
Until the case of L.C. Golak Nath vs. State of Punjab,\footnote{A.I.R. 1967 S.C. 1643.} the Supreme Court had been holding
that no part of the Constitution was unamendable, and that the Parliament might, by passing a
Constitution Amendment Act in compliance with the requirements of article 368, amend any
provision of the Constitution, including the Fundamental Rights and article 368.\footnote{In Shankari Prasad Singh Deo vs. The Union of India (A.I.R. 1951 S.C. 458), the Supreme Court unanimously held: The terms of article 368 are perfectly general and empower Parliament to amend the Constitution without any exception whatever. In the context of article 13, “law” must be taken to mean rules or regulations made in exercise of ordinary legislative power and not amendments to the Constitution made in exercise of constituent power, with the result that article 13 (2) does not affect amendments made under article 368. In Sajjan Singh vs.
The State of Rajasthan (A.I.R. 1965 S.C. 845), the Supreme Court (by a majority of 3:2) held: When article 368
confers on Parliament the right to amend the Constitution, the power in question can be exercised over all the
provisions of the Constitution. It would be unreasonable to hold that the word “Law” in article 13 (2) takes in
Constitution Amendment Acts passed under article 368.} But in
Golak Nath’s case, the Supreme Court (by a majority of 6:5) reserved its own earlier decisions.

\noi
In Golak Nath’s case, the Court held that an amendment of the Constitution is a legislative
process. A Constitution amendment under article 368 is “law” within the meaning of article
13\footnote{Article 13(2): The State shall not make any law which takes away or abridges the right conferred by this Part
and any law made in contravention of this clause shall, to the extent of contravention, be void.} of the Constitution and therefore, if a constitution amendment “takes away or abridges”
a Fundamental Right conferred by Part III, it is void.

\noi
The Court was also of the opinion that Fundamental Rights included in Part III of the
Constitution are given a transcendental position under the Constitution and are kept beyond
the reach of Parliament. The incapacity of Parliament to modify, restrict or impair
Fundamental Freedoms in Part III arises from the scheme of the Constitution and the nature
of the freedoms.

\noi
As a result of the judgment of the Supreme Court in Golak Nath’s case, the Parliament passed
the Constitution (Twenty-fourth Amendment) Act, 1971. This Act has amended the
Constitution to provide expressly that Parliament has power to amend any part of the
Constitution including the provisions relating to Fundamental Rights. This has been done by
amending articles 13 and 368 to make it clear that the bar in article 13 against abridging or
taking away any of the Fundamental Rights does not apply to Constitution amendment made
under article 368.

\noi
In His Holiness Kesavananda Bharati Sripadagalvaru vs. State of Kerala,\footnote{A.I.R. 1973 S.C. 1461.} the Supreme Court
reviewed the decision in the Golak Nath’s case and went into the validity of the 24$^{\rm th}$, 25$^{\rm th}$, 26$^{\rm th}$
and 29$^{\rm th}$ Constitution Amendments. The case was heard by the largest ever Constitution Bench
of 13 Judges. The Bench gave eleven judgements, which agreed on some points and differed
on others. Nine Judges summed up the ‘Majority View’ of the Court thus:

\vspace{-.3cm}

\begin{enumerate}
\itemsep=0pt
\item Golak Nath’s case is over-ruled.

\item Article 368 does not enable Parliament to alter the basic structure or framework of the Constitution. 

\item The Constitution (Twenty-fourth Amendment) Act, 1971 is valid.

\item Section 2(a) and 2(b) of the Constitution (Twenty-fifth Amendment) Act, 1971 is valid.

\item The first part of section 3 of the Constitution (Twenty-fifth Amendment) Act, 1971 is
valid. The second part namely “and no law containing a declaration that it is for giving effect
to such policy shall be called in question in any court on the ground that it does not give effect
to such policy” is invalid.

\item The Constitution (Twenty-ninth Amendment) Act, 1971 is valid.
\end{enumerate}

\vspace{-.3cm}

\noi
The majority of the Full Bench upheld the validity of the Constitution (Twenty-fourth
Amendment) Act and overruled the decision of the Golak Nath’s case holding that a
Constitution Amendment Act is not “law” within the meaning of article 13. Upholding the
validity of clause (4) of article 13 and a corresponding provision in article 368(3), inserted by
the Twenty-fourth Amendment Act, the Court settled in favour of the view that Parliament
has the power to amend the Fundamental Rights also. However, the Court affirmed another
proposition also asserted in the Golak Nath’s case. The Court held that the expression
‘amendment’ of this Constitution in article 368 means any addition or change in any of the
provisions of the Constitution within the broad contours of the Preamble and the Constitution
to carry out the objectives in the Preamble and the Directive Principles. Applied to
Fundamental Rights, it would be that while Fundamental Rights cannot be abrogated,
reasonable abridgement of Fundamental Rights could be effected in the public interest. The
true position is that every provision of the Constitution can be amended provided the
foundation and structure of the Constitution remains the same.

\vspace{-.1cm}

\noi
The theory of basic structure of the Constitution was reaffirmed and applied by the Supreme
Court in Smt. Indira Nehru Gandhi vs. Raj Narain case\footnote{A.I.R. 1975 S.C. 2299.} and certain amendments to the
Constitution were held void.\footnote{In this case, article 329 A inserted by the Constitution (Thirty-ninth Amendment) Act, 1975, came up for challenge. Article 329A put Prime Minister’s and Lok Sabha Speaker’s election outside the purview of the
Judiciary and provided for determination of disputes concerning their elections by an authority to be set up by a
Parliamentary law. The Supreme Court struck down clauses (4) and (5) of the article 329A which made the
existing election law inapplicable to Prime Minister’s and Speaker’s election and declared the pending
proceedings in respect of such elections null and void.}

\vspace{-.1cm}

\noi
Subsequently, based on the Court’s view in Kesavananda Bharati’s case, upholding the
concept of the basic structure, the Supreme Court in Minerva Mills Ltd. vs. Union of India\footnote{A.I.R. 1980 S.C. 1789.}
declared section 55\footnote{Section 55 of the Constitution (Forty-second Amendment) Act, 1976 inserted sub clauses (4) and (5) in article 368 of the Constitution providing that there shall be no limitation on the constituent power of the Parliament and
that the validity of any Constitution Amendment Act, including those amending the Part III, shall not be called in
question in any court on any ground.} of the Constitution (Forty-second Amendment) Act, 1976 as
unconstitutional and void. It held: Since the Constitution had conferred a limited amending
power on the Parliament, the Parliament cannot under the exercise of that limited power
enlarge that very power into an absolute power. Indeed, a limited amending power is one of
the basic features of our Constitution and, therefore, the limitations on that power cannot be
destroyed. In other words, Parliament cannot, under article 368, expand its amending power
so as to acquire for itself the right to repeal or abrogate the Constitution or to destroy its basic
and essential features. The donee of a limited power cannot by the exercise of that power
convert the limited power into an unlimited one.

\vspace{-.1cm}

\noi
The concept of basic structure has since been developed by the Supreme Court in subsequent
cases, such as Waman Rao case,\footnote{Waman Rao vs. Union of India, A.I.R. 1981 S.C. 271.} Bhim Singhji case,\footnote{Bhim Singhji vs. Union of India, A.I.R. 1981 S.C. 234.} Transfer of Judges case,\footnote{S.P. Gupta vs. President of India, A.I.R. 1982 S.C. 149} S.P. Sampath Kumar’s case,\footnote{S.P. Sampath Kumar vs. Union of India, A.I.R. 1987 S.C. 386.} P. Sambamurthy’s case,\footnote{P. Sambamurthy vs. State of A.P., A.I.R. 1987 S.C. 663.} Kihota Hollohon case,\footnote{Kihota Hollohon vs. Zachilhu and others, (1992) 1 S.C.C. 309.} L. Chandra Kumar case,\footnote{L. Chandra Kumar vs. Union of India and others, A.I.R. 1997 S.C. 1125.} P.V. Narsimha Rao case,\footnote{P.V. Narsimha Rao vs. State (CBI/SPE), A.I.R. 1998 S.C. 2120.} I.R. Coelho case,\footnote{I.R. Coelho vs. State of Tamil Nadu and others, (2007) 2 S.C.C. 1.} and Cash for Query case.\footnote{Raja Ram Pal vs. The Hon’ble Speaker, Lok Sabha and others, JT 2007 (2) S.C. 1.} The basic features of the Constitution are not finite. So far about 20 features\footnote{The basic features of the Constitution have not been explicitly defined by the Judiciary. However, Supremacy
of the Constitution; Rule of law; The principle of Separation of Powers; The objectives specified in the Preamble
to the Constitution; Judicial Review; Articles 32 and 226; Federalism; Secularism; The Sovereign, Democratic,
Republican structure; Freedom and dignity of the individual; Unity and integrity of the Nation; The principle of
equality, not every feature of equality, but the quintessence of equal justice; The ‘essence’ of other Fundamental
Rights in Part III; The concept of social and economic justice—to build a Welfare State: Part IV in toto; The
balance between Fundamental Rights and Directive Principles; The Parliamentary system of government; The
principle of free and fair elections; Limitations upon the amending power conferred by Article 368; Independence
of the Judiciary; Effective access to justice; Powers of the Supreme Court under Articles 32, 136, 141, 142;
Legislation seeking to nullify the awards made in exercise of the judicial power of the State by Arbitration
Tribunals constituted under an Act, etc., are termed as some of the basic features of the Constitution.} described ‘basic’ or ‘essential’ in numerous cases, have been incorporated in the list of basic structure. In Indira Nehru Gandhi vs. Raj Naraian popularly known as Election case\footnote{A.I.R. 1975 S.C. 2299 (p. 2465, per Chandrachud J.).} and in Minerva Mills\footnote{A.I.R. 1980 S.C. 1789 (Para 88, per Bhagwati J.).} it
has been observed that the claim of any particular feature of the Constitution to be a ‘basic’
feature would be determined by the Court in each case that comes before it.

\vspace{-.1cm}

\noi
The power and procedure for constitutional amendment in India has some special points of
interest:

\vspace{-.2cm}

\begin{enumerate}
\itemsep=0pt
\item There is no separate constituent body for the purposes of amendment of the Constitution;
constituent power also being vested in the Legislature.

\item Although Parliament must preserve the basic framework of the Constitution, there is no
other limitation placed upon the amending power, that is to say, there is no provision of the
Constitution that cannot be amended.

\item The role of the States in Constitution amendment is limited. The State Legislatures cannot
initiate any Bill or proposal for amendment of the Constitution. They are associated in the
process of Constitution amendment by the ratification procedure laid down in article 368 in
case the amendment seeks to make any change in the any of the provisions mentioned in the
proviso to article 368. Besides, all that is open to them is (1) to initiate the process for creating
or abolishing Legislative Councils in their respective Legislatures\footnote{Article 169 (1): Notwithstanding anything in article 168, Parliament may by law provide for the abolition of the Legislative Council of a State having such a Council or for the creation of such a Council in a State having no such Council, if the Legislative Assembly of the State passes a resolution to that effect by a majority of the total membership of the Assembly and by a majority of not less than two-thirds of the members of the Assembly present and voting.} and (2) to give their views
on a proposed Parliamentary Bill seeking to affect the area, boundaries or name of any State
or States which has been referred to them under the proviso to article 3\footnote{The proviso of article 3 provides that no Bill for the purpose shall be introduced in either House of Parliament
except on the recommendation of the President and unless, where the proposal contained in the Bill affects the
area, boundaries or name of any of the States****, the Bill has been referred by the President to the Legislature
of the State for expressing its views thereon within such period as may be specified in the reference or within such
further period as the President may allow and the period so specified or allowed has expired.} a reference which
does not fetter the power of Parliament to make any further amendments of the Bill.\footnote{See Ruling by the Speaker in Lok Sabha—L.S. Deb., (II) 7 August 1956. The same view was taken by the Supreme Court in Babulal vs. State of Bombay (A.I.R. 1960 S.C. 51).}
\end{enumerate}

\vspace{-.2cm}

\heading{Conclusion}

\noi
One point that stands out before us in the process set down in Article 368 is that the Parliament
seems to have the exclusive right in any direction to change the Constitution. But it is incorrect
to say that the Parliament is independent, so as long as there is a mechanism under Article
368. Parliament cannot be the deciding authority of the constitutional scheme since the
procedure itself restricts the use of the power to amend the Constitution on the Parliament.
The Indian Constitution has been made as a dynamic statute that retains validity over years
without being obsolete and also takes care of the needs of the various classes within the Indian
society. It can be seen to have been drafted considering the best features of the Constitutions
around the world. The doctrine of the Basic Structure proposed by the honourable Supreme
Court is the guiding principle for safeguarding those values and keeping intact the essence of
the Constitution. The contrast with other countries further demonstrates the strong difference
in the amount of complexity and bureaucratic effort needed to change the Constitution in
India, rendering it one of the strongest.

\end{multicols}
