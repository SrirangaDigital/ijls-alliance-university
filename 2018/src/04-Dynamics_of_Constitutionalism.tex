\setcounter{figure}{0}
\setcounter{table}{0}
\setcounter{footnote}{0}

\articletitle{Dynamics of Constitutionalism in India - An Appraisal}\label{2018-art4}
\articleauthor{Mahantesh G. S\footnote{Principal and Program Coordinator, ABBS School of Law, Bangalore – 560091.}}
\lhead[\textit{\textsf{Mahantesh G. S}}]{}
\rhead[]{\textit{\textsf{Dynamics of Constitutionalism...}}}

\begin{multicols}{2}

\heading{Abstract}

\noi
Constitutionalism is a concept having the spirit of abiding the rules enshrined in the Constitution.
What is the constitution? Is it a prerequisite of every State or Nation? How much importance does
it have for the Government? Is it possible to govern a State without a Constitution? What are the
main attributes of a constitution? Does it provide solution to every administrative problem? Can
it ever be changed? Who has the power to change, amend, or expedite the Constitution? How
much extent can it be changed, amended, or expedited up to? Is it the supreme law of the land?
Does it treat every person equally? All the questions abovementioned or other than these questions
come in the mind when the term 'Constitution' comes in debate. There is a great controversy over
this topic in the world and there are some dissents or disagreements but many of the jurists are
seen to be unanimous in this respect. We the students of law are obliged not to accept the opinion
or approach without any appropriate contention otherwise it will be 'the bone of contention'.
Perhaps, this is the beauty of the legal field that everything is measured here by some wellestablished rules. The first and the main rule is that there should be any reasonable cause to make
any law. We will see the controversy and then decide whether the Constitutionalism exists in
India or not? After that we will be able to decide whether it is in existence or not.......

\heading{Introduction}

\noi
A constitution consists of a set of norms creating, structuring, and possibly defining the limits of,
government power or authority. Hence, it is very clear that most of the states have adopted their
own constitutions, and these states are recognized as constitutional states. To be recognized as a
state, it must have some fundamental means of constituting and specifying the limits placed upon
the three cardinal forms of government power i.e., legislative, executive, and judicial power.\footnote{Stanford Encyclopaedia of Philosophy; 'Constitutionalism'. First published on Wed. Jan.10, 2001; substantive revision Tue. Sep.11, 2012.}

\noi
Modern political thought tries to draw a clear-cut difference between 'Constitutionalism' and
'Constitution'. A state may have the 'Constitution' but need not necessarily be 'Constitutionalism'.
For example, in the case of an absolute monarch, Rex, exercising unlimited power in all three
domains. Suppose it is widely acknowledged that Rex has these powers, as well as the authority
to exercise them at his pleasure, the constitution of this state might then be said to have only one
rule, which permits unlimited power to Rex. He is not legally accountable for the wisdom or
morality of his decrees, nor is he obliged by procedures, or any other kinds of limitations or
requirements, in exercising his powers. Thus, Rex decrees is constitutionally valid.\footnote{See supra note 1.} 'Constitutionalism' implies in essence limited Government or a limitation on the Government.
Constitutionalism is the antithesis of arbitrary powers.\footnote{Charles H. Mcilwain, Constitutionalism: Ancient and Modern, 21; S. A. De Smith, Constitutional and Administrative Law, 34(1977); Giovanni Sartori, Constitutionalism: a preliminary discussion. (1962)56 am. Pol.Sc.Rev. 853. Recited in Indian Constitutional Law by M. P. Jain, fifth edition, 2006 at p.5.}

\heading{Prerequisites of Constitutionalism}

\noi
A written Constitution, independent judiciary with powers of judicial review, the doctrine of rule
of law and separation of powers, free elections to legislature, accountable and transparent
democratic government, fundamental rights of the people, federalism, decentralisation of power
are some of the principles and norms which promote constitutionalism in a country.\footnote{See supra note 2 at p. 6.} Besides these, there is the most important prerequisite of the Constitution whereby it becomes expedient according to the need of the hour; that is: 'the amenability'.\footnote{Author's view after analysis of the provision of the amendment to the Constitution of India.}

\noi
There are many characteristic features indicating that the Constitution of India has a real and great
spirit of Constitutionalism. Some of them are as follows:

\vspace{-.1cm}

\noi
{\large \bfseries 1 - Rule Of Law}

\noi
The Constitution of India by and large through many of its provisions seeks to upheld rule of law.
For example, Parliament members and State Legislatures are democratically elected based on
adult suffrage.

\noi
Further, The Constitution through its many of the provisions declare independence of the
judiciary. In fact, Judicial review has been ensured through several constitutional provisions. In Minerva Mills Ltd v. U.O.I,\footnote{AIR 1980 SC 1789 :(1980) 2 SCC 591.}  The Supreme Court has characterised judicial review as a 'basic feature of the Constitution'. Article 14 of the Constitution guarantees right to equality before law. This Constitutional provision has now assumed great significance as it is used to control administrative powers.\footnote{M. P. Jain, A Treatise of Administrative Law, I, Ch. XVIII; M.P. Jain, Indian Administrative Law Cases, and Materials, II, Chapter XV}

\noi
The Supreme Court of India has upheld the rule of law several times in its judgments to emphasise
upon certain constitutional values and principles. For example, in Bachan Singh v. State of
Punjab,\footnote{AIR 1982 SC 1325 :(1982) 3 SCC 24.} \textbf{Justice Bhagawati} has emphasized that rule of law excludes arbitrariness and unreasonableness. To ensure this, he has opined that it is necessary to have a democratic legislature to make laws, but its power should not be unfettered, and that there should be an independent judiciary to protect the citizen against the excesses of executive and legislative power.

\noi
In \textbf{Yusuf Khan v. Manohar Joshi,}\footnote{(1999) SCC (Cri.) 577.} the Supreme Court has ruled that it is the duty of the state to preserve and protect the law and the constitution and that it cannot permit any violent act which may negate the rule of law.

\noi
In \textbf{P. Sambamurthy v. State of Andhra Pradesh,}\footnote{AIR 1987 SC 663: (1987) 1 SCC362.} the Honourable Supreme Court has made it very clear that a provision authorising the executive to interfere with tribunal justice as
unconstitutional, characterising it as “violative of the rule of law which is clearly a basic and the
essential feature of the constitution".

\noi
The two great values which emanate from the concept of rule of law in modern times are:

\vspace{-.3cm}

\begin{enumerate}
\itemsep=0pt

\item No Arbitrary Government; and

\item Upholding Individual Liberty.
\end{enumerate}

\vspace{-.3cm}

\noi
Realising upon these values, \textbf{KHANNA, J.,} observed in \textbf{A.D.M. Jabalpur v. Shivakant Shukla,}\footnote{AIR 1976 SC 1207 at 1254} that "Rule of Law is the most antithesis of arbitrariness......Rule of Law is now the accepted norm of all civilised societies.... everywhere it is identified with the liberty of the
individual. It seeks to maintain a balance between the opposing notions of individual liberty and
public order. In every state, the problem arises of reconciling human rights with the requirements
of public interest. Such harmonising can only be attained by the existence of independent courts
which can hold the balance between citizen and the state and compel Governments to conform to
the law."

\noi
{\large \bfseries 2 - Separation of Powers}

\noi
The Doctrine of separation of powers in a broader way deal with the mutual relations among the
three organs of the Government namely, Legislature, Executive and Judiciary. The origin of this
principle goes back to the period of Plato and Aristotle. It was Aristotle, who for the first time classified the functions of the Government into three categories viz., Deliberative, Magisterial
and Judicial.\footnote{From the Article 'Principle of Separation of Powers and Concentration of Authority' written by Tej Bahadur Singh, Dy. Director (Administration), I.J.T.R., U.P., Lucknow, Published in the I.J.T.R. Journal-Second year issue 4\&5, March 1996.}

\noi
Renowned French \textbf{Jurist Montesquieu} in his work, \textbf{“Spirit of Laws”} published in 1748, for the
first time enunciated the principle of separation of powers. That is why he is known as modern
exponent of this theory. Montesquieu's doctrine implies the fact that one person or body of persons
should not exercise all the three powers of the Government i.e., Legislative, Executive and
Judiciary. In other way, each organ should restrict itself to its own sphere and refrain from
transgressing the province of the other. In the words of Montesquieu, “When the legislative and
executive powers are united in the same person, or in the same body or Magistrate, there can be
no liberty."

\noi
The Doctrine of separation of powers has no place in strict sense in Indian Constitution, but the
functions of different organs of the Government have been sufficiently differentiated, so that one
organ of the Government could not interfere with the functions of another.

\noi
In Constituent Assembly Debates, Prof. K. T. Shah (a member of Constituent Assembly) laid
emphasis to insert by amendment a new Article 40-A concerned with doctrine of separation of
powers. This Article reads as "There shall be complete separation of powers as between the
principal organs of the State, viz. the legislative, the executive and the judicial."\footnote{Constituent Assembly Debates Book No. 2, Vol. No. VII Second Print 1989, p.959.}

\noi
In \textbf{Udai Ram Sharma v. Union of India,}\footnote{AIR 1968 SC 1138 at p. 1152.} The Supreme Court held that "The American doctrine of well-defined separation of legislative and judicial powers has no application to India."

\noi
In \textbf{Kesavananda Bharti v. State of Kerala,}\footnote{AIR 1973 SC 1461 at p.1535.} Hon'ble \textbf{Chief Justice Sikri} observed "separation of powers amongst the legislature, executive and judiciary is a part of the basic structure of the constitution; this structure cannot be destroyed by any form of amendment".

\noi
In \textbf{Smt. Indira Nehru Gandhi v. Raj Narain,} \footnote{AIR 1975 SC 2299 at 2470.} \textbf{Hon'ble Justice Chandrachud} observed: "The \textbf{American Constitution} provides for a rigid separation of governmental powers into three basic divisions: the executive, legislative and judicial. It is essential principle of that Constitution that powers entrusted to one department should not be exercised by any other department. Even, the \textbf{Australian Constitution} follows the same pattern of distribution of powers. Unlike these Constitutions, the \textbf{Indian Constitution} does not expressly vest the three kinds of power in three different organs of the State. But the principle of separation of powers is not a magic formula for keeping the three organs of the State within the strict ambit of their functions."

\noi
In \textbf{Sri Ram v. State of Bombay,}\footnote{AIR 1959 SC 459, 473,474.} it was held that the absolute separation of powers is not possible by any form of Government. In view of the varying situations, the legislature cannot anticipate all the circumstances to which a legislative measure should be extended and applied.
Therefore, legislature is empowered to delegate some of its functions to executive authority. But
one thing is here to note that the legislature cannot delegate its essential legislative power.\footnote{Sri Ram v. State of Bombay,}

\noi
In \textbf{Asif Hameed v. State of J\&K,}\footnote{AIR 1989 SC 1899.} the Supreme Court observed that "Although the doctrine of
separation of powers has not been recognised under the Constitution in its absolute rigidity but
the Constitution makers have meticulously defined the functions of various organs of the State.
Legislature, executive and judiciary must function within their own spheres demarcated under the
Constitution. No organ can usurp the functions assigned to another. The Constitution trusts to the
judgment of these organs to function and exercise their discretion by strictly following the
procedure prescribed therein. The functioning of democracy depends upon the strength and
independence of each of its organs."

\noi
The Government (State) cannot escape from its prime duty (i.e., rendering services for the welfare
of the citizens) showing that it is over-burdened with day-to-day functioning. The functions of a
modern state unlike the police, states of old are not confined to mere collection of taxes or maintenance of laws and protection of the realm from external or internal enemies. A modern
state is certainly expected to engage in all activities necessary for the promotion of the social and
economic welfare of the community.\footnote{Ram Jawaya v. State of Punjab, AIR 1955 SC at p.554.}

\noi
History proves this fact that if there is a complete separation of powers, the Government cannot
run effectively and smoothly. Smooth running of Government is possible only by co-operation
and mutual adjustment of all the three organs of the government. Prof. Garner has rightly pointed
out that, “the doctrine is impracticable as working principle of Government." It is not possible to
categorise the functions of all three branches of Government on mathematical basis. The
observation of Frankfurter is notable in this connection. According to him, "a rigid conception of
separation of powers would make Government impossible."\footnote{Frankfurter-The Public and its Government (1930) quoted by B. Schwartz, in American Constitutional law, 1955 Page 286.}

\noi
{\large \bfseries 3 - Federalism}

\vspace{-.15cm}

\noi
Federalism in essence is a form of government in which sovereign powers are constitutionally
divided between a central government and geographically defined, semi-autonomous regional
governments.\footnote{47 JILI (2005).} In the Indian context, Austin says, "Federalism is an idea and a set of practices, the variety of which depends upon the goals of the citizenry and its leaders, the consequent definition of the term, and the conditions present in the would-be federation."\footnote{G. Austin, Working a Democratic Constitution. The Indian Experience 555(1999).}

\vspace{-.12cm}

\noi
Federalism's commitment to justice and democracy imposes obligation upon power holds to act
with a sense of responsibility to avoid disparities in access to positive rights and welfare. The
features of economic and social asymmetry are to be tackled by the benevolent goal of equal
liberty of all. Effective utilization of the centrally sponsored welfare schemes by the states, fiscal
federalism's focus on development of backward states and equitable apportionment of inter-state
water by using the principle of non-disparity do expand equal rights. The question of diversity of
jurisdiction versus rights can neither be dealt mechanically nor in obstructing the way of social
justice and multiculturalism.

\vspace{-.12cm}

\noi
Cooperative federalism in the matter of combating organised crimes or in rectifying state inaction
tends to create an atmosphere supportive of rights. Asymmetry by special status is prone for
egalitarian influence.\footnote{Taken from "Why and how federalism, matters in elimination of disparities and promotion of equal opportunities for positive rights, liberties and welfare" by 'P. Ishwara Bhat' in 54 JILI 2012 at 324.}

\vspace{-.12cm}

\noi
Federalism, as a type of state, is an enduring expression of the principle of constitutionalism that
retains the unity of federal units and effectuates the constitutional goals through mutual cooperation and co-ordination amidst central and state governments.

\vspace{-.1cm}

\noi
By considering federalism as a part of the basic feature of the Constitution\footnote{C. Rossiter (Ed.), The Federalist Papers XII (New York: The New American Library, 1961; Recited in 54 JILI 2012 at p.327).} and supremacy of the Constitution as a key to the success of federalism, Indian Judiciary has laid emphasis on normative character of federalism's functioning towards equal rights and welfare in addition to mutual checks and balances of powers.

\vspace{-.1cm}

\noi
There are some institutions like Inter-state Council, Finance Commission, Zonal Council,
National Development Council etc. to promote the federalism and the spirit of co-operation and
solve the disputes between two states or among many states. Some like Finance Commission have
constitutional status and some do not have. Whether be it a constitutional institution or not, but it
helps the preservation of the federalism.

\vspace{-.1cm}

\noi
{\large \bfseries 4 - Amendability of Indian Constitution\footnote{S.R. Bommai v. Union of India, AIIR 1994 SC 1918; Kuldip Naya v. Union of India, AIR 2006 SC 3127; State of West Bengal v. Committee for Protection of Democratic Rights, AIR 2010 SC 1476.}}

\vspace{-.2cm}

\noi
Several Articles of the constitution make provisions of a tentative nature, and the Parliament has
been given power to make laws making provisions different from what these Articles provide for.
Such a law can be made by the ordinary legislative process and is not to be regarded as an
amendment of the Constitution and is not subject to the special procedure prescribed in Art. 368.
In most of the cases, the constitutional text remains intact, but Parliament makes different
provisions. These Articles of the Constitution are as follows:

\vspace{-.3cm}

\begin{enumerate}
\itemsep=0pt


\item[(1)] When Parliament admits a new State under Art. 2, it can affect consequential amendments in
Schedules I and IV defining territory and allocating seats in the Rajya Sabha amongst the various
states, respectively.

\item[(2)] Under Article 11, Parliament is empowered to make any provision for acquisition and termination of, and all other matters relating to, citizenship despite Articles. 5 to 10.

\item[(3)] Article 73(2) retains certain executive powers in the States and their officers until Parliament
otherwise provides.

\item[(4)] Arts. 59(3), 75(6), 97, 125(2), 148(3) and 221(2) permit amendments by Parliament of the
Second Schedule dealing with salaries and allowances of certain officers created by the
Constitution.

\item[(5)] Art. 105(3) prescribes parliamentary privileges until it is defined by Parliament.

\item[(6)] Article 124(1) prescribes that Supreme Court shall have a Chief Justice and seven Judges until
Parliament increases the strength of the Judges.

\item[(7)] Article 133(3) prohibits an appeal from the judgment of a single Judge of a High Court to the
Supreme Court unless Parliament provides otherwise.

\item[(8)] Article 135 confers jurisdiction on the Supreme Court (equivalent to the Federal Court), unless
Parliament otherwise provides.

\item[(9)] Under Art. 137, Supreme Court's power to review its own judgments is subject to a law made
by Parliament.

\item[(10)] Article 171(2) states that the composition of the State Legislative Council as laid down in
Article 170(3) shall endure until Parliament makes a law providing otherwise.

\item[(11)] Article 343(3) provides that Parliament may by law provide for the use of English even after
15 years as prescribed in Article 343(2).

\item[(12)] Article 348(1) provides that English as the language to be used in the Supreme Court and the
High Courts and of legislation until Parliament provides otherwise.

\item[(13)] Schedules V and VI deal with administration of the Scheduled Areas and Scheduled Tribes
and Tribal Areas in Assam which may be amended by Parliament by making a law.
\end{enumerate}

\vspace{-.3cm}

\noi
There are certain other Articles in the Constitution which make tentative provisions until a law is
made by the Parliament by following the ordinary legislative process, but before Parliament can
act, the States must take some action. Thus, Art. 3 provides for the re-organisation of the States.
Parliament may pass a law for the purpose and effect consequential amendments in the I and IV
Schedules. Before doing so, however, it is necessary to ascertain the views of the states concerned.

\noi
Under Art. 169, Parliament may abolish a state legislative council, or create one in a State not
having it, if the legislative assembly passes a resolution to that effect by most of its total
membership and by a majority of not less than two-thirds of the members present and voting. The
Parliamentary law enacted for the purpose may contain such provisions amending the constitution
as may be necessary to give effect to it and it is not to be regarded as an amendment of the
constitution for the purposes of Article 368. Corresponding to Articles 75(6) and 105(3), there
are Arts. 164(5) and 194(3) which make tentative provisions until a State Legislature makes other
provisions. They relate respectively to salaries of Ministers in a state and privileges of the Houses.
These are the only Articles in the Constitution which enable a state legislature to make provisions
different from what the constitution prescribes in the first instance.

\noi
The process to amend and adapt other provisions of the Indian Constitution is contained in Article
368. The phraseology of Article 368 has been amended twice since the inauguration of the
constitution. However, the basic features of the amending procedure have remained intact despite
these changes. These basic features are:

\vspace{-.3cm}

\begin{enumerate}[label=(\roman*)]
\itemsep=0pt
\item  An amendment of the Constitution can be initiated only by introducing a Bill for the purpose
in either House of Parliament.

\item After the Bill is passed by each House by most of its total membership, and a majority of not
less than two-thirds of the members of that House present and voting, and after receiving the
assent of the President, the Constitution stands amended in accordance with the terms of the Bill.

\item To amend certain constitutional provisions relating to its federal character, characterised as
the 'entrenched provisions', after the Bill to amend the Constitution is passed by the Houses of
Parliament as mentioned above, but before being presented to the President for his assent, it has
also to be ratified by the legislatures of not less than one-half of the States by resolutions. Leading
cases regarding amendment of the Constitution are:
\end{enumerate}

\vspace{-.3cm}

\noi
{\large \bfseries (a) Shankari Prasad Singh:}

\noi
In \textbf{Shankari Prasad Singh v. Union of India,}\footnote{Indian Constitutional Law by M. P. Jain, Fifth Edi. 2006 at p. 1617-1618.} the first case on amendability of the Constitution, the validity of the Constitution (First Amendment) Act, 1951, curtailing the right to property guaranteed by Article 31 was challenged. The argument against the validity of the First
Amendment was that Article 13 prohibits enactment of a law infringing or abrogating the
fundamental rights that the word 'law' in Article 13 would include any law, even a law amending
the Constitution and, therefore, the validity of such a law could be judged and scrutinised with
reference to the fundamental rights which it could not infringe. The Supreme Court held, “We are
of the opinion that is the context of Article 13 law must be taken to mean rules and regulations
made in the exercise of ordinary legislative power and not amendments to the Constitution made
in the exercise of constituent power with the result that Article 13 (2) does not affect amendments
made under Article 368."

\noi
The Court also held that the contents of Article 368 are perfectly general and empower the
Parliament to amend Constitution without any exception. The fundamental rights are not excluded
or immunised from the process of constitutional amendment under Article 368. These rights could
not be invaded by legislative organs by means of laws and rules made in exercise of legislative
powers, but they could certainly be curtailed, abridged, or even nullified by alterations in the
constitution itself in exercise of the constituent power.

\noi
{\large \bfseries (b) Sajjan Singh:}

\noi
For the next 13 years following Shankari Prasad Singh, the question of amendability of the
fundamental rights remained dormant. The same question was raised again in 1964 in \textbf{Sajjan Singh v. Rajasthan,}\footnote{AIR 1951 SC 458.} when the validity of the Constitution (Seventeenth Amendment) Act,
1964, was called in question. This Amendment again adversely affected the right to property. By
this amendment, several statutes affecting property rights were placed in the Ninth Schedule and
were thus immunised from our review.In the instant case, the Court was called upon to decide the
following questions: 

\vspace{-.3cm}

\begin{enumerate}
\itemsep=0pt
\item[(1)] Whether the amendment of the constitution insofar as it purported to take away or abridged
the Fundamental Rights was within the prohibition of Article 13(2); and

\item[(2)] Whether Articles 31A and 31B (as amended by the XVIIth Amendment) sought to make
changes to Articles 132, 136 and 226, or in any of the lists in the VIIth Schedule of the
Constitution, so that the conditions prescribed in the proviso to Art. 368 had to be satisfied.
\end{enumerate}

\vspace{-.3cm}

\noi
One of the arguments was that the amendment in question reduced the area of judicial review it
thus, affected Article 368 for amending the 'entrenched provisions', that is, the concurrence of at
least half of the states ought to have been secured for the amendment to be validly effectuated.
Such an argument had also been raised in the Shankari Prasad Singh case but without success.
The Supreme Court again rejected the argument by a majority of 3 to 2. The majority ruled that
the \textbf{'pith and substance'} of the amendment was only to amend the Fundamental Right to help
the state legislatures in effectuating the policy of the agrarian reform. If it affected Article 226 in
an insignificant manner, that was only incidental; it was an indirect effect of the Seventeenth
Amendment and it did not amount to an amendment of Article 226. The impugned Act did not
change Article 226 in any way.

\vspace{-.15cm}

\noi
The conclusion of the Supreme Court in Shankari Prasad Singh as regards the relation between
Articles 13 and 368 was reiterated by the majority, it felt no hesitation in holding that the power
of amending the constitution conferred on Parliament under Article 368 could be exercised over
each provision of the Constitution. The majority refused to accept the argument that Fundamental
Rights were "eternal, inviolate, and beyond the reach of Article 368."

\vspace{-.15cm}

\noi
The Court again drew the distinction between an 'ordinary' law and a 'constitutional' law made in
exercise of 'constituent power' and held that only the former, and not the latter, fall under Article 13.

\noi
{\large \bfseries (c) Golak Nath:}

\noi
Perhaps, encouraged by the above stated remarks of the two Judges, the question whether any of
the Fundamental Rights could be abridged or taken away by Parliament in exercise of its power
under Article 368 was raised again in \textbf{Golak Nath}\footnote{AIR 1965 SC 845.} in 1967. Again, the constitutional validity of
the Constitution (Seventeenth Amendment) Act was challenged in a very vigorous and
determined manner. Eleven Judges participated in the decision, and they divided 6 to 5. The
majority now held, overruling the earlier cases of Shankari Prasad and Sajjan Singh that the
Fundamental Rights were non-amendable through the constitutional amending procedure set out
in Article 368, while the minority upheld the line of reasoning adopted by the Court in the two
earlier cases.

\noi
It is true that Article 368 suffers from an anomaly. While for amending certain provisions,
characterised as the 'entrenched clauses', consent of at least half of the State Legislatures is
stipulated in addition to the special majority in Parliament, it is not so with respect to the
Fundamental Rights.\footnote{Golak Nath v. State of Punjab, AIR 1967 SC 1643: 1967(2) SCR762.}

\noi
{\large \bfseries 5- Preamble}

\noi
Unlike the Constitutions of Australia, Canada or the U.S.A., the Constitution of India has an
elaborate Preamble. The purpose of the Preamble is to clarify who has made the Constitution,
what is its source, what is the ultimate sanction behind it; what is the nature of the polity which
is sought to be established by the Constitution and what are its goals and objectives.

\noi
The Preamble does not grant any power, but it gives a direction and purpose to Constitution. It
outlines the objectives of the whole Constitution. The Preamble contains the fundamentals of the
constitution. It serves several important purposes, as for example.

\vspace{-.3cm}

\begin{enumerate}
\itemsep=0pt

\item[(1)] It contains the enacting clause which brings the Constitution into force.

\item[(2)] It declares the great rights and freedoms which the people of India intended to secure to all its
citizens.

\item[(3)] It declares the basic type of government and polity which is sought to be established in the
country.

\item[(4)] It throws light on the source of the Constitution, viz. the People of India. 
\end{enumerate}

\vspace{-.3cm}

\heading{Conclusion}

\noi
Based on the above observation, it is very clear that Constitutionalism, being a vast subject cannot
be confined to some principles only. Its essence is in every nation where the government's
objective is not only to rule but the empowerment, development, and equal opportunities for all.
But to understand the meaning and importance of it, we must analyse some pre-established
doctrines as we have seen above. By analysis of all these contentions in support of
Constitutionalism, we can consider its importance in enhancing nation's integrity, fraternity, and
unity. Rule of law, Separation of Powers are the principles which usher our legislature and
judiciary to provide us a unique, antique, and ever updated or expedited Constitution which is the
base of the democracy which is known as the best democracy of the world.
\end{multicols}
\label{end2018-art4}
