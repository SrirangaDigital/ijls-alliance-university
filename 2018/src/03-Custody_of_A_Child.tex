\setcounter{figure}{0}
\setcounter{table}{0}
\setcounter{footnote}{0}

\articletitle{Custody of A Child -Its Legal Aspects}\label{2018-art3}
\articleauthor{Dr. Heena Ghanshyam Patoli\footnote{Dr. Heena Ghanshyam Patoli, Assistant Professor, Alliance School of Law, Bengaluru}}
\lhead[\textit{\textsf{Dr. Heena Ghanshyam Patoli}}]{}
\rhead[]{\textit{\textsf{Custody of A Child...}}}

\begin{multicols}{2}

\heading{Abstract}

\noi
In modern society divorce has become rampant and the bad effects of that is that where do the
children go or with whom do the children live with. During any breakdown of marriage, the
custody of children is an important question to be resolved. It is said “the child needs a mother
the most, but it needs a father too”. The question that arises is does the Indian law believes this?
Is it that the law believes that the child has a good future just with the mother? Is it possible that
the mother can always be the right person in the life of the child? There are instances when both
the parents want the custody of the children and also when none of the parents want the custody
of the children. There have been cases where the Supreme Court has decided that even when
mother is carrying on illicit, illegal profession the custody is given to her. Issues also arise when
there are foreign jurisdictions and foreign Court judgements involved. Further, there are matters
where the spouse is facing the criminal charges for the murder/ abetment of murder of the other,
then who should be handed over the custody. The Courts have also look into the aspect of working
mothers while deciding on the custody as they may not have time for the children. The Courts
while granting the custody has to look into second marriages of one or both the partners. Many
times the child prefers to alienate himself from both the parents as he is fed up of the fights. Then,
there is a bigger issue of ego clashes between the parents and where they both are harming the
child without realizing it.

\noi
This paper remaps the different legal aspects of custody of a child. The paper discusses the Indian
laws regarding the child custody. The custody of the child affects the child emotionally and
psychologically. The child is caught up in between the parents, its difficult for him/her to choose 
one over the other. In the quarrels the couple forgets that the child is suffering. The paper explores
the legality of custody of a child in India. It brings out the hardships of the parents who have
become slaves of the law and therefore research paper asks a simple question are we even able to
give basic human rights or justice to the parents. Justice can be done to either the father or mother
but not to both and hence justice can be done to just of them and other one is deprived of it. This
paper looks into the laws regarding the custody of the child in India, the amendments required to
remove the superiority of one parent over the other and the kind of impression that is left by the
law on the child. And finally the paper discusses the factors considered by the Courts when
granting custody. Children many times become very aggressive because of the fights of the
parents over the custody issue and many criminals are born, they take into drugs and other vices
and innocent children become criminals after facing the hardships. It becomes very difficult for
such children to survive in the society. The children are also not accepted in the society easily.

\vspace{-.1cm}

\heading{Introduction}

\vspace{-.1cm}

\noi
In modern society divorce has become rampant and the bad effects of that is that where do the
children go or with whom do the children live with. During any breakdown of marriage, the
custody of children is an important question to be resolved. It is said “the child needs a mother
the most, but it needs a father too”. The question that arises is does the Indian law believes this?
Is it that the law believes that the child has a good future just with the mother? Is it possible that
the mother can always be the right person in the life of the child? There are instances when both
the parents want the custody of the children and also when none of the parents want the custody
of the children. There have been cases where the Supreme Court has decided that even when
mother is carrying on illicit, illegal profession the custody is given to her. Issues also arise when
there are foreign jurisdictions and foreign Court judgements involved. Further, there are matters
where the spouse is facing the criminal charges for the murder/ abetment of murder of the other,
then who should be handed over the custody. The Courts have also look into the aspect of working
mothers while deciding on the custody as they may not have time for the children. The Courts
while granting the custody must look into second marriages of one or both the partners. Many
times, the child prefers to alienate himself from both the parents as he is fed up of the fights. Then,
there is a bigger issue of ego clashes between the parents and where they both are harming the
child without realizing it.

\vspace{-.1cm}

\heading{Legal Aspects}

\vspace{-.1cm}

\noi
This paper remaps the different legal aspects of custody of a child. The paper discusses the Indian
laws regarding the child custody. The custody of the child affects the child emotionally and
psychologically. The child is caught up in between the parents, its difficult for him/her to choose
one over the other. In the quarrels the couple forgets that the child is suffering. The paper explores
the legality of custody of a child in India. It brings out the hardships of the parents who have
become slaves of the law and therefore research paper asks a simple question are we even able to
give basic human rights or justice to the parents. Justice can be done to either the father or mother
but not to both and hence justice can be done to just of them and other one is deprived of it. This
paper investigates the laws regarding the custody of the child in India, the amendments required
to remove the superiority of one parent over the other and the kind of impression that is left by
the law on the child. And finally, the paper discusses the factors considered by the Courts when
granting custody. Children many times become very aggressive because of the fights of the
parents over the custody issue and many criminals are born, they take into drugs and other vices
and innocent children become criminals after facing the hardships. It becomes very difficult for
such children to survive in the society. The children are also not accepted in the society easily.

\vspace{-.1cm}

\noi
The law regarding custody finds its place in Sec 26 of the Hindu Marriage Act, Sec 38 of the
Special Marriage Act, The Guardians \& Wards Act 1890, Sec 6 (a) of Hindu Minority and
Guardians Act 1956. The Christians and the Muslims also have their own laws for the custody of
the child. Christian law per se does not have any provision for custody but the issues are well
solved by the Indian Divorce Act which is applicable to all the religions of the country. The Indian
Divorce Act, 1869 contains provisions relating to custody of children. Under Muslim Law, the
first and foremost right to have the custody of children belongs to the mother and she cannot be
deprived of her right so long as she is not found guilty of misconduct.

\vspace{-.1cm}

\noi
‘Child custody’ is a term used in family Law Courts to define legal guardianship of a child under
the age of 18. During divorce or marriage annulment proceedings, the issue of ‘child custody’ often becomes a matter for the Court to determine. In most cases, both parents continue to share
legal child custody but one parent gains physical child custody. Family Law Courts generally
base decisions on the best interests of the child or children, not always on the best arguments of
each parent.

\vspace{-.1cm}

\noi
Divorce has become very common in Indian society these days. When two people cannot get
along maybe it’s the right thing to do, to get separated and live their lives in peace. But the
question arises about the future of children. The custody of the child is an important issue and
needs greater attention by the parents, law and of course the society. According to me the will of
the child should be taken into consideration and should be of paramount importance. Though the
child always wants to stay with both and do not have preference for one parent over the other, the
actual physical custody can be given to just one parent and is generally given to the mother though
the natural guardian is considered as the father and after him the mother. The will of the child
should be taken into consideration and is of paramount importance is mentioned in the Hindu
Minority and Guardians Act, 1956.

\vspace{-.1cm}

\noi
In general, Courts tend to award physical child custody to the parent who demonstrates the most
financial security, adequate parenting skills and the least disruption for the child. Both parents
continue to share legal child custody until the minor has reached the age of 18 or becomes legally
emancipated. Legal custody means that either parent can make decisions which affect the welfare
of the child, such as education, career, religious practices, medical treatments, decision on
marriage etc. Physical child custody means that one parent is held primarily responsible for the
child's housing, educational needs and food apart from other basic needs. In most cases, the noncustodial parent still has visitation rights. But these are judicial statements of general nature and
there is no hard and fast rule which is fair, as having a hard and fast rule may not be appropriate
remedy for the child. As to the children of tender years it is now a firmly established practice that
mother should have their custody since father cannot provide that maternal affection which is
essential for their proper growth. It is also required and accepted for proper psychological
development of children of tender years which makes the mother indispensable.

\vspace{-.1cm}

\noi
The laws governing child custody in India are mentioned in the Guardians and Wards Act 1890,
the Hindu Minority and Guardians Act 1956, Sec 26 of Hindu Marriage Act, Section 38, of the 
Special Marriage Act 1954, Section 41 of the Indian Divorce Act, 1869 contains provisions
relating to custody of children and of course the custody of Muslim children are governed by their
personal laws.

\noi
The Guardians \& Wards Act (GWA) 1890 is a secular law for appointment and declaration of
guardians and allied matters, irrespective of caste, community or religion, though in certain
matters, the Court will give consideration to the personal law of the parties. The provisions of the
HMGA (and other personal laws) and the GWA are complementary and not in derogation to each
other, and the Courts are obliged to read them together in a harmonious way. In determining the
question of custody and guardianship, the paramount consideration is the welfare of the minor.
The word `welfare' has to be taken in its widest sense, and must include the child's, moral as well
as physical well-being, and also have regard to the ties of affection.

\noi
Let us look into what the Hindu Marriage Act says about the custody of the child, Sec 26 of Hindu
Marriage Act says the Court may make provisions regarding the custody, maintenance and
education of the minor child consistently with the wishes of the child whenever possible. So most
of the times the Courts look into the wishes of child but it also takes into consideration the interest
and well-being of the child and sometimes it may be contradictory and in such cases the Court
gives more importance to the interest of the child. At times the mother may not be willing to keep
the child as she may want to remarry or that the job of the mother may require lot of travelling
and taking care of the child may not be possible or that the mother is involved in illicit trade, in
such cases the custody of the child is given to the father.

\noi
Section 38, of the Special Marriage Act 1954; speak of almost the same thing as mentioned in the
Hindu Marriage Act. It says the Court may make provisions regarding the custody, maintenance
and education of the minor child consistently with the wishes of the child whenever possible. The
decision of the Courts is based on the term “Just and proper” with respect to the custody of the
child. What may be just and proper in one case may not be just and proper in another case and
hence the decision of the Court differs from case to case.

\noi
The laws governing child custody in India are mentioned in the Guardians and Wards Act 1890
and the Hindu Minority and Guardians Act 1956. Coming to Section 6 (a) of the Hindu Minority 
and Guardians Act, \footnote{Hindu Minority and Guardians Act, 1956, Sec 6(a) the natural guardian of a Hindu minor in respect of his person
or his property, in case of a boy and an unmarried girl is father and after him the mother,} which says the natural guardian of a Hindu minor in case of a boy and an
unmarried girl is father and after him the mother, but it also says the custody of the minor who
has not completed 5 years shall be ordinarily be with the mother. The Act differentiates between
legitimate child and illegitimate child, it says the guardian will be father in case of legitimate child
but natural guardian will be mother in case of illegitimate child. So the Act does not thrust on
primary responsibility on the father in case of an illegitimate child. In case of a married girl again
the Act differs, it says the natural guardian is the husband. The right which was there before
marriage ceases on her marriage and husband naturally becomes her guardian. Again there is a
difference of natural guardianship and custody in case of step father and step mother. The Act
also says that the person ceases to be a natural guardian of a minor if he ceases to be a Hindu and
if he has completely and finally renounced the world. The natural guardianship of an adopted son,
who is a minor, passes to adoptive father and after him the adoptive mother on adoption.

\noi
The Act also specifies that the natural guardian of a Hindu minor has power to do all such acts
which are necessary or reasonable and proper for the benefit of the minor for the realization,
protection or benefit of the minor’s estate, but the guardian cannot bind the minor by a personal
covenant. The guardian cannot transfer any property without the previous permission of the Court.

\noi
The Act clarifies that the interest of the child will be of paramount consideration. Section 13 of
the Act says in deciding a natural guardian and in custody of the child the welfare of the child
will be of prime consideration and importance.

\noi
There are another set of laws in Muslim law for the Custody of the children. Under Muslim Law,
the first and foremost right to have the custody of children belongs to the mother and she cannot
be deprived of her right so long as she is not found guilty of misconduct. Mother has the right of
custody so long as she is not disqualified. This right is known as right of hizanat and it can be
enforced against the father or any other person. The mother's right of ‘hizanat’ was solely
recognized in the interest of the children and in no sense it is an absolute right. Son among the
Hanafis, it is an established rule that mother's right of ‘hizanat’ over her son terminates on the
latter's completing the age of 7 years. The Shias hold the view that the mother is entitled to the 
custody of her son till he is weaned. Among the Malikis the mother's right of ‘hizanat’ over her
son continues till the child has attained the age of puberty. The rule among the Shafiis and the
Hanabalis remains the same. The mother is entitled to the custody of her daughters, among the
hanafis, till the age of puberty and among the Malilikis, Shafiis and the Hanabalis the mother's
right of custody over her daughters continues till they are married. Under the Ithna Ashari law the
mother is entitled to the custody of her daughters till they attain the age of 7. The mother has the
right of custody of her children up to the ages specified in each school, irrespective of the fact
whether the child is legitimate or illegitimate. Mother cannot surrender her right to any person
including her husband, the father of the child. Under the Shia school after the mother the right of
‘hizanat’ belongs to the father. In the absence of both the parents or on their being disqualified
the grandfather is entitled to custody. Among the Malikis the custody of the child, in the absence
of mother goes to the maternal grandmother, maternal great grandmother, maternal aunt and great
aunt, full sister, uterine sister, consanguine sister, paternal aunt i.e. Father's sister. All the schools
of Muslim law recognize father's right of ‘hizanat’ under two conditions that are on the
completion of the age by the child up to which mother or other females are entitled to custody. In
the absence of mother or other females who have the right to ‘hizanat’ of minor children. Father
undoubtedly has the power of appointing a testamentary guardian and entrusting him with the
custody of his children. Other male relations entitled to right to ‘hizanat’ are nearest paternal
grandfather, full brother, consanguine brother, full brother's son, consanguine brother's father, full
brother of the father, consanguine brother of the father, father's full brother's son father's
consanguine brother's son Among the Shias hizanat belongs to the grandfather in the absence of
the father.

\noi
Christian law per se does not have any provision for custody but the issues are well solved by the
Indian Divorce Act which is applicable to all of the religions of the country. The Indian Divorce
Act contains provisions relating to custody of children. Section 41 of the said Act provides with
the powers to make orders as to custody of children in suit for separation. -In any suit for obtaining
a judicial separation the Court may from time to time, before making its decree, make such interim
orders, and may make such provision in the decree, as it deems proper with respect to the custody,
maintenance and education of the minor children, the marriage of whose parents is the subject of
such suit, and may, if it think fit, direct proceedings to be taken for placing such children under
the protection of the said Court.

\noi
There are numerous connotations this can take, some of these are that the law reflects our
patriarchal social structure and that small children are always better off with the mother. Matters
are also complicated by a legal process that does not view legal guardianship to be coterminous
with physical custody of a child. The Supreme Court of India has consistently held that in deciding
the cases of child custody the first and paramount consideration is the welfare and interest of the
child and not the rights of the parents. The Supreme Court has several times has held that no
statute on the subject can ignore or obliterate the vital factor of the welfare of the child. In a
landmark judgment,\footnote{Ms. Githa Hariharan \& Anr v. Reserve Bank of India \& Anr, 1999} the petitioner, Ms Githa Hariharan and Dr Mohan Ram were married in
Bangalore in 1982 and had a son in July 1984. In December 1984 the petitioner applied to the
Reserve Bank of India (RBI) for 9\% Relief Bond to be held in the name of the son indicating that
she, the mother, would act as the natural guardian for the purposes of investments. RBI returned
the application advising the petitioner either to produce an application signed by the father or a
certificate of guardianship from a competent authority in her favour to enable the bank to issue
bonds as requested. This petition was related to a petition for custody of the child stemming from
a divorce proceeding pending in the District Court of Delhi. The husband petitioned for custody
in the proceedings. The petitioner filed an application for maintenance for herself and the minor
son, arguing that the father had shown total apathy towards the child and was not interested in the
welfare of the child. He was only claiming the right to be the natural guardian without discharging
any corresponding obligation. On these facts, the petitioner asks for a declaration that the
provisions of S. 6(a) of the Hindu Minority and Guardianship Act of 1956 along with S. 19(b) of
the Guardian Constitution and Wards Act violated Articles 14 and 15 of the Constitution of India.
The Supreme Court brings to fact the equality of the mother to fulfil the role of a guardian held
that gender equality is one of the basic principles of our Constitution and therefore the father by
reason of dominant personality cannot be ascribed to have a preferential right over the mother in
the matter of guardianship since both fall within the same category. It was like saying gender was
not a consideration in deciding matters of child custody and guardianship interest of the child was
more important.

%~ \vspace{-.1cm}

\noi
In Re Kamal Rudra Das J.\footnote{1949 2 I.L.R. 374} expressed the same view vividly that the mother's lap is God's own
cradle for a child of this age, and that as between father and mother, other things being equal, a
child of such tender age should remain with mother. But he also said that a mother who neglects
the infant child as she does not want to sacrifice the type of life she is leading can be deprived of
custody. In respect of older children, the Courts take the view that the male children above the
age of sixteen years and female children above the age of fourteen years, should not ordinarily be
compelled to live in the custody to which they object. However, even the wishes of the mature
children will be given consideration only if they are consistent with their welfare. \textbf{In the case
of Rukmangathan v J. Dhanalakshmi\footnote{16 December, 1997, (1998) 1 MLJ 628], Madras High Court} it was laid down that the} male above 16 years and female child above 14 years cannot be compelled to live in the custody where are do not wish to live. In Venkataramma v. Tulsi,\footnote{(1949) 2 MLJ 802} the Court disregarded the wishes of the children as it found that it was done wholesale persuasion and were even tortured. Ordinarily,
custody to third persons should not be given except to either of the parents. But where welfare so
requires, custody may be given to a third person. In Baby Sarojam v. S. Vijayakrishnan Nair \footnote{AIR 1992 Ker 277, I (1994) DMC 79} granting custody of two minor children to maternal grandfather, the Court observed that even if the father was not found unfit, custody might be given to a third person in the welfare of the child.

\vspace{-.1cm}

\noi
In the case of Rosy Jacob v. Jacob A. Chakramakkal\footnote{1973 AIR 2090, 1973 SCR (3) 918} the Court held that all Orders relating to the custody of the minor wards from their very nature must be considered to be temporary Orders made in the existing circumstances. With the changed conditions and circumstances, including the passage of time, the Court is entitled to vary such Orders if such variation is considered to be in the interest of the welfare of the wards. Orders relating to custody of wards even when based on consent are liable to be varied by the Court, if the welfare of the wards demands variation. The Court after a decree of judicial separation, may upon application (by petition) for this purpose make, from time to time, all such Orders and provision, with respect to the custody, maintenance and education of the minor children, the marriage of either of the parents is the subject of the decree, or for placing such children under the protection of the Court. The Court may from time
to time, before making its decree absolute or its decree make such interim Orders.

\noi
In a habeas corpus, in Punjab and Haryana High Court\footnote{Mandeep Kaur v. State of Punjab, 2021 SCC OnLine P\&H 1060, decided on 10-05-2021}, case regarding custody of the child the Bench of Anupinder Singh Grewal, J. refused to consider extra-marital affair as a ground to deny custody of child to the mother. The Bench was of the opinion that extra marital affair cannot be
the reason to deny custody to the mother. The court emphasized that the mother is the natural
guardian of the child till the age of five years in terms of Section 6 of the Hindu Minority and
Guardianship Act, 1956, and that the child would require love, care and affection of the mother
for her development in the formative years. Similarly, the support and guidance of the mother
would also be imperative during adolescence. The Bench remarked that it is common to make
allegations on the moral character of a woman. Therefore, allegations against the petitioner being
wholly unsubstantiated were not considered relevant to adjudicate the issue of custody of the
minor child. Furthermore, the petitioner had permanent residency in Australia. She was earning
Rs 70,000/- Australian dollars per annum and a handsome sum would be payable to her for the
maintenance of child as well by the Australian authorities. The father was also an Australian
citizen but right now had come to India and so the child would be doing better with mother.

\noi
In the facts of the case the mother/wife had sought the issuance of a writ in the nature of habeas
corpus for the release of her minor daughter who was alleged to be in the custody of her husband.
The husband was an Australian citizen and the petitioner later joined him in Australia. Out of the
wedlock, a girl was born. Later on, the couple developed matrimonial differences which led to
their separation. The parties arrived in India and by a foul play the child was taken away by
husband/father when the petitioner had gone to her parental village. It was further contended by
the petitioner that the husband, instead of acceding to the request of the petitioner to handover the
child, started threatening her and the petitioner fearing her safety, fled back to Australia. She filed
a petition for the custody of the minor child in the Federal Circuit Court, Australia and the court
had passed an interim order directing the husband /father to return the minor child to Australia.
On the other hand, the husband submitted that the petitioner was involved in a relationship with 
his brother-in-law which had led to marital discord between the parties. The local Panchayat
intervened, and it was agreed that as the petitioner had permanent residency in Australia, the
custody of the child would be handed over to the husband. He further submitted that after her
return to Australia, the petitioner had preferred an application for the custody of the child and in
the application, the Australian address of the husband had been mentioned although she knew
that he along with their child was in India. Relying on the judgment \textit{Ranbir Singh v. Satinder
Kaur Mann},\footnote{2006(3) RCR (Civil) 628} the husband submitted that a decree, which had been obtained from a foreign court on the basis of a fraud would not be enforceable in India.

\noi
\textbf{According to the} Bench the principle of comity of courts had been followed by the Courts in
India to honour and to show due respect to the judgments obtained by the Courts abroad.
However, the judgment of a foreign court could not be the only factor while considering the issue
of custody of a child to a parent. The Court referred on the decision of Supreme Court in \textit{Yashita
Sahu v. State of Rajasthan,} wherein the bench had held that \textit{in the fast changing world where
adults marry and shift from one jurisdiction to another there are increasing issues of jurisdiction
as to which country’s courts will have jurisdiction. In many cases the jurisdiction may vest in two
countries, though here also the interest of the child is extremely important and is, in fact, of
paramount importance, the courts of one jurisdiction should respect the orders of a court of
competent jurisdiction even if it is beyond its territories. When a child is removed by one parent
from one country to another, especially in violation of the orders passed by a court, the country
to which the child is removed must consider the question of custody and decide whether the court
should conduct an elaborate enquiry on the question of child’s custody or deal with the matter
summarily, ordering the parent to return the custody of the child to the jurisdiction from which
the child was removed, and all aspects relating to the child’s welfare be investigated in a court
in his/her own country.}

\noi
Accordingly, the custody of the girl child was handed over to the petitioner. However, the
petitioner was directed to arrange interaction of the child with the father regularly through video
conferencing and the parties were directed to abide by the orders of the Federal/Family Court in Australia. The statute indicates a preference for the mother, so far as a child below five years is
concerned.

\noi
In another case\footnote{Meenakshi v. State of U.P., 2020 SCC OnLine All 1475, decided on 02-12-2020} a petition was filed for a writ of habeas corpus, instituted by Master Anav’s
mother, the first petitioner, asking the Court to liberate the minor from his father’s custody by
entrusting the minor into hers, is about a young child’s devastating world. Petitioner 1 states that
during her stay with her husband, she was tortured physically and mentally, both. Her mother
even gave dowry. Later, petitioner 1 realised that her husband had an amorous relationship with
her sister-in-law and another girl from the village to which she objected in vain. She was even
forced to abandon the marriage and go back to her mother’s home. The discord between parties
was mediated and finally ended in mutual divorce. Further, it was stated that the 1$^{\rm st}$ petitioner
after the above settlement went back to her mother’s home along with her young son. After some
time petitioner 1 claimed that there was an unholy alliance between her brother and her estranged
husband to oust her minor son from her mother’s home. The 1$^{\rm st}$ petitioner was beaten up and son was taken away because he thought that she may claim a share for her son in her ancestral property.

\noi
The court decided that since the child was of tender years, he is not capable of expressing an
intelligent preference between his parents, in whose custody, he would most like to be. Also, the
Court noticed is the fact that the father is not, particularly, interested in raising the minor. The
above-stated discloses the disinclination of the father to bear a whole-time responsibility for the
minor’s custody and the complementary inclination of the mother to take that responsibility.

\noi
The Supreme Court Decision in \textit{Ratan Kundu v. Abhijit Kundu,}\footnote{(2008) 9 SCC 413} wherein it was held that A court while dealing with custody cases, is neither bound by statutes nor by strict rules of evidence or procedure nor by precedents. In selecting proper guardian of a minor, the paramount consideration should be the welfare and well-being of the child. But the general rule about custody of a child, below the age of five years, is not to be given a go-by. If the mother is to be denied custody of a child, below five years, something exceptional derogating from the child’s welfare is to be shown. It was noted that nothing on record was placed where it could be stated that the
mother was unsuitable to raise the minor. But since the child needs both the parents, he must have
his father’s company too, as much as can be, under the circumstances. The Court must, therefore,
devise a suitable arrangement, where the minor can meet his father and have sufficient visitation
while the minor stays with his mother.

\noi
In an Allahabad High Court judgment,\footnote{Shaurya Gautam v. State of U.P., 2020 SCC OnLine All 1372, decided on 10-11-2020.} it was decided that the minors not be given into the
father’s custody who has instituted the instant petition. Even if the father is a natural guardian
but faces criminal charges relating to death of spouse, the custody of children or visitation rights
cannot be granted to the natural guardian. In the present matter, Court stated that the custody
which is given currently cannot be termed as unlawful. The custody is with the grandmother of
the minors’ who has been given custody in the presence of the Station House Officer. The father
of the minors’ could say that being the natural guardian of the two minors’ he has the right to seek
their custody from the grandmother.

\noi
It is precisely this right which the father asserts, by virtue of Section 6 (a) of the Hindu Minority
and Guardianship Act, 1956. He says he is the sole natural surviving guardian, and therefore,
entitled to the minors’ custody. It is, no doubt, true that the father is the minors’ natural guardian
under Section 6 (a) of Act, 1956, but the issue about the minors’ custody is not so much about the
right of one who claims it, as it is about the minors’ welfare.

\noi
The issue of welfare of the child cannot be mechanically determined. It is to be sensitively
approached, taking into consideration both broad and subtle factors that would ensure it best. The
totality of the circumstances on record shows that unless acquitted, it would not be appropriate to
place the two minor children in their father’s custody.

\noi
Bench held that the father is not entitled to the minors’ custody when he is facing criminal charges.
Once he is acquitted, it would be open to him to make an appropriate application seeking their
custody to the Court of competent jurisdiction under the Guardians and Wards Act, 1890. In the totality of the circumstances obtaining for the present, this Court did not find it appropriate to
grant any visitation rights to the father.

\noi
In Sashanka v. Prakash case\footnote{2020 SCC OnLine Bom 3497, decided on 27-11-2020.} it was decided by Bombay High Court that Welfare of child as
paramount consideration and the custody given to father of minor for mother not being able to
take care of the child.

\noi
In an another case, the Court decided in Faisal Khan v. Humera\footnote{2020 SCC OnLine Del 572, decided on 1-5-2020.} that Second marriage of a mother is by itself not sufficient to deprive her of custody of her biological child.

\noi
In S.K. Rout v. Ministry of Health and Family Welfare, Union of India,\footnote{2020 SCC OnLine Del 575, 05-05-2020} the SC in this case has coined a new term ‘mirror Order’\footnote{“The mirror order is passed to ensure that the courts of the country where the child is being shifted are aware of the arrangements which were made in the country where he had ordinarily been residing. Such an order would also safeguard the interest of the parent who is losing custody, so that the rights of visitation and temporary custody are not impaired.”} which stresses on interjurisdictional child custody matters.
\noi
Mirror orders are passed to safeguard the interest of the child who is in transit from one
jurisdiction to another. The primary jurisdiction is exercised by the court where the child has been
ordinarily residing for a substantial period of time and has conducted an elaborate enquiry on the
issue of custody. The court may direct the parties to obtain a “mirror order” from the court where
the custody of the child is being shifted. Such an order is ancillary or auxiliary in character, and
supportive of the order passed by the court which has exercised primary jurisdiction over the
custody of the child. In international family law, it is necessary that jurisdiction is exercised by
only one court at a time. These orders are passed keeping in mind the principle of comity of courts
and public policy.

{\normalsize\bfseries {Factors Considered by the Courts when Granting Custody.}}

\noi
The welfare of the minor is very broadly defined and includes many diverse factors, notably:

\begin{enumerate}
\item Apart from the age, sex and religion of the minor, Courts consider the personal law of the father. The welfare of younger children is generally regarded as being in the mother's custody.

\item The character and capacity of the proposed guardian, Courts usually reject baseless
allegations against mothers. The wishes, if any, of a deceased parent, for example
specified in a will is taken into consideration.

\newpage

\item Any existing or previous relations of the proposed guardian with the minor's property,
Courts do not look kindly on guardians seeking custody just in order to have control over
the minor's property. But if, for example, the minor's property is shared with the mother
and she is otherwise a suitable guardian, the Court will regard the property relationship as
an additional factor in the mother's favor.

\item The minor's preference if she/he is old enough to form an intelligent preference, usually
accepted as about 9 years old. Courts prefer to keep children united and award custody of
both to either the mother or the father.

\item Whether either/both parents have remarried and there are step-children, Although the
mother's remarriage to someone who is not the children's close blood-relative often means
the Court will not grant her custody, this rule is not strictly followed. Although the father's
remarriage usually denies him custody, sometimes the Courts agree to grant him custody
especially when the children's step-mother cannot or will not have her own children.

\item Whether the parents live far apart, Courts sometimes do not give the mother custody
because she lives very far away from the father who is the natural guardian. But in 1994
an Uzbek woman living in Uzbekistan was given custody; the judge said modern transport
had shortened distances and meant that the father could depart from his home in the
morning and return by evening.

\item The child's comfort, health, material, intellectual, moral and spiritual welfare this very
broad category includes the adequate and undisturbed education of the child.

\item However, the mere fact that the mother is economically less secure than the father, or that
she suffers from ill-health or a disability is not usually reason enough to deny her custody
because maintenance is the father's responsibility irrespective of who holds custody.

\item The mental and psychological development of the minor should not be disturbed and the
parents and the courts must maintain status quo; Courts will take into account the likely
impact of a change in guardians and the child's reaction to this change.
\end{enumerate}


\vspace{-.4cm}

\heading{Conclusion}

\noi
The legislation has not changed much when considering the ‘custody of children’ provisions. but
its good that the Supreme Court has given new dimensions to the child custody matters. It is
righteous that the mothers are not looked by the Courts from the lens of character, financial
stability, distance, career-oriented mothers, we have come a long way from Geeta Hariharan case.
The supreme is expanding its arenas and delving into new facets and incorporating the new socio
and legal changes happening in the society. The custody is given to mothers inspite of issues
relating to extramarital affairs, issues of long distance or mothers with financial stability. The
fathers have also been given custody inspite of what law says.\footnote{Sec 6 (a) Hindu Minority and Guardians Act, 1956.} The only consideration now stands is ‘interest of child’ and not much law has been looked into.

\heading{References:}

\vspace{-.3cm}

\begin{enumerate}

\itemsep=0pt

\item The Hindu Marriage Act 1955

\item The Special marriage Act 1956

\item The Guardians \& Wards Act 1890

\item The Hindu Minority and Guardians act 1956.

\item The Indian Divorce Act, 2001 (amended) 
\end{enumerate}

\vspace{-.3cm}

\heading{Websites:}

\vspace{-.3cm}

\begin{enumerate}
\item \url{http://www.legalserviceindia.com/article/l34-Custody-Laws.html#sthash.4d9CTUr0.dpuf}
\end{enumerate}
\end{multicols}
\label{end2018-art3}
