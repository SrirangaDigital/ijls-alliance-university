\setcounter{figure}{0}
\setcounter{table}{0}
\setcounter{footnote}{0}

\articletitle{Teaching Criminal Law With Latest Developments : A Case for Inclusion of Corporate Crimes and Corporate Manslaughter}
\articleauthor{Benarji Chakka\footnote{Professor of Law and Dean (Interim), Alliance School of Law, Alliance University, Bangalore, previously British Chevening Scholar (2015-16) at School of Oriental and African Studies (SOAS), University.}}
\lhead[\textit{\textsf{Benarji Chakka}}]{}
\rhead[]{\textit{\textsf{Teaching Criminal Law With...}}}

\begin{multicols}{2}

\heading{Abstract}

\noi
Teaching criminal law in a law school is a unique experience to a teacher must have
the required skills to teach the subject in a most unique manner. Teaching and
learning criminal law gives a very fascinating experience by involving the students
actively in a classroom environment. However, today in a law school set up most of
the teachers are equipped to teach the subject based on the syllabus prepared or
prescribed by the university authorities and Bar Council of India’s model syllabus. In
this context, it is an important question whether the prescribed syllabus gives a space
and opportunity for the law teacher to teach in an innovative manner while keeping
abreast with the challenges and development, which are posed by society. It is
perceived that teachers do not have space and opportunity to include the on going
developments and challenges to teach the subject in a more contemporary manner.
This may involve various factors, which may include a lack of opportunity for the
teachers to prepare the syllabus by themselves while engages to teach the students.
Lack of expertise on the subject and not having an opportunity to teach the same
subject continuously. Some of these are only indicative, and many other reasons are
available in the Indian legal education system. The current essay is an attempt to
study some of the pertinent issues and suggest teachers of criminal law to be more
innovative and adoptive to the changes, which are taking in society.

\vspace{-.4cm}

\section{Teaching Criminal Law – An Introduction}\label{section-1}

\vspace{-.2cm}

\noi
Teaching criminal law in law schools is a unique and interesting experience for a law
teacher. For Students, learning criminal law in a law school is a fascinating
experience, because most of the students come with some basic ideas, preconceived
notions and real-time experiences based on the day to day events witnessed by them,
general reading of news papers and watching some movies. In the midst of having a
basic and pre-knowledge, teaching criminal law to the students is certainly a
challenging task and the teacher has to play a key role in imparting knowledge and
realising the dreams of students. Further, the task of a teacher is to prepare students to
become committed lawyers, law teachers, judicial officers, civil servants and political
leaders. Once a committed teacher imparts comprehensive knowledge, better and
informed ideas of criminal law to the students of law, they may not go away from the
ideals, which have been taught in the classroom. Therefore it is a challenge for the
law teacher to teach and impart the knowledge of criminal law in a law school.
Further, the prime object and purpose of teaching substantive criminal law are to
produce students capable of representing clients and practising with social
commitment in order to translate the ideals of the constitution into practice. Sanford
H. Kadish, the American Criminal Law jurist describes that “the aim of teaching criminal law as to produce ‘good, sensitive, aware, socially conscious’ citizens”.\footnote{Angela P. Harris and Cynthia Lee, “Teaching Criminal Law from a Critical Perspective”, \textit{Ohio State Journal of Criminal Law,} Vol. 7 (2009), p. 264.} It is also worth mentioning the words of Andrew E. Taslitz, professor of criminal law that “the joy of teaching is part of what makes a law professor’s life so fulfilling”.\footnote{Andrew E. Taslitz, \textit{Strategies and Techniques for Teaching Criminal Law,} Wolters Kluwer Law \& Business, New York, 2012}

\vspace{-.3cm}

\section{Teaching Techniques and methodologies}\label{section-2}

\vspace{-.2cm}

\noi
As stated above teaching criminal law to undergraduate students is a unique
experience, first we need to question us, what is unique about the teaching of criminal
law? It is unique because it has relevance in everyday life, while dealing with the
uniqueness of the criminal law; it is not an attempt to underrate or undermine the
importance and relevance of other subjects or to overrate the criminal law. The
uniqueness of the criminal law teaching can be easily related to each and every aspect
of human life and bring on going development and contemporary challenges to the
classroom discussion while teaching criminal law and make more interesting to the
students. This subject can easily create interest among the students in a classroom
learning process by analysing and critically examining with the participation of
students. Apart from that, a law student can think and relates the subject easily to their
real-time experiences and take good practices by reading of the subject seriously,
which eventually facilitate them to become an outstanding criminal lawyer. Further
students can critically examine the scientific and technological developments and
their influence in society and the role of criminal law.

\noi
In the light of this background, this current essay focuses to suggest the law teachers
who are teaching criminal law to bring some of the on going challenges and
developments in society while teaching criminal law in law schools. The current
development may vary from corporate manslaughter or corporate homicide, sedition
laws, the law relating to limiting freedom of speech and expression, criminalising
expression of art forms, socio-economic crimes, crimes committed online using
cyberspace, growing crimes like mob-lynching and many others. These developments
have to be brought to the classroom while teaching criminal law to the students. It is
very important to note that the Indian Penal Code was adopted in 1862, since then not
many developments and changes are brought to the Penal Code, except the Criminal
Law Amendment act of 2013, which was made substantial changes in the criminal
law in the wake incidents, which took place in Delhi.

\noi
However, criminal law requires keeping abreast with changes taking place in the
wake of economic globalisation. These changes are prompted by the developments,
which took place in 1995. In the post liberalised economic world, big corporations
and multinational companies are expanding their business across the shores. The
business and investment activities are facilitated by various bilateral and multilateral
agreements, which give concessions and subsidies to the corporation while they are
establishing their entities. Whereas, these corporations during their operations in third
world countries, they work directly with the local communities and they do involve in
the exploitation of natural resources. In some places, while engaging in exploitation
by the corporations they experience resistance from the local populations. During 
such circumstances, the company or corporation may involve in using force to the
extent of violating the basic rights of people, killing and abuse of people. It is not only
limited to it but also includes environmental violations and pollutions of the
environment and climate in the region, fuelling the conflict by directly sponsoring the
non-state armed actors and many others. These may lead to committing crimes, which
are otherwise prohibited by the laws. However, in circumstances, criminal law and
other domestic law are silent in third world countries including India. It is warranted
that the Indian criminal must look into it and take the issues and circumstances which
lead to the violations of the rights of the people which happened in the Bhopal gas
leak tragedy and other industrial disasters.

\noi
Now the argument is that it is very important to look into these challenges in changing
circumstances and include them in the teaching of criminal law in the classroom. It is
an urgent necessity the Indian academia must focus on these aspects and take steps to
address and engage the students and make them aware of these issues.

\vspace{-.3cm}

\subsection{Teaching General principles of Criminal law in relation to other public law\\ subjects}\label{subsection-2.1}

\vspace{-.2cm}

\noi
As we are all aware, most substantive law for that matter teaching of public law
discussion revolves around elements of law, in the case of teaching criminal law, it
should start by imparting ideas about the general principles of criminal law or
elements of a crime. Because, most of the students tend to take interest in learning
criminal law due to several reasons. Teaching elements of a crime or general
principles of criminal law play a more critical role in learning criminal law, these
rules explain applicable rules to all crimes and the general nature of the major crimes.
This discussion also reveals around the critical linkages between public law domains
such as criminal law and constitutional law. For that matter, most of the offences in
criminal law have been designated where the offenders violate the substantial rights
of the people and the state has the duty to protect the basic fundamental rights of the
citizens. Once the basic right is violated and it will take a convenient place in criminal
law. It is important here to recollect the definition of crime provided by Blackstone,
he defines crime as “an act committed or omitted in violation of public law forbidding
or commanding it” this definition will cover the constitutional or political offences.

\noi
This definition attracted several criticisms, in response to which, he developed a
modified version of the definition, he provides, "a crime is a violation of the public
rights and duties of the own community". According to the improved version of the
definition, "crime is a violation of public right or duty". This definition makes it clear
that criminal law is designed to punish the offenders who violate the public rights of
the people. It is also important to note that the state represents the victims in the
criminal justice system, since state has an obligation to protect the rights of
individuals in the society.

\noi
Based on the above definition and analysis it is pertinent that a law teacher while
teaching criminal law necessarily bring the relevance of constitutional law. Without
teaching the close linkages between the protections of constitutional rights in the light
of criminal law mechanisms, students may not appreciate the efforts and relevance of
the study of criminal law in law schools. Hence, it is an important duty of the law
teacher to impart the criminal law knowledge to suits the requirements of the constitutional law obligation. Further, it equips and prepares the students adequately for practice in criminal law.

\vspace{-.3cm}

\subsection{Teaching Criminal law in broader\\ perspectives}\label{subsection-2.2}

\vspace{-.2cm}

\noi
There is no uniform set pattern in teaching criminal law. It is depending upon the
background in which one person comes from and the skill, which the person
possesses, makes the teaching difference. The methodology also differs from culture
to culture and country to country. However, one useful style or method could be
providing coherence and logical progression makes a big difference in understanding
of the criminal law in general. It should be substantiated by providing or discussing
some issues, which includes, the relevance of criminal law, purposes and object of
criminal law in a legal system, steps in the criminal justice system, the purposes and
object of sentencing in the legal system and teaching criminal law keeping abreast
with the latest changes and challenges.

\noi
If we take the Indian criminal law or Indian Penal Code into consideration in the light
of the above discussion, it is very important to see the emergence of the Indian Penal
Code (IPC) and its relevance in contemporary society. As we know that IPC was
adopted in 1860 during the British colonial regime to suits the requirements and
political needs. This step was hailed and glorified as the unification of criminal law in
India to render effective criminal administration of the justice system to the Indian
masses. Since, 1862 the IPC applied and followed without making any changes or
amendments until this day, except in the case of the criminal law amendment act of
2013 as illustrated above in the earlier paragraphs. In this situation, it is very pertinent
to see the relevance of IPC in contemporary use. It is also warranted to see the
application of certain rules, which have been there in IPC to the free and independent
India, where free, and Independent India is governed by the principles and rules of the
Indian Constitution. As we are aware, the Indian Constitution is based on the
fundamental principles of Rules of Law and Democratic Governance.

\noi
Further, the governance in India has been changed drastically since its independence,
due to changes in the international economic climate and onward march to
globalisation. In the 1990s, when India has signed and ratified World Trade
Organisation, the ideals on which Indian Constitution is based were more or less
compromised. In the age of neo-liberalism, corporations and private business houses
have dominated the governance and they are having direct contact with the people in
certain areas. In such circumstances, if corporations have committed certain crimes or
if they are found involved in certain industrial disasters, what would be the ideal
mechanism and solution to address the situation?

\noi
Since 1980s India has been witnessing industrial disasters along with it, on going
rights violations of people by the corporations in certain areas. These rights violations
and industrial disasters have not been addressed either by the general laws or by the
criminal law mechanism in India. There is a conspicuous gap existed in this situation.
Otherwise, in some developed economies these incidents have been termed as
“Corporate Manslaughter or Corporate Homicide” and legal regulations are designed
to address them effectively. The other issue along with corporate homicide or
corporate manslaughter, there are mass killings and mass murders of innocent
individuals in communal riots, which have been widespread in India since its 
independence. People who are victims of mass murders and communal riots are still
languishing to seek justice in the administration of the criminal justice system in
India.

\noi
Therefore, the law teacher while teaching criminal law to their students such kind of
gaps has to be identified and address them in an effective manner. It is also required
to encourage the students to conduct advanced study and research in such areas in
order to address the existing gaps and challenges. Now, here we will discuss issues
relating to corporate manslaughter or corporate crimes and the requirement of
amendments in Indian criminal law.

\vspace{-.3cm}

\section{Corporate Crimes\protect\footnote{Corporate Crime means, "An illegal conduct that is linked to a human rights abuse, including conduct
that should be criminalized in order to meet requirements under international law even if the state has
failed to do so." See Justice Ian Binnie and Anita Ramasastry, Advancing Investigations and
Prosecutions in Human Rights Cases: A Report of the Members of the Independent Commission of
Experts, Amnesty International and International Corporate Accountability Roundtable. \protect\url{http://www.commercecrimehumanrights.org/wp-content/uploads/2016/10/CCHR-0929-Final.pdf} Available accessed on 18th June 2018} and Corporate\\ Manslaughter}\label{section-3}

\vspace{-.2cm}

\noi
The general principles of criminal law with regard to offences against the human body
provide serious homicide offences against the human body are murder and
manslaughter. As we know the unjustified killing of another human being has
traditionally been regarded as the most serious offence known to the law.
Manslaughter is an unlawful killing that does not involve malice aforethought
(blameable mental condition). The absence of malice aforethought means that
manslaughter involves less moral blame than either first or second degree of
murderThus, manslaughter is a serious crime, the punishment for manslaughter is
generally lesser than it would be for murder.

\noi
Further, we are aware that ‘murder is an intentional killing that is unlawful (in other
words, the killing is not legally justified), andcommitted with "malice aforethought"
(intent to harm or kill or reckless disregard towards life). The offence of murder and
manslaughter have a common physical act \textit{(actus reas)} which is the unlawful killing of
another human being.

\noi
Therefore, based on the above principles if a natural person commits an act which is
intentional or reckless disregard towards life it would be prosecuted. In such
circumstances, if a legal or juridical person commits the same offence how should
they be treated under law or not. Should that be treated as an offence equal to that of
murder or manslaughter. As it is discussed, if there is blameable mental conditions
and there is an unjustified killing of another human or reckless disregard to human
life, that should be treated as manslaughter. In Indian criminal law, there is no such
provision to deal with an offence of manslaughter, hence, there is an urgent need to
address this problem by amending criminal law.

\noi
As we all know that globalisation has accelerated economic growth by permitting
unbridled access to multinational companies or corporations to boost the economy.
These multinational companies act across borders with ease due to development in technology as well as favourable, trade and investment laws, they exercise significant
power and influence in the countries where they operate. Sometimes they do get
contacted directly with people and commit wrongful acts which, normally, go
unnoticed and sometimes the legal system may not have sufficient rules to bring them
to justice. This situation arises due to various reasons based on legal and non-legal
issues. These issues may include, legal personality, limited shareholder liability and
accountability, origin and registration of corporations in offshores and operating
through a local subsidiary or joint venture in the host states. This conspicuous
governance gap has created rather facilitated an environment in which the corporate
actors commit serious human rights abuses with little or no accountability and with
impunity.\footnote{Justice Ian Binnie and Anita Ramasastry, \textit{Advancing Investigations and Prosecutions in Human Rights
Cases: A Report of the Members of the Independent Commission of Experts,} Amnesty International and International Corporate Accountability Roundtable. Available \url{http://www.commercecrimehumanrights.org/wp-content/uploads/2016/10/CCHR-0929-Final.pdf}\\ accessed on 18th June 2018}

\vspace{-.3cm}

\subsection{Nature of Corporate Homicide or\\ Manslaughter}\label{subsection-3.1}

\vspace{-.2cm}

\noi
In developing countries like India, there is no such efforts have been made to define
the offence of corporate homicide or manslaughter. However, the developed and
industrialised countries have adopted laws to address the problems of corporate
homicide or manslaughter. The common law country like the United Kingdom has
adopted in 2007 the corporate manslaughter and corporate homicide act. The acts
explain the offence:\footnote{Corporate Manslaughter and Corporate Homicide Act 2007, available at \url{http://www.legislation.gov.uk/ukpga/2007/19/pdfs/ukpga_20070019_en.pdf}, accessed on 19$^{\rm th}$ June 2018}

\vspace{-.45cm}

\noi
\begin{quoting}
An organisation is guilty of an offence if the way in which its activities are
managed or organised causes a person's death, and amounts to a gross breach of
a relevant duty of care owed by the organisation to the deceased.
\vspace{-.1cm}
\end{quoting}

\vspace{-.6cm}

\noi
\begin{quoting}
The organisation to which this section applied are: a corporation, a department
or other body listed in its schedule or a police force apartnership or a trade
union or employers associations that is an employer.

\vspace{-.15cm}

\end{quoting}

\vspace{-.7cm}

\noi
\begin{quoting}
An organisation is guilty of an offence under this section only if the way in
which its activities are managed or organised by its senior management is a
substantial element in the breach
\vspace{-.1cm}
\end{quoting}

\vspace{-.5cm}

\noi
This definition offer a comprehensive analysis of corporate crimes or homicide and
which may range from industrial disasters such as Bhopal gas incidents, to abuse
of child labour, human trafficking, abduction, arbitrary detention, beating and
violence, complicity, death threats, denial of freedom of expression, denial of
freedom of movement, disappearances, displacement, genocide, injuries
intimidation and threats, rape, sexual abuse and sexual harassment, torture and illtreatment. The list of the crimes provided above is not an exhaustive but to name a
few crimes.

\noi
Some of the crimes, which are listed above, have been included in special
legislation but enforcement requires the will of the state to cooperate and
coordinate with international actors. For example in transnational human
trafficking, cybercrimes, transnational organised crimes, money laundering and
corporate manslaughter needs international cooperation and also it is important
states must comply with the obligations of the international treaty mechanism. In
absence of international cooperation prosecution of the crimes of the above nature
becomes challenge. These are some of the issues that must be brought to the
knowledge of the students in the classroom while teaching in criminal law.

\vspace{-.5cm}

\subsection{Corporate Crimes and Possible Legal\\ Solutions}\label{subsection-3.2}

\vspace{-.3cm}

\noi
The commission of the above-mentioned crimes such as corporate homicide and
corporate manslaughter have witnessed by the international community during the
World War II. During wartime, some of the senior officials of business houses and
corporations were heavily involved in helping the Nazi regime to commit some of
the worst crimes by supplying poisonous gas to concentration camps. Some of the
corporations actively promoted and engaged in slave labour to work in their
factories, some of them have ill treatment of slave workers and accumulated and
enriched their companies by plundering property in occupied territories.

\noi
This sort of criminal activities committed by corporations did not stop or limited to
the World War II, it resurfaced in the age of neo-liberalism where transnational
companies have committed brutal atrocities and knowingly assisting governments,
and rebel armed groups and other non-state armed groups to commit gross human
rights abuses.\footnote{A Report of the International Commission of Jurists Expert Legal Panel on \textit{Corporate Complicity in
International Crimes: Facing the Facts and charting a Legal Path,} vol. 1, International Commission of Jurists, Geneva, 2008.} Oil, mining companies from the advanced countries that seek concessions and security from the third world countries have been allegedly involved in supplying arms, ammunition, money, vehicles to the government as
well as armed rebel forces. These state actors and non-state armed actors with the
help of transnational corporations committed systematic violence and killed
civilians and were involved in the disappearance of civilian populations. These
abuses are keeping on happening in most of the third world countries including
India.

\noi
In the current economic climate of interdependence, social, political and cultural
aspects are heavily impacted by the behaviour of business entities. The empirical
evidence reveals that there exists a complex relationship between businesses,
individuals, communities and governments. Further, it is understood this is the
kind of nexus between government and multinational corporations greatly
affecting the lives and rights of individuals. As described above, the companies
mostly enjoy the culture of impunity, due to the considerable political influence,
which they have on the host government. Many of the corporations have developed
close political relationships with those in power, including governments or nonstate armed groups that perpetuate human rights abuse and commit criminal
activities against the local population and innocent people.

\noi
In this background, it is pertinent to understand the close nexus between the
multinational corporations and the political corridors in the country. It is also
pertinent to understand the operation of the legal framework, which can offer
better protection to the individual those who are victims of the current system.
Hence it is urged that the criminal law teacher must be aware of the current and on
going challenges and to provide space for the student to debate and discuss them in
the classroom. It is also warranted that the teachers should motivate themselves
and students to conduct advanced research on these areas, which is very pressing
and to locate solutions to the emerging challenge in order to fill the gaps in the
existing criminal law mechanisms.

\noi
Further, the criminal law teacher is expected to clarifying the legal and policy
meaning of new age crimes such as mass murders, corporate manslaughter and
corporate homicide to the students. It is also expected that to explain to the
students that the complicity on part of the governments and corporations in
addressing these crimes within the legal framework. Along with the governments,
business groups enjoy the culture of impunity and this aspect needs to be informed
to the students and encourage them to engage in research to address the measures,
which can end the culture of impunity within the jurisdictions. If it possible engage
students to the extent of writing research briefs and policy documents, so that these
briefs and documents may be submitted to the appropriate forums to undertake
necessary changes in the legal system or amendments which are required.

\vspace{-.3cm}

\section{The way forward}\label{section-4}

\vspace{-.2cm}

\noi
As elucidated above, the Indian criminal legal system is not providing many
solutions to emerging and new challenges. As we have witnessed since 1980s, the
Bhopal gas victims still waiting for justice, along with that victims of the state
perpetrated violence in places like Chhattisgarh, victims are knocking on the door
of the apex court, but their voices have not been heard and justice has not been
done to them within the constitutional mechanisms. Now the time has come for the
law teachers and students to start thinking on these lines to conduct advanced study
and research to provide justice to the victims of corporate crimes. Further, it is the
role of a teacher to make aware of the students to such kinds of gaps, which are
existed in the criminal law subject and encourage them to look for alternatives in
order to strengthen the criminal legal system. Finally, research blended teaching
would help the teacher to update their knowledge and to provide and to
disseminate the advanced knowledge to the students. This would certainly make
students and new generations of individuals and professionals aware of the new
challenges and also provide a platform to think and equip them to fight for the
rights and justice to the victims. It is also important to note that criminal law
teachers must updates their knowledge and contribute research articles in order to
create awareness among the legal fraternity. This may lead to a greater debate
among law students, legal scholars and judiciary and policymakers. These debates
may eventually help in the inclusion of the above-mentioned acts in the discourse
and pedagogy of criminal law.



\end{multicols}
