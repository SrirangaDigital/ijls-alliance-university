\setcounter{figure}{0}
\setcounter{table}{0}
\setcounter{footnote}{0}

\articletitle{Media Debates, Asymmetrical Argumentation, and the Nature of Violence in Indian Politics}
\articleauthor{Keerthiraj\footnote{Assistant Professor Alliance School of Law, Alliance University.}}
\lhead[\textit{\textsf{Keerthiraj}}]{}
\rhead[]{\textit{\textsf{Media Debates, Asymmetrical Argumentation...}}}

\begin{multicols}{2}

\heading{Abstract}

\noi
Dialogue is often considered as a solution to reduce violent conflicts between different competing
parties. Applying this logic of dialogue to reduce the possibility of violence holds validity. On the
other hand, empirical evidence show that dialogues also resulted in violent conflicts, instead of
ameliorating the situation. This paper focuses on this unique problem of applying dialogue as a
remedy to violent conflicts in India with a special reference to media debates. Extended media
including both mainstream and social media provided large space for dialogues regarding issues
in the contemporary world. India is no exception to this fact. This paper critically analyses this
entire discourse of media debates on contentious issues in India to test the hypothesis regarding
the validity of dialogue as a remedy to prevent violence and chaos in Indian political context.

\noi
\textbf{Key Words:} Dialogue, Media, Indian Politics, Conflict, Violence, Post-colonial

\heading{Introduction}

\noi
Contemporary society has a strong conviction that the dialogue is the finest solution to any kind
of difference. There is a flaw in this general perception of dialogue as an effective remedy to
violence in social, political, or cultural domains. This paper considers empirical evidence from
the socio-political and cultural experiences in India to show the ineffectiveness of dialogue in
addressing the problem of violence. Majority of the examples prove that dialogues not only fail
to resolve conflict/violence but also act as a complementary force to provide a space to violence.
The general claim that dialogue produces remedies to conflicts or claims that process of dialogue
itself as the solution of violence etc. should be verified with objective inquiry. This paper
specifically considers media debates to verify this hypothesis regarding the relationship between
dialogue and violence. To understand this contradiction, one needs to frame conceptual
understandings regarding the core issues involved in this area like nature and presumptions of
dialogue, nature of violence in Indian socio-political setup, perception shaping in India, nature of
media dialogues in India and other related issues.

\heading{Research Problem \& Research Question}

\noi
Generally, dialogue is an effective solution to prevent violence in any socio-political setup. But
the empirical evidence from Indian socio-political reality shows that dialogues not only failed to
establish peace but also worked to reinforce the elements of violence among competing parties.
In this context, this paper examines the question, \textbf{why media debates increase the possibility of
violence in India, even though dialogue is much paraded solution to violence?}

\heading{Hypothesis}

\noi
Intense media debates directly contributed to the increase in violence as dialogue in Indian society
fails to provide space for symmetrical argumentation.

\heading{Objectives of the Study}

\vspace{-.2cm}

\begin{enumerate}[label=$\bullet$]
\itemsep=0pt

\item To understand the effectiveness of dialogue in settling violence in Indian society.

\item To analyze the nature of violence in Indian socio-political system.

\item To evaluate the role of media debates in Indian public discourse.
\end{enumerate}

\vspace{-.2cm}

\noi
{\normalsize\bfseries Scope of the Study}

\noi
This research paper deals with the media debates and its impact on Indian society with a special
inquiry of the effectiveness of dialogue in settling violence in Indian socio-political scenario.
Even though the paper deals with the larger issues like ‘media’ and ‘Indian Socio-political
scenario’, focus of the study will be limited to investigate the relevance of dialogue in preventing
violence as applicable to these larger discourses.

\noi
{\normalsize\bfseries Methodology}

\noi
This research paper will make efforts to test the hypothesis and to achieve the objectives of the
study through different methods. Primarily, the proposed study will adopt theoretical, historical,
descriptive, and analytical study design. The study depends on secondary sources of data,
information, and literature.

\noi
{\normalsize\bfseries Discussion}

\noi
The discussion part of this paper is organized into four parts to address the research problem and
to verify the hypothesis offered by this paper. The first part ‘Understanding the nature of
Dialogues in India’ deals with the basic conceptual understandings about the nature of dialogue
in India. Second part, ‘Asymmetrical argumentation and the Source of Violence’ shows the core
problem within the process of dialogue which makes obvious space for conflict. Third part, ‘Flaws
of Universal Rationality and the Problems of Attribution’ critically throw light on the problematic
use of reason in dialogue and the unjustifiable imposition of attributions, which leads to an
asymmetrical dialogue. Finally, ‘Media Debates and the Myth of Preventing Violence’ focuses
on contemporary media debates to justify the claims made by earlier parts of this discussion.

\noi
{\large \bfseries 1. Understanding the Nature of Dialogues in\\ India}

\noi
The empirical evidence on Indian socio-political scenario makes it clear that dialogues in different
forms increased the probability of violence, instead of reducing conflict/violence. In this case a
serious academic research consideration is very much necessary to analyze this contradictory
situation. Particularly, the origin of ‘dialogue’ traces back to the emergence of liberal political
values in European society. This idea of rational dialogue originated and evolved with the events
like enlightenment, reformation etc. in European history. Many scholars considered that violence
can be reduced only with the help of critical reasoning and rational dialogue. On the other hand,
Indian socio-political context is witnessing contradictory results as debates go hand in hand with
conflicts and violence. At the very outset, one can assume that the reason for this unique
experience of India is that scholars without having the original experience of India studied and
built knowledge on Indian socio-political realities. But this assumption fails to explain the
situation because one cannot guarantee a different outcome even if an Indian scholar studies
Indian society, as he will be using the same tools to explain Indian society, the results tend to be
the same. Now the problem lies with the tools used to understand Indian society, to be precise Freudian psychoanalysis involved in dialogues to understand the psyche of Indian society.
(Balagangadhara, 2012)

\noi
Academicians or scholars are trying to initiate a dialogue about ‘Hindu traditions’, ‘culture’,
‘practices’ etc. through their writings, speeches, or any other form of expression. But these
expressions quickly attract a violent rebuttal from the respondent and the further dialogue fuels
more conflict. For example, Paul Courtright’s portrayal of Ganesha in his book ‘Ganesa: Lord of
Obstacles, Lord of Beginnings’ (Courtright, 1989) received harsh responses from Hindu
community (Singh A. , 2009). Academia and scholars termed this response as inimical to
academic freedom connecting it to the rising Hindu Fundamentalism. But the question of dialogue
remains unanswered, if academia thinks Courtright’s effort is to initiate a dialogue with the
respondent community i.e., Hindu, why dialogue created so much of violent results? We cannot
simplify this phenomenon as it is the problem of outsider studying Indian culture, as there won’t
be any difference in the outcome even if an Indian studies about the same case using the same set
of psychoanalytical tools. Other set of arguments limit this phenomenon within the domain of
Hindu fundamentalism also fail to describe the phenomenon, as it questions the whole
effectiveness of dialogue as a panacea to violence/conflict between two competing parties.

\noi
Any effort by academicians, journalists, or artists to interpret an Indian phenomenon becomes a
controversial domain. Religious and cultural domains are specifically more sensitive than other
areas, where a dialogue seems almost impossible. In fact, a comment on Sabarimala issue, Triple
Talaq article 370, JNU, CAA or any other issue is an attempt to initiate the dialogue with the
other party. But, in reality this attempts to initiate a dialogue ends up with death threats, violent
conflicts, chaos etc. These aggressive and violent reactions cannot be and should not be easily
labeled as fundamentalist, anti-academic freedom forces. As this attitude of looking at different
phenomena in a binary vision of black and white yielded us no or negative results in most of the
cases. Now this is the time to relook into the nature of dialogues in India with a critical
understanding about how they work in nonwestern societies.

\noi
{\large \bfseries 2. Asymmetrical Argumentation and the\\ Source of Violence}

\noi
Dialogue has a basic structure of argumentation with loaded assumptions. When western scholar
or even an Indian scholar is initiating a dialogue regarding a phenomenon in Indian context, they inquire the status with some interrelated cognitive moves. These interrelated set of cognitive
moves, psychoanalysis etc. make the argumentation imbalanced between the scholar and subject
of his/her inquiry. In the process of using psychoanalytical tools, the scholar ignores the fact that
the rationality of those tools evolved in a particular context in western philosophical domain and
they are alien to Indian context.

\noi
The paper illustrates this with an example of dialogue between ‘Party A’ (Scholars/academicians
initiating a dialogue about Indian polity/society) and ‘Party B’ (The people who are expected to
answer the claims of ‘Party A’ i.e., Indians). Here ‘Party A’ tries to invoke rational logic of the
other party by attributing some assumptions on the other party through explanations and
interpretations. Dialogue requires several assumptions from one party on the other party to keep
the dialogue alive. Interestingly, the party attributing these assumptions (i.e., Party A) is not
accountable to the assumptions made by it. But on the other hand, ‘Party B’ will be burdened with
the responsibility to prove its actions by continuing the process of dialogue without knowing the
assumptions attributed by ‘Party A’. This uneven distribution of responsibility/ accountability
leads dialogue to the violent end.

\noi
Logical reason and psychoanalytical tools might be appropriate to understand western
psychology, but they appear inappropriate to deal with non-western societies. But logical reason
and psychoanalytical tools become inevitable to have a dialogue. This contradiction in
intercultural encounters make dialogue more problematic with attributed assumption on nonwestern societies (Party B) with an additional burden on them to justify their actions without
understanding nature of assumptions attributed on them. On the other hand, attributor of these
assumptions (Party A) has no onus of providing evidence to their assumptions. In other words,
‘Party A’ escapes from the onus of providing evidence as it can switch between explanation and
interpretation, but ‘Party B’ must stick on to explanation to justify their cause. This asymmetric
argumentation puts ‘Party B’ in an unfavourable situation, as it cannot defend its stand/action
within the dialogical discourse. The situation makes it clear that the dialogue increases the
frustration and anger of ‘Party B’ leading to violence. In this context of violence, a demand for
more dialogue will only bring more violence.

\noi
{\large \bfseries 3. Flaws of Universal Rationality and the\\ Problems of Attribution}

\noi
Asymmetric relationship between two parties in the dialogue makes it difficult to ‘Party B’ to
remain within this discourse of dialogue. This cognitive asymmetric relation enables the ‘Party
A’ to defend their point in the name of psychoanalytical analysis, as this analysis has been used
to understand different issues and religions including Christianity. In this established situation,
any set of argumentations will be inclined in favour of ‘Party B’. This process of attributing
assumptions to the practice of ‘Party B’ makes the dialogue possible, but such attributions won’t
make such sense to ‘Party B’. So, this process also makes sure that ‘Party B’ is not intellectually
fit for dialogue. In this way, tools evolved in a particular context of western history concludes
non-western societies like Africans or Indians as inferior species without even having a basic
understanding of their own socio-political experiences. Thus, ‘Party B’ must counter and question
this core logic of psychoanalysis and the universality of logical reasoning.

\noi
Psychology has an established assumption that rationality is universal, which has its roots in
enlightenment and reformation phases in European history. This psychoanalysis firmly believes
in the relationship between reasons and actions, beliefs, and behaviour etc. Most importantly these
theories of rationality are paraded as universal beyond their philosophical context, where they
really originated. But the failure of dialogue to prevent violence in societies like India has
something significant to say regarding this phenomenon. A reasonable discussion or a rational
dialogue is only possible when two parties stand on the symmetrical position sharing common
sense and common folk psychology. This is true when rational dialogue happens within the
domain of western culture, but it fails miserably in intercultural encounters. Reasonable
discussion will not remain as a neutral mechanism when west is dealing with non-western
societies. In this unique context much celebrated liberal idea of ‘reason’ fails to acknowledge one
more celebrated liberal idea i.e., ‘pluralism’.

\noi
{\large \bfseries 4. Media Debates and the Myth of Preventing Violence}

\noi
Contemporary media debates in India regarding various issues are getting complex day by day
attributing a binary view of Indian society based on the claim of rationality. Media debates and
discussions are representing the conceptual issues of dialogue discussed in the earlier three
sections. Media is on the frontline along with other driving forces like literature, social sciences
etc. in suggesting dialogue among contesting parties to reduce the possibility of violence/conflict. With the advent of internet media/ social media platforms, the space for dialogue is increasing
than ever before.

\noi
No one can deny the space produced by media and especially by internet media in facilitating
dialogue among different communities in India. Internet media played an important role in
increasing the mass participation media in the place of elitism. This also brought a claim of
objectivity to the discussions happening around, but unfortunately increasing the violence. Again,
it is not the elite media, not surely the mass participation in media, nor is the access of internet
the reason for the escalation of violence. Background ideas of different communities involved in
a particular dialogue are producing a black and white binary vision to look at the issues in India
neglecting the original experience.

\noi
There is no dearth of debates as Indian socio-political discourse provides a rich list of debates to
analyze the problems of dialogue. A close observation on media behavior and media debate on
recent contentious issues on Indian society proves the Hypothesis of this paper right. Debates on
Citizenship Amendment Act, National Register of Citizens, National Population Register, women
entry to Sabarimala temple, academic freedom of JNU and other universities, Section 377 of the
Indian Penal Code and many other issues were still being debated pushing communities towards
more hostility. Each of the issues raised here, despite of the differences in their nature meets up
at common point i.e., two communities opposing each other. A focused concentration on these
media debates will tell us one more important fact that the parties involved in these debates are
busy in defending their stand and not really interested in the facts or the actual process and
outcome of that issue. Even when facts are used in such dialogue, such use will have mere purpose
of supporting the stand of respective competing parties. Such debates and dialogues help nothing
but in escalating the violence/conflicts.

\noi
This problem is much bigger than it seems to be. Generally accepted accusations against media
like corporate control, political influences, TRP ambitions can give a temporary relief to the
questions raised here. These reasons look impressive as there are huge communities to believe in
these allegations, when they are made against the media which is on the opposite camp. Right,
Left, Centre, liberal, conservative, feminist, theist, atheist or any other group is not an exception
to this. Media seems objective to a particular group when it suits their narrative and the allegations mentioned above will be reserved for the media, which contradicts their narrative. But these
allegations are only help us to show media as a self-aggrandizing selfish parasite and nothing else.
Only deeper understandings of background ideas which work behind media narratives help us to
go beyond the age-old allegations against media. Otherwise, a push for more dialogue comes with
an increased amount of violence.

\heading{Conclusion}

\noi
This paper is an attempt to explore the contradictory complex phenomena of the interrelation
between dialogue and violence rather than offering a solution to the problem of violence in Indian
socio-political setup. With one voice, academic scholarships, political system, in fact the whole
public discourse suggested dialogue as the solution to violence in India. On the other hand, it is
also true that India failed to achieve its objective through dialogue, as they became more
problematic with the passing time. Even though, this paper lacks enough space, both in technical
and intellectual level to provide concrete solutions to this problem, it provides an abstract if not a
concrete base to relook this phenomenon with new academic rigor. This paper concludes by
opening a vast research gap on the appropriateness of dialogue in media studies, public policy
and other socio-political discourses of India.

\heading{References}

\vspace{-.2cm}

\begin{enumerate}[label=$\bullet$]
\item Burns, L. S. (2013). Understanding journalism. London : SAGE Publications.

\item Chauhan, S., \& Chandra, N. (2005). Modern journalism : issues and challenges. New Delhi: Kanishka Publishers.

\item Courtright, P. B. (1989). Ganesa: Lord of Obstacles, Lord of Beginnings. Oxford: Oxford University Press.

\item Herman, Said, E., \& Chomsky, N. (1988). Manufacturing Consent: The Political Economy of the Mass Media. New York: Pantheon. 

\item Maras, S. (2013). Objectivity in Journalism. Cambridge: Polity.

\item Keerthiraj. (2018). Role of Media in Constructing Feminist Discourse. In B. Salma, C. Dhanalakshmi (Ed.), Higher Education in Digital Era: A Multidisciplinary Approach. Bengaluru: GFGC Anekal.

\item Meyers, C. (2010). Journalism Ethics: A Philosophical Approach. New York: Oxford
University Press.

\item Keerthiraj. (2016, August). Clash of Civilizations Thesis: Some Reflections. Third
Concept: An International Journal of Ideas, 30(354), 10-14.

\item Patterson, T. E. (n.d.). The Corruption of Information. Freedom Magazine. (D. Luzadder, Ed.) Los Angeles, California, US. Retrieved May 15, 2018, from \url{https://www.freedommag.org/issue/201409-under-influence/media-and-ethics/the-corruptionof-information.html}

\item Balagangadhara, S. N. (2012). Reconceptualizing India Studies. New Delhi: Oxford
University Press.

\item Singh, A. (2009, June 13). Much ado about Ganesha: Paul Courtright, Wendy Doniger, and
the Hindu Right (Rajiv Malhotra) by Amardeep Singh - (w/ Courtright review). Retrieved February 13, 2018, from Press Habor: \url{http://www.debashishbanerji.com/sawiki/sciy/www.sciy.org/blog/_archives/2009/6/13/4221234.html}
\end{enumerate}

\end{multicols}
