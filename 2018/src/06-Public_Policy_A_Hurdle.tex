\setcounter{figure}{0}
\setcounter{table}{0}
\setcounter{footnote}{0}

\articletitle{Public Policy A Hurdle Under the Indian Arbitration Law: Critical Analysis}
\articleauthor{Vishal Ranaware\footnote{Assistant Professor,Symbiosis Centre for Distance Learning, Pune(Email: adv.vishalranaware@gmail.com)} and Amol Shelar\footnote{Faculty of Law,Sinhgad Law College,Pune}}
\lhead[\textit{\textsf{Vishal Ranaware and Amol Shelar}}]{}
\rhead[]{\textit{\textsf{Public Policy A Hurdle...}}}

\begin{multicols}{2}

\heading{Abstract}

\noi
The traditional role of public policy was to limit the scope of foreign law, recognition, and
enforcement of foreign judgments or awards. Sometimes domestic courts use this doctrine to
strike down the foreign arbitral awards. Though the disputing parties are free to choose
applicable laws in international commercial arbitration, when it comes to the recognition and
enforcement of an award they rely on the domestic laws and courts. If the court thinks that an
award before them deals with a matter violates public policy, the court may refuse to recognise
and enforce it. There is no uniformity in public policy notion among the states, it has been
interpreted in different ways in different jurisdictions so it becomes very difficult to say which
award will be allowed and which will violate the principle. Therefore, it becomes a big hurdle
in the way of international commercial arbitration. To deal with this issue Indian judiciary
took a step to define it and limit the scope doctrine of public policy. Finally, in 2015 Indian
Parliament amended the Arbitration and Conciliation Act, 1996, and clarified the term ‘public
policy’.

\heading{Keywords}

\noi
Public Policy, International Commercial Arbitration, the Arbitration and Conciliation Act
1996, party autonomy, arbitral award, Renusagar.

\heading{Introduction}

\noi
Alternative dispute resolution mechanism brings hope for the disputing parties as it is flexible
and informal compared to the judicial system, also have other advantages. Arbitration is one
of the ways to resolve the dispute outside the court. It gives autonomy to the parties on certain
aspects, like appointing arbitrators, deciding their qualification, place, date, and time, and most
importantly finalising a set of procedural rules and laws applicable to the dispute. Also, the
international and national laws provide for minimum intervention of the judiciary. In the
arbitral proceedings, the judiciary can intervene only under the limited grounds provided by
the Arbitration and Conciliation Act, 1996\footnote{The Arbitration and Conciliation Act, 1996, No. 26, Acts of Parliament, 1996 (India).}  (the Act). It is to protect the rights of the parties
to resolve their dispute through arbitration, a recognised mode of dispute resolution, as per the
agreement and this should not be hampered by unwelcome judicial intervention. However,
this autonomy is not absolute, there are certain provisions that work as a limitation on the
concept of party autonomy given by national and international law. Here, in this research paper
researcher has discussed the concept of ‘public policy’ which works as a limitation on party
autonomy. However, in 2015, Section 34\footnote{The Arbitration and Conciliation Act, 1996, § 34, No. 26, Acts of Parliament, 1996 (India).} and 48\footnote{The Arbitration and Conciliation Act, 1996, § 48, No. 26, Acts of Parliament, 1996 (India).} have amended to limit the scope of ‘public policy’.

\heading{Finality of Arbitral Award}

\noi
As per the national and international laws, the decision of an arbitral tribunal is final and
binding on parties and persons claiming under it. Judicial intervention is allowed in defined
circumstances; therefore, if aggrieved party wants to set aside an arbitral award it can be done
by the court ‘only’ on the grounds defined under Section 34, Part I of the Act. According to
Section 34(2)(a), the party has to establish that:

\begin{enumerate}[label=$\alph*.$]
\item the party is under some incapacity,

\item the arbitration agreement is invalid under the laws applicable,

\item the arbitrator has appointed without giving due notice to the party,

\item the constitution of the arbitral tribunal is not as defined by parties, unless otherwise,

\item the dispute or the matter covered by an arbitral award is not within the scope of arbitration according to the submission agreement,

\item the arbitral tribunal has not followed the procedure defined by parties, unless otherwise.\footnote{The Arbitration and Conciliation Act, 1996, § 34(2)(a), No. 26, Acts of Parliament, 1996 (India).}
\end{enumerate}

\noi
Apart from the above mentioned, there are two more grounds on which the court may set aside an arbitral award; arbitrability and public policy.\footnote{The Arbitration and Conciliation Act, 1996, § 34(2)(b), No. 26, Acts of Parliament, 1996 (India).} This provision is based on Article 34 of the UNCITRAL Model Law on International Commercial Arbitration, 1985 which is the first recourse against an award at the seat of arbitration.

\noi
Part II of the Act gives effect to the New York Convention 1958 (NY Convention), the Geneva
Protocol 1923, and Geneva Convention 1927 which deals with the recognition and
enforcement of foreign arbitral awards under the Act. Sub-section (1) and (2) of Section 48 of
the Act based on Article V of the NY Convention define more or less similar grounds stated
under Section 34 of the Act on which a local court where the recognition and enforcement
sought may deny it.

\noi
Among all grounds, the principle of public policy gives scope for interpretation also works as a limitation on party autonomy.\footnote{The Arbitration and Conciliation Act, 1996, § 34 (2)(b)(ii), No. 26, Acts of Parliament, 1996 (India).} Let’s see how!

\heading{Doctrine of Public Policy}

\noi
The term ‘public policy’ has not been defined under the Act nor under any convention which
makes it difficult to interpret and this gives an opportunity to judge to decide its course. It has
been defined in different ways in different jurisdictions across the globe. The House of Lords
in 1853 defined public policy as the legal principle which forbids the subject from doing
something which is injurious to the public or against the public good.\footnote{Egerton v. Brownlow, (1853) 4 HLC 1.} It means the things which are injurious to the public, against the good morals or public good are not allowed to do in that particular jurisdiction. So, if an arbitral award deals with such matters, contrary to laws or standards, violate the notion of morality and justice prevail in the court’s jurisdiction
such awards will be vacated by the domestic court. For example, there is a dispute between
parties over casino profit. The disputing parties may resolve it through arbitration. Now, some
states will consider it as a commercial dispute and will allow its enforcement. However, the
states with strict rules against gambling may not be enforced on the ground of public policy
as it is illegal in that particular jurisdiction. A similar approach has been adopted by the US
Second Circuit Court of Appeals in Parsons \& Whittemore Overseas Co., Inc. v. Societe
Generale de I’Industrie du Papier\footnote{Parsons \& Whittemore Overseas Co., Inc. v. Societe Generale de I’Industrie du Papier, 508 F.2d 969 (2d Cir. 1974).} while affirming the arbitral award against an American Company. The court stated that the term public policy should be interpreted narrowly and the enforcement of foreign awards under the NY Convention may be denied if it goes against the basic idea of morality and justice.\footnote{Id.}

\noi
The Supreme Court of Korea stated that the basic tenet of the public policy principle is to
protect the fundamental moral beliefs and social order of the country where recognition and
enforcement are sought from being harmed.\footnote{Adviso NV (Netherlands Antilles) v. Korea Overseas Construction Corp., XXI YBCA 612 (1996).} Here, the Korean court gave a narrow
interpretation and on the same note, the Swizz court in K S AG v. CC SA\footnote{K S AG v. CC SA, XX YBCA 762 (1995).} upheld the constrained approach of the public policy principle. Apart from international commercial arbitration, in the US, courts from states like Ohio, South Carolina, and North Carolina consider that the binding arbitration agreements between parents to resolve child support disputes violates public policy.\footnote{Cohoon v. Cohoon, 770 N. E. 2d 885.}

\heading{Indian Legal System and Doctrine of Public Policy}

\noi
Section 34(2)(b)(ii) and 48(2)(b) states that the court may set aside and refuse to enforce an
arbitral award, respectively, if it contradicts with the notion of ‘public policy’ prevailed in the
state.\footnote{The Arbitration and Conciliation Act, 1996, No. 26, Acts of Parliament, 1996 (India).} The Act is silent on its meaning however, the judiciary has taken an initiative to decode. It denotes the fundamental policy of law, justice, and morality.\footnote{Bharti Airtel Limited v. Union of India, 231 (2016) DLT 71.} The Supreme Court
discussed this issue in a number of cases. In Renusagar Power Co. Limited v. General Electric
Company\footnote{Renusagar Power Co. Limited v. General Electric Company, 1994 Supp (1) SCC 644.} (Renusagar), the Apex Court interpreted the term ‘public policy’ defined as the
ground for setting aside an award under the Foreign Awards (Recognition and Enforcement)
Act, 1961.\footnote{The Foreign Awards (Recognition and Enforcement) Act, 1961, § 7(1)(b)(ii), No. 45, Acts of Parliament, 1961 (India).} The Court held that the term used in a very restricted sense, therefore an arbitral award cannot be barred under public policy principle merely on the ground of violation of Indian laws. The judges have to look for something more to apply the bar of public policy to
foreign arbitral awards. It observed that to refuse the enforcement of foreign awards on the
ground of public policy the court should find that award contrary to:

\begin{enumerate}[label=$(\alph*)$]
\item Fundamental policy of Indian Law; or

\item The interest if India; or

\item Justice or morality.
\end{enumerate}

\noi
In furtherance of the above observation, the Indian judiciary has expanded the scope of the
term ‘public policy’ by adding few more grounds to it. In 2003, in Oil \& Natural Gas
Corporation Ltd. v. SAW Pipes Ltd\footnote{Oil \& Natural Gas Corporation Ltd. v. SAW Pipes Ltd, (2003) 5 SCC 705.} (ONGC) the Supreme Court observed that the role of
the court under Section 34 of the Act is appellate/revision court therefore, the vast powers are
conferred by the Act. It stated that the ‘patent illegality’ could be a valid ground to set aside
an arbitral award. As per the decision, to call an award ‘patently illegal’ has to disregard the
substantive provisions of law or contradict the terms of the contract. If the given condition is
satisfied, the court can intervene and pass an order under Section 34 of the Act. The court
further added that the narrow approach would make some provisions of the Act insignificant,
so an extensive interpretation is a prerequisite of the statute to vacate ‘patently illegal’
awards.\footnote{Id.}

\noi
There was a difference between Renusagar\footnote{Renusagar Power Co. Limited v. General Electric Company, 1994 Supp (1) SCC 644.} and ONGC\footnote{Oil \& Natural Gas Corporation Ltd. v. SAW Pipes Ltd., (2003) 5 SCC 705.} as the earlier one was dealing with
enforcement of an award under Section 7 of Foreign Awards (Recognition and Enforcement)
Act, 1961\footnote{The Foreign Awards (Recognition and Enforcement) Act, 1961, § 7, No. 45, Acts of Parliament, 1961 (India).} (since it is repealed Section 48 of the Act govern this field) and later with validity
under Section 34 of the Act. However, it increases the burden of the Indian judiciary. Now,
every award with an error of application of legal provisions could be challenged under Section
34 of the Act by virtue of newly added ground. Indian courts re-heard the awards on merits
which defeated the very basic purpose of the arbitration.

\noi
In 2011, one more case related to the ‘public policy’ under Section 48 of the Act was filed
before the Supreme Court. In Phulchand Exports Ltd. v. OOO Patriot\footnote{Phulchand Exports Ltd. v. OOO Patriot, (2011) 10 SCC 300.} (Phulchand) the
Supreme Court held that the test given in ONGC\footnote{Oil \& Natural Gas Corporation Ltd. v. SAW Pipes Ltd., (2003) 5 SCC 705.} must be followed for foreign awards as the
expression ‘public policy’ under Section 34 and 48 of the Act are the same. The Supreme
Court brought foreign awards and domestic awards on the same page without specifying
reasons for ignoring the difference between these two drawn by the Act. It expands the
meaning of the term ‘public policy’ in India.

\noi
However, Phulchand\footnote{Phulchand Exports Ltd. v. OOO Patriot, (2011) 10 SCC 300.} ruling had a short span, it was overturned in Shri Lal Mahal Ltd. v.
Progetto Grano Spa\footnote{Shri Lal Mahal Ltd. v. Progetto Grano Spa, (2014) 2 SCC 433.} (“Lal Mahal”). The Apex Court held that the term ‘public policy’,
defined as ground under Section 48 of the Act, doesn’t cover the ‘patent illegality. This
decision restored the position held in Renusagar\footnote{Renusagar Power Co. Limited v. General Electric Company, 1994 Supp (1) SCC 644.} with respect to enforcement of the foreign
award and ceased application of ONGC\footnote{Oil \& Natural Gas Corporation Ltd. v. SAW Pipes Ltd., (2003) 5 SCC 705.} to Section 48 cases. It ended strikes on the foreign
awards on the ground of ‘patent illegality’ by narrowing down the scope of the term ‘public
policy’ in India. The court observed that an application of the term ‘public policy’ under
Section 48 of the Act is restricted to the arbitral awards contradicting the fundamental policy of India, the interest of India, and justice and morality. Section 48 of the Act doesn’t give an
opportunity to review the awards on the merits.

\noi
Further, it was expected from the highest judicial forum that ONGC v. Western Geco
International Ltd.\footnote{ONGC v. Western Geco International Ltd., (2014) SLT 564.} (Western Geco) will review the explanation of the term ‘public policy’
under Section 34 of the Act and override the ONGC.\footnote{Oil \& Natural Gas Corporation Ltd. v. SAW Pipes Ltd., (2003) 5 SCC 705.} However the Apex Court broadened
the scope of ‘public policy’ and observed that the term ‘public policy’ must include all such
fundamental principles as providing a basis for the administration of justice and enforcement
of law in this country. According to the court, the fundamental policy of Indian law includes
three distinct and fundamental juristic principles, those are:

\begin{enumerate}[label=$\alph*)$]
\item the adjudicating authority must adopt a judicial approach while defining the rights of the citizens,

\item the adjudicating authority must follow the principles of natural justice and consider relevant facts of the case to determine the rights and duties of parties,

\item the court should not allow the enforcement of perverse or irrational awards.
\end{enumerate}

\noi
These are the judgments that widened the scope of the expression ‘public policy’ referred
under Sections 34 and 48 of the Act. To limit the scope of interpretation of the term ‘public
policy’, the legislature added explanation to Section 34 and 48 through the Arbitration and
Conciliation (Amendment) Act, 2015.\footnote{The Arbitration and Conciliation (Amendment) Act, 2015, No. 3, Acts of Parliament, 2016 (India).}

\heading{The 246$^{\rm th}$ Law Commission Report and the 2015 Amendment}

\noi
The Law Commission of India (the Law Commission) responded to these judgments in
February 2015 by issuing a supplement to the 246th Law Commission Report, published in
August 2014. The Law Commission criticised the Supreme Court decisions in ONGC\footnote{Oil \& Natural Gas Corporation Ltd. v. SAW Pipes Ltd., (2003) 5 SCC 705.} and
Western Geco\footnote{ONGC v. Western Geco International Ltd., (2014) SLT 564.} for broadening the scope of the term ‘public policy’ and “opening the floodgates”. The Law Commission highlighted that the exhaustive list of grounds defined
under Section 34 and 48 of the Act are related to the procedural issues and the courts are not
supposed to go into the substantive problem. The Law Commission recommended the
definition of public policy given by the Supreme Court in Renusagar.\footnote{Renusagar Power Co. Limited v. General Electric Company, 1994 Supp (1) SCC 644.}

\noi
Considering the recommendations of the Law Commission on this particular issue, the
Parliament amended the Act through the Arbitration and Conciliation (Amendment) Act,
2015.\footnote{The Arbitration and Conciliation (Amendment) Act, 2015, No. 3, Acts of Parliament, 2016 (India).} It adds explanation to the Section 34 and 48 of the Act. According to the amended
provision, the court may set aside an arbitral award or deny enforcement if it conflicts with
the public policy of India, only if:

\begin{enumerate}
\item The making of the award was induced or affected by fraud or corruption or was in violation of section 75\footnote{The Arbitration and Conciliation Act, 1996, § 75, No. 26, Acts of Parliament, 1996 (India).} or section 81;\footnote{The Arbitration and Conciliation Act, 1996, § 81, No. 26, Acts of Parliament, 1996 (India).} or

\item It is in contravention with the fundamental policy of Indian law; or

\item It conflicts with the most basic notions of morality or justice.
\end{enumerate}

\noi
This amendment limits the scope of ‘public policy’ and reduced the scope of judicial
intervention.

\heading{Conclusion}

\noi
Arbitration is one of the ways of alternative dispute resolution. It has been preferred by the
parties for commercial disputes especially international because of its unique features, like,
party autonomy and minimal court intervention. Parties to an arbitration agreement are
permitted to select the applicable laws to the subject matter of dispute as well as procedural
aspects of the arbitration. However, the doctrine of ‘public policy’ limits the party autonomy
as ultimately the finality and enforcement of the arbitral award depend on the laws prevailing
at the seat of arbitration and place the party seeking enforcement. Interpretation of the term
‘public policy’ varies from state to state, time to time as stated by the Supreme Court in
Murlidhar Agarwal and another v. State of U.P. and others.\footnote{Murlidhar Agarwal and another v. State of U.P. and others, 1974 (2) SCC 472.} It was observed by the court that the notion of public policy changes with time, generation, community, and state. Even in one
generation, it may change its course. It never remains the same or static. It became useless it
didn’t change or remain in fixed moulds. So, an award finalised in one state may be denied its
enforcement at another. The term ‘public policy’ gives power to the court to decide its future
course and the same was observed in India since Renusagar\footnote{Renusagar Power Co. Limited v. General Electric Company, 1994 Supp (1) SCC 644.} to Western Geco.\footnote{ONGC v. Western Geco International Ltd., (2014) SLT 564.}

\noi
This issue has been resolved by the Arbitration and Conciliation (Amendment) Act, 2015
which is giving a positive result. Since the amendment, the courts have refused to examine the
Section 34 and 48 cases on the merits, act as an appellate authority, or give a wide
interpretation to expression ‘public policy’. In Venture Global Engineering LLC and Ors v.
Tech Mahindra Ltd. and Ors,\footnote{Venture Global Engineering LLC and Ors v. Tech Mahindra Ltd. and Ors, (2018) 1 SCC 656.} the Hon’ble Supreme Court held that the grounds specified
under Section 34 of the Act are the ‘only’ grounds on which the court can set aside the awards.
While dealing with Section 34 cases the court should not act like an appellate court, they are
not supposed to examine the legality of an award on merits of claims by entering into a factual
arena.\footnote{Id.} The same approach has been adopted by the judiciary in other cases like Sutlej
Construction v. The Union Territory of Chandigarh.\footnote{Sutlej Construction v. The Union Territory of Chandigarh, (2017) 14 SCALE 240 (SC).} Now, the courts are realising that they
have to intervene in the arbitral process ‘only’ in specified conditions and grounds defined by
the Act and give some free-way so an arbitration can achieve its intended objectives.
\end{multicols}
