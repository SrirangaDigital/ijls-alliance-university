\setcounter{figure}{0}
\setcounter{table}{0}
\setcounter{footnote}{0}

\articletitle{Investigative Journalism in India: Case studies of prominent Journalists}\label{2017-art6}
\articleauthor{Siddharth Negi\footnote{Research Scholar, Department of Media Studies, Punjab University Chandigarh.}}
\lhead[\textit{\textsf{Siddharth Negi}}]{}
\rhead[]{\textit{\textsf{Investigative Journalism in India...}}}

\begin{multicols}{2}

\heading{Introduction:}

\noi
Indian journalism took the investigative route with hesitant yet effective steps. The government
and media organizations gave little importance to investigative journalism. It is only after the
emergency that investigative journalism gained importance with the efforts of a few committed
journalists. Several journalists such as Arun Shourie, challenged the might of the then
government and thus had to pay a price. The socio-political conditions of the country helped in
the growth of the Indian investigative journalism. Later, the work and effort of journalists
institutionalized investigative journalism in India. The lives and times of these journalists can\footnote{Trehan, Madhu (2005). Tehelka as Metaphor. New Delhi: Roli Books.}
throw light on the development of this genre in India. The researcher would like to study the
works of major investigative journalists and also the media history of the period in bringing
out the unique contribution of the journalists. In the art ofstatecraft, ‘Arthshastra’ is one of the
fundamental and historical books\footnote{Puri, Rajinder (1971). A Crisis of Conscience. New Delhi: Orient Paperbacks.} written by Chanakya. According to this ancient text, even
the honesty of judges should be periodically tested by the agent provocateurs. Arthshastra can
be called as the best example of investigative journalism in Indian history. It can be said that
modern day sting used in investigative journalism derives same authority from the writings of
Chanakya.

\vspace{-.1cm}

\noi
After getting freedom from the British yoke our leaders promised us a free press. Even the tall
promises regarding the freedom of expression and speech have been made in our Indian
Constitution in Article 19. Paradoxically, Indian Constitution nowhere gives any independent
provision regarding press. It is considered to be the part and parcel of Article 19 whereas
American Constitution specifically provides a right to press. The historian Robin Jeffery has
pointed that “the first amendment to the US constitution guaranteed freedom ofspeech and the
press. The first amendment to the Indian constitution, passed in June 1951, curtailed those
rights. It permitted governments to ban publication of material likely to disturb public order,
incite people to commit a crime as harm relations with foreign powers”.

\vspace{-.1cm}

\noi
Ironically, press is also considered to be the fourth pillar of democracy in India yet it has not
been able to rise from a partisan role. The press from 1947-1967 can be called free but it was
always in cooperation with the government. During this time the press hardly tried to evaluate the decisions and policies of the State. The status of press after independence can be described
clearly in the words of Rajender Puri: “The national press after independence has been
traditionally obedient. In the early years this was partly due to momentum of the freedom
struggle, partly due the towering personality of Nehru. All our national newspapers also happen
to be patterned after the quality press of Great Britain and the upper-class restraint that the
English exercised there was caricatured into an obsequious obedience to authority by the
journalists here”.

\vspace{-.1cm}

\noi
This period fromthe 1947-1967 can be termed asthe black period in the historyof investigative
journalism in India. Journalists used to support the State instead of being watchdogs of Indian
democracy. The Times of India, Hindustan Times, Indian Express, The Statesman and The
Hindu also called Pachyderms were the five big and important newspapers after independence.
These all five newspapers were the voice ofthe State and none ofthem believed in investigative
journalism. These newspapers believed in the bureaucratically given information and worked
hand in glove with the State in the so-called development of the nation. Rajender Puri critiques
the late sixties as, “The newsmen of Delhi certainly proved themselves to be committed
journalists in those days, but their commitment was not to the ethics of their profession, or to
the skill of their craft, or even to the will of the people, but only the will of the government”.
Till 1968 that is 21 years after the independence, not much had changed in the Indian media
scenario. The Pachyderms worked\footnote{Subramaniam, Chitra (1993). Bofors: The Story Behind News. New Delhi: Viking Publishers.} as usual and cared little about the investigative journalism.
There was no real competition for these five big newspapers. There were only a few reporters
working as watchdogs. Kuldip Nayar and Inder Malhotra were outstanding reporters of their
times. Nayar’s column ‘Between the lines’ was extremely well informed and popular among
the masses. It was only after the emergency that the investigative journalism gathered pace in
India. National emergency was declared by the Prime Minister Smt. Indira Gandhi in June
1975. “She had advised the president to proclaim emergency without consulting her cabinet.”
Due to the emergency, fundamental rights guaranteed by Article 19 were suspended.

\vspace{-.1cm}

\noi
The paper focuses on the unique contribution of four journalists who have contributed
immensely to this genre of journalism in India. The lives and works of these journalists
illuminate development of investigative journalism in India.\footnote{Singh, Manorma (2007). Sting Operation. New Delhi: Sachin Printers.}

\vspace{-.1cm}

\noi
{\large \bfseries (1) Kupdip Nayar}

\vspace{-.1cm}

\noi
Kuldip Nayar in the early seventies wrote against the then prime minister of India, Indira
Gandhi. In his articles he challenged the authoritarian government of Indira Gandhi. Indira
Gandhi was hailed as ‘Durga’ by the opposition leader Atal Bihari Vajpayee, after her victory
against Bangladesh war in 1971. According to Nayar, Indira Gandhi felt herself as a roman
emperor returning after a triumphant war . This supreme power and a sense of authority gave
Indira Gandhi an impression that she was above all authority.

\noi
Nayar says, “The basic departure in her approach was that the politics which since
independence had had a consensual element began bearing her personal domineering stamp.
She, in fact, began going out of her way to rub her political opponent the wrong way” (Nayar,
2012, 220). Article 19 1(a) which gives the power of speech and expression was completely
blocked. Barring few media houses (Indian Express, Seminar) the media in general took the
line of the state. It is in such difficult times that real journalist are born. Kuldip Nayar is one
such journalist who stood for the press values and also went to the prison during the emergency.
The courage and quest for truth that Nayar showed gave encouragement to other journalists to
indulge in investigative reporting. The emergency can be understood from the book written by
Nayar called “The Judgement” (1977).

\noi
One of the reasons for the emergency as stated by Kuldip Nayar in his autobiography “Beyond
the lines” was Justice Krishna Iyer. “I suspect that being a leftist in his leaning he held a soft
corner for Indira Gandhi who said\footnote{Rangarajan, L. N. (1992). Kautilya The Arthashastra. New Delhi: Penguin Books India} to be left of centre. The left generally supported her because of her nationalization of banks and insurance companies. Some responsibility for what happened during the Emergency rested on the shoulders ofJustice Iyer because he gave her the stay” . This seems to be an interesting reason neglected by the historians.

\noi
It is believed that Nayar was arrested by the state for being bold in his opposition ofEmergency.
The son of the then prime minister Indira Gandhi wanted to silence the top journalist after
imposing press censorship. He remained in the jail for a period of three months. Not a single
journalist in the country supported Nayar’s family when he was in the prison. “It had been
tough on my family because no relative or journalist or personal friends of mine had dared to contact them during the three months I was in jail” . The isolation of Nayar by his fraternity
and the state oppression seems to be absolute. In the release of Nayar the judiciary played an
important role. Later, he was released by the judgement of the court. But those judges who
gave decision in favour of Nayar had to face severe consequences.

\newpage

\noi
After being released from the prison Nayar once again raised his voice against the state
oppression. His effort saw no success as none in his fraternity supported him. “I tried to pick
up the thread of protest where I had left it off before I was jailed. However, my efforts bore no
fruit. I found journalists were afraid to say anything about the Emergency in public, and editors
proved to be the most cowardly. It appeared asif they had been co-opted bythe system” (Nayar,
2012, 244).

\noi
Sanjay Gandhi played an important and powerful role in the Emergency. He has given very
few interviews and little is known about his side of the story. In his autobiography Nayar has
for the first time written about his interview with Sanjay. This interview was taken in the first
place for the book called “The Judgement”. But it was not included in that book as per the wish
of Sanjay Gandhi. “Sanjay had heard about the book I was writing. Strangely, he asked me not
to include any part of our conversation..................I have however no obligation after his death
and therefore disclosing the gist of our conversation for the first time”.\footnote{Shorie, Arun (1978). Symptoms of Fascism: New Delhi: Vikas Publishing House Pvt Ltd.}

\noi
This interview brings out the real side of Sanjay Gandhi the architect of Emergency who
wanted the “Emergency for at least 20 to 25 years or more until they have changed the people’s
way of thinking”. Sanjay also believed in a state completely devoid of fundamental rights,
freedom of speech and expression.

\noi
Indira Gandhi ended the emergency and the congress lost the elections held after the
emergency. It was in the response to the incapability shown by the leading newspapers that
journalism saw the trend\footnote{Pandey, J.N (2006). Constitutional Law of India. Allahabad: Central Law Agency.}
of daring publications like Nayar’s “The Judgement”. The book
presents the authentic inside story of the emergency. It is an important document in the area of
Indian media history. The book starts from 12 June 1975, the day Allahabad High Court
judgement unseating Mrs Gandhi came. The volume traces the decline in quality of the
democratic values in India with the implementation of emergency. In the conclusion the book also brings up the judgement rendered by the people in favour of democracy. The people who
ruled during the emergency lost the elections. In addition Nayar also brings out the Maruti
scandal, the fate of MISA (Maintenance of Internal Security Act) prisoners, and the role of
intelligence organization and police.

\noi
The press of that time supported and applauded Nayar and his book. “Top of the heap was the
newly sainted (after his jail spell) Kuldip Nayar whose judgement was an instant bestseller.”
This trend was followed by Janaradan Thakur who became famous for his book ‘All the Prime
Minister’s Men’. The title seems to be inspired by the famous book on Watergate, ‘All the
President’s Men’.

\noi
The art of writing books to present important issues was further developed by the father of
Indian investigative journalism, Arun Shourie. Nayar throughout his long career has used books
as a powerfulsource of examining intricate subjects. In his autobiography he has broughtout the
problem of commercialization and other challenges faced by the media. In the history of
investigative journalism Nayar emphasised the fact that even a dictatorial regime can be
challenged by the force of pen. The heroism of Nayar at the time of emergency motivated the
journalists to take the watchdog role. It also gave birth to the narrative of dissent and challenged
the authority in the name of freedom of expression. Thus, Nayar paved the path for
investigative journalism in India with many other journalists of his times.

\noi
During 1977, emergency was the most significant incident. Judiciary, executive, legislature,
and press had to face the authoritarian rule of Mrs Gandhi. But Indira Gandhi again came to
power and was able to re-establish the dynastic rule. She was able to regain the support of
judiciary, executive, and legislature however she failed to garner the support of the press.
Emergency led to three things, “Firstly the influence of pachyderms declined. In 1977, there
was a general feeling that the major dailies had let the country down by supporting the
dictatorial government of Mrs Gandhi. Secondly, the emergency created hunger for a new kind
of news. The news that went beyond the blended official statements and dug out what was
really happening is investigative reporting. Thirdly, as the papers went into decline in the
immediate post-Emergency phase, a new crop of slick, professionally produced magazines
mushroomed to take their place”.

\noi
{\large \bfseries (2) Arun Shourie}

\vspace{-.2cm}

\noi
Arun Shourie can be called the real product of emergency in the field of media. He is considered
as the father of investigative journalism in India. Shourie used both newspaper and books as
the medium to investigate important and challenging matters. The best thing about the Shourie
brand of investigative journalism is that it believes in meticulous research of the documents or available material. It is on the basis of these documents that Shourie was able to expose many
wrong doings of the regime.

\noi
In the beginning Shourie was encouraged by his organization but eventually he was fired
for his courageous style of writing. Shourie started his journalistic career by writing opinion
pieces for the magazine India today. Shourie was hired by the owner of the Indian Express, Mr.
Ram Nath Goenka, as an executive editor. Goenka later came to know about the superb
journalistic skills that Shourie possessed. Shourie did not change much in the Indian Express.
The single significant thing he did was that he called his journalists to investigate every
bureaucratically given information.

\noi
Shourie exposed A R Antulay the then chief minister of Maharashtra in 1981. Mr. Antulay was
involved in corrupt practices. He was allotting cement quotas to builders in lieu of some
contributions towards the trusts that he had founded. He had to resign after the stories about
him were reported. Another important story done by the Indian express under Shourie was
about the flesh trade in Madhya Pradesh. The story was broken in1981.The reporter Ashwini
Sarin,\footnote{Ibid.} exposed the flesh trade racket by buying a girl named ‘Kamla’ for Rs. 2300(Lakshman
(ed.), 2007). Writing for ‘The Illustrated Weekly of India’ in 1985, Malavika and Vir Sanghvi
highlighted the impact of Shourie on indian journalism.

\noi
According to the writers, “Arun Shourie had an impact on Indian journalism that few will be
able to equal. Firstly, he found the power of the newspapers: if you have a story on the front
page of every one of your editions when parliament is in session, then the government has to
take notice. Secondly, he showed journalists how to push themselves, how to always
investigate further and to document everything. Thirdly, he became the first Indian journalist
to become a nationally respected figure and thereby attested the entire perception of journalism
as profession”. Shourie after his term in the Indian Express, could not keep his job intact for a
long time. Shourie had to resign for his courage and brave ideas. Shourie after his newspaper
work started writing books in order to investigate important matters of public concern which
again became a great success.

\noi
According to Martha C. Nussbaum, “Shourie has been a prolific author. All his books are
recognizably the creation of a smart, determined muckraking journalist. They are polemical,
ad hominem, often extremely shrill in tone.” She further writes, “The books written by Shourie
are obviously the work of a brilliant man, with wide if idiosyncratic learning, a passion for freedom of speech and press, and a desire to get beneath current events to address underlying issues.”

\noi
The two important booksthat bring out the important elements of Shourie’s journalism
are ‘Symptoms of fascism’ and the other ‘Worshipping False God’. Symptoms of fascism bring
out the “abuses of power” at the time of emergency. Shourie in his 1978 book ‘Symptoms of
Fascism’ seems to be hardly impressed with the press ofthat time. An important issue regarding
the deaths of hundreds of young men and women in alleged encounters in Andhra Pradesh was
given little importance by the press. Shourie and eight members of his team investigated the
matter and came up with two interim reports namely ‘Encounters are Murders’ and ‘The Killing
in Guntur’. According to Shourie, “press did not follow up work on its own. A couple of
investigative reports and that is all. Contrast this with the space and effort it devoted to the
Nanavati murder case some years ago and to the Vidya Jain murder case more recently.”
(Shourie, 1978, 322) It proves that sensationalism has always been present in Indian media and
is not a recent phenomenon.

\noi
Shourie also have an activist zeal and believes that change can only happen when every citizen
contributes towards it. With his investigative works Shourie also tried to educate and forewarn
the citizens about the dangers of dictatorship and authoritarianism.

\vspace{-.15cm}

\noi
{\large \bfseries (3) Chitra Subramaniam}

\vspace{-.15cm}

\noi
Chitra Subramaniam is one of the most renowned investigative journalist in India. Famous for
her meticulous investigation of the Bofors-Indian howitzer deal. Chitra started her journalist
career by reporting in India today (news magazine) in 1979. After this she moved to
Switzerland in 1983. Chitra was in Greece as a united nation (UN) correspondent when she
came to know about the Bofors pay off through radio. A foreign radio station can be given the
credit for the most important investigative story in the history of Indian investigative
journalism. The Swedish radio broadcast claimed that AB Bofors paid money to key Indian
policy makers and top defense official to secure the deal. The news of this broadcast was first
reported by Chitra Subramaniam for ‘The Hindu’ (newspaper). With the assistance of ‘The
Hindu’ newspaper and its editor N. Ram she was able to unearth important secrets regarding
corruption in the Bofors deal between India and Sweden. Colombia school of journalism has
rated this story as one of the best among the hundred best stories. This story of political
corruption which started on 16 April 1987 is still in news and has not reached the conclusion.
In her book “Bofors: The Story Behind the News” Chitra Subramaniam gives the complete
story of the Bofors scandal. This volume is significant because it disputed the official version.
It is able to establish the truth with the aid of proper investigation and documents. It was an international scandal involving Sweden, Switzerland and India. The script brings out all the
subtle nuances regarding this scandal at a single post. It can be called a genuine work in the
area of media history.

\vspace{-.15cm}

\noi
In the area of investigative journalism this case brings out the importance of two issues. The
first is the role of whistle-blowers and the second is the importance of documents. In the words
of Subramaniam, “I would have had no story to narrate-and no document to back that story
without ‘Sting’ my principle source in Sweden”.Sting provided Subramaniam, “300 documents
that established massive fraud in the Bofors-India howitzer payoff scandal. Among them were
bank documents, credits slips, payment instructions from Bofors to the banks in Sweden and
Switzerland”. It can be stated that this scandal could only be unearthed due to the aid of the
whistle-blower and diligent work of investigative journalist. In this story documents played an
important role. All kinds of papers related to the case were studied in order to determine the
truth. In the words of Subramaniam, “Nothing connected with the story was ever thrown out
and the early notes, doodling, and first drafts ofstories soon grew into a large pile”. In this case
government of India tried its best to cover-up the whole case. In those days there was no
‘whistle-blower act’ and ‘right to information act’ in India. One significant impact of the probe
was that due to it Rajiv Gandhi government lost the elections (1989). The work of
Subramaniam proved corruption charges against the government. A single journalist brought
down a mighty but a crooked regime. It can be summed up in the words of Subramaniam, “My
phone rang off the hook with friends calling from India. ‘Your stories sealed the fate of his
government” said Indian journalist Vir Sanghavi.

\vspace{-.15cm}

\noi
Chitra Subramaniam wrote another book called ‘India is for sale’ in 1997. It contains nine
satirical essays. These essays highlight the corrupt practices in Indian political organization.
Subramaniam herself states that the reason and crux of the book was that “If I were to tell you
something about this book in a few words, I’d say it’s a little about growing up pains and a lot
about those needless pains that plague you when you refuse to grow up. It is about asking the
Vedas to decipher your past and the World Bank to predict your future while your present falls
apart in front of you. It’s about behaving like an irresponsible adolescent at 50” (Subramaniam,
1997, 3). This book was written in\footnote{Nussbaum, Martha C. (2008). The Clash Within. United States of America: Harvard University Press.} 1997 when India turned 50. This book takes account of
the challenges and failure that India faced till 1997. From the starting till the end, she maintains
the muckraking and the satirical tone in her writing. She is critical of our politicians who have
little knowledge regarding the economic matters. Subramaniam uses very strong words while mentioning the understanding of our leaders in economic matters. According to her, “Davos is
a big showroom where you display your country, your industry, and your services…. To this
forum India sends politicians who are armed not just with their ignorance of Indian and
international politics but also long list of complexes……. No other country’s politicians beg
for money from one side of the mouth while criticising the ways of their donors with the other
as our politicians do”. In the essay ‘Intellectual Singh Paperwala’ Subramaniam criticises the
pseudo-Indian intellectual who is not aware of the real challenges that the country is facing.
“In other parts of the world, intellectuals come from allsections ofsocieties. In India they come
from circles so closed and incestuous that ultimately, they become irrelevant to the country’s
needs”. The book is autobiographical in nature and “the situation drawn in this book are from
real experience, the name and characters portrayed are the product ofthe author’s imagination”.
It seems Subramaniam have used this device in the book in order to uphold the privacy of
people she has written about.

%~ \vspace{-.2cm}

\noi
{\large \bfseries (4) Mathew Samuel and use of technology in investigative journalism}

\vspace{-.1cm}

\noi
The basic tools for print reporters are the pencils and notebook and for television reporters are
camera and videotapes. Due to the technological innovation in present times, we are witnessing
the rising importance of the new technology. This new technology has changed the way in
which investigative journalism is being conducted. Undercover investigative reporting has
become the new form of broadcast reporting. This has become possible due to miniature audio
and video technology.\footnote{Nayar, Kuldip (2012). Beyond The Lines: An Autobiography. New Delhi: Roli Books.} Today a reporter can enter any place. For instance, WikiLeaks
(wikileaks.org) proved the fact that even the privacy of the most powerful nation (USA) is not
secure.

\vspace{-.2cm}

\noi
The use of technology has started a debate regarding the privacy laws. The focus now is on the
practices of investigative reporters and their use of hidden cameras instead of the information
that has been uncovered. ‘Food Lion’ case is an important example. In this case, the court
found\footnote{Tully, Mark (2002). India In Slow Motion. Great Britain: Penguin Books.} that ABC News was guilty of using fraudulent tactics when going undercover with a
hidden camera to investigate the reports of unsanitary conditions in a local Food Lion
supermarket. As the result of the Food Lion verdict, many new organizations have modified
their use of undercover reporters and hidden cameras. Indian courts have taken the side of the
undercover journalists as privacy laws are still at a nascent stage. The judgement of the Delhi High court in the ‘Aniruddha Bahal v. State’ case has supported the undercover journalism in
order to unearth corruption. The following quote from the judgement, “ I consider that in order
to expose corruption at higher level and to show to what extent the state managers are corrupt,
acting as agent provocateurs does not amount to committing a crime” brings out the Indian
position. In India, Tehelka sting is one case where the matter of technical usage and privacy
came into conflict for the first time. It seems that sting as an investigative tool that has been
institutionalized in India.

\noi
The book named “Tehelka as Metaphor” by Madhu Trehan tries to bring out the truth of this
sting. In March 2001, the website Tehelka broke “Operation West End”, the biggest sting
operation and undercover news story in Indian journalism. Tehlaka’s reporters infiltrated the
Indian government, bribed army officers, and gave money to the President of the ruling party.
This eventually forced ministers to resign. In a rigorously researched and searing authentic
account of the Tehelka expose’ and its aftermath, Madhu Trehan did a forensic study of the
imperative at the root of it, the characters and heroes and villains of the Tehelka sting operation
story, and of how the system got back: by obfuscating, by attempting to destroy Tehelka and
its investors. The book brings out the point that how the government used instruments of
democracy to destroy the investors without leaving any footprints. In the style of Rashomon,
the Tehelka operation story is related by numerous participants of the same incidents and, of
course, none of the stories tally . This level of uncertainty is always attached with stings.It can
be called the “Rashomon” effect. During the Tehelka sting operation the most important work
was done by a lesser-known journalist called Mathew Samuel.

\vspace{-.2cm}

\noi
Mathew Samuel is a Delhi based investigative journalist. In the beginning Samuel worked for
Mangalam newspaper and Midday for a short period. He then worked as an investigative
journalist for a dot-com company called Tehelka, which means ‘sensation’. Samuel and
Aniruddha Bahal together took part in the sting operation “Operation West End”. This led to
the resignation of the ministers of the national democratic alliance and it nearly brought down
the government of India in 2001.This sting operation was the brain child of Mathew Samuel.
According to him: “We began by creating fake brochures for a fake firm in London,\footnote{Lambeth, E. B (1990). Waiting for New St. Benedict: Alasdair MacIntyre and the Theory and Practice of Journalism. Retrieved from \url{http://www.jstor.org/stable/27800034.}} ostensibly dealing in equipment for the armed forces. We called our firm Westend. We printedvisiting cards and began by meeting people in the bottom rung. We told them we had a product – handheld thermal imagers – that could be of immense use to the military. We paid money at
every level and worked our way up to several officers in the defence ministry. The entire
operation lasted eight months. We could have gone even farther but for the lack of resources.
We were also scared that our bluff will be called if we kept pushing our luck.”One hundred and
five (105) tapes were recorded by the Tehelka team and most of them were done by Mathew.
It is strange that Tehelka sting is known by Tarun Tejpal, who did the little investigative work.
It seems that he was not able to market himself like Tarun Tejpal, who started a weekend
newspaper, Tehelka. In an interview to Madhu Trehan the failure on his part could be
understood.

\vspace{-.15cm}

\noi
Madhu: “But Samuel, if you had given interviews and let people take your picture, you would
have also become a star.”

\vspace{-.15cm}

\noi
Samuel: “I never need like that, one thing. I never want to be celebrity, or anything. I have my
own limitations. I’m warring. I’m not like an order of the day, anywhere. I’mworking like that.
I got many crawls now also to give interviews. But I refuse to give interview.”

\vspace{-.15cm}

\noi
Journalists like Mathew Sameul seem to be misfit in the market driven media. Marketing and
public relations have become the inseparable part of media. The inner core is being neglected.
There are very few journalists who want to stay with the story after the initial expose`. The
case of Mathew Sameul is important because he was neglected by his fraternity and punished
by the state against whom he acted. The good thing is that Mathew has broadened the space for
investigative journalism in India.

\vspace{-.15cm}

\noi
{\large \bfseries (5) Important Reasons for the Later\\ Development of Investigative Journalism in\\ India}

\vspace{-.15cm}

\noi
Investigative journalism in India is not growing at a steady rate. In comparison to America and
England we are lagging behind. Investigative journalism in India is a very recent phenomenon.
The investigation of Bofors guns was the high point in the late eighties. The use of new
technology has its benefits, but technology cannot replace the thorough investigation required
for investigative journalism. This part attempts to point out the causes for the delay of
investigative journalism in India. It will also elaborate the reasons for slow growth of this trade.

\vspace{.1cm}

\heading{Development Journalism and Lack of\\ Competition\footnote{Nayar, Kuldip (1977). The Judgement: Inside Story of The Emergency In India. New Delhi: Vikas Publishing House Pvt Ltd.}}

\vspace{-.1cm}

\noi
After independence the journalists\footnote{Jeffrey, Robin (2000). India’s Newspaper Revolution. New Delhi: Oxford University Press. Lakshman, Nirmala (ed.) (2007). Writing A Nation: An Anthology of Indian Journalism. New Delhi: Rupa \& Co.} in India spent most of the time on development stories.
The journalist felt it was their duty to propagate the policies of the state. One important reason was also the charisma of Nehru who was able to charm the media. Thus, an atmosphere was
created in which media started respecting the authority. There were very few journalists who
played the role of watchdog. The other reason for the slow growth after Independence is lack
of competition among the leading newspapers.

%~ \newpage

\heading{Political Pressure}

\vspace{-.15cm}

\noi
In India media was never free. Our leading papers have always been owned by the
industrialists. The politicians were able to influence the owners of media. Indira Gandhi said
about press, “How much freedom can the press have in a country like India fighting poverty,
backwardness, ignorance, disease and superstition”. She then “answered her own question by
making it clear that the less freedom the press had, the better it was, in her book” (Malvika,
Sanghvi). In Indian context the state has always tried to keep its control over the press.

\vspace{-.15cm}

\heading{Legal Causes for the Late Development of\\ Investigative Journalism in India}

\vspace{-.15cm}

\noi
In India, after independence the government kept strict control over the official information.
The law of Official secret act was used to prevent a citizen from scrutinizing official
information. This also prevented journalists from working on investigative stories. The
scenario changed in June 2005 with the passing of ‘Right to information act. The Passing of
this act has given a solid advantage to investigative journalism.’ Watergate’ and ‘Bofors’ both
happened with the help of the whistle-blowers called ‘Deep-throat’ and ‘Sting’. In India for
more than sixty years there was no whistle-blower act. In India this act has come into form only
on 9 may 2014. These two laws are decisive for investigative journalism. In India the rightof
freedom of speech and expression is same for all. The journalists working under difficult
situation should be considered for some special rights.

\vspace{-.15cm}

\heading{Commercialization in Media}

\vspace{-.1cm}

\noi
In the present-day media has become a big industry. The most important\footnote{Malvika \& Sanghavi, Vir (1985, November 10). The Typewriter Guerrillas. The Illustrated Weekly of India. Retrieved From \url{http://www.cscarchive.org.}} goal of media industry is to earn profit. It is indulging in activities like paid news. “It is a scandalous phenomenon in Indian media, in which mainstream media (with a few exceptions) was found to be systematically engaged in publishing favourable articles in exchange for payment”. In such a scenario it is difficult to think about the growth of investigative journalism.

\vspace{-.15cm}

\heading{(6) Conclusion}

\vspace{-.1cm}

\noi
In India it has taken a considerable time for the establishment of investigative journalism. This
journalistic genre is practiced by very few journalists in India. The journalist involved in
investigative journalism face political, legal and fiscal obstacles. The work and times of the
four important journalists discussed in this paper, gives us an opportunity to understand the
history and development of investigative journalism in India. In India \footnote{Aucoin, L. James (2005). The Evolution of American Investigative Journalism. University of Missouri Press.} it is by the effort of few
individual practitioners that investigative journalism has become a respectable, coveted and
distinct genre of journalism. The state of investigative journalism in India\footnote{Thakur, Janardan (1977). All The Prime Minister’s Men. New Delhi: Vikas Publishing House Pvt Ltd.} can best be
understood by Alasdair Macintyre’s approach, who has described how social practices develop
and persist overtime. Social practice according to Macintyre\footnote{Lambeth, E. B (1990). Waiting for New St. Benedict: Alasdair MacIntyre and the Theory and Practice of Journalism. Retrieved from \url{http://www.jstor.org/stable/27800034.}} is “a coherent complex cooperative human activity in a social setting. He says members of practice obtain goods that are specific to the practice by carrying out activities in pursuits of standard of excellence. Macintyre further argues that a social practice is sustained and indeed progresses through the efforts of practitioners to meet and extend the practice’s standard of excellence.” (Aucoin, 2005, 5). In India we find that journalists have on their own made efforts and have exercised courage, justice and honesty. According to Macintyre, “They are acquired human traits the exercise of which contribute materially to the development and extension of practices” (Lambeth, 1990). In India the journalists on their own are developing the field of investigative journalism but they are not able to develop powerful institutes for sustaining the practice like IRE (Investigative Reports and editors) and CIJ (Centre for investigative Journalism). The real
growth as argued by Macintyre can happen when both institute and journalists are strong.\footnote{Subramaniam, Chitra (1997). India Is for Sale. New Delhi: UBS Publishers’ Distributors Ltd}
\end{multicols}
	
\label{end2017-art6}
