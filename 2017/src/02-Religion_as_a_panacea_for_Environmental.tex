\setcounter{figure}{0}
\setcounter{table}{0}
\setcounter{footnote}{0}

\articletitle{Religion as a panacea for Environmental Protection: A long overdue revisit of Rights, Duties and Law}\label{2017-art2}
\articleauthor{Manjeri Subin Sunder Raj\footnote{Assistant Professor, NLSIU, Bangalore}}
\lhead[\textit{\textsf{Manjeri Subin Sunder Raj}}]{}
\rhead[]{\textit{\textsf{Religion as a panacea for Environmental...}}}

\begin{multicols}{2}

\heading{Introduction:}

\vspace{-.15cm}

\noi
An anecdote with its timeless beauty depicts accurately the dependence of humankind on
Mother Earth and the nature of their relation. Justice O. Chinnappa Reddy in \textit{Shri Sachidanand
Pandey and Anr. v. The State of West Bengal and Others}\footnote{A.I.R.1987 S.C.1109} had quoted the same in paragraph two of his judgment and taking into account the beauty of the same, a very short part is reproduced. The backdrop of the story is in the USA, circa 1854, when ‘the wise Indian Chief of Seattle’ retorted to ‘The great White Chief in Washington’ who wanted to buy their land, that \textit{if one does not own natural features and entities, how can one buy it?}\footnote{* Independent Member, Knowledge Network Experts, United Nations Harmony with Nature. This article, as the name suggests is a revisit - more of an introspection which aims at putting across suggestions to use religion as a tool to foster environment protection. The first part of this article has been published by CEERA, NLSIU in 2013. Manjeri Subin Sunder Raj, \textit{Fostering Environment Protection: Is Religion the MuchAwaited Panacea?} CEERA Newsletter 2013, p. 7.}

\vspace{-.15cm}

\noi
If one takes a walk back in time, there is no doubt that one would come across the myriad
debates and deliberations that have taken place to fathom, realise and curb the impact Mother
Nature endures, as a result of the ‘developmental activities’ that are ever on the increase. It has
held fort for quite a long time, ever changing. The time has exceeded for a remedy to pop up,
just like that, almost like a magic potion, and present hope and succour, and translate itself to
a much needed, much awaited, saviour for humankind as it is ever dependent on Mother Nature,
without whom it has no future.

\vspace{-.15cm}

\noi
The first way out that comes across one’s mind, when one talks about a saviour, is obviously, law - a factor that has played a great role in bringing about some sort of a control over mankind over time; a factor that still plays a great role, if not the greatest to ensure that man’s activities are kept under control. It is built on a structure composed of a large number of component concepts. These are identified to be the elements of law and as a result are perched on a high pedestal.

\heading{Meaning Demystified}

\vspace{-.15cm}

\noi
The inimitable perception relating to \textit{‘Rights’} and \textit{‘Duties’} as parts of law also arises from the said \textit{‘elements of law’}.\footnote{For more see, F Schmidt, The Four Elements of Law. The Cambridge Law Journal, 33(2), 1974, 246-259.accessed on 19/02/2017.} These components, by its sheer presence tend to clarify the stand of law and the need for the same, as compared to the needs of the society in which it exists.\footnote{Joseph W. Bingham, The Nature of Legal Rights and Duties, Michigan Law Review 12, no. 1 (1913): 1-26. Accessed 24/02/2017. For a sociological take see, Christopher N. J. Roberts, Human Rights and Sociological Duties", Sociological Forum 32, no. 1 (2017): 213-16, accessed on 19/02/2017.} Their presence becomes an absolute necessity. The very spirit and vigour of law which helps such a system to command supremacy over the society and binds it materialises with these concepts. This in turn helps the very functioning of every legalsystem; ensuring that it works in a smooth
and effective manner. Ever since there was the birth of ‘state’, the population looked upon it
and saw it as a ‘provider of rights. What was envisaged by the whole populace, i.e., the
governed, was that in return of them surrendering before the sovereign, the sovereign who
would be the all-powerful, would provide the necessary ‘rights’ and adequate protection. The
corresponding part of ‘Rights’ were the ‘Duties’ that were cast upon the populace and they are
known as correlative of rights.\footnote{Ori J. Herstein, A Legal Right to do Legal Wrong, Oxford Journal of Legal Studies 34, no. 1 (2014): 21-45, accessed February24, 2017. See also, John S Mackenzie,. Rights and Duties, International Journal ofEthics 6, no. 4 (1896): 425-41. See also, Abbye Atkinson, Borrowing Equality, Columbia Law Review120, no. 6 (2017): 1403-470.} A legal duty is the legal condition of a person whom the law
directs to do or not to do an act.\footnote{Susan S. Silbey, The Availabilityof Law Redux: TheCorrelation of Rights and Duties, Law \& SocietyReview 48, no. 2 (2014): 297-310. See also, Henry T. Terry, Legal Duties and Rights, 12 The Yale Law Journal 185 (1903), at p.186, available at \url{http://www.jstor.org/stable/781938,} accessed on 19/02/2017.} The act commanded or forbidden is known as the content of
the duty.\footnote{Jules L Coleman,. Legal Duty and Moral Argument, Social Theory and Practice 5, no. 3/4 (1980): 377-407, accessed on 19/02/2017; Henry T. Terry, The Correspondence of Duties and Rights, 25 The Yale Law Journal 171, at p. 172, available at \url{http://www.jstor.org/stable/786397,} accessed on 19/02/2017.} It is the subject matter of the duty.\footnote{Henry T. Terry, Duties. Rights and Wrongs, 10 American Bar Association Journal 123 (1924), available at  \url{http://www.jstor.org/stable/25711521, accessed on 19/02/2017.}} They apply regardless of whether or not one desires to do that which one has a duty to do. If a person has a duty to carry out an act he is supposed to do, even if he likes it or not, he has to do it because it is his duty.\footnote{Joseph Raz, Liberating Duties, 8 Law and Philosophy 3 (1989), at p. 5, available at  \url{http://www.jstor.org/stable/3504627, accessed on 19/02/2017.}} Corresponding duties which are owed both to the sovereign as well as to the other people arise in response to the rights that we have which are given by the sovereign. Thereby the concept of duties has
some sort of a relation to the concept of rights and they are correlative. But then, Austin had,
said that all rights have correlative duties but not vice versa.

\noi
It is quite crucial that for a legal system to be effective and smooth functioning, a healthy bond
should exist between the rights and duties that are prevalent in the society, both individually as
well as collectively. Not being superior but to be able to exist together, symbiotically is what
is needed. This in turn ensures the relation between members be healthy as well as foster better
societal needs in general.

\noi
Human tendency is concerned more on rights and not on duties. Human, in general, are
concerned about what they are eligible to and in very few circumstances do they think beyond
the \textit{right} aspect and cross over to the \textit{duty} aspect. The time that one understands that duties too
are an integral part of the system and they hold an equal, if not more importance as when
compared to rights, which in turn ensures the smooth functioning of the society and the legal
system which control the society, has long gone by.

\heading{Religion: A Saviour?}

\noi
Human race has a duty to protect the environment. This has taken shape as a co-relative to the
right that we have as regards the environment, i.e., the right to live in a healthy environment.\footnote{This right is included in the Indian Constitution, under Article 21.}
The time is not far, before we lose our environment, to the uncontrollable expanse of human
activity and the problems that it gives rise to.

\noi
Having come face to face with such a circumstance, whether religion is able to ensure a higher
degree of compliance, is an area that needs attention. The role it has played is understood well
and is quite definitive. Starting from determining a man’s attitude and way of life, to affecting
almost all his day-to-day activities; religion has been able to carve a niche for itself over
mankind. An attribute of religion which differentiates it from law can be understood to be that there is no fear of sanction \textit{as} far as religion is concerned as is present in law. Religion urges
one to do his duty that has been imposed by the principles and puts it across that if one fails to
do so, one would definitely suffer divine consequences that are beyond man’s comprehension.
This, to a religious person is much bigger a reprimand than anything that can be accredited to
law. The fear that a divine sanction would work against him in some way ensures that a person
would not do anything which are against his religion. This in turn leads to an observation that
religion plays a greater role in ensuring that man does his duty. As opposed to Law which
speaks of ‘rights’ more often than ‘duties’, religion speaks more of the ‘duties’. Whether
religion squarely fits into the scheme of environment protection is debatable. Since one can
relate religion in so far as it has instigated the individual to perform solemn obligations of social
life from time immemorial,\footnote{Varapa Rakrachakarn, George P. Moschis, Fon Sim Ong, and Randall Shannon, Materialism and Life Satisfaction: The Role of Religion, Journal of Religion and Health 54, no. 2 (2015): 413-26; J Hill, Rejecting Evolution: The Role of Religion, Education, and Social Networks. Journal for the Scientific Study of Religion, 53(3), 575-594; Colleen Murphy, "Religion \& Transitional Justice." Daedalus 149, no. 3 (2017): 185-200.} there is little to think of it being considered as one of \textit{the} factors that determine man’s behaviour and shapes his conduct. But what needs to be realised is the way in which it surpasses its spiritual form and converts into physical action, wherein followers are exhorted to perform their duties thereby auguring better levels of environment protection.
The fear of divine sanction gives religion an enviable position, because it is the reason why
religion has been able to exert a solid control over the man’s actions and behaviour.

\heading{Theoretical underpinnings}

\noi
Lynn White in his essay, ‘The Historical Roots of Our Ecological Crisis’, had opined that
religion does play a great role in shaping one’s attitude.\footnote{.All about Religion and the Environment, available at \url{http://www.uvm.edu/~gflomenh/ENV-NGOPA395/articles/Lynn-White.pdf, accessed on 19/02/2017.} See also, Matthew B. Arbuckle, and David M. Konisky, Social Science Quarterly 96, no. 5 (2015): 1244-263.} Quite understandably, religion has
had a powerful impact on mankind. It has ensured that mankind as a whole act in ways which
are favourable to the environment thereby guaranteeing protection and safety.
To realise the relation religion and law has, it is quite imperative that one falls back on decided
cases which have played a great role in bringing out opinions of various jurists. \textit{In The
Commissioner, Hindu Religious Endowments, Madras v. Sri Lakshmindra Thirtha Swamiar of
Sri Shirur Mutt},\footnote{AIR 1954 SC 282} the court was of the opinion that \textit{“religion is a matter of faith with individuals or communities, and it is not necessarily theistic”.} Similarly in \textit{Davis v. Beason},\footnote{(1888) 133 US 333, at p. 342 G} it was opined that religion does have an innate connection with an individual’s idea and conception of the all-powerful. The role that needs to be played by religion, in the betterment of the society, too was dealt with by the Supreme Court of India, in \textit{Sri Adi Visheshwara of Kashi Vishwanath Temple Varanasi and others v. State of U.P. and others}.\footnote{(1997) 4 SCC 606} Religion, it was opined, was to \textit{“guide community life and advise people to follow the tenets laid down so as to ensure that an egalitarian social order can be created”.}

\noi
But then, it has not always been a cake-walk. Religion and religious rites have in quite a large
number of circumstances drawn the ire of the judiciary and the people alike. In \textit{Minersville School District Board of Education v. Gobitis},\footnote{310 US 586, 594-595, 84 L. Ed. 1375, 60 S Ct 1010 (1940)} Justice Frankfurter had to deal with the
intricacies of religion and society. He said that people should never be relieved from obeying
general law. If a person has a religious belief or conviction which contradicts the rules of the
society, it does not take away the duty to follow the general law which casts political and social
responsibilities. The Himachal Pradesh High Court in \textit{Ramesh Sharma v. State of Himachal
Pradesh}\footnote{MANU/HP/0934/2014} dealt with animal sacrifice in places of worship and laid down that such practises
need be done away with. The Court went to the extent to say that \textit{“we must permit gradual
reasoning into religion”}.\footnote{\textit{Ibid,} para 78}

\heading{An Introspection}

\noi
Environment protection is the need of the hour; it can be said without any hesitation
whatsoever. The harm that has been caused by us, humans, to the environment has been
unfathomable. Ever since man settled down, he has been in one way or the other causing some
harm to the environment. The growth of science furthered this, as man got to use a lot of
hitherto unknown, unseen and unheard ways of endangering the environment.

\noi
The greatest peril that has ever befallen on mankind has cataclysmic powers. Time was not far
away when we, humans faced the wrath of nature. It was only at a much later stage that man 
realized his folly and soon he was frantically seeking steps to overcome the harm he had
caused. It wasthen, that he realized that it was not an easy task, but an arduous one that had to
be tackled meticulously, lest all life perish. The eco-system was on the verge of a collapse that
would possibly put an end to all life; and here was man, resting on his laurels and achievements.
He was basking in the glory of his accomplishments; least cared about the consequences that
his wanton acts would one day bring. And bring it did! Soon he was running helter-skelter.
Realizing the effects that such a situation would give rise to, man engaged himself in a lastditch effort to save nature. Selfishness still rules, one can argue, for man, who is straining every
nerve of his to save the Earth, is engaging himself in such activities, when he was at the
receiving end. Little did he bother, when his actions were suffocating all other life forms.

\noi
Turning back, we can see that man’s actions were to be controlled and it was precisely as a
result of this line of thought laws were made by the state. They aimed to keep a check on the
activities of man and regulate them for common good. Punishments were present to ensure that
such unwanted actions were not done. But throughout history it was seen that laws however
stringent they may be, are seen as having a compulsory nature and has its limitations in
reforming man. Reformative punishment aims to make man understand the wrongdoing of his
acts and tries to create a sense of repentance in man when he realizes his mistake. Law, as said
earlier, many a time lacks this sort of an effect on man. Law has not been able to stop man
from destroying the environment, owing to his wanton acts. There are a lot of laws that were
passed to protect the environment, when it was realized that it needs to be protected. But were
they successful? Laws emerged as a result of the growing sense of environment protection
amongst world nations. World conferences were held and declarations were made regarding
the hot topic of ‘environment protection’. Punishments for environmental degradation, as said
earlier, were galore and they were made out for a variety of offences, under numerous laws.
The enforcement mechanism, though not satisfactory in implementation, made its presence
felt. Various bookings under various Acts were made and lots of people were punished.
Environmental protection was the watchword. Thus being so, people were aware of the
consequences they would face, if they do anything which was in contravention of the law
present.

\heading{An Analysis}

\vspace{-.15cm}

\noi
But was the so called ‘law’ successful in protecting the environment by deterring and more
importantly reforming man from his anti-environment acts? The answer is a big and blatant, on the face, ‘NO’! This is made evident by the innumerable circumstances that have been
caused by man, wherein he was plundering the resources given by nature, shamelessly and
consciously. Mother Nature was suffering at the very behest of her ‘loving’ sons and daughters!
Law, with all the sanction that it prescribed was not able to restrict man from harming the
environment.

\vspace{-.15cm}

\noi
For sure, law made its presence felt when people were punished for actions harming the
environment. But then still people continue doing so. Law, it can be said has had a very limited
role in making man aware; making him realize the gravity of the situation. This is because
formal laws do not have the effect of creating awareness; as for the common man, it is made
by the rulers and imposed on him. This situation happens even in a democratic society, as the
common man’s participation is very limited in law making. Law thus remains aloof from the
people for whom it is made. The compliance is ensured not through the hearty co-operation of
people, but through coercive measures. This is the greatest limitation of law in any case, and
particularly so in the case of protection of environment.

\vspace{-.15cm}

\noi
Law has had a very restricted effect in the sense that we all are aware of the fact that
environmental pollution and degradation is on the rise and has reached at levels which are
threatening the very existence and continuation of all life on planet Earth. This situation has
arisen due to mankind’s irresponsibility and uncontrolled wanton acts. Though laws were
present, they were not able to ensure that man stop his activities that harm nature. Even in the
presence of such a large number of laws, that purportedly were enacted to provide some sort
of a succor to the tormented environment, there was still no dearth of instances wherein man
continued his rampage and destroyed nature.

\vspace{-.1cm}

\noi
Law, it can be argued, had miserably failed in creating awareness in man as regards
environment protection. More importantly, it has failed in creating a thought in man that he is
part of the environment, and it is his duty to ensure its protection, owing to his superior place
on Earth. Man had this sort of a feeling long time back; that he was part and parcel of nature
and was to revere nature. This feeling that existed has been clearly depicted in the musings of
the Indian Chief.\footnote{Supra n.1} It was this sense of oneness that was to be created so as to ensure that man
had an inner feeling, an innate urge, to live in tune with nature and take care of it.

\vspace{-.1cm}

\noi
But why has law failed? This called for a thorough introspection. With the onset of laws, it
can be seen that man was concerned more about his rights. Duties, a component of similar
status, if not more, were completely overlooked by man. Overshadowed by rights, duties had
been treated with ‘contempt’, if liberty can be taken to use such strong words. This ‘alien’
treatment of duties is what has been the main reason for the failure of law.

\vspace{-.1cm}

\noi
The reason for this failure can be attributed to the compulsory nature of law; the ‘imposition’
of it on man. Law is an external agency. It is a child of the state. It has not come from within.
Had it had come fromwithin; it would have definitely played a greater part in controlling man’s
actions. Though it can be said without an inkling of doubt that religion has the necessary
attributes to foster human duty to protect the environment, whether it would transform itself
and be successful or just be an over hyped panacea, only time would tell! But if it is so, what
legal systems, around the world, can imbibe from religion to guarantee better protection of the
environment can be looked into and action can be taken to help law enjoy a higher level of
compliance as well as create an inborn affinity, which it previously did not have.

\vspace{-.2cm}

\heading{Religion and its relevance}

\vspace{-.1cm}

\noi
Religion was mooted as the alternative for law. Religion, as we all are aware of, has a control
over its followers that is far beyond common man’s perception. It has been able to mould man
and restrict his activities in a way that is conducive to its teachings. Religion thereby exerts a
much higher influence on man than laws that are present.

\noi
Religion, it can be found has an impact on man that law is not capable of exerting. Law has
its limitations, owing to man treating it as an external factor; a factor that is imposed on man
by reason of him being a subject of a particular land. It is this aspect of religion that helps it
big time to score in an area where law has not been able to make its mark. Stamping its
authority, religion it can be said, has been highly successful in checking the actions of man and
creating in him a sense of togetherness with all things present in nature in general and all life
in particular.

\heading{Suggestions}

\noi
Law’s relevance cannot be done away with. To have the sanction of the State is one of the
essential features of law. That is where the strength of law lies. Religion has a drawback when
compared to law in the sense that its principles are ‘binding’, rather followed, only by
followers. It lacks the power to unite people from all faiths. So what is proposed is that an 
atmosphere be made so as to enable the principles of religion to be incorporated into law; as
well as to protect by law what has been protected by religion. By such a step it can be made
applicable to all, irrespective of the religion that one belongs to. Steps, a few which have been
listed out, can be initiated so as to ensure that such principles are incorporated in law.

\vspace{-.3cm}

\begin{enumerate}[label=$\bullet$]
\item A brief study of various religions should be made compulsory in schools, so that the
students get an idea about various religions that are practiced and the principles that are
present in them. The incorporation of principles as regards environment protection and
ways in which the environment is to be treated that are embodied in religious texts
should be compulsorily taught at schools. The students should be made aware of the
consequences of damage caused to the environment. Everything in nature responds to
our acts, irrespective of it being good or bad. The fact that trees respond to the love and
affection showered on them; that they, on realizing man’s presence, tries to catch his
attention like animals is to be realized. An opportunity for the children to realize this
should be created in the schools as a part of their curriculum. This will enable the
children to love the nature as their own friend or mother and be successful in instilling
the attitude at a very young age itself.

\item Studies which are restricted between the four walls of the classes should be done away
with and an education system that proliferate the growth of an environment friendly
approach should be created whereby the students get an opportunity to be close to
nature. By doing so an affinity would be developed inherently.

\item Keeping in mind the fact that a large majority of children lack primary education, steps
should be initiated at the grass root level so as to ensure the message of environment
protection is spread.

\item Religious practices that foster environment protection and the like should be given
much prominence and be highlighted, and people made more aware of the
consequences of such acts. In short, an education or a creating of awareness in people
as regards the benefits of environment protection is what is needed. Such awareness
should not be restricted to any age group as it would not serve the purpose. Only if the
grownups are aware would they pass on to their young the same values.

\item Curbing of religious activities that cause harm to the environment should be implemented through law after making the people aware of the harmful effects.

\item Religious leaders should come forward and play an important role in instilling in the
minds of people the need to take care of the environment.
\end{enumerate}

\vspace{-.2cm}

\heading{Concluding Remarks}

\noi
Religion to a great extent helps man realize himself, and thereby other life forms as well. It
helps him channelize his energy and intelligence. It ‘awakens’ man’s inner being. The essence
of all religions is sacred and environment friendly. While there are laws to prohibit the evil
effects of religion like Sati, child marriage etc., it is sad that there is no legal initiative in
harnessing the positives of religion and making it available to all, irrespective of religion.
Lawyers, religious leaders, social reformers, academicians, politicians and common man
together can do a lot to protect environment, by creating a feeling of oneness in the minds of
people, not by making legislations, but by arranging camps, seminars, workshops, and many
other such programmes, which should be sponsored by the State. After all, creating an innate
sense of love for the environment and all that it encompasses is the only sure way to win this
last battle; to ensure that mankind continues to roam on this Earth.
\end{multicols}
	
\label{end2017-art2}
