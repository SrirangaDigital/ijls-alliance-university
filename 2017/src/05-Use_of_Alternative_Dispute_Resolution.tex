\setcounter{figure}{0}
\setcounter{table}{0}
\setcounter{footnote}{0}

\articletitle{Use of Alternative Dispute Resolution methods for effective consumer protection- A Critical Analysis}
\articleauthor{Sree Krishna Bharadwaj H\footnote{Research Scholar, National Law School of India University, Bengaluru}}
\lhead[\textit{\textsf{Sree Krishna Bharadwaj H}}]{}
\rhead[]{\textit{\textsf{Use of Alternative Dispute Resolution...}}}

\begin{multicols}{2}

\heading{Introduction:}

\noi
Dispute resolution has two prominent categories, namely Adversarial and Non-Adversarial i.e. Adjudicatory and Non-Adjudicatory. One of the most formal methods is Adversarial which includes trial and arbitration. This method, by virtue of being the most formalized methods, is also the most widely used methods of dispute resolution.\footnote{Rick Sarre, Alternative dispute resolution and non-adversarial regulation: why are they still not mainstream and can they ever become mainstream? Asia-Pacific Mediation Forum Conference, Adelaide, 29 November – 1 December 2001.} The Non-Adjudicatory methods of dispute resolution include Negotiation, Mediation, Conciliation, and Lok Adalat.

\heading{The differences between Adversarial and Non- Adversarial process}

\noi
One of the most important differences between the two methods is that while the first method
involves a legally set up, due procedure and procedural laws, the latter does not involve any
due process.\footnote{Australian Law Reform Commission, Discussion Paper 62: Review of the Federal Civil Justice System. Sidney (1999).} The Non-Adversarial system is not coercive and the parties can withdraw
whenever they want.

\noi
The parties involved in a dispute have full control over the non-Adversarial system of dispute
resolution. In the Adversarial system, the focus is on facts and goes strictly by the facts
available on the table, whereas the non-Adversarial system considers relationships between the
disputing parties.

\noi
The Adversarial system looks at past and is regressive in nature. It goes by set precedents and
then arrives at a conclusion The Non-Adversarial system seeks to resolve issues in a manner
that maintains the goodwill relationship of the parties involved and keeps their future in
mind.\footnote{Judy Gutman, “The Reality of Non-Adversarial Justice: Principles and Practice”, 14 Deakin Law Review 29-51 (2009).} The Adversarial system establishes liability, whereas the non-Adversarial system focuses primarily on maintaining or in fact utilizing the relationship between the two parties.

2nd page
\end{multicols}
	
