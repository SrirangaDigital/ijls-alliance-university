\setcounter{figure}{0}
\setcounter{table}{0}
\setcounter{footnote}{0}

\articletitle{Prostitution: Sexual autonomy or Exploitation of Women?}\label{2017-art3}
\articleauthor{Rachita Agrawal\footnote{Law student,NLU Delhi.}}
\lhead[\textit{\textsf{Rachita Agrawal}}]{}
\rhead[]{\textit{\textsf{Prostitution: Sexual autonomy...}}}

\begin{multicols}{2}

\heading{Introduction:}

\noi
Prostitution is perhapsthe most stigmatized and condemned line of work in the society in which
women is engaged. In most of the third world countries talking about sexuality and sexual
preferences in itself considered as a sin. Indeed, it is women who majorly take part in
prostitution as work; the over whelming majority of prostitutes are female, while nearly all of
their customers are male.\footnote{Karen Peterson Iyer, Prostitution: A Feminist Ethical Analysis, Journal of Feminist Studies in Religion, Fall, 1998, Vol. 14, No. 2 (Fall, 1998), pp. 19-44} Though largely in many parts of the world, women are considered
as property of men and perceived as a sexual commodity, but the level of perception declines
more in case of a prostitutes as they are well-thought-out as a sick person, who has no sense of
character, purity and is physically as well as morally ill.

\noi
Prostitution around the world is accepted with varying degrees of legality. In some places,
prostitution itself is legal but the varying degrees of legality. In some places, prostitution itself
is legal but the activities that make it possible-such as “soliciting” or “pandering”- are not.
Prostitution is illegal in the vast majority of areas in the United States as a result of state laws
rather than federal laws. It is, however, legal in some rural counties within the state of Nevada.
In Germany, Switzerland, Austria, Greece, Turkey, the Netherlands, Hungary, etc. prostitution
is legal and regulated.

\noi
In Indian context, traffic in human beings and forced labour is prohibited under article 23 of
the Constitution of India \footnote{Constitution of India, 1950}. As regards the Directive Principles of State Policy under Part IV of
the Constitution, it is significant to refer Article 39 which particularly contains certain
directives. Under clause (e) of Article 39\footnote{Ibid.}
, one of the directives is that the State should, in
particular, secure that childhood and youth are protected against abuse and against moral
abandonment. The objectives reflect the intent of the Constitution makers to safeguard the
welfare of the children of our country, who often become victims of immoral trafficking and
forced illegal acts.

\noi
‘Freedom’ and ‘liberty’ is usually understood to be virtually synonymous terms and defines them
both as the absence of coercion.\footnote{By defining coercion as subjection to the arbitrary will of another Hayek creates the im- pression that liberty is diminished only by the discretionary action of persons granted wide powers by the law. Hayek says that 'coercion
implies both the threat of inflicting harm and the intention thereby to bring about certain conduct' (p. 134).} Women in prostitution is often looked upon as they are
deprived of basic notion of liberty which is considered to be a state of being free from
oppressive restrictions imposed on one's way of life, behavior, profession, social or political
views by people in the mainstream society.

\noi
This research paper tries to analyze the areas where liberty of sex workers is compromised
along with how persistent inequalities have led to devastating and harmful situation in a way
that injustice is a gift to them from the society.

\heading{Transition from non-prostitute to a prostitute}

\noi
Young women or children are majority of trafficked persons who have been forced into sex
work or sexual activities. Forced human trafficking for sexual purposes is outcome of poverty,
illness in family, worse economic conditions, sexual activity for enjoyment, peer association,
family neglect often when they are in age group of 12-20 years.

\noi
Not only this but domestic clashes, drug addiction in husbands and in involuntarily, forced
rape, sexual assault, early marriages, trafficking, deceived by family or lover are also factors
contributing in substandard and worse conditions of women.

\vspace{-.15cm}

\noi
Published literature further points to an growing demand for younger children and virgins led
to overall sophistication of human trafficking systems controlled by organized crime and
development of new destinations for trafficked persons partly fueled by the fear of HIV/AIDS.\footnote{International Labor Organization. Demand side of human trafficking in Asia: Empirical findings Regional Project on Combating Child Trafficking for Labour and Sexual Exploitation (TICSA-II), Bangkok, Thailand; 2006 available at  \url{http://www.humantrafficking.org/uploads/publications/ilo06_demand_side_of_human_tiaefpdf;} (last accessed on 13-12-2017) Asian Development Bank, Combating trafficking of women and children in south Asia - Regional synthesis paper for Bangladesh, India, and Nepal. 2003 available at \url{http://www.adb.org/Documents/Books/Combating_Trafficking/Regional_Synthesis_Paper.pdf;} (last accessed on 13-12-2017) Asian Development Bank, Combating trafficking of women and children in south Asia: Guide for integrating trafficking concerns into ADB operations. 2003 available at \url{http://www.adb.Org/Documents/Guidelines/Combating_Trafficking/GuideJntegrating_Trafficking_Concems.pdf.} (last accessed on 13-12-2017)} There are certain freedoms and fundamental rights that are significant part by virtue of being born as a human being. These inalienable principles and fundamentals are violated at large extent in forced organized crimes against children at very tender age.

\vspace{-.15cm}

\noi
Prostitution as a vicious form of violence against women and maintains that, where prostitution
is legal, human trafficking will upsurge to meet the open demand for sex.\footnote{Nichols, Andrea. (2017). 10. Sex Trafficking and Exploitation of LGBTQ+ People: Implications for Practice.
10.7312/nich18092-011.} It can be noted that
it is intrinsically a form of violation of human rights of women perpetuating men's dominance
in society and forms the basis of gender inequality serving as social evil. From ancient times,
man arbitrary deprives women from material and non- material sources that she is entitled to,
behind the alleged aseptic exchange ofsex for money lies a relation of power in which women
are required to sacrifice their respect and dignity.\footnote{Agustin Vicente, \textit{Prostitution and the Ideal State. A Defense of a Policy of Vigilance;} Ethical Theory and Moral Practice, April 2016, Vol. 19, No. 2 (April 2016), pp. 475-487} Stigma, discrimination and marginalization
demeaning very existence violate the fundamental equality of all persons associated with
prostitution. In a country like India where practicing untouchability in any form is considered
to be a constitutional violation, still prostitutes are looked upon and treated as untouchables.
There might be claims that society have headed towards progressive change but at the
grassroots level, in villages or small towns women are pushed into forced sex in exchange of
money and are suffering as a victim of male dominating social order and patriarchy.

\vspace{-.15cm}

\noi
Whoredom is “the great social evil” representing a flagrant defiance of common decency and
social norms governing women’s sexuality. The harlot is a threat to the family, societal predetermined structure and she corruptsthe young. To engage in prostitution signifies a total loss
of social image and character.\footnote{Lars O. Ericsson, \textit{Charges Against Prostitution: An Attempt at a Philosophical Assessment,} 90(3) Ethics. 335,366 (1980)} With so many negative aspects it’s hard to believe that there exists element of freedom of choice for women in prostitution. It is hard to believe that transition of most of the women into non-dignified figure in the society is a liberty that she
wants to exercise, for many liberties might have a meaning to get her body secured from
objectifying and implementation of fundamental protection provided in Article 17 and 23 of
Constitution of India.

\vspace{-.15cm}

\heading{Legal and societal response to prostitution}

\vspace{-.1cm}

\noi
Major section of society considered talking about prostitution a shameful act. Such discomfort
is not limited only to backward or uneducated people but also exists among highly educated
persons, working in apex education, administrative and health institutions, who carefully
evades pronunciation of the words like prostitution and sex-workers etc. The limited scope of
education on sexuality in schools and condemning sex talks in public makes it clear that sex is
considered a taboo in countries like India. And, in a social and cultural structure that makes sex
a taboo, legalizing sex work is almost blasphemous.\footnote{Palak Sharma, Legalizing sex work: both sides of the debate (2017) available at \url{https://blogs.lse.ac.uk/socialpolicy/2017/12/10/legalising-sex-work-both-sides-of-the-debate/} (last accessed on 13-12-2017)} The prostitute body carries with herself
the imposed burden of countless identities by the mainstream society dominating her existence.
An identity is a mirror ofthe perception of what societydictatesthe norms of what is acceptable
and unacceptable behaviour in the society.\footnote{Shannon Bell, Reading writing and rewriting prostitute body 1 (1994)}

\vspace{-.15cm}

\noi
An Act named as The Immoral Traffic (Prevention) Act, 1956,\footnote{Act No. 104 of 1956} has been promulgated in
pursuance of the implementation of International Convention signed at New York on the 9th
May, 1950, for the deterrence of immoral traffic. The preamble of convention\footnote{Convention for the Suppression of the Traffic in Persons and of the Exploitation of the Prostitution of Others,
came into force on 25 July 1951. It was signed by Shri Gopala Menon on behalf of India on the 9th May, 1950.} states:

\vspace{-.1cm}

\noi
\textit{"Whereas prostitution and the accompanying evil of the traffic in persons for the purpose of
prostitution are incompatible with the dignity and worth of the human person and endanger
the welfare of the individual, the family and the community"}

\vspace{-.1cm}

\noi
The Suppression of Immoral Traffic Act, 1956 was put into operation in the year 1958 (SITA
for short),\footnote{By Section 3 of the Suppression of Immoral Traffic in Women and Girls (Amendment) Act, 1986 (44 of 1986) the nomenclature of the Act has been changed with effect from 20th January, 1987. Now it stands asthe Immoral Traffic (Prevention) Act 1956 (104 of 1956).} aimed at defeating the evils of prostitution, and to provide opportunity to fallen
women and girls to acclimatize and rehabilitate themselves as decent members of the society.\footnote{State of Uttar Pradesh v. Kaushailiya, AIR 1964 SC 416 at 419, 420} But contrary to all expectations, the said Act has failed to prove an effective deterrent to the commercialized vice of prostitution which still continues unabated, although surreptitiously, in the different industrial belts.

\vspace{-.15cm}

\noi
Judicial activism and public interest litigations has emerged over the years, judiciary has a
significant influence and manifestation on how an issue is perceived. There has been a constant
effort by court to examine into situations of lawlessness that demand consideration, and tried to give a valued well-reasoned judgement. In \textit{Vishal Jeet vs Union of India}\footnote{Vishal Jeet v. Union Of India And Ors , 1990 AIR 1412} the court claims the forced prostitution is “destructive of moral value” and also states that rehabilitative measures would be taken for “these fallen women.”

\vspace{-.15cm}

\heading{Different Moral standard for men and women}

\vspace{-.1cm}

\noi
Women from time memorial have been considered as a social entity and classified on the basis
of morality, chastity, purity and promiscuousness, her individuality is denied and suppressed
in different forms by the prevailing structure. However, promiscuousness cannot be viewed as
prostitution, when there was no system of marriage in the early Vedic era.\footnote{S.N. Sinha \& N.K. Basu , History of Marriage and Prostitution, Vedas to Vatsyayana56(1992)} It is ironical to
note that even in the Rigveda the first text that indicates prostitution in the name of female as
illicit lover, there is no question on the character and chastity of the male counterpart. Women
who could not find husbands, or had grieved early widowhood, or who had been forcefully
violated, abducted or enjoyed therefore, deprived of an honourable status in the main stream
society.\footnote{Sukumari Bhattacharji, Prostitution in Ancient India, 15(2) Social Scientist. 32,61 (1987)} The situation of women as prostitute in past was miserable and same is today, all the
burden ofshame and disgrace is contemplated upon the women in contrast there are no morality
standards are set for men. Women identity in the societal set up has been gradually abridged
from “a whole person to vagina and womb” rendering them fit only for sex-labor and perceived
as a sexual property of men.\footnote{Dworkin, A. (1983). Right-Wing Women. New York: Putnam. p. 79}

\newpage

\noi
There are both male and female commercialized sex workers within the continuum of
commercialized sex practices.\footnote{Debra Satz, Markets in Women's Sexual Labor, Vol. 106, No. 1 (Oct., 1995), pp. 63-85} However, according to Hochschild (1983),\footnote{Hochschild, A. (1983).The Managed Heart: Commercialization of Human Feeling, London: University of
California Press} there is a significant and more widespread expectation for women to engage in the field of commercialized sex than there is for men. Female sex workers are more vulnerable to violence, exploitation, stigmatization and disease than men. On comparing arrests by police officials in cases of prostitution, it is observed that commercial sex workers, who are predominately woman, are arrested ten times more often than their clients, who are predominately man.\footnote{World Health Organization, Regional Office for the Western Pacific. (2001). Sex work in 4 Asia. Manila,
Philippines: Author.}

\vspace{-.15cm}

\noi
Males who are often involved in prostitution think they are really big and real brave as they
can dominate sexuality of a women and treat them as their slaves. They're very proud of themselves-they brag a lot and for satisfaction of their ego, there have been instances that
sometimes they get involved in abusive sexual activities.\footnote{Andrea Dworkin, Prostitution and Male Supremacy, 1 MICH. J. GENDER \& L. 1 (1993)} There is a necessity to look into the
role of men in romanticizing prostitution and in this process its cost to women as she becomes
culturally invisible and subjected to harsh inequalities. Often male use the power of societal
structure, the economic power, the cultural power and the social power, to create silence among
those who have been injured, miffed and the silence of the women who have been used for
sexual pleasures.\footnote{Ibid.}

\vspace{-.15cm}

\noi
Women as a prostitute is being subject of humiliation and symbol of dirt in the mainstream
society and over the years have been overpowered by mail dominance and desires.

\vspace{-.15cm}

\heading{Identity as a prostitute – Choice or force?}

\vspace{-.15cm}

\noi
There have been contrasting views whenever it comes to identity of a women as a prostitute, is
it a symbol of sexual autonomy and empowerment or as a result of bleak factors resulting in
this demean recognition.

\noi
A prostitute body by people at large in the society is understood as “flesh and blood body hired
in exchange of payment for sexual interaction.”\footnote{Supra Note 13} It is also not at all surprising to learn that
wherever women choose to engage in sex work, they are forced to live on the margins of the
law. A women enforced in prostitution resign herself to regular rape, thrashings, discernment
and all other forms of mistreatment that a human being do not deserve.\footnote{Swagata Sen and Sushmita Choudhary, Gen-Me: Word by Word, India Today (20/02/2006). Available at \url{http://indiatoday.intoday.in/story/ideas-and-trends-that-explain-the-indian-youth/1/181898.html} (last accessed on 13-12-2017)} Fundamental dignity
and respect as a human being is sacrificed on all fronts.

\noi
Fundamental feminists condemning the world’s oldest profession, therefore do not view sex
work as a legitimate form of labor. The reason is very much apparent and substantial as sex
work embodies male domination over women sexuality and private parts. In problematic
historical reference to the anti-slavery movement, these fundamental feminist advocates call
themselves “abolitionists”. They seek to supposedly liberate women from shackles of
patriarchy controlling and commodifying sexual organs by ‘abolishing prostitution’.\footnote{International Committee on the Rights of Sex Workers in Europe, 2016, “Feminism Needs Sex Workers, Sex Workers Need Feminism: For A Sex Worker Inclusive Women’s Rights Movement” available at \url{http://www.sexworkeurope.org/news/general-news/feminism-needs-sexworkers-sex-workers-needfeminismsex-worker-inclusive-women} (last accessed on 13-12-2017)}

\noi
Women prostitutes had to face worse and severe consequences in respect of their health like
unwanted pregnancy, abortion, undesirable children, HIV/AIDS as a result of unprotected
sexual activities which are part and parcel of this profession. Drug addiction, forcefully abusive
acts, violence, threatening, involvement with criminals, ovarian issues, some other
psychological and social issues in respect of customers' behavior like refusing of payment,
beating, kidnapping and sexual assault have magnified impact on their mental well-being as
well.\footnote{Shahid Qayyum, Causes and decision of women'sinvolvement into prostitution and its consequences in Punjab, Pakistan, Vol - 4 Academic Research International}

\noi
With the intensification of the HIV/AIDS epidemic in the 1980s, commercial prostitutes again
became scapegoats to proliferate sexually transmitted diseases.\footnote{Sacks, V. Women and AIDS: An analysis of media misrepresentations, Social Science \& Medicine, 42, 59–73 (1996)} Many commercial sex
workers were exposed to greater risk of HIV/AIDS and other sexual transmitted infections as
they were unable to enforce the use of condoms by the clients.\footnote{Supra 24} Absence of consent and
autonomy while performing sexual acts is a kind of forced violence against female mind, body
and reputation. In societalset up like India, where having sex before marriage is considered to
be sin and women are deemed as characterless, impure etc. if they do so. Acceptance of
prostitutes as a dignified human being is a far from reality in such societal and mentalset up.

\noi
Fundamental feminism views female sexuality as intimately connected to norms establishing
male dominance in the society. Some fundamental feminists argue that over the years all sexual
relationships of women with men are inherently subordinating and controlled by desire of
man.\footnote{Leeds Revolutionary Feminist Group, Love Your Enemy? – Debate Between Heterosexual Feminism and Political Lesbianism, (London: Only women Press, Ltd., 1981)} Also, trading sex for money is contemplated to be demeaning women and involve the unacceptable commodification of female sexuality.

\noi
The counterargument views hold that prostitution is a form ofsexual emancipation, expression
ofsexual desires and women’s agency. Moen is his study \textit{‘Is prostitution Harmful’}\footnote{ Ole Martin Moen, Is prostitution harmful? Journal of Medical Ethics , February 2014, Vol. 40, No. 2 (February 2014), pp. 73-81} suggested
that the way people stigmatize and disvalue those who work as prostitutes it is the primary
source of harm to them. Virtually any job or activity would correlate with significant
tribulations to its practitioners if the social order and culturally established norms treated them
the way prostitutes are typically perceived and treated. Stigma associated with women as a prostitute also averts sex workers from discover employment and clients in the formal labor market.

\noi
‘Sex-work is work’ feminists differ in the degree to which they perceive sex work as something
which may be empowering or an overall positive experience (sex-radical feminists), and those
who think that sex work is largely a negative experience, similar to many other forms of
precarious work under capitalism.\footnote{International Committee on the Rights of Sex Workers in Europe, 2016, “Feminism Needs Sex Workers, Sex Workers Need Feminism: For A Sex Worker Inclusive Women’s Rights Movement.”}

\noi
Long held debate that the agency of prostitutes must be recognized and protected have been
supported by sex workers’ rights activists, feminist allies and human rights advocates. That all
aspects of sex work should be decriminalized bringing a change into already established
societal norms, and that sex in exchange of money should be recognized as work and regulated
under legal frameworks.\footnote{Sex Work and Gender Equality, Global Network of Sex Work Projects 2017}

\noi
The constant tussle between both the views is unending, what is more important is freedom as
what it exists. Freedom of women to choose between living a life as prostitute or as a not
prostitute is what seems most significant. The path a woman chooses as a right of her freedom
should be respected and look upon with dignity. Acts imposed on women by coercion in any
form are anti-thesis to their freedom of choice.

\newpage

\heading{Conclusion}

\noi
All individuals irrespective of their gender, caste, profession etc. must be afforded
social/economic justice, equal educational, employment opportunities and access to necessary
health care (including mental health and substance abuse treatment). The taboo created around
sex work, is leading to marginalization of prostitutes from main stream society. There exists no
element of sexual autonomy, prostitution is result of forced factors for instance worse financial
conditions, debt, sex for pleasure, peer association, family abandonment, domestic clashes,
betrayal from lover etc.

\noi
Internationally, United Nations Protocol to Prevent, Suppress and Punish Trafficking in Persons adopted in 2000,\footnote{Violence against women Definition and scope of the problem available at  \url{https://www.who.int/gender/violence/v4.pdf (last accessed on 13-12-2017)}} United Nations Declaration on the Elimination of Violence against Women in 1993, the United Nations Convention on the Elimination of all Forms of Discrimination against Women in 1979,\footnote{The Convention on the Elimination of All Forms of Discrimination against Women (CEDAW) available at \url{https://www.un.org/womenwatch/daw/cedaw/cedaw.htm} (last accessed on 13-12-2017)} and these are instruments which tries to condemn forced sex work and support the elimination of human trafficking for sexual abuse and exploitation.

\noi
There should be access to all possible health care facilities and HIV prevention activities for
sex workers ensuring equality and liberty of their community, acceptance in the mainstream
society as a human being along with reduction in negative factors contributing in forced
prostitution to secure justice.

\noi
There should be access to all possible health care facilities and HIV prevention activities for
sex workers ensuring equality and liberty of their community, acceptance in the mainstream
society as a human being along with reduction in negative factors contributing in forced
prostitution to secure justice.

\noi
Many injecting drug users, male and female, work as a sex worker to fund their drug addiction
habit. The requirement to earn money for drugs can often supersede the desire to perform safe
sex and dignified choice. Many clients who are aware of this weakness, will attempt to procure
person for sexual interaction in exchange of less money, sometimes without a condom,
exposing the person to sexually transmitted risks. Programs for drug treatment and reducing
the risk of HIV should be initiated to with a aim of increasing awareness and understanding
about these sensitive issues, breaking the vicious circle of sex work for drugs.

\noi
Child and adolescent victims and survivors of marketable sexual manipulation and sex
trafficking should be given community-based training and enduring education along with
proper health consultancy. The focus should not only be on sex workers, but the customers of sex
work who have been criminalized should be given regular sessions, as to what they are doing
is not natural rather it is an act of subjugation, violence equivalent to harassment and rape,
demeaning the life of person by pushing them into darkness where no autonomy to oneself
exists.
\end{multicols}
	
\label{end2017-art3}
