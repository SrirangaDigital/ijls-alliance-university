\setcounter{figure}{0}
\setcounter{table}{0}

\articletitle{A Critical Comparative Analysis on Unfair Trade Practice in India with Special Reference to United States \& United Kingdom}
\articleauthor{Anita A. Patil\footnote{Research Scholar, National Law School of India University, Bengaluru}}
\lhead[\textit{\textsf{Anita A. Patil}}]{}
\rhead[]{\textit{\textsf{A Critical Comparative Analysis...}}}

\begin{multicols}{2}

\heading{Introduction:}

\noi
The world today, unlike the world before the industrial revolution and the rapid growth of
international trade and trade, is an interconnected global economy. This industry boom has
resulted in a wide variety of consumer productsto appeal to newly evolving consumer demands
and a range ofservices as well. At the same time, distributors and suppliers have become more
and more organized. This created a discrepancy in the negotiating power of the weaker party
in the customer transaction. A variety of legal initiatives have been adopted and guidelines and
regulations have been framed, both at the international and domestic level, in order to
discourage and protect the abuse of consumers. India has a special statute, the Consumer
Protection Act 1986, to protect the interests of the consumer.

\noi
The author has attempted to understand the ground level reality of these practices especially in
places like airports, five star hotels, shopping malls and multiplexes where the balance of the
bargaining power is heavily tilted in the favour of these institutions and study the treatment of
the same under the laws currently in effect. The has focused on the major differences that are
prevalent between the consumer protection legislations of the two major powers, the United
States of America and the United Kingdom paper to evaluate the efficacy of their laws \textit{vis-à-vis} each other. Ultimately, an attempt has been made to study the current consumer protection
legislation that is present in India for the protection of the consumers against the unfair trade
practices and Consumer Protection Bill, 2015 has also been touched upon and suggestions to
improve the same have been provided thereto.

\noi
Unfair Trade Practice includes a wide variety of violations, many of which include economic
harm caused by misleading or wrongful actions. Claims such as trade secret misappropriation,
unfair competition, misleading advertising, palming-off, dilution and disparagement are among
the legal theories that can be claimed. Although attempts have been made to strengthen the
customer's status and attempts have been made to make the consumer the king, he or she is still
vulnerable to many unfair market practices implemented by retailers and manufacturers to achieve greater profits and sales revenues. There is a great deal of manipulation that revolves
around product pricing.

\noi
Price is often the most important determinant in translating a consumer’s interest in the product
into sale. The market forces of demand and supply may sometimes make the consumer
vulnerable and the retailer could easily exploit the situation to extract a larger price. In order to
prevent this from happening, there are some requirements like declaration of the maximum
retail price on the packaging of products. However, this is insufficient as there are still
widespread instances of non-declaration, overpricing and differential pricing.

\heading{Unfair Trade Practice}

\noi
Before delving into a critical analysis of the unfair trade practices specifically related to the
maximum retail price, it is important to obtain a deeper understanding of these two crucial
terms. As it is known, the legal definition and interpretation of a lot terms differ from the usage
they have in common parlance.

\noi
Section 2(1)(r) of the Consumer Protection Act, 1986\footnote{Consumer Protection Act, 1986.} defines the term “unfair trade practice” Section 2(1)(r) of the Consumer Protection Act, 1986 defines the term “unfair trade practice” and includes under its ambit a wide array of practices deemed to be disadvantageous to consumers.\footnote{\textit{“a trade practice which, for the purpose of promoting the sale, use or supply of any goods or for the provision of any service, adopts any unfair method or unfair or deceptive practice including any of the following practices,”} namely; —
(1) the practice of making any statement, whether orally or in writing or by visible representation which, —
\begin{enumerate}[label=(\roman*)]
\item falsely represents that the goods are of a particular standard, quality, quantity, grade, composition, style or
model;

\item falsely represents that the services are of a particular standard, quality or grade;

\item falsely represents any re-built, second-hand, reno•vated, reconditioned or old goods as new goods;

\item representsthat the goods or services have sponsor•ship, approval, performance, characteristics,
accesso•ries, uses or benefits which such goods or services do not have;

\item represents that the seller or the supplier has a spon•sorship or approval or affiliation which such seller or
supplier does not have;

\item makes a false or misleading representation concern•ing the need for, or the usefulness of, any goods or services;
\end{enumerate}}

\noi
The above is a partial extract from Section 2(1)(r) and includes, inter alia, the unfair denial of
a transaction, the exclusion of unfairly justified rivals, the unfair solicitation of customers, the
intimidation of customers, the unequal treatment of a transacting party, the unfair abuse of a
specific bargaining position by the transacting party, the conduct of business and conducting
business under terms and conditions that unfairly curtail a negotiating party's trading ventures, "disrupting another company's business activities, and unequal provision of financial, assets,
manpower, etc.\footnote{P.K. Majumdar, LAW OF CONSUMER PROTECTION IN INDIA, 102.}

\noi
Unfair trade practice has much to do with unjustified injury to the consumer: such injury must
be substantial, not outweighed by any benefit to the consumer, not adverse to competition, and
the injury must be such that the consumer could not have reasonably avoided it.\footnote{J.N. Barowalia, COMMENTARY ON CONSUMER PROTECTION ACT, 1986, 13, (2008)} It therefore follows that any injury will not be considered to be ‘unfair’. Substantial injury includes monetary harm, damage to the health of consumers, coercion, and information asymmetry amongst other things.\footnote{Majumdar, Supra note 3, p. at 102.}

\noi
Emotional disturbance or mental injury is not usually considered to be grounds for unfair trade
practice, although there are exceptions to this. The Consumer Forums and Commissions take
into account the economic costs incurred by the company in question, and coststo society such
as reduced innovation and more regulation on the flow of information.\footnote{Lydia Kerketta, Unfair Trade Practices in India, (July 13, 2015), LEGAL SERVICES INDIA, available at  \url{http://www.legalservicesindia.com/article/article/unfair-trade-practice-in-india-1861-1.html} (Last visited on
December 1, 2015).}

\noi
Until the Consumer Protection Act, and after it the Monopolies and Restrictive Trade Practices Act,\footnote{Monopolies and Restrictive Trade Practices Act, 1969.} came into force, the principle behind buyer-seller interactions in India was that of \textit{caveat emptor} or “let the buyer beware”. The buyers were expected to obtain sufficient information about the product/service and only then purchase it.

\noi
However, in this day and age, rising competition and globalisation has led to numerous cases
of unfair trade practices. Some may even say that we are moving into the era of \textit{caveat venditor,}
i.e. “let the seller beware”. The author would not go as far to agree with that sentiment;
however, we must admit that a large benefit has accrued to consumers with the genesis of the
Consumer Protection Act, 1986.

\noi
One amongst many unfair trade practices is that of charging a higher Maximum Retail Price
than that printed on the outer packaging. There have also been instances where a higher MRP
is printed on the same product, with the same quality, but to be sold at a different place. Such
places commonly include airports, malls, multiplexes, and five star hotels. Various aspects related to this are explored in the upcoming parts of the paper. Section 2(1)(r) of the Act is in
furtherance of free and informed C2B (consumer-to-business) transactions and interactions.

\heading{Maximum Retail Price}

\noi
Today, one can see all packaged goods in India, be it beverages, electronic goods or cosmetics,
stamped with a price. This is dictated by the manufacturer as the maximum allowable cost to
the consumer. However, this system of Maximum Retail Price (MRP) is exclusive to India and
Sri Lanka. Even in India this was not always the case. Most other countries have a system of
‘suggested retail price’.\footnote{J. Blythman, RETAIL MARKETING, 197.}

\noi
India has seen the evolution of several pricing mechanisms over several decades. Up to 1974,
the Market determined pricing system was in force. This was replaced by the Administrative
pricing system. It remained in force for 14 years until it was decided to follow the Import Parity
Price mechanism in 1998. Then this was replaced with the Import Parity Price Mechanism.
Finally, the present pricing mechanism is the one adopted in 2006 and it is the Trade Parity
Price Mechanism.\footnote{A. Kalhan and M Franz, Regulation of retail: comparative experience, 44(32) E. P. W. 56, 57.}

\noi
The present-day version of the MRP was adopted towards the end of 1990. Before this, the
manufacture could print the price of the products in two ways. The first was in the form of
‘Retail Price + Local Taxes (extra)’. The second was ‘Maximum Retail Price (inclusive of all
taxes)’. This practice was done away with due to widespread allegations from consumers and
organisation that retailers were overcharging them under the pretext of adding additional local
taxes. In reality, the actual rate of local taxes was far lower. This malpractice resulted differing
rates – a consumer would end up paying up a far higher price than someone in the neighbouring
town.\footnote{\textit{Ibid.}} Both of these events prompted the Ministry of Consumer Affairs, Food and Public
Distribution, formerly known as the Ministry of Civil Supplies, and its executive branch, the
Department of Legal Metrology, to make amendments to the Rules of Weights \& Measures
Act (Rules of Packaged Commodities)\footnote{Standards of Weights \& Measures Act, 1976.}

\noi
This move was meant to bring the concerns and issues of overcharging customers to a full stop.
This did not fix any problems absolutely. In order to cut competition, there have been allegations of undercharging goods compared to MRP. However, the emphasis of this paper is
limited to the problems of differential pricing and overcharging from a consumer-centred point
of view.

\noi
The Consumer Goods (Mandatory Printing of Cost of Production and Maximum Retail Price)
Act 2006\footnote{The Consumer Goods (Mandatory Printing of Cost of Production and Maximum Retail Price) Act, 2006.} has established such guidelines to avoid charging the consumer more than the
maximum price printed by the producers on the packaging of the goods.

\noi
Market products, manufacturing cost, printing and maximum retail price are specified by these
guidelines. The legislative requirement to print the retail price at a noticeable place and in the
languages of English, Hindi and the local languages was created by these definitions. This act
has made it mandatory for printing.\footnote{Rule 6(1)(e), Legal Metrology (Packaged Commodity) Rules, 2011.}

\noi
Also, in compliance with Rule 6 of Chapter II of the 2011 Rules on Legal Metrology (Packaged
Commodity), the manufacturer is required to make certain declarations on the package.

\noi
Rule 6(1)(e) mandates that the retail selling price of the package be announced. Such
exceptions are also provided for.\footnote{“Provided that for packages containing alcoholic beverages or spirituous liquor, the State Excise Laws and the rules made there under shall be applicable within the State in which it is manufactured and where the state excise laws and rules made there under do not provide for declaration of retail sale price, the provisions of these rules shall apply.”} Provision for prohibiting any product charging more than the MRP \& obligations on the product packaging respectively under Rule 18(2)\footnote{Rule 18(2) which states “No retail dealer or other person including manufacturer, packer, importer and wholesale dealer shall make any sale of any commodity in packed form at a price exceeding the retail sale price thereof”} \& Rule 18(5)\footnote{“No wholesale dealer or retail dealer or other person shall obliterate, smudge or alter the retail sale price, indicated by the manufacturer or the packer or the importer, as the case may be, on the package or on the label affixed thereto.}

\heading{Unfair Trade Practice}

\noi
It is important to understand the historical reasons for certain practices being termed unfair and
how the definitions have undergone changes with time in order to understand the intent of the
legislators and its current application.

\noi
For the longest time, there was little to no remedy available to consumers in India. They could
not even approach the courts to seek redressal for any grievances.

\noi
Although The Monopolies and Restrictive Trade Practices Act was intended to protect
consumers, it failed to adequately do so. Due to this, it was amended in 1984 with the
recommendations ofthe Sachar Committee, and a part on unfair trade practices was introduced:

\begin{enumerate}
\item[1)] Making false or misleading claims about a product or service through an advertisement or otherwise.

\item[2)] Offering bargain prices and bait advertisements.

\item[3)] Offering pseudo prizes or gifts and conducting sales promotion contests, lotteries, or games of chance/skill.

\item[4)] Supplying hazardous or unsafe products.

\item[5)] Hoarding or destroying goods, or refusing to sell goods, resulting in a price increase.\footnote{D.P.S. Verma, Developments in Consumer Protection in India, 25, JOURNAL OF CONSUMER POLICY, 107, p.110 (2002).}
\end{enumerate}

\noi
Earlier, the government had to take \textit{suo moto} cognizance of any unfair trade practice and pursue
the issue themselves. Certain specific enactments did allow the consumer to approach the courts
directly, but only for the particular product mentioned in such enactment. Another amendment
to the MRTP Act in 1986, gave consumers the right to approach the Commission directly,
where previously only a group of 25 members or by a consumer association with 25 members
or more could do so.\footnote{Sec 36, Monopolies and Restrictive Trade Practices Act, 1969.}

\noi
Then 1986 came, and brought with it the Consumer Protection Act, which was hailed as one of
the greatest steps towards consumer protection, being one of the most progressive socioeconomic legislations. In Section 2(1)(r) of the Consumer Protection Act, unfair trade practices
were defined, and consumers were now free to contact the forums and commissions
themselves. Also, after the Consumer Protection Act came into effect, the MRTP Commission
has tried a large number of cases relating to unfair trade practices.\footnote{Verma, \textit{Supra} note 17 at 114.}

\noi
The Consumer Protection Act kick started a long line of positive steps: the Department of
Consumer Affairs was set up in 1991; the exemption with respect to unfair trade practice
enjoyed by public sector undertakings, cooperative societies, and financial institutions was
withdrawn in 1991; and certain provisions in the MRTP Act were strengthened.\footnote{The power to compel parties to appear before it (Monopolies and Restrictive Trade Practices Act, Sec 13B),
to punish for contempt of itself (Monopolies and Restrictive Trade Practices Act Sec 12A), to issue ex parte decisions (Monopolies and Restrictive Trade Practices Act, Section 36 D(1)).} Further, issues of medical practice were also included under the ambit of Consumer Protection Act.
Such is the evolution of unfair trade practice in India.

\heading{Legal Position in India of Unfair Trade Practices}

\noi
The author seeks to provide a clear picture of the current legal position regarding the
malpractices of overpricing and hiking the maximum retail price in these sections by touching
upon the relevant sections of the various related legislations. The definitions and provisions
related to the terms ‘unfair trade practice’ and ‘maximum retail price’ have already been dealt
with in detail in the preceding sections.

\noi
Further, malls often alter the MRP on products by using price printing machines, in order to
lure customers with the pretence of a discount on the MRP.\footnote{M.S. Sreedhar, Modus operandi: How you get cheated, THE HINDU (March 28, 2013), available at  \url{http://www.thehindu.com/news/cities/Hyderabad/modus-operandi-how-you-getcheated/article4557695.ece?ref=relatedNews (Last visited on December 1, 2015).}} The goods are also sold at this
inflated price in many instances. The vendors claim to have no choice but to do this as rentals
at malls are very high.

\noi
Prahlad was traveling along the Mumbai-Goa highway in January 2009 and stopped at the
Kamat Hotel to buy a bottle of water. He charged Rs 25/- for it, but later realized that the same
bottle was being sold at other stores on the highway for Rs.15/- without any difference in
quality or quantity.\footnote{ A.J. Lakade, Hotel can’t sell bottled water at higher price: Consumer Forum, THE INDIAN EXPRESS (February 22, 2012), available at \url{http://indianexpress.com/article/cities/pune/hotel-cant-sell-bottled-water-at-higher-priceconsumer-forum/ (Last visited on November 28, 2015).}} Prahlad transferred the Pune-based Dhariwal Industries Limited (the producers of the bottled water in question) and Vitthal Kamat of Kamat Hotels to the Raigad
District Consumer Disputes Redressal Forum against Rasiklal Dhariwal.

\noi
By arguing that the extra fee was for the environment offered by the hotel, they contested the
argument. The Forum denied its appeal, arguing that it was devoid of substance and that there
were no distinguishable qualities that could justify the higher MRP. It is notable that this
decision is only applicable to retail outlets on the counter and not for sit-in customers.

\noi
In another such instance involving hotels, the National Consumer Dispute Redressal
Commission (NCDRC)\footnote{The apex consumer body with respect to C2B dispute redressal in India.} ordered Nyay Mandir, in Gujarat, to pay 1.5 lakh to the consumer
welfare fund and upheld the Gujarat State Commission's decision ordering Nyay Mandir to pay
Rs. 6,000/- as compensation to complainant Ishwar Lal Jinabhai Desai, who had approached the forum for having been charged Rs 18 for the popular carbonated drink 'Miranda', despite
its MRP being marked at only Rs. 12.50/.25\footnote{}

8th page
\end{multicols}
