\setcounter{figure}{0}
\setcounter{table}{0}

\articletitle{A Critical Comparative Analysis on Unfair Trade Practice in India with Special Reference to United States \& United Kingdom}
\articleauthor{Anita A. Patil\footnote{Research Scholar, National Law School of India University, Bengaluru}}
\lhead[\textit{\textsf{Anita A. Patil}}]{}
\rhead[]{\textit{\textsf{A Critical Comparative Analysis...}}}

\begin{multicols}{2}

\heading{Introduction:}

\vspace{.1cm}

\noi
The world today, unlike the world before the industrial revolution and the rapid growth of
international trade and trade, is an interconnected global economy. This industry boom has
resulted in a wide variety of consumer productsto appeal to newly evolving consumer demands
and a range ofservices as well. At the same time, distributors and suppliers have become more
and more organized. This created a discrepancy in the negotiating power of the weaker party
in the customer transaction. A variety of legal initiatives have been adopted and guidelines and
regulations have been framed, both at the international and domestic level, in order to
discourage and protect the abuse of consumers. India has a special statute, the Consumer
Protection Act 1986, to protect the interests of the consumer.

\vspace{.1cm}

\noi
The author has attempted to understand the ground level reality of these practices especially in
places like airports, five star hotels, shopping malls and multiplexes where the balance of the
bargaining power is heavily tilted in the favour of these institutions and study the treatment of
the same under the laws currently in effect. The has focused on the major differences that are
prevalent between the consumer protection legislations of the two major powers, the United
States of America and the United Kingdom paper to evaluate the efficacy of their laws \textit{vis-à-vis} each other. Ultimately, an attempt has been made to study the current consumer protection
legislation that is present in India for the protection of the consumers against the unfair trade
practices and Consumer Protection Bill, 2015 has also been touched upon and suggestions to
improve the same have been provided thereto.

\vspace{.1cm}

\noi
Unfair Trade Practice includes a wide variety of violations, many of which include economic
harm caused by misleading or wrongful actions. Claims such as trade secret misappropriation,
unfair competition, misleading advertising, palming-off, dilution and disparagement are among
the legal theories that can be claimed. Although attempts have been made to strengthen the
customer's status and attempts have been made to make the consumer the king, he or she is still
vulnerable to many unfair market practices implemented by retailers and manufacturers to achieve greater profits and sales revenues. There is a great deal of manipulation that revolves
around product pricing.

\noi
Price is often the most important determinant in translating a consumer’s interest in the product
into sale. The market forces of demand and supply may sometimes make the consumer
vulnerable and the retailer could easily exploit the situation to extract a larger price. In order to
prevent this from happening, there are some requirements like declaration of the maximum
retail price on the packaging of products. However, this is insufficient as there are still
widespread instances of non-declaration, overpricing and differential pricing.

\heading{Unfair Trade Practice}

\noi
Before delving into a critical analysis of the unfair trade practices specifically related to the
maximum retail price, it is important to obtain a deeper understanding of these two crucial
terms. As it is known, the legal definition and interpretation of a lot terms differ from the usage
they have in common parlance.

\noi
Section 2(1)(r) of the Consumer Protection Act, 1986\footnote{Consumer Protection Act, 1986.} defines the term “unfair trade practice” Section 2(1)(r) of the Consumer Protection Act, 1986 defines the term “unfair trade practice” and includes under its ambit a wide array of practices deemed to be disadvantageous to consumers.\footnote{\textit{“a trade practice which, for the purpose of promoting the sale, use or supply of any goods or for the provision of any service, adopts any unfair method or unfair or deceptive practice including any of the following practices,”} namely; —
(1) the practice of making any statement, whether orally or in writing or by visible representation which, —
\begin{enumerate}[label=(\roman*)]
\item falsely represents that the goods are of a particular standard, quality, quantity, grade, composition, style or
model;

\item falsely represents that the services are of a particular standard, quality or grade;

\item falsely represents any re-built, second-hand, reno•vated, reconditioned or old goods as new goods;

\item representsthat the goods or services have sponsor•ship, approval, performance, characteristics,
accesso•ries, uses or benefits which such goods or services do not have;

\item represents that the seller or the supplier has a spon•sorship or approval or affiliation which such seller or
supplier does not have;

\item makes a false or misleading representation concern•ing the need for, or the usefulness of, any goods or services;
\end{enumerate}}

\vspace{-.15cm}

\noi
The above is a partial extract from Section 2(1)(r) and includes, inter alia, the unfair denial of
a transaction, the exclusion of unfairly justified rivals, the unfair solicitation of customers, the
intimidation of customers, the unequal treatment of a transacting party, the unfair abuse of a
specific bargaining position by the transacting party, the conduct of business and conducting
business under terms and conditions that unfairly curtail a negotiating party's trading ventures, "disrupting another company's business activities, and unequal provision of financial, assets,
manpower, etc.\footnote{P.K. Majumdar, LAW OF CONSUMER PROTECTION IN INDIA, 102.}

\vspace{-.15cm}

\noi
Unfair trade practice has much to do with unjustified injury to the consumer: such injury must
be substantial, not outweighed by any benefit to the consumer, not adverse to competition, and
the injury must be such that the consumer could not have reasonably avoided it.\footnote{J.N. Barowalia, COMMENTARY ON CONSUMER PROTECTION ACT, 1986, 13, (2008)} It therefore follows that any injury will not be considered to be ‘unfair’. Substantial injury includes monetary harm, damage to the health of consumers, coercion, and information asymmetry amongst other things.\footnote{Majumdar, Supra note 3, p. at 102.}

\vspace{-.15cm}

\noi
Emotional disturbance or mental injury is not usually considered to be grounds for unfair trade
practice, although there are exceptions to this. The Consumer Forums and Commissions take
into account the economic costs incurred by the company in question, and coststo society such
as reduced innovation and more regulation on the flow of information.\footnote{Lydia Kerketta, Unfair Trade Practices in India, (July 13, 2015), LEGAL SERVICES INDIA, available at  \url{http://www.legalservicesindia.com/article/article/unfair-trade-practice-in-india-1861-1.html} (Last visited on
December 1, 2015).}

\noi
Until the Consumer Protection Act, and after it the Monopolies and Restrictive Trade Practices Act,\footnote{Monopolies and Restrictive Trade Practices Act, 1969.} came into force, the principle behind buyer-seller interactions in India was that of \textit{caveat emptor} or “let the buyer beware”. The buyers were expected to obtain sufficient information about the product/service and only then purchase it.

\noi
However, in this day and age, rising competition and globalisation has led to numerous cases
of unfair trade practices. Some may even say that we are moving into the era of \textit{caveat venditor,}
i.e. “let the seller beware”. The author would not go as far to agree with that sentiment;
however, we must admit that a large benefit has accrued to consumers with the genesis of the
Consumer Protection Act, 1986.

\noi
One amongst many unfair trade practices is that of charging a higher Maximum Retail Price
than that printed on the outer packaging. There have also been instances where a higher MRP
is printed on the same product, with the same quality, but to be sold at a different place. Such
places commonly include airports, malls, multiplexes, and five star hotels. Various aspects related to this are explored in the upcoming parts of the paper. Section 2(1)(r) of the Act is in
furtherance of free and informed C2B (consumer-to-business) transactions and interactions.

\vspace{-.1cm}

\heading{Maximum Retail Price}

\vspace{-.1cm}

\noi
Today, one can see all packaged goods in India, be it beverages, electronic goods or cosmetics,
stamped with a price. This is dictated by the manufacturer as the maximum allowable cost to
the consumer. However, this system of Maximum Retail Price (MRP) is exclusive to India and
Sri Lanka. Even in India this was not always the case. Most other countries have a system of
‘suggested retail price’.\footnote{J. Blythman, RETAIL MARKETING, 197.}

\vspace{-.1cm}

\noi
India has seen the evolution of several pricing mechanisms over several decades. Up to 1974,
the Market determined pricing system was in force. This was replaced by the Administrative
pricing system. It remained in force for 14 years until it was decided to follow the Import Parity
Price mechanism in 1998. Then this was replaced with the Import Parity Price Mechanism.
Finally, the present pricing mechanism is the one adopted in 2006 and it is the Trade Parity
Price Mechanism.\footnote{A. Kalhan and M Franz, Regulation of retail: comparative experience, 44(32) E. P. W. 56, 57.}

\vspace{-.1cm}

\noi
The present-day version of the MRP was adopted towards the end of 1990. Before this, the
manufacture could print the price of the products in two ways. The first was in the form of
‘Retail Price + Local Taxes (extra)’. The second was ‘Maximum Retail Price (inclusive of all
taxes)’. This practice was done away with due to widespread allegations from consumers and
organisation that retailers were overcharging them under the pretext of adding additional local
taxes. In reality, the actual rate of local taxes was far lower. This malpractice resulted differing
rates – a consumer would end up paying up a far higher price than someone in the neighbouring
town.\footnote{\textit{Ibid.}} Both of these events prompted the Ministry of Consumer Affairs, Food and Public
Distribution, formerly known as the Ministry of Civil Supplies, and its executive branch, the
Department of Legal Metrology, to make amendments to the Rules of Weights \& Measures
Act (Rules of Packaged Commodities)\footnote{Standards of Weights \& Measures Act, 1976.}

\vspace{-.1cm}

\noi
This move was meant to bring the concerns and issues of overcharging customers to a full stop.
This did not fix any problems absolutely. In order to cut competition, there have been allegations of undercharging goods compared to MRP. However, the emphasis of this paper is
limited to the problems of differential pricing and overcharging from a consumer-centred point
of view.

\vspace{-.1cm}

\noi
The Consumer Goods (Mandatory Printing of Cost of Production and Maximum Retail Price)
Act 2006\footnote{The Consumer Goods (Mandatory Printing of Cost of Production and Maximum Retail Price) Act, 2006.} has established such guidelines to avoid charging the consumer more than the
maximum price printed by the producers on the packaging of the goods.

\vspace{-.1cm}

\noi
Market products, manufacturing cost, printing and maximum retail price are specified by these
guidelines. The legislative requirement to print the retail price at a noticeable place and in the
languages of English, Hindi and the local languages was created by these definitions. This act
has made it mandatory for printing.\footnote{Rule 6(1)(e), Legal Metrology (Packaged Commodity) Rules, 2011.}

\noi
Also, in compliance with Rule 6 of Chapter II of the 2011 Rules on Legal Metrology (Packaged
Commodity), the manufacturer is required to make certain declarations on the package.

\noi
Rule 6(1)(e) mandates that the retail selling price of the package be announced. Such
exceptions are also provided for.\footnote{“Provided that for packages containing alcoholic beverages or spirituous liquor, the State Excise Laws and the rules made there under shall be applicable within the State in which it is manufactured and where the state excise laws and rules made there under do not provide for declaration of retail sale price, the provisions of these rules shall apply.”} Provision for prohibiting any product charging more than the MRP \& obligations on the product packaging respectively under Rule 18(2)\footnote{Rule 18(2) which states “No retail dealer or other person including manufacturer, packer, importer and wholesale dealer shall make any sale of any commodity in packed form at a price exceeding the retail sale price thereof”} \& Rule 18(5)\footnote{“No wholesale dealer or retail dealer or other person shall obliterate, smudge or alter the retail sale price, indicated by the manufacturer or the packer or the importer, as the case may be, on the package or on the label affixed thereto.}

\heading{Unfair Trade Practice}

\noi
It is important to understand the historical reasons for certain practices being termed unfair and
how the definitions have undergone changes with time in order to understand the intent of the
legislators and its current application.

\noi
For the longest time, there was little to no remedy available to consumers in India. They could
not even approach the courts to seek redressal for any grievances.

\noi
Although The Monopolies and Restrictive Trade Practices Act was intended to protect
consumers, it failed to adequately do so. Due to this, it was amended in 1984 with the
recommendations ofthe Sachar Committee, and a part on unfair trade practices was introduced:

\vspace{-.2cm}

\begin{enumerate}
\itemsep=0pt

\item[1)] Making false or misleading claims about a product or service through an advertisement or otherwise.

\item[2)] Offering bargain prices and bait advertisements.

\item[3)] Offering pseudo prizes or gifts and conducting sales promotion contests, lotteries, or games of chance/skill.

\item[4)] Supplying hazardous or unsafe products.

\item[5)] Hoarding or destroying goods, or refusing to sell goods, resulting in a price increase.\footnote{D.P.S. Verma, Developments in Consumer Protection in India, 25, JOURNAL OF CONSUMER POLICY, 107, p.110 (2002).}
\end{enumerate}

\vspace{-.2cm}

\noi
Earlier, the government had to take \textit{suo moto} cognizance of any unfair trade practice and pursue
the issue themselves. Certain specific enactments did allow the consumer to approach the courts
directly, but only for the particular product mentioned in such enactment. Another amendment
to the MRTP Act in 1986, gave consumers the right to approach the Commission directly,
where previously only a group of 25 members or by a consumer association with 25 members
or more could do so.\footnote{Sec 36, Monopolies and Restrictive Trade Practices Act, 1969.}

\vspace{-.1cm}

\noi
Then 1986 came, and brought with it the Consumer Protection Act, which was hailed as one of
the greatest steps towards consumer protection, being one of the most progressive socioeconomic legislations. In Section 2(1)(r) of the Consumer Protection Act, unfair trade practices
were defined, and consumers were now free to contact the forums and commissions
themselves. Also, after the Consumer Protection Act came into effect, the MRTP Commission
has tried a large number of cases relating to unfair trade practices.\footnote{Verma, \textit{Supra} note 17 at 114.}

\vspace{-.1cm}

\noi
The Consumer Protection Act kick started a long line of positive steps: the Department of
Consumer Affairs was set up in 1991; the exemption with respect to unfair trade practice
enjoyed by public sector undertakings, cooperative societies, and financial institutions was
withdrawn in 1991; and certain provisions in the MRTP Act were strengthened.\footnote{The power to compel parties to appear before it (Monopolies and Restrictive Trade Practices Act, Sec 13B),
to punish for contempt of itself (Monopolies and Restrictive Trade Practices Act Sec 12A), to issue ex parte decisions (Monopolies and Restrictive Trade Practices Act, Section 36 D(1)).} Further, issues of medical practice were also included under the ambit of Consumer Protection Act.
Such is the evolution of unfair trade practice in India.

\heading{Legal Position in India of Unfair Trade\\ Practices}

\noi
The author seeks to provide a clear picture of the current legal position regarding the
malpractices of overpricing and hiking the maximum retail price in these sections by touching
upon the relevant sections of the various related legislations. The definitions and provisions
related to the terms ‘unfair trade practice’ and ‘maximum retail price’ have already been dealt
with in detail in the preceding sections.

\noi
Further, malls often alter the MRP on products by using price printing machines, in order to
lure customers with the pretence of a discount on the MRP.\footnote{M.S. Sreedhar, Modus operandi: How you get cheated, THE HINDU (March 28, 2013), available at\\  \url{http://www.thehindu.com/news/cities/Hyderabad/modus-operandi-how-you-getcheated/article4557695.ece?ref=relatedNews (Last visited on December 1, 2015).}} The goods are also sold at this
inflated price in many instances. The vendors claim to have no choice but to do this as rentals
at malls are very high.

\noi
Prahlad was traveling along the Mumbai-Goa highway in January 2009 and stopped at the
Kamat Hotel to buy a bottle of water. He charged Rs 25/- for it, but later realized that the same
bottle was being sold at other stores on the highway for Rs.15/- without any difference in
quality or quantity.\footnote{ A.J. Lakade, Hotel can’t sell bottled water at higher price: Consumer Forum, THE INDIAN EXPRESS (February 22, 2012), available at \url{http://indianexpress.com/article/cities/pune/hotel-cant-sell-bottled-water-at-higher-priceconsumer-forum/ (Last visited on November 28, 2015).}} Prahlad transferred the Pune-based Dhariwal Industries Limited (the producers of the bottled water in question) and Vitthal Kamat of Kamat Hotels to the Raigad
District Consumer Disputes Redressal Forum against Rasiklal Dhariwal.

\noi
By arguing that the extra fee was for the environment offered by the hotel, they contested the
argument. The Forum denied its appeal, arguing that it was devoid of substance and that there
were no distinguishable qualities that could justify the higher MRP. It is notable that this
decision is only applicable to retail outlets on the counter and not for sit-in customers.

\noi
In another such instance involving hotels, the National Consumer Dispute Redressal
Commission (NCDRC)\footnote{The apex consumer body with respect to C2B dispute redressal in India.} ordered Nyay Mandir, in Gujarat, to pay 1.5 lakh to the consumer
welfare fund and upheld the Gujarat State Commission's decision ordering Nyay Mandir to pay
Rs. 6,000/- as compensation to complainant Ishwar Lal Jinabhai Desai, who had approached the forum for having been charged Rs 18 for the popular carbonated drink 'Miranda', despite
its MRP being marked at only Rs. 12.50/.\footnote{Hotel fined Rs 1.5 lakh for charging Rs 5 above MRP costs, THE ECONOMIC TIMES (Decemeber 27, 2010), available at\\ \url{http://articles.economictimes.indiatimes.com/2010-12-27/news/28392816_1_mrp-consumer-forum-hotel} (Last visited on November 29, 2015).}

\noi
While this wasn’t a five-star hotel, it is astounding that little to no data in terms of news
coverage and consumer reports was available on customers seeking redressal in any manner
against five star hotels which engage in similar practices. The NCDRC also ruled in favour of
Delhi’s DK Chopra when he filed a complaint regarding Snack Bar at Chennai’s Kamaraj
Domestic Terminal.

\noi
They charged him Rs. 300/- for 2 cans of Red Bull, an energy drink, when the price printed on
the can was Rs.75/-. Previously, the Chennai District Forum, and the Tamil Nadu State
Commission, had both dismissed the complaint. The NCDRC also ordered Snack Bar to pay
50 lakhs towards the consumer welfare fund for unjust enrichment.

\noi
The Commission also distinguished between restaurants and hotels which provide a service,
and stalls or shops which do not do so, stating that the former could levy a charge above the
MRP. They observed that the Airports Authority of India seemed to be colluding with retailers
and turning a blind eye to such practices so that they could charge a higher license fee which
would be otherwise unaffordable.\footnote{J. Gai, How not to get exploited by dual pricing, overcharging, BUSINESS STANDARD (June 1, 2014), available at\\ \url{http://www.business-standard.com/article/pf/how-not-to-get-exploited-by-dual-pricing-overcharging114060100725_1.html} (Last visited on December 1, 2015).}

\noi
Vindya Ramachandran was in transit at an airport, and stopped to buy a drink and some snacks.
Much to her surprise, she was asked to pay double to triple of the MRP mentioned on the bottle
of water (Rs. 10/-), and on the bar of chocolate (Rs. 20/-). She also observed that all the items
sold at the store were being uniformly overcharged. The clerk at the counter ignored her query
regarding the matter, so Vindhya proceeded to register a written complaint with the Airport
Duty Manager.\footnote{B. Jairaj, Check before you pay, THE HINDU (May 22, 2006) available at \url{http://www.thehindu.com/todayspaper/tp-features/tp-metroplus/check-before-you-pay/article3193002.ece} (Last visited on December 1, 2015).}

\noi
The section in which the researcher have presented the findings of the empirical study and
interviews conducted has also dealt with a few case laws in the course of comparison to draw
upon the experiences of the complainants – namely, \textit{Hotel Nyay Manndir v. Ishwar Desai,}\footnote{Hotel Nyay Mandir v. Ishwar Lal Jinabhai Desai, Revision Petition No. 550 of 2006, National Consumer Disputes Redressal Commission, New Delhi.} \textit{Kamat Hotels v. Prahlad Padalikar,}\footnote{Kamat Hotels \& Anr v. Prahlad Padalikar, First Appeal No. a/09/890 Before the Hon’ble State Consumer Disputes Redressal Commission, Maharashtra.} \textit{Adithya Banavar \& Ors v. Pepsi Co \& Ors,}\footnote{Adithya Banavar \& Ors v. Pepsi Co \& Ors, Complaint No. 2010 Before the Hon’ble Urban District Consumer Disputes Redressal Forum, Cauvery Bhavan, Bangalore.} and \textit{D.K. Chopra v. Snack Bar}\footnote{D.K. Chopra v. Snack Bar, Revision Petition No. 4090 of 2012, National Consumer Disputes Redressal Commission, New Delhi.} among others. A prelude to the Legal Metrology Act (Packaged Commodity Rules) 2011 has also been provided through the interviews with the Assistant Controller as well as with many aggrieved consumers and precedents.

\noi
This section will deal with the Packaged Commodities Rules with reference to the definition
of the term ‘institutional consumers’ in greater detail. Further, there will deeper analysis and
comparison of the relevant sections of the Consumer Protection Act. Three specific case laws
will be dealt with in depth. Through the case law, it will also be attempted to provide a
simplistic understanding of how a consumer use the redressal mechanism in force to prevent
future instances of exploitation.

\noi
One of the famous illegal practices used by shopping malls is to put stickers over the original
label with different MRPs. The shop owner does not place the new price tag over the original
one in the event of offering a discount.

\noi
Under the provisions of the Packed Commodity Rules dealt with earlier, this widespread
practice of affixing new MRP stickers is punishable. The method of registering complaints via
telephone and electronic mail has also recently been implemented by the Legal Metrology
Department. In addition, toll-free numbers for customers to come forward and lodge
complaints have also been provided.\footnote{Now, stopping that MRP misuse is just an SMS away, THE HINDU (March 22, 2013), available at  \url{http://www.thehindu.com/news/cities/Hyderabad/now-stopping-that-mrp-misuse-is-just-an-smsaway/article4539027.ece} (Last visited on December 1, 2015).}

\noi
The National Consumer Disputes Redressal Commission, New Delhi, decided what would go
on to be a beacon of light for consumers everywhere. In, Hotel \textit{Nyay Mandir v Ishwar Lal
Jinabhai Desai}\footnote{Hotel Nyay Mandir v. Ishwar Lal Jinabhai Desai, Revision Petition No. 550 of 2006, National Consumer Disputes Redressal Commission, New Delhi.} the complainant was travelling on the Mumbai Goa highway and stopped for refreshment at Hotel Nyay Mandir.\footnote{Hotel Nyay Mandir v. Ishwar Lal Jinabhai Desai, Revision Petition No. 550 of 2006, National Consumer Disputes Redressal Commission, New Delhi.}

\noi
He was charged in excess of the MRP printed for the aerated drink ‘Mirinda’, i.e. he had to pay
Rs.18/- per bottle instead of Rs.12.50/-. Mr Ishwar filed a complaint with the District Forum,
on the grounds of deficiency of service, and prayed for a refund of the excess money among
other things. The District Forum found in favour of the complainant; the decision was appealed
to the State Commission who upheld the decision of the District Forum.\footnote{Hotel Nyay Mandir v. Ishwar Lal Jinabhai Desai, Revision Petition No. 550 of 2006, National Consumer Disputes Redressal Commission, New Delhi.}

\vspace{-.15cm}

\noi
The NCDRC too upheld the same decision and approved of District Forum’s decision. The
petitioner argued that the complaint could not have been filed without the written permission
of the District Forum as it was being filed on behalf of numerous consumers with the same
interests, as under Rule 8, Order 1 of the Civil Procedure Code of 1908.\footnote{Hotel Nyay Mandir v. Ishwar Lal Jinabhai Desai, Revision Petition No. 550 of 2006, National Consumer Disputes Redressal Commission, New Delhi.}

\vspace{-.15cm}

\noi
Both the State and National Commissions held that this proviso was to be applied only in case
of identifiable complainants who could have their claims \textit{“canvassed against one another”} and
not in cases such as this, where the consumer has filed a grievance due to an unfair trade
practice not only against him but many unidentified consumers.\footnote{Hotel Nyay Mandir v. Ishwar Lal Jinabhai Desai, Revision Petition No. 550 of 2006, National Consumer Disputes Redressal Commission, New Delhi.}

\vspace{-.15cm}

\noi
In \textit{Rupasi Multiplex v. Mautusi Chaudhuri \& Ors},\footnote{REVISION PETITION NO. 3972 OF 2014 (Against the Order dated 19/09/2014 in Appeal No. 16/2014 of the State Commission Tripura)} the respondents bought tickets to watch a movie in the petitioner's cinema hall. They paid Rs 330 for the same sum. For safety reasons, they were not permitted to carry drinking water inside the cinema hall. This was despite the ticket containing no express prohibition that prevented the same from happening. It was also claimed that the water facility was available at the entry gate of the hall in the lobby. The complainants addressed the District Forum in question, accusing infrastructure failures and the petitioner's implementation of unfair trade practices.\footnote{ Rupasi Multiplex v. Mautusi Chaudhuri \& Ors, Revision Petition No. 3972 of 2014, National Consumer Disputes Redressal Commission, New Delhi.}

\vspace{-.1cm}

\noi
This complaint was dismissed by the District Forum. The aggrieved plaintiff filed the same appeal before the Commission of the State concerned. The State Commission allowed the appeal and ordered the opposite party to pay the claimant for the deficiency of service a total of Rs 10,000 as compensation along with another Rs 10,000 quantified as the expense of litigation.

\vspace{-.1cm}

\noi
In addition, the payment of compensation had to be rendered within thirty days of the date on
which an interest of 9\% per annum was to be charged. Rupasi Multiplex was also required to deposit an amount of Rs 5,000 in the State Commission Legal Aid Account.\footnote{Rupasi Multiplex v. Mautusi Chaudhuri \& Ors, Revision Petition No. 3972 of 2014, National Consumer Disputes Redressal Commission, New Delhi.}

\vspace{-.1cm}

\noi
Aggrieved with the verdict of the State Commission, Rupasi Multiplex appealed before the
National Consumer Disputes Redressal Commission. The arguments furthered by them were
that, they imposed the restriction on carrying beverages inside the cinema hall to ensure the
safety of the customers. Water was available inside the cinema hall – available for sale in the
cafeteria and for free. There was a disagreement whether the term beverages included drinking
water. The honourable bench referred to several reliable authorities on the English language to
conclude that the term beverages did not include water.\footnote{Rupasi Multiplex v. Mautusi Chaudhuri \& Ors, Revision Petition No. 3972 of 2014, National Consumer Disputes Redressal Commission, New Delhi.}

\vspace{-.1cm}

\noi
The bench also drew the observation that it cannot be reasonably expected for a movie watcher
to remain without water for the entire duration of the movie. This would cause substantial
discomfort. The demographics of the movie watching population includes small children and
the infirm. Water is a basic necessity, and it has to be made available inside the cinema hall if
they prohibit carrying. Non-availability of potable drinking water would be classified as a
deficiency in rendering services.\footnote{Rupasi Multiplex v. Mautusi Chaudhuri \& Ors, Revision Petition No. 3972 of 2014, National Consumer Disputes Redressal Commission, New Delhi.}

\vspace{-.1cm}

\noi
The high cost of the drinking water sold in the cinema halls makes its availability insufficient
since it may beyond the affordability of many movie watchers who would not be able to shell
such a huge amount for water when it is available at a price several times lower in the market
outside the cinema halls. It can reasonably be concluded that the profit from the sale of the
exorbitantly priced water would be going to the pocket of the cinema hall owner.\footnote{Rupasi Multiplex v. Mautusi Chaudhuri \& Ors, Revision Petition No. 3972 of 2014, National Consumer Disputes Redressal Commission, New Delhi.}

\vspace{-.1cm}

\noi
Unfair Trade Practice within the scope of Section 2(1)(r) of the Consumer Protection Act 1986
would constitute a ban on the carrying of drinking water within the hall where free drinking
water is not made available to them, requiring them to buy the same at a considerably higher
price.

\noi
The bench also observed that the unfair and deceptive practices listed in Section 2(1)(r) are not
exhaustive but inclusive and that there may be practices that are unfair or deceptive business
practices other than those expressly listed in them.\footnote{Rupasi Multiplex v. Mautusi Chaudhuri \& Ors, Revision Petition No. 3972 of 2014, National Consumer Disputes Redressal Commission, New Delhi.}

\noi
In \textit{Adithya Banavar \& Ors v. Pepsi Co \& Ors,} the complainants submitted the arguments that
identical products were being sold with carrying MRPs and that such variations might have
been practiced at the manufacturer’s level also. The opposite partyargued that the manufacturer
had marked two different prices on identical products to cover the service charges of the
outlet.\footnote{Adithya Banavar \& Ors v. Pepsi Co \& Ors, Complaint No. 2010 Before the Hon’ble Urban District Consumer Disputes Redressal Forum, Cauvery Bhavan, Bangalore.}

\vspace{-.15cm}

\noi
The complainants submitted that the differential marking of the MRPs is not only an unfair
trade practice under a purposive reading of Section 2(1)(r) of the Consumer Protection Act but
also defeated the very purpose of requiring a manufacturer to mark his products with MRP. It
was a blatant practice of cheating of consumers who are unaware of this differential marking
of the MRPs.\footnote{Adithya Banavar \& Ors v. Pepsi Co \& Ors, Complaint No. 2010 Before the Hon’ble Urban District Consumer Disputes Redressal Forum, Cauvery Bhavan, Bangalore.}

\vspace{-.15cm}

\noi
The lack of warning either on the product or separately fromthe outlet regarding the availability
of an identical product at a much cheaper rate at other retail shops was an unfair trade practice
which affected the whole body of consumers and led to unjust enrichment of the opposite
parties. In addition to this it also caused mental agony to individual consumers due to the
inflated bills. The entire sequence or chain of action starting from the manufacturer to the final
retailer constituted an unfair trade practice.\footnote{Adithya Banavar \& Ors v. Pepsi Co \& Ors, Complaint No. 2010 Before the Hon’ble Urban District Consumer Disputes Redressal Forum, Cauvery Bhavan, Bangalore.}

\vspace{-.15cm}

\noi
The complainants won their consumer forum action against Pepsi Co, the opposite party, for
charging differential MRPs on drinks, with the forum finding the 'unfair' and 'illegal' practices.
In an order dated April 1, 2011, the consumer forum ruled that printing different MRP leads to
Unfair Trade Practice.\footnote{“These printing different rent MRPs for the same material without any change in the material either in the contents or in the quantity is nothing but an unfair trade practice and selling it to the consumers is really unfair trade practice and also deficiency in service. This has to be curtailed. How can the same material have a different M.R.P. at different places? There is no answer. If the retailer wants to sell it for a higher price, it is his business and he has to satisfy the customers that he is selling it at a particular price in case customers wants to take it they may take, if they may not take or they may reject it, but the manufacturer cannot print different prices for the same commodity, it is nothing but an unfair trade practice. As the prints different M.R.P. will allow the retailer to gain more profit for the same material which is impermissible in law. The complainants are the customers. The material purchased at a particular place has a particular M.R.P. the same material must have the same M.RP. at different places also. It cannot have two different M.R.Ps. Hence, printing different M.R.Ps is bad in law, is unfair trade practice.”} Furthermore, complainant Adithya Banavar claimed that Pepsi had been instructed by the Forum to avoid the differential labelling of MRPs on the same quantities of goods. They were further made to award Rs 5000 as damages and Rs 2000 to cover the cost of litigation. On the basis that its position was limited to leasing and collecting rent, the shopping mall was relieved of its obligation. It had no involvement in selling the food or drink in the conduct of business and did not engage in fixing or charging the taxes impugned.\footnote{\textit{Ibid.}}

\vspace{-.15cm}

\noi
The case of \textit{Federation of Hotels \& Restaurants Association of India \& Ors v. Union of India \& Ors}\footnote{1988 AIR 1291 (High Court of Delhi).} specifically deals with subject matter of maximum retail price with reference to five star hotels and their right or lack there of to charge over and above the maximum retail price.\footnote{Federation of Hotels \& Restaurants Association of India \& Ors v. Union of India \& Ors, 1988 AIR 1291 (High Court of Delhi).} In this case Justice Vikramjit Sen held \& observed that\footnote{“I hold that charging prices for mineral water in excess of MRP printed on the packaging, during the service of customers in hotels and restaurants does not violate any of the provisions of the SWM Act as this does not constitute a sale or transfer of these commodities by the hotelier or Restaurateur to its customers. The customer does not enter a hotel or a restaurant to make a simple purchase of these commodities. It may well be that a client would order nothing beyond a bottle of water or a beverage, but his direct purpose in doing so would clearly travel to enjoying the ambience available therein and incidentally to the ordering of any article for consumption.”}

\vspace{-.15cm}

\noi
The provisions of the Consumer Protection Act, in particular Section 2(1)(d), do not extend to
an individual who goes to a hotel or restaurant and orders and consumes those items while he
or she is there. The act of providing such goodsshould not be regarded as a transfer of property,
but is in reality an activity to facilitate the person's provision of service.\footnote{Federation of Hotels \& Restaurants Association of India \& Ors v. Union of India \& Ors, 1988 AIR 1291 (High Court of Delhi).}

\vspace{-.15cm}

\noi
The term ‘retail package’ under the Legal Metrology (Packaged Commodities) Rules 2011
prior to the 2013 amendment included only those packages intended for retail sale to the
ultimate consumer for the purpose of consumption. Rule 3 of the Rules specifically excluded
the applicability of Chapter II to packaged commodities to be sold to industrial or institutional
consumers.\footnote{Rule 3, Legal Metrology (Packaged Commodities) Rules, 2011.}

\vspace{-.15cm}

\noi
Institutional consumers refer to \textit{“transportation, airways, railways, hotels, hospitals or any
other service institutions who buy packaged commodities directly from the manufacturer for
use by that institution”.}\footnote{Rule 3, Legal Metrology (Packaged Commodities) Rules, 2011.} Due to this exclusion, the institutional consumers were exempt from the requirement of the declaration of a retail sale price on the principal display panel. They were not required to comply with the mandated rules of Chapter II. However, after the amendment they have been included in the ambit of the term ‘retail package’ and are required to make the necessary declarations as per Rule 6 of Chapter II.\footnote{Rule 6, Legal Metrology (Packaged Commodities) Rules 2011}

\vspace{-.15cm}

\noi
It is also relevant to look into some of the recommendations of the Sachar committee regarding
unfair trade practices. Although, there is no exception to Section 2(1)(r) of the Consumer
Protection Act, the Committee felt a few should be adopted to ensure a more balanced approach
which would prevent the manufacturers from feeling that the Act was skewed in the favour of
the consumers.\footnote{High Powered Expert (Sachar) Committee on Companies Act and MRTP Act, 252 (1978).} This would only increase their willingness to comply with the requirements
of the Act since it would create a possibility for them to defend themselves on reasonable
grounds.

\vspace{-.15cm}

\noi
The first exception is where the retailer or producer took appropriate measures to procure a
quantity of the product that would have been reasonable according to the purpose of the
advertising but was unable to do so because of circumstances outside its control that could not
have been reasonably expected.\footnote{High Powered Expert (Sachar) Committee on Companies Act and MRTP Act, 252 (1978).}

\noi
The second exception is similar to the first exception in all respects except that it refers to the
quantity of the product and the lead time of acquiring the same. He or she cannot be held liable
in case they are unable to meet the demand of the product because demand thereof surpassed
reasonable expectations.\footnote{High Powered Expert (Sachar) Committee on Companies Act and MRTP Act, 252 (1978).}

\noi
The third exception refers to the specifications of the product including the quality and bargain
price. The retailer should be absolved of the liability if he undertook the efforts expected of a
reasonable man to supply the same product or any equivalent product of the same or better
quality.\footnote{High Powered Expert (Sachar) Committee on Companies Act and MRTP Act, 252 (1978).}

\noi
While these recommendations were made in the context of the responsibility and obligations of a retailer or manufacturer in case ofthe advertisements launched by them and their fulfilment of the promises made to the consumers and their expectations thereof, it has relevance with respect to the unfair trade practices related to pricing. It shows the importance of not transposing the entire burden onto the opposite party which could have the effect of making them seem liable even before the complaint is examined.

\heading{Comparison of the US and UK Consumer\\ Protection Regimes}

\noi
The United Kingdom has central legislations governing consumer protection against unfair
trade practices. It has a uniform regime of regulation and enforcement. The USA has general
consumer protection legislation, but each state has its own laws to combat unfair trade practices.
In some US states, there is no private cause of action and individuals cannot file suitsin courts.
This was the case in the UK too where only enforcement authorities could take up cases until
the Consumer Protection (Amendment) Regulations 2014 were passed which allowed direct
action.\footnote{\url{http://www.freshfields.com/uploadedFiles/SiteWide/Knowledge/00805_BS_MBD_COM_Consumer_Law_V1.pdf.}} The meaning of ‘unfair’, as expressed in both legal systems is essentially the same. In
the UK it is that which is contrary to professional diligence and honestbusiness practice while
in the USA it is differently worded as that which is against normal business behaviour or is
unreasonable. However, the standard to hold a trader liable is very different in both cases. In
the US, intention to commit a breach is necessary to be shown. In the UK, Breaches of the
general prohibition will require proof that the trader acted knowinglyor recklessly. Other
breaches do not require any proof of a specific state of mind.\footnote{\url{http://www.out-law.com/page-9002}} In fact, forthe rest of the offences, strict liability exists.\footnote{Office of Fair Trading, Guidance on the Consumer Protection from Unfair Trading Regulations, 2008, (2008), available at\\ \url{https://www.gov.uk/government/uploads/system/uploads/attachment_data/file/284442/oft1008.pdf}} The intention is not so important in UK law as the fact that an act had the effect of breaching CPRs. The scope of unfair trade practice is verynarrow in the USA while in the UK it is, on purpose, as broad as can be, to deal with all possiblepresent situations and those arising in the future. UK authorities are permitted to frame future regulations but not their US counterparts. Part 3 of the UK Consumer Rights Act contains special provisions for digital content along with goods and services but the US general consumer legislation has no such provisions. Both countries have statutory enforcement authorities like the Federal Trade Commission in the US and the erstwhile Office of Fair Trading in the UK which have powers to regulate, investigate and prosecute in cases of unfair trade practices. In most US states, criminal liability is absent in consumer law cases but in the UK, even individuals in managerial positions are criminally liable. If pronounced guilty in summary proceedings, they can be fined or imprisoned for a maximum period of two years. American law does not provide any remedy for unique, new cases that may arise in the future.

\noi
Any new offence must come under the already established heads of unfair trade practices. UK
law, being, by its very nature so broad, allows for new cases of unfair trade practice to arise in
addition to the 31 already blacklisted practices. In the USA, a mandatory arbitration clause
imposed by companies in standard term contracts would be absolutely binding on consumers.
On the other hand, the British legislation encourages arbitration as it is faster and involves
lesser costs but does not make it binding on the consumer. In fact, the 2015 Act, extends the
scope of arbitration to all sectors now. With regard to unfair terms in standard form contracts,
insurance companies in the US are fully exempted from all liabilities if there is such a clause
excluding liability in the contract. In the UK, generally firms cannot evade liability by imposing
unfair terms in small print as the consumer may not have the time to read them carefully, but
even here, insurance firms are an exception.

\noi
In conclusion, UK consumer law for protection from unfair trading practices is far more
consumer-friendly than the one in the USA. Their scope is wider, there are far less loopholes
and exceptions, stricter provisions for enforcement and far more easily available remedies. The
British OFT is believed to have saved consumers about 100 million pounds by combating
unfair trade practices.\footnote{Office of Fair Trading: Protecting the Consumer from Unfair Trading Practices  \url{https://www.nao.org.uk/report/office-of-fair-trading-protecting-the-consumer-from-unfair-trading-practices/.}}

\vspace{.1cm}

\heading{Suggestions for the Indian Consumer\\ Protection Bill, 2015}

\vspace{.1cm}

\noi
There are several positives in the UK law which the draftsmen of the current bill would do well
to incorporate. The 2015 Consumer Rights Act in the UK contains, in addition to goods and
services, a special chapter on digital content. There are provisions for its replacement and repair
too. The 2015 bill also contains similar sections on e-commerce and electronic intermediaries.
Thus, both laws have taken into account changes in business to consumer contracts brought
about by electronic transactions. The UK legislation has a much wider ambit for what
constitutes unfair trade practice while the Indian legislation is more specific and contains
particular definitions of different kinds of unfair trade practices. The problem with this is that
it does not take into account the possibility of future developments which may not fit into any
of the existing categories. It must also lay more emphasis on unfair terms which may be foisted
on consumers who could not make an informed decision as to their implications. The CPRs in
the UK contain clear definitions of misleading acts, omissions and aggressive commercial
practices which could also be made part of the upcoming Indian legislation. Specific remedies 
like the right to reject, discount, refund or unwind could be added. It must also be more flexible
with respect to the standard of care which can vary as per the establishment and also according
to the average consumer of a particular grouping. Unlike India, UK consumer law enforcement
authorities like the Trading Standards Services are far more proactive while Consumer Courts
in India suffer from a backlog of cases. For this reason, perhaps, a new Consumer Protection
Authority is being established for better enforcement. Any new enforcement authority must
have wide-ranging investigative powers but to grant them the same powers as their UK
counterparts is a point to be debated. As arbitration has been made the preferred form of
Alternate Dispute Resolution in the UK, mediation has been incorporated in chapter 5 of the
Indian Bill. Other forms of ADR can also be explored to ensure speedy and inexpensive
redressal of grievances and clear the backlog of cases.

\heading{Conclusion}

\noi
The paper introduces the broad legislation for consumer protection of consumers in the India,
United Kingdom and the United States of America particularly against the unfair trade
practices. It also brings to attention the very fact that in the United Kingdom, by reason of the
chaos spread by the World War II, there was a dire need for consumer protection, in the light
of which Monopolies and Restrictive Practices Enquiry and Controls Act was adopted in 1948.
Due to the advancement in technology a need arose to address the same and then the
Government introduced the Consumer Rights Act, 2015 which inculcated a very wide
definition of unfair trade practice and novel remedies in the interest of the consumers. The
ambit of the powers given to the enforcement authorities in the United Kingdom are quite large
and can be construed to be consumer friendly as against the position in United States.

\noi
It has been concluded through the study that the position of the legislation in the United States
in stark difference to the position in the United States. In order to prevent the interest of the
consumers against malpractices that take place in the market every state in the United States
has a separate statute namely the Unfair and Deceptive Acts and Practices to address the same
which in itself have been adopted to suit their own needs and naturally there are differences in
the different legislations. The paper also brings to attention the very fact that in the consumer
law in certain industries in the United States works in favor of the producers as against the
consumers and in the light of this certain suggestions have been provided which if adopted
could render the consumers in a better position.

\noi
In addition, the paper concentrated on the major disparities that are present between the two major forces, the United States of America, and the United Kingdom, in consumer protection laws. The new consumer protection law in India for consumer protection against unfair and restrictive trade practices has also been debated and suggestions have been made.
\end{multicols}
