\setcounter{figure}{0}
\setcounter{table}{0}
\setcounter{footnote}{0}

\articletitle{Gender based Discrimination and Human Rights: A Gift of Goddess Lakshmi}\label{2017-art4}
\articleauthor{Shubha Vats and Sejal Talgotra\footnote{Assistant Professors, The Law School, University of Jammu, Jammu, India. (Corresponding Author) shubhapadha@gmail.com Research Scholar, Department of English, University of Jammu, Jammu, India. Most of the views so expressed in this article by the authors are the sole creation and thought process of assimilations of the concerned revelations and facts and therefore such ideas and thoughtprovoking analysis of the title of this article could be considered as genuine contribution by the authors. The authors are indeed indebted to the discussions and deliberations they had with the concerned stakeholders while writing this article as their inputs really made the allegorical significance of this article very decisive and worth understanding. The authors express their gratitude to them.}}
\lhead[\textit{\textsf{Shubha Vats and Sejal Talgotra}}]{}
\rhead[]{\textit{\textsf{Gender based Discrimination...}}}

\begin{multicols}{2}

\heading{Introduction:}

\noi
Every human being should be treated equally and given the choice of gender. Gender will never
be a determining factor in the pursuit of fundamental rights. As a result, no gender disparities
can exist. A dignified life, in addition, necessitates education for proper personality growth. A
Gift of Goddess Lakshmi, a biography of India's first transgender school principal, was cowritten by Manobi Bandyopadhyay and Jhimli Mukherjee Pandey. Manobi Bandyopadhyay's
quest for self-identity via education is depicted in this biography. The tension between
Manobi's gender identity and her biography is the central theme. It's her fight and struggle that
we're talking about over here. Humans have some natural and natural rights as a result of their
being human. These rights are referred to as human rights. These rights are theirs simply by
virtue oftheir being, and they become effective with their birth, regardless of caste, creed, faith,
sex, or nationality. These rights are consistent with human dignity and equality and promote
physical, moral, social, and spiritual well-being. By establishing favourable circumstances,
they assist citizens in advancing materially and morally. Fundamental rights, inherent rights,
natural rights, and birth rights are all terms used to describe human rights.

\noi
The term "Loka Samastha Sukhino Bhawanthu," which translates to "Loka Samastha Sukhino
Bhawanthu," Sama is a character in the filmSama Sam is a man who enjoys doing stuff. Human
rights are the greatest cultural and civilizational gift of classical and contemporary human
thought. Every society is still fighting for the security, promotion, and preservation of human
rights. The womb of the old gives birth to new privileges.\footnote{Sharma Shobharam, “Transgender in India: Human Rights and Social Exclusion” All India Reporter 86 (2013).}

\noi
The first clause of the Universal Declaration of Human Rights says that all human beings are
born free and should be treated equal as far as dignity and individuality is concerned. They are
gifted with reason and conscience, and must behave in fraternal manner with each other.
Everyone has the right to all of the Declaration's rights and freedoms, irrespective of their race,
colour, gender, religion, political beliefs, nation, society, property, birth and class. No
distinction will be made based on a person's political influence or the country's international
status. Thirty Articles of the Declaration are divided not only into civil and political rights, but
also talks about economic, social, and cultural rights of mankind.

\vspace{-.15cm}

\heading{Developmental Phase}

\vspace{-.15cm}

\noi
Many of the rights mentioned above are available to humans. As a result, transgender people,
as well as everyone who may not fit into the binary gender mould are also considered as
humans. Almost all international documents claim that everyone is born with an intrinsic right
to life and that law shall be protecting this right; and that no one's right to life will be violated
arbitrarily. Transgender people and other sexual minorities will also benefit from these rights.
Transgender people, on the other hand, have been persecuted, mocked, and despised all over
the world. Even if they've had a bad life, transgender people are entitled to all of the protections
outlined in international treaties. The United Nations has been a leader in the fight to protect
and advance the rights of sexual minorities, including transgender people.\footnote{Ghoshal \& Somak. “The Brave but Heart-breaking Journey of India’s First Transgender College Principal.” Huffpost. (2016).}

\noi
The Office of the High Commissioner for Human Rights released its first report on LGBT
people's human rights in December 2011. This study examines various types of discrimination
and violence against LGBT people around the world, including discrimination in the
workplace, health care, education, imprisonment, and torture. The Human Rights Council made
two significant decisions in response to the publication of this report. A resolution calling for
the decriminalization of homosexuality was signed by 85 countries in March 2011. Second,
South Africa passed a resolution in favor of gay rights in June 2011. In the battle against
homophobia and transphobia, the study emphasizes the value of shared community
responsibility, and it calls on countries to defend citizens of all sexual orientations and gender
identities.\footnote{Web source: (PDF) Transgender Rights as Human Rights (researchgate.net)( Last accessed on 24 Dec 2017).}

\noi
Discernment on the basis of sexual orientation and gender identity is a violation of the universal
principle of equal dignity and rights for all people. Discrimination of this kind is prohibited by
international law, both explicitly and implicitly. Almost every country recognizes man and
woman as the two traditional gender identities and socialroles, while all other gender identities
and expressions are ignored. Some nations, however, have laws that recognize a third gender.
As a result of a deep and broader understanding of the breadth of concepts outside of the
conventional definitions of man and woman, many self-descriptions such as pander,
polygender, genderqueer, and nonbinary are now becoming part of literature. Sex
reassignments are now recognized in many countries, which encourage people to change their
legal gender on their birth certificate.

\heading{Different Perceptions}

\noi
All humans should be treated as equals and given the option regarding gender. In the pursuit
of fundamental rights, gender can never be a deciding factor. As a consequence, there should
be no gender inequality. Furthermore, for proper personality growth, a dignified life
necessitates education. For children aged 6 to 14, the Indian Constitution recognizes the right to free and compulsory education.\footnote{Chakrapani Venkatesan, “Hijras/Transgender Women in India: HIV, Human Rights and Social Exclusion” Tg Issue Brief 8 (2010)} When education is withheld, discrimination exists. All should learn it because it gives people a sense of dignity and self-identity. Only a small number of transgender people have access to education, and they are fighting for their rights.

\noi
Manobi Bandyopadhyay and Jhimli Mukherjee Pandey collaborated on A Gift of Goddess
Lakshmi, a biography of India's first transgender principal. Manobi Bandyopadhyay's quest for
identity and individuality through education is depicted this biography. In June 9, 2015, she
became India's first transgender principal at Krishnagar Women's College in West Bengal.

\noi
"Education: If we understand, all of our problems will be solved," she tells her group. The plot
follows a transgender woman who fights the myth that transgender people are strange, repulsive
beings that are probably criminals and definitely worthy of contempt. Despite the factthat she
was born transgender, Manobi sees herself as a woman trapped in a man's body. Despite her
humiliation, Manobi tries to accept herself for who she is. Through the influence of education,
she was able to change the trajectory of her life by gaining self-acceptance and social
acceptance.\footnote{Web source:\\ \url{ijlljs.in/wp-content/uploads/2017/12/ARTICLE_ON_TG_1-1.pdf (Last accessed on 24 Jan 2017).}}

\noi
In June 2015, Manobi Bandyopadhyay hit the headlines when she became the world's first
transgender principal of any educational institution. The government-aided Krishnagar
Women's College in Nadia district, about 100 kilometers north of Kolkata, was taken over by
Bandyopadhyay, who was 50 at that time. The state was appreciated for this progressive
decision to uplift the suppressed trans community. Bandyopadhyay was the first transgender
person to get the degree of doctorate and to be the professor (she transitioned while teaching).

\heading{Formative viewpoints}

\noi
A year before she was appointed principal, the Supreme Court issued a landmark decision
granting third-gender status to transgender people if they so desired and ordering governments
to safeguard the community by giving them the equal opportunities following years of
indifference and discrimination. Bandyopadhyay is now a member of West Bengal's
Transgender Development Board. Bandyopadhyay, the youngest ofthree siblings born in 1964,
rose to national prominence as a result of the news. As Bandyopadhyay's biography to
journalist Jhimli Mukherjee Pandey reveals, he was already a trailblazer and a public figure.
When you are a transgender person, it is impossible to be non-transgender.

\noi
Bandyopadhyay's socially odd status was only augmented by her academic accomplishments,
which she defied gender norms. She refused to join any hijra gharana and continued to live
with her family despite their inability to assist her and was subjected to exploitative
relationships as well as psychological trauma from colleagues.

\noi
Bandyopadhyay's life\footnote{Bandyopadhyay, Manobi \& Jhimli Mukherjee Pandey. A Gift of Goddess Lakshmi: A Candid Biography of India’s First Transgender Principal. India: Penguin Books, (2017)} is a complicated story in several respects. It's a fascinating look at the origins of queer Indians: From the mid-1990s onwards, Bandyopadhyay published Abomanob (sub-human), a magazine that dealt with transgender issues. The fact that transgenderism was already being debated disproves the commonly held idea that millennials are the first generation to question gender and sexuality norms. Through the narrative of Jagadish, Bandyopadhyay's friend, we learn that some segments of Bihar's society are more accepting of homosexuality. The biography's sections on Jagadish emphasize Bandyopadhyay's heteronormative perspective on sex and marriage. Millennials are the first generation to question gender and sexuality norms. Through the narrative of Jagadish, Bandyopadhyay's friend, we learn that some segments of Bihar's society are more accepting of homosexuality. The biography's sections on Jagadish emphasize Bandyopadhyay's heteronormative perspective on sex and marriage.\footnote{Mousumi Padhi \& Purnima Anjali Mohanty, “Securing Transgender Rights through capability development”, Economic and Political Weekly Web source: pdf (epw.in) (Last accessed on 21 Dec 2017).}

\noi
Additionally, the biography includes a list of transgender individuals' accounts of being bullied
and discriminated against. She discusses her battle with the state's higher education department,
which denied her promotion due to a discrepancy between the names on her pre-transition
academic certificates and those on her PhD certificate. Additionally, she lists the number of
times she has been forced to relocate due to conflict with her neighbors or the community.
Bandyopadhyay encountered hostility in liberal spaces such as Jadavpur University (JU), where
she studied in the 1980s. "Just because JU culture was more refined does not mean it was not
divided into two sexes, as it was elsewhere in the world," she writes.

\heading{Conclusion}

\noi
Simultaneously, the assumptions\footnote{Web source: 4115GI.p65 (socialjustice.nic.in)( Last accessed on 21 Jan 2017).} and biases of the narrator must be critically examined. When addressing one of her more serious relationships, she extols the domestic gender position that women must play. "I had already started looking after his general well-being and asked him to stop cooking," she writes, "even though I wasn't married to him yet." This may be just what she wanted, but it still perpetuates a myth that oppresses women all over the world. In December, Bandyopadhyay was back in the news. Due to a lack of support from the faculty and some students, she had resigned from the college. This biography places her point in detail. A Gift of Goddess Lakshmi depicts the extraordinary and courageous struggle of a
transgendered woman to develop her identity and reach new heights. The core theme of
Manobi's biography is her conflict with her gender identity. It is her fight and struggle.\footnote{Ibid.}

\end{multicols}
	
\label{end2017-art4}
