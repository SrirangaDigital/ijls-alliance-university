\setcounter{figure}{0}
\setcounter{table}{0}

\articletitle{Challenges of Criminal Justice system in adjudicating Cases relating to Sexual\\[4pt] Offences}\label{2016-art1}

\vspace{-.5cm}

\articleauthor{Dr. Kiran D. Gardner\footnote{Dean, Parul University, Vadodara Gujrat }}
\lhead[\textit{\textsf{Dr. Kiran D. Gardner}}]{}
\rhead[]{\textit{\textsf{Challenges of Criminal Justice system...}}}

\begin{multicols}{2}

\heading{Introduction}

\noi
The criminal justice system is pivotal in upholding the faith of people in the judiciary.
When the justice administration does not provide reprieve to societies pressing issues,
it causes anguish in the mind of people. A weak criminal justice system has no deterrent
effect on criminals, instead, it encourages criminality. A meagre rate of prosecution and
conviction bolsters the miscreants to undermine the law creating a favorable
environment for delinquents and deviants to thrive and risking the safety and security
of law-abiding citizens with the constant threat to their life and liberty, freedom,
dignity, and property.

\noi
A time has come when we need to shift our attention from the rights of the accused to
the rights of the victim. It has been long since our justice delivery system has
concentrated more on the rights of the accused and somewhere down the line rights of
the victims have been compromised. Gang rape and mutilation of 23 years old medical
student, 'Nirbhaya' in Delhi on 16th December 2012 have shaken the conscience of the
dormant population of India and recently 80 years 0ld lady from-Sonipat Haryana faced
similar sexual brutality bleeding to death.\footnote{Times of India, 6$^{\rm th}$ January 2017, Incident occurred on 4$^{\rm th}$ January 2017.}
It has encouraged the author of this paper to
look into challenges faced by rape victims in getting justice. All credit goes to the media
for igniting nationwide protest, leading to demands of reforms in rape laws. Similarly,
media has taken an active role in sensitizing society. It has been observed that there is
an impact of public pressure on the judiciary. Public pressure through mass media has
ensured justice rather than stricter legislation.

\vspace{.6cm}

\heading{Society's protest}

\noi
In ‘Nirbhaya's Case' people have been demanding that rapists should be given the death
penalty, castration, public stoning, hanging, or immediate justice in any other
appropriate way. They want the rapist should be punished as soon as possible. In this era of instant solutions, people have come out on street seeking immediate remedies.
Magic wand solution was expected from Justice J. S. Verma Committee that is
constituted to suggest reform in Criminal Justice System. Public Interest Litigation is
filed in Supreme Court by former Indian Administrative Services officer Promilla
Shanker ask for directions to the government that would safeguard women and fasttracking of court proceedings and speedy disposal in all rape cases. She has submitted
that cases of rape and crime against being gender-specific must be investigated by a
police officer of female gender and trials be conducted by female judges\footnote{\url{http://www.dnaindia.com/india/report_sc-agrees-to-hear-pil-seeking-speedy-trials-in-rapecases_1784591} visited on 11th January 2013.}. Seminars
and conferences are being held all over to suggest reforms in the criminal justice
administration system so that such gruesome sexual offences are avoided in the future.
Before suggesting any reforms in rape laws, it would be appropriate to understand
existing laws in India concerning rape. 

\heading{Rape laws in India}

\noi
Indian Penal Code was enacted in 1860. For more than 120 years rape laws had
remained unchanged. It was in 1983 when rape laws were amended for the first time.
Sections 375 and 376 of the Indian Penal Code deal with rape laws. These laws were
amended for the first time after the controversial {\bf Mathura rape case}\footnote{{\it Tukaram v State of Maharashtra} [1979 AIR 185 SC]}, which took place
in 1972.

\newpage

\noi
A tribal girl of 16 years named Mathura was raped in the custody of the police by two
police personnel in the premises of the police station. Session Court acquitted the
accused relying on the argument that the victim was in habit of sexual intercourse. High
Court convicted the policemen and held that passive submission under coercion does
not amount to consent. However, Supreme Court set aside High Court judgment on
grounds that Mathura had no injury marks on her body and did not raise alarm for her
protection.

\noi
The Judgment had triggered a nationwide campaign for changes in rape laws. It brought
forth the major drawbacks of the Criminal justice system, the major being that it was
required for a rape victim to prove that she did not consent for sexual approval. Prof. Upendra Baxi from Delhi University along with his colleagues wrote an open letter to
the Chief Justice of India criticizing the judgment.

\vspace{-.1cm}

\noi
Amendments to rape laws were enacted in 1983 after 11 years from the Mathura rape
case incident. According to Sections 376 B, 376C, 376D of IPC onus of proof regarding
the consent of women was shifted to the accused in some specific instances as in
custodial rape cases i.e. rape by policemen, public servants, managers of public
hospitals, and remand homes and wardens of the jail. Insertion of section 228 A of IPC
disclosed the identity of the victim in a rape case was made punishable. Similarly, as
per the amended section 114A of The Indian Evidence, if the rape victim says that she
did not consent to the sexual intercourse, the court shall presume that she did not
consent even if the woman is of easy virtue.

\vspace{-.1cm}

\noi
In, {\bf State of Maharashtra v Madhukar Narayan Mardikar}\footnote{AIR 1991 SC 207}, the Supreme Court
held:

\vspace{-.1cm}

\noi
\textit{“Even a woman of easy virtue is entitled to privacy, and no one can invade her privacy
as and when he likes. So also, it is not open to any and every person to violate her
person as and when he wishes. She is entitled to protect her person if there is an attempt
to violate it against her wish. She is equally entitled to the protection of the law.
Therefore, merely because she is a woman of easy virtue, her evidence cannot be
thrown overboard”.}

\noi
Section 155(4) of the Indian Evidence Act has been omitted by Act 4 of 2003. Earlier
this section permitted the victim to be questioned about her past sexual history, which
the defense used to humiliate the victim in the courtroom.

\noi
The campaign resulted in changes in the rape laws, the most significant being the
mandatory minimum punishment of seven years which could extend to life
imprisonment. It was expected that it would have a deterrent effect. But the rape cases
reported since 1983 have increased and hence the amendment has failed to deliver the
expected outcome. Moreover, the conviction rate is negligible. Even in the cases that
have led to conviction trial courts have seldom awarded minimum mandatory seven
years and given as low as 2 years and in some cases lesser.

\noi
Many laws have been enacted for the benefit of women victims of sexual offences.
These laws are illusory. Stringent penal provisions are provided to ensure the safety
and security of women. Despite laudable legislation, there is no deterrent effect of these
laws.

\noi
The following table will give insight into the status of rape cases in Bombay

\vspace{-.3cm}

\subsection*{Disposal of Rape in Bombay 1885 to 1989}
\begin{tabular}{|l|l|l|l|l|l|}
\hline
Description & 1985 & 1986 & 1987 & 1988 & 1989\\
\hline
\end{tabular}

\begin{tabular}{|l|l|l|l|l|l|}
\hline
Registered & 101 & 102 & 85 & 108 & 108\\
\hline
Charge & 93 & 96 & 76 & 104 & 100\\
Sheeted &   &    &    &     &    \\
\hline
Convicted & 8 & 1 & 2 & 1 &1 \\
\hline
Acqitted & 4 & 3 & 1 &-& 2\\
\hline
Pending & 81 & 91 & 72 & 102 & 95\\
Trial   &    &    &    &     &    \\
\hline
\end{tabular}

\noi
Source: The Lawyers, April 1991.

\noi
The table points out the minuscule cases that have led to the conviction and a plethora
of pending cases. Usually, it takes five to ten years for a rape case to be decided. As our
justice system is more inclined to the rights of the accused the delay also benefits the
accused, Sentence of the accused is reduced on the ground that many years have passed.
Thus, justice is delayed and denied.

\heading{Why are rape laws ineffective?}

\noi
We do have laws in place yet other factors influence the decision of the court. Some of
them are discussed hereunder:

\heading{1. Procedural Flaws:}

\noi
Sexual offences fall within the ambit of criminal law. The state is prosecuting the party
in a criminal trial. The accused is a defendant and has the liberty to hire an advocate of
his choice. The victim who suffers embarrassment, humiliation, stigma, mental agony,
emotional disturbance, economic constraint, and at times physical injuries, ends up
being just a witness in the court.

\noi
(The victim should be provided with the freedom to choose the prosecutor from the
panel and should be able to monitor the conduct of the case.)

\noi
The outcome of the Criminal justice system depends a lot on the kind of evidence
produced in court. The style of working of functionaries of the criminal justice
administration has not kept pace with the changing society and thus has not been able
to win the confidence of the masses. People's indifference undermines the effectiveness
of the criminal justice system. Though lots are said about rights, the time has come
when the duty of a person towards society needs to be taken seriously. Any person who
fails to render assistance according to his ability in the prevention of crime should be
appropriately punished. Another issue that needs to be addressed is witnesses'

\noi
protection and compensation of victims. Witnesses are intimidated, enticed, influenced
resulting in witnesses turning hostile.

\noi
The confidence of people in the justice delivery system is reducing. The faith of people
will naturally erode due to Advocates seeking frequent adjournments, using delay tact
tics, find ways and means to get the accused acquitted. Their focus is entirely on making
money rather than justice delivery. Justice V. R. Krishna Iyer, former judge, Supreme
Court of India, in his book, “Access to Justice", has beautifully summed up the state of
justice delivery system as,

\noi
\textit{“…If all the judges and the lawyers of India pull down the shutters of their law shops
nationwide, injustice may not anymore escalate. Indeed, around 80 to 90 percent of
crimes investigated end in discharge or acquittal. So much so, the law is dead, and if
the Bench and Bar go on a long holiday, litigative waste of human and material
resources may be obviated.”}

\heading{2. Judiciary:}

\noi
It has been observed that there has been no uniformity in court decisions. The outcome
of the rape trial cannot be predicted with certainty. The attitude of an individual judge
has influenced the outcome of judicial decisions.

\vspace{-.4cm}

\subsection*{i) Minimum mandatory punishment is not awarded}

\vspace{-.2cm}

\noi
Amendments in 1983, made it mandatory for the sentencing of minimum punishment
of 7 years/ 10 years for rape depending on the gravity of the offence. A study of the
subsequent judgments reveals that routinely less than a minimum mandatory sentence
is awarded. The provision in S 376 states: \textit{“Provided that the court may, for adequate
and special reasons to be mentioned in the judgment, impose a sentence of
imprisonment for a term less than 7 This provision years” ......} Further, it mentions 10
years where the minimum punishment is 10 years is been frequently used for awarding
less punishment.

\noi
In \textbf{Suman Rani's Case}, the Supreme Court in custodial rape case reduced the sentence
from a minimum of 10 years to 5 years. The review petition filed against the reduction
of sentence w rejected.	

\noi
In Bhansingh v State of Haryana\footnote{1984 Cri. LJ. 786}, a 7-year-old girl was raped by 18 years old boy.
She was severely injured and left in an unconscious state. An appeal to enhance the
sentence was dismissed by High Court on the ground that the accused was only 18 years
of age, and it would not be in the interest of justice to enhance the sentence of five years
imposed by the trial court.

\noi
Thus, there is no deterrent effect of stringent punishment for sexual offences unless the
mindset of the implementers and more specifically judiciary are sensitized towards such
heinous offences.


\vspace{-.3cm}

\subsection*{ii) Corruption in Judiciary}

\noi
Corruption in the judiciary affects the outcome of judicial decisions. Mr. K. R. Narayan,
former President of India, addressing the Golden Jubilee function of Supreme Court in
January 2000, stressed the need for an accountable judiciary in the country to dispense \textit{“quick, affordable and incorruptible justice”} to the people to sustain their faith in the
courts. The President hit out against falling standards in the judiciary. Quoting a famous
English saying, he said, \textit{“Courts are no longer cathedrals. They have become casinos
where throw of dice matters.”}

\vspace{-.3cm}

\subsection*{iii) Attitude of the Judges}

\noi
Views of individual Judges have influenced the decision in the cases of sexual offences.
Many times, the judiciary has considered rape from the perspective of men and
considered it as an offence of man's uncontrollable lust rather than violation against
women’s body and a forceful act without consent, an act of sexual violence against
women.

\noi
The judgment in \textbf{Phul Singh v State of Haryana}\footnote{AIR 1980 SC 249} describes the offence as, \textit{“A
philanderer of 22, appellant Phul Singh, overpowered by sex stress in excess, hoisted
himself into his cousin's house next door, and in broad day-light, overpowered the
temptingly lonely prosecutrix of twenty-four, Pushpa, raped her in hurried heat and
made an urgent exist having fulfilled his erotic sortie.”}

\noi
Sentence of 4 years rigorous imprisonment was reduced to 2 years on the ground that,
\textit{“The appellant is a youth barely 22 with no criminal antecedents save this offence. He
has a wife and a farm to look after. Given correctional courses through meditational
therapy other measures, his erotic aberration may wither away.”}

\noi
In \textbf{Mohammed Habib v State}\footnote{1984 Cri LJ 137}, a 9-year-old girl was raped by 21 years old boy in a
pit near a bus stop. On the evidence of eyewitnesses and the statement of prosecutrix
the accused was convicted and awarded life imprisonment. The Delhi High Court set
aside the conviction of the session court on the ground that there was injury to the
accused only on the body and not on the penis. The court ruled that in the rape of a
minor by a fully developed male, injury to the penis is essential."

\noi
Somewhere the court has overlooked the provision that mere penetration is enough to
constitute the offence and secondly the victim was a girl of merely 9 years, while the
accused was 21 years robust man.

\noi
In \textbf{Prem Narayan v State of Madhya Pradesh}\footnote{1989 Cri LJ 707}, The accused dragged a 9-year-old
girl near the bushes and tried to penetrate. The girl was severely injured. Due to pain,
the girl did not allow the doctors to carry out the internal examination. Hence the exact
vaginal tear could not be determined. Giving the benefit of doubt to the accused, the
trial court convicted the accused only of an attempt to commit rape. On appeal, the High
court commented that the accused had erroneously escaped punishment for rape but
held that since the State had not appealed against it, it was not proper to look into this
question.

\noi
These are a few randomly picked cases where it is very evident that the judiciary has
been lenient towards offenders of sexual offences. A radical change in the attitude of
defence counsel and judges to sexual assault is required.

\heading{3. Role of investigating agency:}

\noi
The success or failure of the criminal justice delivery system entirely depends on the
quality of the investigation. The case will stand or fall depending on how diligently
investigation is conducted and with what precision the evidence is collected and
produced in the court. Efficiency and effectiveness of investigation and charge sheeting
the accused with exactitude will ensure the prevalence of justice. Deterioration in its
function will not only endanger the life and liberty of the people but will pose threat to
the very survival of the Constitution.

\heading{Mr.~M.~N.~Singh Commissioner of police,\\ Mumbai, opines:}

\vspace{-.1cm}

\noi
“It is an accepted fact that conviction rates have declined over the years. There are
several reasons behind poor conviction rates. Firstly, the quality of police investigations
has gone down. Secondly, our cumbersome legal system causes delays in judicial
proceedings. Thirdly, Public prosecutors are not part of the police department and we
do not have control over them. As a result, it becomes difficult to get convictions in
criminal cases."

\vspace{-.1cm}

\noi
The latest figures show that all States and Union Territories had collectively reported
police vacancy to the tune of nearly 4.20 lakhs during 2011. The actual strength of the
police force in the country as of December 2011, was 16.60 lakhs as against the sanctioned strength of 20.86 lakhs, with Uttar Pradesh leading the pack in reporting
police vacancies. The other States which report huge vacancies include Andhra
Pradesh, Bihar, Jharkhand, Maharashtra, Haryana, and West Bengal. This shortage is
further compounded by the excessive deployment of police personnel for VIP security.

\vspace{-.1cm}

\noi
There is a need to strengthen institutional mechanisms and create standard operating
procedures to prevent sexual offences. The state should impart training to cops for
handling cases relating to sexual offences: The state should fill up the police vacancies
as soon as possible and proportionately increase the police force to meet the
requirement of the growing population.

\heading{Guidelines for the trial of rape cases}

\vspace{-.1cm}

\noi
Over and above the existing rape laws, Supreme Court in various cases has provided
guidelines for conducting trials in rape cases. In \textbf{Delhi Domestic Working Women
Forum v Union of India}\footnote{ (1995) 1 SCC 14}
, have laid down guidelines for the trial of rape cases. They
are as follows: 

\noi
\textit{1) A provision for legal representation for the complainants of sexual assault
cases. An experienced person with knowledge of the functioning of the justice system.
The role of the victim's advocate would not only be to explain to the victim the nature
of the proceedings, to prepare her for the case, and to assist her in the police station
and court but to provide her with guidance as to how she might obtain the help of a
different nature from other agencies, for example, mind counseling or medical
assistance. It is important to secure continuity of assistance by ensuring that the same
person who looked after the complainant's interests in the police station represents her
till the end of the case.}

\noi
\textit{2) Legal assistance will have to be provided at the police station since the victim
of sexual assault might very well be in a distressed state upon arrival at the police
station, the guidance and support of a lawyer at this stage and whilst she was being
questioned would be of great assistance to her}

\noi
\textit{3) The police should be under a duty to inform the victim of her right to
representation before any questions were asked of her and that the police report should
state that the victim was so informed.}

\noi
\textit{4) A list of advocates willing to act in these cases should be kept at the police
station for victims who did not have a particular lawyer in mind or whose own lawyer
was unavailable.}

\noi
\textit{5) The advocate shall be appointed by the court, upon application by the police at
the earliest convenient moment, but to ensure that victims were questioned without
undue delay, advocates would be authorized to act at the police station before leave of
the court was sought or obtained.}

\noi
\textit{6) In all rape trials, the anonymity of the victim must be maintained, as far as
necessary.}

\newpage
\noi
\textit{7) It is necessary, having regard to the Directive Principles contained under
Article 38(1) of the Constitution of India to set up the Criminal Injuries Compensation
Board. Rape victims frequently incur substantial financial losses. Some, for example,
are too traumatized to continue in employment.}

\noi
\textit{8) Compensation for victims shall be awarded by the court on conviction of the
offender and by the Criminal Injuries Compensation Board whether or not a conviction
has taken place. The Board will take into account pain, suffering and shock as well as
loss of earnings due to pregnancy and the expenses of childbirth if this occurred as a
result of the rape.”}

\noi
The National Commission for women is asked to frame schemes for compensation and
rehabilitation to ensure justice to the victims of such crimes.

\heading{Recommendations}

\noi
If the faith of the mass is to be retained in the criminal justice delivery system there is
a need for qualitative improvement in the functioning of each unit. A safer society
cannot be imagined without significant enhancement in efficiency and speed. Dr. A. S.
Anand has most aptly has stated:

\noi
“One of the greatest challenges that stare us in the face as we approach 21 century is
the failure of the judiciary to deliver justice expeditiously, which has brought about the
sense of frustration among the litigants......."

\noi
The victim of a sexual offence has to prove the case beyond doubt. All the advantages
in the trial have to be granted to the accused. The benefit of delay in the justice delivery
system also goes to the accused. The sentence of the accused is reduced on the ground
that many years have passed. The criminal justice system needs to revamp to be more
victim-friendly than accused-friendly. The inquisitorial method of the criminal justice
system seems more suitable for trying cases of sexual offences.

\noi
With the increase in populations and the number of laws, the criminal justice system is
overloaded with responsibilities other than maintaining law and order. Thus, Criminal
Justice System has not been able to keep crime under control and deliver speedy justice
to people. There is an urgent need to increase the number of courts to ensure no case is
pending for more than 3 months for trial. There is an urgent need to strengthen
investigating agencies and provide rigorous training for efficient handling of such
sensitive cases.

\noi
When the minimum mandatory punishment is prescribed for the offence, there should
be no scope for granting lesser punishment. As such concession removes the fear of
punishment from the minds of criminals and provides a loophole that is used for the
advantage of the accused. If existing laws are implemented in letter and spirit there
would be no need for stricter punishment. Moreover, the stricter the punishment, the
lesser would be the conviction rate. It is not the severity of punishment but the certainty
of punishment that deters the criminals.

\end{multicols}




\label{end2016-art1}
