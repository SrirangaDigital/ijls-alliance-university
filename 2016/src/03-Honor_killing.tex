\setcounter{figure}{0}
\setcounter{table}{0}
\setcounter{footnote}{0}

\articletitle{“Honor Killing”: Whether Honorable or Dishonorable}\label{2016-art3}

\vspace{-.3cm}

\articleauthor{Vineetha. S and Kannan Balakrishnan\footnote{Post Graduate in Criminology and Criminal Justice Administration, University of Madras, Chennai,
India. E-mail: \url{tara.vinita@gmail.com}, \url{balakrishnankannan84@gmail.com}}}
\lhead[\textit{\textsf{Vineetha. S and Kannan Balakrishnan}}]{}
\rhead[]{\textit{\textsf{“Honor Killing”: Whether....}}}



\begin{multicols}{2}

\heading{Introduction}

\noi
Instances of honour-related crimes have seen to be increasing in the Northern part of
India, as our societies are generally intolerant and rigid when it comes to the practices
associated with marriages, especially when women make choices of one’s partner\footnote{Judge, Paramjit S, 2012, Love as Rebellion and Shame: Honour Killings in the Punjabi Literary
Imagination. \textit{Economic and Political Weekly} Vol. 47, No. 44}. Among 60 cases heard by the Vacation Bench of the Punjab and Haryana High Court
in 2006, approximately 27 cases involve young couples who had sought protection or
have applied for anticipatory bail, on fear of apprehension by the police (ibid). Most of
these couples got married against the “bhaichara” (brotherhood) principle of the caste
panchayats and hence considered as to bring dishonor to the society. The past two
decades have witnessed thousands of cases where young couples were victimized for
reasons of bypassing the boundaries set by their communities, or families. In 2009, the
then Union Home Minister, P. Chidambaram, while addressing the Rajya Sabha and
replying to a call for attention on increasing honour-related offences have called such
crimes as a blot on the country\footnote{Srivastava, Mihir. 2009. Honour killings shame Chidambaram. India Today,
\url{https://www.indiatoday.in/latest-headlines/story/honour-killings-shame-chidambaram-53048-2009-07-
28}}.
.
\noi
Honour related crimes have been reported from across the world, with a report showing
increased incidences in South Asian countries. In patriarchal societies, predominantly
in the Middle East and parts of South Asia, the literature suggests that sexual
imprudence, especially by women, is assumed to bring disgrace to their families, who
are forced to disburse terrible prices. The United Nations Commission on Human
Rights also illustrates instances of honour killings in the rest of the world, with pieces
of evidence from Bangladesh, Great Britain, Brazil, Ecuador, Egypt, India, Israel, Italy,
Jordan, Pakistan, Morocco, Sweden, Turkey, and Uganda\footnote{Mayell, Hillary, 2002. Thousands of Women Killed for Family "Honor",
\url{https://www.nationalgeographic.com/}}
. There are also pieces of
evidence of honour-related crimes in Afghanistan, Iraq, and Iran.

\noi
A large extent of literature on honour-related crimes understand it as an Islamic concept
where families impose control over their women, however, Brown writes that there is
nothing Islamic on honour-related crimes\footnote{Brown, Jonathan, 2016, Islam is not the Cause of Honor Killings.~It’s Part of the Solution.
\url{https://yaqeeninstitute.org/read/paper/islam-is-not-the-cause-of-honor-killings-its-part-of-the-solution}}
. Dogan (2005) opines that even amongst the
popular rhetoric, the problem is not found to be specific for Muslims and the most
serious and concentrated occurrences of such offenses do not include Muslims at all.
Labeling it as an Islamic crime tends to dilute the seriousness of the crime and portray
it as a crime committed by Islamic men against their women (Brown 2016). There is
also no uniformity in the methods adopted while imposing honour-related crimes. It
could range from murder to attempted murder to rape to other physical and mental
injuries. However, only a small percentage of women get subjected to such extreme
forms of punishment, although most communities are ruled by notions of honour and
shame. Such links between the behavior and expressions of women and honour of the
community are a result of a faulty and distorted interpretation of various socio-cultural
factors, including religion (ibid). 

%~ \vspace{-.3cm}

\newpage

\noi
‘Crimes of Passion,’ a synonym of the same, are treated with exceeding lenience in
Latin America\footnote{ Goldstein, Matthew A, 2002, The biological roots of heat-of-passion crimes and honor killings. Politics
and the Life Sciences 21(2):28-37}. Widney Brown, the Advocacy Director for Human Rights Watch in
Islamist countries, draws similarities between ‘dowry deaths’ and ‘crimes of passion’.
Society perceives these crimes as excusable and understandable\footnote{Mayell, Hillary, 2002. Thousands of Women Killed for Family "Honor",
\url{https://www.nationalgeographic.com/}}. The prevalence of
these crimes is diverse and spreads across cultures and religions. Similar to dowry
deaths, many women either get profusely injured or lose their life as a result of honourrelated crimes. But, many of these instances remain unreported which also results in
leaving perpetrators of such crimes unpunished. A major reason for the under-reporting
of such offences is since society normalizes the same and conceives it as a justified way
to ‘tackle unruliness’. Additionally, although it is generally committed against women,
other women in the family and community also accept it as a general norm,
strengthening the notions of ‘woman being a material possession’ and ‘violence against
family members is a family issue and not a judicial issue.’ As a result, honour killings become justified by both the perpetrators and the supporters, as an attempt to uphold
morals and behavioural codes set by religion or popular notions in the society.

\noi
Historical pieces of evidence for honour killings can be traced from the Mexican Jewish
legal interpretations. For example, the punishment for women who committed adultery
in the valley of Mexico during 150 BCE - 1521 CE was death either by stoning or by
strangulation. Interpretations to Halakha (the Jewish Law) by Leviticus and
Deuteronomy show that courts used to impose severe penal provisions like the death
penalty, for sexual aberrations exhibited by both men and women. Similar laws also
existed in other jurisdictions as well. For instance, until 1975, the French Penal Code
exempted the husband who murders the wife from any legal consequences, if the
murder has been committed on the grounds of adultery. 


\heading{Honour Crimes in Indian Context}

\noi
Although honour crimes across the world have been conceived as crimes committed to
protecting family honour, in the Indian rural context they have a different underpinning
altogether. Standard definitions of honour killing illustrate it as crimes committed
against the women folk by their own family members, generally male, who are,
compelled to eliminate the disgrace brought to the family by the female members. In
India, rejecting to marry someone the family has proscribed, choosing one’s sexual
partner, being a victim of sexual assault, or even demanding a divorce is believed to
bring disgrace to the family. In addition, the notions of family honour are believed to
be threatened when the woman explores her sexual freedom. Unlike the notions of
crime, where the commission of the act becomes the evidence of the disgrace, here the
mere perception that ‘dishonour’ is being brought to the family by a woman could result
in her being victimized (Human Rights Watch, 2001)\footnote{ Human Rights Watch (2001) “Integration of the human rights of women and the gender perspective:
Violence Against Women and "Honor" Crimes” \url{https://www.hrw.org/news/2001/04/05/item-12-
integration-human-rights-women-and-gender-perspective-violence-against-women}}.

\noi
An analysis of the instances of reported cases of honour-related crimes in India shows
the reasons of such crime perpetration to be varied. It could range from engaging in a
romantic relationship with someone, especially when the person belongs to a different
ethnic or religious affiliation, to a woman who chooses to convert herself into another
religion or embrace the customs or practices of another community. The practice of homosexuality, or being a victim of rape was also found to be reason for the
perpetuation of honor related crimes in India.

\noi
The incidences of honour related crimes have been witnessed more in the rural
communities than in urban centers. The reason for the same could be closely knit social
structures in rural areas where the notions of honour are also upheld in high spirits.
Honour killings stigmatize the Indian rural communities as the crime rate is highly
increasing. Until the recent decade, the Indian society never acknowledged the incidents
of honour-related crimes in India and has sidelined it as a phenomenon limited to
Islamic societies. However, a survey conducted in mid-2007 by a Delhi-based
organization, named Indian Population Statistics Survey (IPSS), around 655 cases of
honour killing have been registered as a case of homicide, as honour-related crimes are
not recognized as a separate crimes in India. The survey shows increasing rates of
honour related murders in Punjab and Delhi, with Muzaffarnagar district of Uttar
Pradesh reporting the highest number of similar crimes until mid-2007. Further, in
2015, the Times of India reported that 215 people were killed in the name of honour in
the year, which is a sharp increase of 796 percent from the year 2014. The state of Uttar
Pradesh (131) reported the highest number of cases during the year followed by Madhya
Pradesh, Punjab, and Haryana (George 2016)\footnote{George, Nirmala, 2016, “India records huge spike in ‘honor killings’ in 2015”, The Seattle Times
\url{https://www.seattletimes.com/nation-world/india-records-huge-spike-in-honor-killings-in-2015/}}
. It needs to be noted that the report points
to the murders that occurred and not other crimes, for example missing individuals.

\heading{Honour Crimes and its relationship with\\ Social Structures}

\noi
The Hindu caste has under it various sub-castes which have developed their own codes
of conduct which are expected to be followed by its members. Such codes could be
developed for many conducts, including dress codes specific to each gender, choosing
one's partner in marriage, sexual orientations, sexual freedoms, etc. Any violation of
these codes would result in strong repulsions from the community to maintain and
restore the sanctity of these codes. Honour crimes in India, unlike that, is practiced in
other parts of the world, are one such repulsion to buttress the societal codes. 

\noi
Honour related crimes that occur in India can be divided into two broad categories:
those committed by individual members of the family against their own family members to remove the disgrace brought to that family and those committed by the
members of the community for protecting the honour of the community and prevent
others from committing similar ‘disgraces’ (Ramakrishnan 2009). In the Northern part
of India, such a group comprising of members of the community form the Khap
panchayats. Such panchayats have empowered themselves as quasi-judicial bodies,
who adhere themselves to the traditional customs and values and impose stringent
punishments so that the cultural values can be upheld. These punchayats act as bodies
that could settle disputes and disagreements within the community.

\noi
Certain acts by the members of the community are considered more serious than others
and the panchayats use all measures to curb them. Codes governing customary
marriages are taken more seriously than a dress code and hence invite more stringent
punishments. The punishments pronounced by such panchayats in case of ‘prohibited’
marriages include declaring the couple as siblings, annulment of marriage, ostracizing
the couple and their families, ordering for rape or parading naked, killing of the people
involved, etc. ‘Prohibited’ nature of the relationship is determined by the principle of
‘kinship exogamy’, which considers people belonging to same gotra/caste as siblings
and prevented from engaging in marital relationships. The customary marriages in
many parts of North India follow caste endogamy and adopt the rule of ‘got’ exogamy\footnote{Most caste follow got exogamy of three or four gotra, which prevents marital relationship between
members of these gotras. This principle prevents marriage between one’s own gotra, one’s father’s gotra
and mother’s gotra, or two generations earlier.}

\noi.
In addition to kinship exogamy, is the extension of the same principle, that is, the rule
of territorial exogamy. According to this principle marriages among people belonging
to the same villages are prohibited. Both these principles together lead to the
elimination of a large number of gotras from marrying each other. As generations
expand, the inhabitants of one village will not be able to marry within their village but
also from neighbouring villages.

\noi
Evidences show that in many cases of honour-related crimes in India, the crimes were
committed as a result of the violation or suspected violation of codes governing
customary marriage or claiming sexual autonomy. Indian history also shows that many
times families are forced to commit the same by the caste groups to retain their position
in the community. Literature suggests that honour-related crimes in India have emerged
as an attempt to retain the dominance of higher castes over the lower castes in the society. Many of the victims were unaware of their spouses or romantic partners to be
their brothers/sisters as part of the principle of kinship exogamy or territorial exogamy.
Further, recent instances have also shown that the phenomenon of honour killing is not
limited to rural or Northern India. There are examples like the case of Bibi Jagir Kaur,
a high profile minister and the first woman president of Shiromani Gurdawara
Prabhandak Committee in 2000, case of D P Yadav, an MP of Uttar Pradesh (2002),
the case of Todi family (2007), etc. Another characteristic of honour-related crimes is
that the perpetrators of the crime believe that they have performed their responsibility
as member of the family or community to uphold their honour. These facts make it
different from the other crimes. 

\heading{Legal Framework on Honour Crimes in India}

\noi
India does not have specific legislation to deal with honour crimes and they are
generally recorded under the general Indian Penal Code (IPC) offences. The
Constitution of India ensures various rights to citizens of India, including the right to
equality before the law (Article 14), the right against discrimination (Article 15(1)),
creating special provisions for women and children (Article 15(3)), right to freedom of
speech (Article 19), right to life and personal liberty (Article 21) and offering equal
opportunities to grow safely and protecting young people from abuse (Article 39 (f)).
Honour related crimes can be seen as violative of all the above-mentioned rights. Since
honour crimes are largely directed against women, which lead to inequality. Such
crimes against women also result in discrimination against them leading to violation of
Article 15 (1) of the Constitution of India. Further, if such instances have resulted in
the loose of life of the person or result in fatal injuries, then it leads to violation of
Article 21 of the Constitution of India. The Government of India also observed the
urgency of the situation suggesting to amend IPC and the Evidence Act and also to
come up with a new law to deter such instances. The amendments were proposed to
sections 300 and 354 of IPC; Sec 105 of Evidence Act and provisions of the Special
Marriage Act (1954). 

\noi
The fifth clause of Section 300 was suggested to be amended to include “persons or a
member of the family or member of a group of a specific clan or community or caste
panchayat believing that the victim has brought either real or perceived dishonor”.
Further, an explanation of dishonor was suggested to include the adoption of dress codes that are unacceptable to the community. The amendment to the Evidence Act
suggested shifting the onus of innocence to the offender or group. Further, there were
also suggestions to abolish the period of notice, which is currently 30 days, in cases of
inter-religious/inter-caste marriages. 

\noi
A report of the Law Commission of India on “Prevention of Interference with the
Freedom of Matrimonial Alliances (in the name of Honour and Tradition): A Suggested
Legal Framework” was proposed to arrest the interference of caste councils towards the
life and liberty of young persons. In the introduction to the proposed bill, it is noted that
the terms honour killings and honour crimes are many a time loosely used to refer to
incidents of violence against young people intending to get married against the wishes
of their family members. The report also realizes the need for new legislation to address
the growing number of crimes committed in the name of honour. The Supreme Court
of India has also observed in several cases, including \textit{Arumugam Servai vs. State of
Tamil Nadu}\footnote{ 2011 6 SCC 405}\textit{and Lata Singh vs. the State of U.P.}\footnote{2006, 5 SCC 475} had condemned the activities of such
caste-based communities that take law in their hands and propagate violence against
young couples. The proposed bill recognises honour crimes as a socio-cultural problem
that is rooted in superstitious beliefs and authoritarianism and argues that such causes
should not result in the violation of the right to life and liberty of young couples. This
Bill tried to bring such violent activities of caste panchayats punishable under our
criminal law and to make the officials accountable for failing to perform their duties.
Further, the bill shifts the burden of proof on the mere accusation of involvement in a
serious offence. However, even after the draft Bill and stressing upon special
legislation, there are no new laws introduced. Currently, honour crimes and recorded
under a range of crimes, including culpable homicide, culpable homicide amounting to
murder, attempt to murder, rape, hurt, grievous hurt, criminal conspiracy, abetment, etc,
of the IPC. 

\heading{Political Systems and Honour Crimes}

\noi
Khap Panchayats have currently emerged as parallel institutions that perform the
functions of the executive, legislature, and judiciary. Evidences also show that the
current criminal justice framework in India is not capable to challenge the authority of these parallel bodies. Such institutions also receive high levels of support as they act as
community organizations for settling ‘issues’ faced by the community, which are
largely the functions of the Panchayati raj institutions and the criminal justice system.
It needs to be acknowledged that such parallel bodies have emerged due to the
weakening of Panchayati Raj institutions of respective states. The failure of such
institutions has led to khap panchayats taking control over various ranges of social
intercourses. In addition, these community groups have also become vote bank
determinants which also prevent political parties to take cognizance over the matter.

\heading{Conclusion}

\noi
Honour related crimes have been a common phenomenon in India in the past decade
claiming the lives of many young couples. Analysis of the instances in India shows that
such crimes do not receive the required attention from either the executive or the
legislature. The judiciary has time again condemned such occurrences and has also
formed a Law Commission to draft model legislation to combat such offences.
However, a lack of a political will could be seen as a reason for the non-adoption of the
legislation. In addition, the structures in the society and underlying caste hierarchies
have also overtly supported the caste-based communities who have spearheaded the
commission of such instances in India. Since there is no support from the society nor
the legislature, the law enforcement agencies also could not do much towards their
prevention or prosecution of the guilty. 

\noi
In addition, a blame game between the law enforcement officers and the political parties
could also be seen in the growing instances of honour-related crimes in India. When
the police officers and political parties blame the political parties to be non-complacent,
the political parties even go to the extent of defending such practices claiming it as a
part of the social custom and tradition. The political parties never dare towards
declaration of such caste panchayats as illegitimate extra-judicial bodies fearing loss of
their support in retaining political power. 

\noi
So by considering the seriousness of the present scenario and the lack of political will
of the major parties, a strong legal enactment can only help to nail the perpetrators of
such serious crimes. Such a legal enactment should consider not only the act of honour
killings as offences but also glorification as crime and it should give Magistrate special powers to take cognizance and injunctions against caste panchayats. There is a need for
sufficient research into the crime to define it clearly along with the modus operandi and
to have overall statistics of this particular crime. Hope the elected representatives will
be sensitive to these social issues and the political system will not use these social
injustices for political mobilization.

\end{multicols}
\label{end2016-art3}
