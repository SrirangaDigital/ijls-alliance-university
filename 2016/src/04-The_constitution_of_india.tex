\setcounter{figure}{0}
\setcounter{table}{0}
\setcounter{footnote}{0}

\articletitle{The Constitution of India: The Grand Experiment}\label{2016-art4}
\articleauthor{Satamita Ghosh\footnote{Advocate, Delhi High Court}}
\lhead[\textit{\textsf{Satamita Ghosh}}]{}
\rhead[]{\textit{\textsf{The Constitution of India:....}}}



\begin{multicols}{2}

\heading{Introduction}

\noi
65 years since the drafting of the Constitution of India, the experiment called India has survived but how well did the grand experiment of the Constitution of India with an assembly of varied people work and how well has it performed in the past 65 years and how did it work and for whom did it work well would be explored in this paper. The workings of the constitution would be dealt around with three different perspectives of people from three diverse strata of the society, a common man (person from the lowerincome group), a person of the middle class, and the business class.

\noi
Winston Churchill talked about how India would not survive post and independence from the British rule and stated that “If Independence is granted to India, power will go to the hands of rascals, rogues, freebooters; all Indian leaders will be of low caliber and men of straw. They will have sweet tongues and silly hearts. They will fight amongst themselves for power and India will be lost in political squabbles. A day would come when even air and water would be taxed in India.” But the fact is there is a country called India still existing. So the nation has survived but it is not purely attributable to just political intellect, on the contrary, it was the judiciary that has held the country together though with some decisions the divisions in the country have become prominent.

\noi
Before independence, India was divided into two Political categories: the provinces of British India and the Indian states, which came together after the independence and adoption of the Constitution. Since 26 January 1930, it was the day on which thousands of people, in villages, in mohallas, in towns, in small and big groups would take the independence pledge, committing them to the complete independence of India from British rule. It was only fitting that the new republic should come into being on that day, marking from its very inception the continuity between the struggle for independence and the adoption of the Constitution that made India a Republic. Its origin lie deeply embedded in the struggle for independence from Britain and the movements for responsible and Constitutional government in the princely states.

\noi
Ours is a written Constitution, which is also the longest in the world and quite a few of
the provisions are influenced by the Constitutions of other countries.

\noi
Under the Constitution of India, the President occupies the same position as the King under the English Constitution. He is the head of the State but not of the Executive. He represents the Nation but does not rule the Nation. He is also bound by the suggestion of his council of ministers. The Constitution is a dual polity with single citizenship. It is partly flexible and partly rigid in nature because the procedure of amendment of the Constitution is neither very easy nor very difficult, it strikes a golden balance. It is only a few of the amendments of provisions that require the ratification of the state legislatures and even ratification by only $\frac{1}{2}$ of them, and the rest of the Constitution may be amended by a special majority of the Union Parliament i.e. majority of not less than $\frac{2}{3}$ of the members of each house present and voting. It also provides for a universal franchise without any communal representation.

\noi
The Constitution of India has been successful in striking a balance between the powers
of the union and the state. Whenever any conflict has arisen between the same, the
judiciary has always come to the rescue.

\noi
The Constitution’s importance has been a question of great significance; it is important because it structures the norms that govern our politics and also maps out the rules to be followed. It also establishes the apex bodies like the Supreme Court and the parliament of the country and lays down the key roles that they are required to play in the governance of the country. It also lays down the powers and functions of the office of president, Governor, prime minister, and other such important offices of the nation. Therefore, it clearly defines the institutional structure of the country.

\heading{Initial Criticism of Constituent Assembly}

\noi
The enormous task of drafting the Indian Constitution was taken up by the constituent assembly. The assembly brought into existence with the help of the British, drafted the Constitution in three years, from December 1946 to December 1949. The assembly got legal status in the year 1947. In the period it had 11 sessions and 165 days of actual work. As many as about 2500 amendments were made in the said draft of the Constitution.

\noi
The term constituent assembly has been defined in a variety of ways. It is a representative body chosen to consider and either adopt or propose a new Constitution or change in the existing Constitution. According to Abbe Sizes, it is an assembly of extraordinary representatives, to which the nation shall have entrusted the authority to make a Constitution or at any point define its contents. It is a democratic device for formulating or adopting a new Constitution by free people.

\noi
In India, the constituent assembly was formed with the help of the British, with the help of The Cabinet Mission Plan. A detailed procedure for the formation of the constituent assembly was laid down in the plan. A specific number of seats for every province was also proposed in the plan. In all, the constituent assembly was to have 389 seats, out of which 296 of them were to be elected from British India, and 93 were to be the representatives of the native states. Initially, the plan was opposed by both congress and the Muslim League. But later, both the parties contested the election. Congress won with an overwhelming majority in the elections. The constituent assembly had a total of 15 committees with a membership of more than 80 people. Various reports were submitted by the said committees, based on which the Drafting committee was constituted on 27th August 1947. The Drafting Committee was formulated to cater to the needs of people from all the social groups of the country then. The committee was headed by Dr. B.R. Ambedkar. Also, Maulana Azad, Jawahar Lal Nehru, Rajendra Prasad, and Vallabh Bhai Patel played a key role in the assembly. It is said that congress had a great hold on the assembly and that although the issues raised were openly debated but the influence of congress on the same is irresistible.

\heading{Features of the working process of the constituent Assembly}

\begin{enumerate}
\item Decision making by consensus:

\noi
The manner of deciding in the constituent assembly is by consensus or unanimity of the opinions of the people present in the assembly. The approach was adopted in a variety of ways. The most important among them were the congress assembly party meetings, where each provision of the Constitution was subjected to frank and searching debates. The primary example of the decision by consensus were the provisions relating to federalism and languages.

\item Principle of Accommodation:

\noi
This is the ability to reconcile inconsistent concepts. India’s Constitutional structure is a great example of the same. It has reconciled various concepts such as the federal and unitary system, membership of the commonwealth and republican status of the government, etc.

\item Art of selection and modification:

\noi
Although various provisions of the Constitution have been borrowed by the constituent assembly the same has been modified in a way that suits the Indian conditions. An example of the same is the provision for Constitutional amendments. Three mechanisms have been developed by the assembly to keep the Constitution flexible but at the same time protect the interest of the states.
\end{enumerate}

\heading{Criticism of the Constitution by the Assembly}

\noi
One of the members of the assembly charged that the Constitution was largely of foreign origin and would be unworkable in the Indian conditions. The Constitution made by the committee is said to have no manifest relation with the fundamental spirit of India. It was said to be suitable for India as it was not represented by the ancient polity of India. On the contrary, the Constitution of the country has been accepted by the masses of the nation and the goals of the Constitution have been accepted as one’s own goals by the people of the country.

\noi
Another issue raised by the members of the assembly against the Constitution is the overcentralization of the power, i.e. state according to the critics has been reduced to Municipalities only. To this, it was replied that the question is based upon a misunderstanding and that the basic principle of federalism is that the legislative authority and executive authority are partitioned between the center and the state. The state under our Constitution is not dependent upon the center for its legislative and executive authority and that the two are co-equals.

\noi
The next charge is that the center has been vested with the powers to override the states, which was admitted. But it was stated that the overriding feature is not a basic feature of the Constitution it is to be exercised only during emergencies.

\noi
The Constitution has been stated to be very long and rigid. The same has been accepted to be long and detailed but it’s not rigid. There were such great details because acc. To B R Ambedkar, it would have been very easy to pervert the Constitution by simply perverting the administration.

\heading{The Landmark Cases}

\noi
The first amendment to address \textit{Romesh Thappar vs The State Of Madras,} the petitioner was the printer, publisher, and editor of an English journal called Crossroads. Crossroads was printed in and circulated from Bombay (now Mumbai). It was considered critical towards various policies of the government. The government of madras had declared the communist parties to be illegal. Madras state government, using their powers as provided for under Section 9 (1-A) of the Madras Maintenance of Public Order Act, 1949 issued an order No MS 1333 dated March 1, 1950. In such order, they imposed a ban on the distribution of crossroads in Madras. Romesh Thapar approached the Supreme Court of India and alleged that such a ban violated his freedom of speech and expression as provided for under Article 19 (1) (a) of the Constitution of India. The court, in this case, declared that such a ban prima facie constitutes a violation of the fundamental right of freedom of speech and expression unless it is proven that such restriction fell under the ambit of exceptions as provided for through Article 19 (2) of the Constitution of India. The issue in this matter, therefore, was whether Section 9 (1-A) of the Madras Maintenance of Public Order Act was protected by Article 19 (2) of the Constitution. Section 9 (1-A) allowed the Government “to secure the public safety or the maintenance of public order, to prohibit or regulate the entry into or the circulation, sale or distribution in the Province of Madras or any part thereof of any document or class of documents”. Given that Article 19 (2) did not include the expression ‘public safety’ or ‘public order, the issue then was whether it would fall under the ambit of Article 19 (2) and therefore it could be considered as a “law relating to any matter which undermines the security of or tends to overthrow the state”. The government therein submitted that the phrase “public safety” as it appears in the said Act, which is a legislation relating to law and order, which connects to, “the security of the state” which is within the ambit of Article 19 (2) as “state” is defined in Article 12 of the constitution on India including, \textit{inter-alia,} the government and the legislature of each of the Provinces. The court noticed that the expression ‘public safety’ had a broader implication than ‘security of the state’, as ‘public safety includes such trivial matters which may not be necessarily as critical as the implications of the ‘security of the state’. The court concluded “unless a law restricting freedom of speech and expression is directed solely against the undermining of the security of the state or the overthrow of it, such law cannot fall within the reservation under clause (2) of Article 19, although the restrictions which it seeks to impose may have been conceived generally in the interests of public order. It follows that Section 9 (1-A) which authorizes imposition of restrictions for the wider purpose of securing public safety or the maintenance of public order falls outside the scope of authorized restrictions under clause (2), and is therefore void and unconstitutional”.

\noi
The First amendment in The Constitution of India was effected in the year 1951. This amendment was aimed at strengthening the state to make regulations to restrict the freedom of speech and expression by adding to the scope of Article 19 (2. This ‘crisis of media’ in the nascent stage of the country was viewed as a crisis of the country itself, and this structure of ‘national crisis’ is still prevalent as a threat that penetrates to multiple cross-sections of the history of the media in independent India. Prof. Upendra Baxi has termed this as “Constitutionalism as a site of state formative practices” (1). Article 19 (1) (a) in its original form read as follows: “All citizens shall have the right to freedom of speech and expression.” This fundamental right is, restricted through Article 19 (2) of the Constitution of India which provides for: “Nothing in sub-clause (a) of clause 1 shall affect the operation of any existing law insofar as it relates to or prevents the state from making any law relating to libel, slander, defamation, contempt of court or any matter which offend against decency or morality or which undermines the security of the state or tends to overthrow the state.”

\noi
The very first amendment was made to the provisos to Article 19 (2). The amended proviso read: Article 19 (2) “Nothing in sub-clause (a) of clause 1 shall affect the operation of any existing law insofar as such law imposes reasonable restrictions on the exercise of the right conferred by the sub-clause in the interests of the security of the state, friendly relations with foreign states, public order, decency, or morality or in relation to contempt of court, defamation, or incitement to an offence.”

\noi
The three significant changes brought through this amendment were: i) Replacing ‘reasonable restrictions’ with ‘restrictions’ b) addition of ‘friendly relations with foreign states as a ground under Article 19(2), and c) ‘public order’.

\noi
The then prime minister of India Mr. Jawahar Lal Nehru was unhappy with such interpretation by the court. He communicated to Dr. Bhimrao Ambedkar “expressing the view that the Constitution’s provisions on law and order and subversive activities needed to be amended. Reflecting the difficulties, the government was having with the courts over the fundamental rights, Nehru added that the provision affecting zamindari abolition and nationalization of road transport also needed to be amended”. In February 1951, Mr. Jawahar Lal Nehru established a cabinet committee to review the amendment which was proposed. The ministry of home affairs recommended to the committee that ‘public order and ‘incitement to a crime should be added with the other exceptions to article 19. It favored removing ‘to overthrow the state’ to effectuate a wider formulation through adding ‘in the interests of the security of the state’ instead. The original text of Article 19 (2) did not contain the ‘reasonable’ as a qualifier for ‘restrictions’, and hence the ministry of law and justice was believed that ‘reasonable’ in Article 19 should be preserved and included in Article 19 (2). The word reasonable was added to the article. Such adding of ‘reasonable’ may not have been a step welcomed by the ruling dispensation, as it was evident that if the government had the option, the government would have chosen not to have any qualifications to the restrictions. In a letter to T T Krishnamachari, Nehru stated that he did not like the word ‘reasonable’ for the reason that word ‘Reasonable’ was ambiguous in meaning and therefore it can lay open the chances of invocation of powers of the court being to interpret reasonability of a particular act.\footnote{Dasu Krishnamoorty, \textit{Nehru’s Tryst With Press Curbs,} Thehoot,, may 29,2007, \url{http://asu.thehoot.org/free-speech/media-freedom/nehru-s-tryst-with-press-curbs-2572}}

\noi
\textit{State of Madras v. Champakam Dorairajan} (Discrimination based on caste) in this case a brahmin candidate (girl) filed a petition for issuance of a writ of Mandamus restraining the State of Madras from enforcing communal Government Order issued by the state that provided for the reservation In the electoral constituencies. A full bench of the Madras High Court upheld the petitioner's plea. The state appealed to the Supreme Court. The said appeal was dismissed by a seven-judge bench. This judgment necessitated the 1st amendment to the Indian Constitution, which added clause (4) to Article 15.

\noi
The Supreme court of India in this case held “the communal G.O. constituted a violation of the fundamental right guaranteed to the citizens of India by Article 29(2) of the Constitution of India and was therefore void under Article 13- The directive principles of State Policy laid down in Part IV of the Constitution cannot in any way override or abridge the fundamental rights guaranteed by Part III- On the other hand, they have to conform to and run as subsidiary to the fundamental rights laid down in Part III.” The preamble to the Constitution of India reflects the will of the people of India for the establishing a new structure of “security- social, political and economic,” for all its citizens based on “justice, liberty and equality”. These are the great objects that the Constitution \& the Govt. established by it are intended to serve \& promote. The preamble of the constitution, however, leaves out any definition of the correlating rights of people and the State who exercise their power under the constitution. Architects of the Constitution as on behalf of the people of the country were devising an apparatus for the governance of a free democracy. In such a process, they were worried about the threat to the liberty of the individual and civil rights from the exercise of powers by government and therefore gave the same a place at the beginning of the Constitution in the preamble and the dedicated the chapter on "Fundamental Rights." They experienced an extensive and costly experience of the previous regime with its frequent encroachments on the personal liberty of citizens, especially during the period of the last world war; its emphasis on, if not encouragement of communal \& other differences which seriously weakened national unity; \& its discriminating practices in favour of individuals \& communities designed to win their support.

\noi
The people of this country had also become painfully aware of the evils of communal discord \& distrust culminating as they did in the partition of the country \& were presumably keen on eradicating the virus of communalism that had infected the body politic. Chapter III of the Constitution of India reflects these widely prevalent feelings \& ideas of the time \& is both a reaction to the evils of the past \& a guarantee of Constitutional liberty to the citizen in the future. The rights singled out for such protection \& guarantee are such as might be regarded as highly important to a citizen in a free civilized State \& are appropriately styled "fundamental rights." One cannot shut 'one's eyes to the fact that inequality is a fundamental or basic fact in actual life. Absolute equality, there is not, among human beings. It is a matter of common sense that you cannot treat an adult \& a child, a sane man \& an idiot or lunatic, a millionaire \& a pauper, a convict \& an innocent man, a literate \& an illiterate person, an engineer \& a bricklayer, a qualified physician or surgeon \& a quack, as occupying the same or equal position in actual life. Though Article 14 recognizes a general or Constitutional equality among all human beings, some distinction, some classification, some gradation or differentiation either in legislative practice or in day to day administration is inevitable if one has to reconcile Constitutional or legal equality with the facts of life \& the needs of public administration.

\noi
According to the Supreme Court, \textit{“the directive principles of the State policy, which by article 37 are expressly made unenforceable by a Court, cannot override the provisions found in Part III which, notwithstanding other provisions, are expressly made enforceable by appropriate Writs, Orders or directions under article 32. The chapter of Fundamental Rights is sacrosanct and not liable to be abridged by any Legislative or Executive Act or order, except to the extent provided in the appropriate article in Part III. The directive principles of State policy have to conform to and run as a subsidiary to the Chapter of Fundamental Rights. In our opinion, that is the correct way in which the provisions found in Parts III and IV have to be understood. However, so long as there is no infringement of any Fundamental. Right, to the extent conferred by the provisions in Part III, there can be no objection to the State acting in accordance with the directive principles set out in Part IV, but subject again to the Legislative and Executive powers and limitations conferred on the State under different provisions of the Constitution.\footnote{Madras v. Smt. Champakam Dorairajan AIR 1951 SC 226}”}

\noi
\textit{Kameshwar Singh v. State of Bihar,} AIR 1951 Pat. 91 in this case the Bihar land reforms act 1950 was challenged which made provision for the transference of interest in the land to the state and lease of such interest including interest in trees etc. The ground of challenge is the classification of Zamindars made to give compensation was discriminatory and denied equal protection of laws guaranteed to the citizens under Article 14 of the Constitution. The Patna high court struck down the Bihar Act, as unconstitutional and void as it was in contravention with the provision laid down under article 14 of the Constitution. The central government felt that such a judicial pronouncement would endanger the whole zamindari abolition programme. Therefore, to overcome this difficulty, a new provision under art 31A was introduced in the Constitution vide the 1st amendment.

\subsection*{Constitution from the perspective of the poor}

\noi
Nehru wanted to work for the benefit of and for the upliftment of the poor and hence almost all things went fine in terms of poor and property initially, but later due to political reasons and improper application of schemes poor are left out and have become the victims of improper administration of the system, at the same time criminal justice system also been anti-poor in the country and did not provide enough relief to the accused who are poor, they are tortured in the custody, they also do not get the basic necessities in the custody and are deprived of the basic humane treatment by the authorities. Many of the people remain in the prisons because they are unable to find themselves advocates who can represent them in the cases, as a result, they remain in the prisons for a period more than the actual imprisonment for the offence.

\noi
\textit{Sunil Batra vs. Delhi Administration;} The petitioner, a convict under a death sentence, through a letter to one of the Judges of this Court alleged that torture was practiced upon another prisoner by a jail warder, to extract me from the victim through his visiting relations. The letter was converted into a habeas corpus proceeding. The Court issued notice to the State and the concerned officials.

\noi
The Judgment stated that No prisoner can be personally subjected to deprivation not necessitated by the fact of incarceration and the sentence of the court. All other freedoms belong to him to read and write, to exercise and recreation, to meditation and chant, to comforts like protection from extreme cold and heat, to freedom from indignities such as compulsory nudity, forced sodomy, and other such unbearable vulgarity, to movement within the prison campus subject to requirements of discipline and security, to the minimal joys of self-expression, to acquire skills and techniques. A corollary of this ruling is the Right to Basic Minimum Needs necessary for the healthy maintenance of the body and development of the human mind. This umbrella of rights would include: Right to proper Accommodation, Hygienic living conditions, Wholesome diet, Clothing, Bedding, timely Medical Services, Rehabilitative and Treatment programmes”.

\noi
Prisoners are peculiarly and doubly handicapped. For one thing, most prisoners belong to the weaker segment, in poverty, literacy, social station, and the like. Secondly, the prison house is a walled-off world that is incommunicado for the human world, with the result that the bonded inmates are invisible, their voices inaudible, their injustices unheeded. So it is imperative, as implicit in Art. 21 that life or liberty shall not be kept in suspension animation or congealed into animal existence without the fresh flow of air.

\noi
According to Justice Chandrachud "Convicts are not, by mere reason of the conviction, denuded of all the fundamental rights which they otherwise possess. A compulsion under the authority of law, following upon a conviction, to live in a prison-house entails by its force the deprivation of fundamental freedoms like the right to move freely throughout the territory of India or the right to 'practice' a profession. A man of profession would thus stand stripped of his right to hold consultations while serving out his sentence. But the Constitution guarantees other freedoms like the right to acquire hold and dispose of property for the exercise of which incarceration can be no impediment. Likewise, even a convict is entitled G to the precious right guaranteed by Article 21 of the Constitution that he shall not be deprived of his life or personal liberty except according to procedure established by law."

\noi
The only way that we will ever have prisons that operate with a substantial degree of justice and fairness is when all concerned with that prison staff and prisoners alike share in a meaningful way the decision-making process, share the making of rule and their enforcement. This should not mean three "snitches" appointed by the warden to be an "inmate advisory committee". However, if we are to instill in people a respect for the democratic process, which is now the free world attempts to live, we are not achieving that by forcing people to live in the most totalitarian institution that we have in our society. Thus, ways must be developed to involve prisoners in the process of making the decision that affects every aspect of their life in the prison. Imprisonment and other measures which result in cutting off an offender from the outside world are afflictive by the very fact of taking from the person the right of self-determination by depriving him of his liberty.

\noi
Therefore the prison system shall not except as incidental to justifiable segregation or the maintenance of discipline, aggravate the suffering inherent in such a situation. The institution should utilize all the remedial, educational, moral, spiritual, and other forces and forms of assistance that are appropriate and available, and should seek to apply them according to the individual treatment needs of the prisoners.

\noi
\textit{Hussainara Khatoon v. the State of Bihar,} AIR 1979 SC 1369, It was brought to the notice of the Supreme court that an alarmingly large number of men, women, and children were kept in prison for years awaiting trials in courts of law. The offences that they were charged with were trivial and if convicted the imprisonment would not have been more than a few months. But they were deprived of their freedom for a period ranging from three to ten years, without the commencement of the trials. The court held that: \textit{“the procedure under which a person may be deprived of his life or liberty should be 'reasonable fair and just.' Free legal services to the poor and the needy is an essential element of any 'reasonable fair and just procedure. A prisoner who is to seek his liberation through the court's process should have legal services available to him.\footnote{Hussainara Khatoon \& Ors vs Home Secretary, State Of Bihar 1979 AIR 1369}”}

\noi
Article 39A provides that free legal services are a prominent part of 'reasonable, fair and just procedure’ because deprived of it any citizen in distress by economic or other debilities would not be able to secure justice for himself. The right of free legal aid is, hence, an obvious element of the 'reasonable, fair and just process for the one who has been accused of any offence. Therefore, it has to be an inherent guarantee under Art. 21 of the constitution of India.

\noi
The poor in their contact with the legal system has always been on the wrong side of the law. They have always come across "law for the poor" rather than "law of the poor". The law is regarded by them as something mysterious and forbidding-always taking something away from them and not as a positive and constructive social device for changing the socio-economic order and improving their life conditions by conferring rights and benefits on them. The result is that the legal system has lost its credibility for the weaker sections of the community. It is, therefore, necessary to inject equal justice into legality and that can be done only by a dynamic and activist scheme of legal services.

\noi
The State cannot avoid its constitutional obligation to provide speedy trial to the accused by pleading financial or administrative inability. The State is under a Constitutional mandate to ensure speedy trial and whatever is necessary for this purpose has to be done by the State. It is also the Constitutional obligation of this Court, as the guardian of the fundamental rights of the people as a sentinel on the qui-vive, to enforce the fundamental right of the accused to a speedy trial by issuing the necessary directions to the State which may include the taking of positive action, such as augmenting and strengthening the investigative machinery, setting up new courts, building new courthouses, the appointment of additional judges and other measures calculated to ensure speedy trial.

\noi
The powers of this Court in the protection of the Constitutional rights are of the widest amplitude and this Court should adopt an activist approach and issue to the State, directions which may involve taking of positive action to secure enforcement of the fundamental right to a speedy trial. But to enable the court to discharge this constitutional obligation, the court must have the requisite information bearing on the problem.

\noi
When an inmate is cruelly restricted in a manner that supports no such relevant purpose the restriction becomes unreasonable and arbitrary and unConstitutionality is the consequence. Traumatic ` futility is obnoxious to pragmatic legality. Social defense is the reason for the penal code and bears judicial control over prison administration. If the whole atmosphere is that of constant fear of violence, frequent torture, and denial of opportunity to improve oneself is created or if medical facilities and basic elements of care arid comfort necessary to sustain life are refused then also the humane jurisdiction of the court will become operational based on Art. 19.

\noi
\textit{M .H.Hoskot vs. State of Maharashtra} AIR 1978 SC 154, the Supreme court laid down the right to free legal aid at the cost of state to an accused who could not afford who cannot afford legal services for the reasons of poverty, indigence, or incommunicado situations. This has been declared to be a duty of the state and not government charity.

\noi
The Constitutional obligation is to provide free legal service to an indigent accused not only arises when the trial commences but also attaches when he is for the first time produced before the magistrate. That is the stage where the accused needs proper legal advice and representation. It was also held that the magistrate or the session judge before whom the accused appears is under the obligation to inform the accused that if he is unable to engage a lawyer, he is entitled to obtain legal aid at the cost of the state.

\heading{Constitution from the perspective of the middle class}

\noi
\textit{The middle class has always been the worst sufferers both politically and judicially. People of this stratum of the society become the testers for both the legislature and the policymakers of the nation. According to rulings of the Supreme Court in various vases, legal service has been granted only to the people of lower classes or such people who are indigent. The Supreme Court in the year 1978 had formulated a legal aid committee under the chairmanship of Hon’ble Justice D.A. Desai, specifically to help the poor people who approached the apex court for the want of justice but were unable to pay the huge amount to the advocate required to represent them. The Committee formed und chairmanship of justice Desai would provide Free Legal Aid to citizens of the country having total income below rupees 12000 a year.}

\noi
\textit{Supreme Court Middle Income Group Legal Aid Society}

\noi
\textit{The only exception to this practice as prevalent in the country is the “Supreme Court Middle Income Group Legal Aid Society”. The target group of this legal aid society is specifically the middle-income group or the middle-class people of the society who cannot engage leading advocates on their condition. They could make their payment for such legal services at more affordable rates under the MIG (Middle Income Group) Scheme.} The advantage of this Scheme would be availed by such people whose annual income is more than Rs 12,000 but less than Rs 1,20,000, and by government employees at all central, state, and municipal levels, and employees of public sector undertakings (PSUs). The fixed prices were given for advocates’ charges at various stages including court charges and overheads.

\noi
\textit{It can be observed that a huge number of people who would fall in the ambit of the income group as given above would not be able to pay the rising cost of litigation. This Scheme made provisions for disbursement of Rs 3300 as an upper limit in charges to an Advocateon-Record up till the stage of admission of the case and, further, Rs 3300 till the stage of the final hearing. Hence, the charges that an Advocate-on-Record could levy from a client till the stage of disposal in the Supreme Court are capped at Rs 6600 only. Likewise, charges a senior advocate can take including everything till the admission of the case is capped at Rs 4300. Additionally, Rs 5000 could be charged as fees for the final hearing. Therefore, the services of a designated senior advocate may be availed at the cost of Rs 9300 till the final decision in any matter. Also according to the scheme, the advocates cannot charge for the adjournments, the court fees are given under the Supreme Court Rules.}

\noi
\textit{The charges to be paid for services of an Advocate-on-Record or senior, if such services are requested and availed by the applicant, were provided for in the schedule of the aforementioned scheme as well.}

\noi
\textit{The scheme also established a panel of the advocates which also included senior AoR and senior advocates including one or two of such advocates who were well versed in the regional language so that they were able to understand the documents relating to the lower courts. The penal advocates so chosen would have to give an undertaking of compliance to the scheme as well for any case that would be given to them under the scheme.}

\noi
\textit{The scheme provided for an office of the secretary who would be appointed under the said scheme. This secretary was the designated authority for receiving applications by persons who were desirous of availing services of legal aid under the scheme. A charge of Rs. 350 was required to be deposited along with the application to the secretary for such purposes as charges to the legal aid committee. After vetting of the suitability of the case for litigation, the litigant would be required to pay the other charges as mentioned in the foregoing paragraphs.}

\subsection*{Mandal commission case}

\noi
\textit{The story of affirmative action through Reservation in India has had a long tale. Article  16(4) of the Constitution of India warrants that the government may create reservations for the purposes of employment for educationally and socially backward classes. An important query thereof would be the definition of backward class. The solution to the query was investigated by the Kalelkar Commission which was established in the year 1953. Kelkar commission linked six markers for the identification of the backward classes, which were: 1) Traditional occupation and profession, 2) Literacy, 3) Population, 4) Distribution and concentration, 5) Social position in the caste hierarchy, and 6) Representation in the Govt. service, or the industrial sphere.}

\noi
\textit{These observations of the Kelkar commission were disapproved by Government through a memorandum issued in the year 1956. Later, in the year 1961, the Government decided to allow all the Governments at the state level to come up with their list of backward classes.}

\noi
\textit{In the year 1979, through Presidential Order issued under Article 340, Mandal Commission was established to ascertain all backward classes for the purpose of affirmative actions. Following extensive evaluation, Mandal Commission provided for 11 indicators for “social and educational backwardness,” which then were categorized in 3 distinct headings – social, economic, and political. Social indicators of backwardness consisted of – “castes/classes considered backward by others, caste/classes depending upon manual labour for their livelihood, castes/classes with low average ages of marriage and castes/classes with a low proportion of the female workforce.” Educational indicators consisted of - “percentages of school attendance, dropouts, and matriculation. Economic criteria included value of family assets, number of families living in Kucha houses, the distance of sources of drinking water, and households having taken consumption loans.” Social markers were assessed at three points each, educational indicators assessed at two points, and economic were assessed at one point. Making the total of such points as 22. Any caste or classes which scored above 11 on the scale were to be considered as “socially and educationally backward”.}

\noi
The Mandal commission also utilised additional criteria to distinguish \textit{“other backward classes,”} and also from the non-Hindu section of the population. As an outcome of such exercise, the Mandal Commission found that the percentage of OBCs in India was 52\%.

\noi
The Mandal Commission report came to a lot of controversies right from inception. It stirred up anxieties in the entire country. Ultimately, these memoranda were questioned in the Hon’ble Supreme Court of India. A nine-judge bench of the Supreme Court heard the matter, and \textit{Indra Sawhney} became the litigant in the matter. Though the principle of the Creamy layer was evolved in this case to mitigate and rationalize a just outcome, the executive rendered it practically useless by setting the creamy layer requirement in such a fashion that practically there was no real effect on the introduction of the creamy layer concept.

\noi
The cost of success of the middle class has remained too high and proves to be very costly every day.

\heading{Constitution from the Perspective of the Rich}

\noi
From the perspective of the Rich class, it was a mixed bag in terms of how the result went. Initially from the case of \textit{Kameshwar Singh v. the State of Bihar, AIR 1951 Pat. 91,} where the Patna high court gave a judgment in favour of the rich landlord interpreting that the fundamental rights are at a higher pedestal than directive principles of state policy, essentially taking a very literal approach that directive principles are not justiciable but the fundamental rights are, the government of India grew very restive and to get over the high court judgment and to ensure few other judgments of various high courts, introduced the first amendment to the constitution of India where the laws for land reforms are attempted to be made immune to the attack on the grounds of violation of fundamental rights.

\noi
The fourth amendment to the Constitution of India was essentially made again to acquire land more specifically under the land reforms initiative and to not have the state burdened with the just compensation that the courts have declared as a right of the land owner. The change in the perspective came by during the case of Golaknath v. State of Punjab AIR 1967 SC 1643 where the court again corrected the position and introduced the doctrine of prospective overruling yet the concept and the compensation of the land acquisition remained as per the previous judgments that it must be some compensation not necessarily with a prominent or strong nexus to the current market value.

\noi
By the 1970s, all the judgments of the Supreme Court wherein major issues were challenged came directly from the rich class as they were the ones who could afford comfortably to litigate the matter to the Supreme Court. Until about this time, the Supreme Court was giving literal interpretations to the Part III enforceability and Part IV non-enforceability, and then came the era where the Supreme Court started harmoniously constructing the Directive Principles of State Policy along with the Fundamental Rights, this led to the directive principles being elevated and it was during this stage that gradual corruption in the governance of the country became rampant and then the judgments of the Supreme Court even though were favoring the public good more towards the benefit of the poor were sidelined while implementing them due to the stronghold the rich class had in the government and corruption had aided them and shielded them from the adverse effects which could have affected their property.

\noi
Essentially all of the litigation related to property was fought by rich litigants. The landmark case of Kesavananda Bharati v. State of Kerala AIR 1973 SC 1461, though has the name of the seer from Kerala as the prominent name, what remains a fact is that most of the luminaries and legal experts in the field of law like Nani A Palkhivala, Fali S Nariman, etc, have appeared for the rich clients like Tata Group to protect their interests. The motivations of the advocates are not the focus here in this comment, but what is evident is the fact that it was only the rich litigants and rich litigation which shaped considerable portions of the rights regime in India.

\noi
The right to travel abroad or move about freely was held to be a fundamental right covered under Article 21 of the Constitution of India, in the case of Menaka Gandhi v. Union of India. Here again, the litigant was a popular person and belonged to the rich class.

\noi
As one can see as far as the rich class is concerned they were able to protect their rights under the regime of our constitution in a more proper fashion, even though in the initial stages they faced hurdles in civil matters. Later even though the judgments of the Supreme Court went against their interests, they were able to contort it in their favour by mitigating the effects through the management of the executive.

\noi
On the criminal side, the rich always had the best representation, and at least until recently, none of the rich class had ever received neither a punishment of death nor any proper life term. Indira Gandhi avoided any criminal law repercussions of her electoral violations, the then Senior Advocate R.K. Anand got away with a lenient sentence of reprimand and loss of seniority designation when caught in the sting operation with clear evidence of witness tampering.

\noi
With the explosive growth of corruption and the rich class taking the whole country for a ride, the Supreme Court in the latest times has given some landmark decisions (albeit interim) where the licenses of 2G spectrum were canceled. People involved even though rich and powerful have remained in jail even though for an interim period. Yet it is to be seen as to how it turns out as the executive machinery is in cahoots with the rich class and there is just so much a court can do without proper investigation and evidence.

\heading{Conclusion}

\noi
Despite all the criticism of the Constitution of India, during the various changes which have occurred all along in the past 65 years, the primary purpose of the Constitution of India is to keep intact the experiment called India which is being called a country with absolutely no single factor in its favor in the traditional sense of a country has survived and remains one. This can be attributed to the Constitution and the Institutions under the Constitution alone. The efficiency and the amount of good or development which could have occurred had the institutions worked efficiently would be debatable but what cannot be questioned is the result that India remains a single country.



\end{multicols}
\label{end2016-art4}
