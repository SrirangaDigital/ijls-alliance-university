\setcounter{figure}{0}
\setcounter{table}{0}
\setcounter{footnote}{0}

\articletitle{Judicial Review of Legislative actions: A study with reference to the principles\\ formulated by the Indian Judiciary in the context of Judicial Review\\ of Amending Power of the Parliament}\label{2016-art2}
\vspace{-.3cm}
\articleauthor{Dr.~Sanjay Bang\footnote{The Author is Reader of Law at Lal Bahadur Shastri National Academy of Administration, Mussoorie,
Uttarakhand. The Author can be reached at sanjaysbang27@gmail.com}}
\lhead[\textit{\textsf{Dr. Sanjay Bang}}]{}
\rhead[]{\textit{\textsf{Judicial Review of Legislative actions: A study with reference....}}}



\begin{multicols}{2}

\noi
\textit{“I am of the view that if there is one feature of our Constitution which, more than any
other is basic and fundamental to the maintenance of democracy and the rule of law,
it is the power of judicial review and it is unquestionably, to my mind, part of the basic
structure of the Constitution”- Justice Bhagwati}

\heading{Introduction}

\noi
A legal interpretation of constitutions of countries implies the supreme law of a nation,
an underlying normative document that is the source of all other secondary normative
documents i.e.., statutes, delegated legislation, and ordinances. When these secondary
norms do not fit in with the constitution, the courts or constitutional tribunals are
compelled to declare such laws unconstitutional. Judicial or constitutional review
relates to this procedure.\footnote{ The Blackwell Encyclopedia of Political Institutions, (1987), page 142.}
Judicial Review, therefore, is one facet of the power
conferred to the judicial bodies in a nation, which is applied by judicial bodies to
establish the legitimacy of a law or an action of the state or any agency thereof. In the
legal framework of modern democratic countries, it has very wide implications. The
judiciary performs a critical role as guardian of the values enshrined in the constitution
that the founding fathers have given to the people. They attempt to negate the harm
that is caused through actions of the legislature or executive. They also make an effort
to offer to every citizen whatever is guaranteed through the Constitution. All the
aforesaid becomes possible due to the power vested in the judiciary in form of judicial
review.

\noi
The Republic of India is blessed to receive a constitution which provides for
fundamental rights and for the safeguarding which a judiciary has been established which is independent and acts as custodian of the constitution and guardian of the
liberties of the people and hence provides ammunition against any force of
authoritarianism. In a pure form of democracy, provision for a courageous, objective,
and impartial judiciary is vital and therefore its importance of it cannot be overestimated. Judicial review is an outcome of two of the most basic aspects of the
Constitution of India. The first, two-tier system of legislation mandated: the
constitution which functions as the Supreme law of the land, and law as in ordinary
legislation, the validity of which is dependent on conformity with the constitution. The
other is the division of the legislative, executive, and judicial functions of the state.
gaining powers from the constitution, the legislatures in India enact statutes. There is a
two-prong limitation on the legitimacy of any statute, i.e., competency of the
legislature and conformity of the constitution. 

\heading{Meaning of Judicial Review:-}

\noi
The term ‘review’ implies any action of inspecting or examining anything to
correct or improve the same. This demonstrates that anything previously done by any
person whose correction or improvement is envisaged in the meaning of ‘review’.
‘review’ in ‘judicial review’ implies an action by a competent judicial body to assess
the validity or accuracy of an action of an agency. Thus, the authority of the Judiciary
to review and establish the validity of any normative law or any executive order may
be defined as the power of “Judicial review”. It implies that the constitution is the
\textit{‘supreme law of the land’} and any law not in conformity therewith shall be void.
‘Judicial Review’ legislation or executive action can be defined as “Judicial review is
the ultimate power of any court to declare any act of legislatures or executives as
unconstitutional and hence unenforceable as a) any law. b) Any official action based
upon the law and c) any other action by a public official that it deems to conflict with
the constitution.”

\noi
In \textit{L. Chandra Kumar vs. Union of India}\footnote{
(AIR 1997 SC 1125)}
, the Supreme Court of India held that
“Henry J. Abraham’s definition of judicial review in the American constitution is,
subject to a few modifications, equally applicable to the concept as it is understood in
Indian constitutional law. Broadly speaking judicial review in India comprises three aspects. Judicial review of legislative action, judicial review of judicial decisions and
judicial review of administrative action.”

\noi
Thus, Judicial Review is defined as \textit{‘the power of the court to hold
unconstitutional any law or official action that it deems to conflict with the basic law
or the Constitution.’}\footnote{Ferguson \& McHenry: The American Federal Government, Ed. X,, p. 12} Professor Henry J Abraham describes Judicial Review as “the
power of any court to hold unconstitutional and hence unenforceable any law, any
official action based upon it, any illegal action by a public official that it deems to
conflict with the Basic Law.”\footnote{ Abraham: ‘The Judicial Process’, page 251}

\heading{Theoretical perspectives of amending power and interpretation:-}

\noi
In an ordinary sense word ‘amend’ means ‘to improve’. However, when it is
used in the context of a constitutional amendment, its meaning becomes quite technical
as well comprehensive. Thus studied, the phrase ‘constitutional amendment’ covers
anything relating to modification, addition, or subtraction of any part that affects the
body of the fundamental law of the land. The term ‘amendment of the constitution has
both wider and narrower implications; it has technical as well as ordinary senses also.
It is a different matter that its technical or particular meaning is invoked while dealing
with the issue of changing a constitution. Thus, the constitutional amendment means
making any improvement in or removing any defect of the fundamental law of the land,
even if it amounts to the very abrogation of its real form, then a question arises as to
whether there should be some limitations, express or implied, on the power of amending
it or not.

\noi
As in the case of ordinary legislation in the case of constitutional amendments also
depends upon the view taken by the court in interpreting the legislation, and in
interpreting the amendment the courts are guided by certain well-known principles.
These principles are the principles of interpretation that are relevant not only to the
exercise of judicial power in determining the rights and duties of individuals but also
concerning the powers and procedures of public authorities in the matter of making the
laws. Further, some principles are such that they have their origin in foreign
jurisdictions and they are followed by the courts if they suit our conditions. In some
cases, the courts have refused to follow the principles of the foreign jurisdiction.

\heading{The Principles followed by the Courts in the United States of America
and India regarding Judicial review of amending power}

\vspace{-.4cm}

\subsection*{I. Principles formulated by the American Supreme Court:-}

\noi
As described by Chief Justice Marshall, the founder of the Judicial Review in the United
States, in the leading case of in \textit{Marbury v. Madison}\footnote{5 US(Cranch)137(1803)}
, “It is emphatically the province
and duty of the judicial department to say what the law is … If then, the courts are to
regard the Constitution, the Constitution is superior to any ordinary act of the
legislature; the Constitution and not such ordinary act must govern the case to which
they both apply.”

\noi
Thus, the Judicial Review is a creation of the American Supreme Court at the hands of
Chief Justice Marshall. Ever since 1803, the ‘American judiciary has made use of this
power in several leading cases that have gone to lay down the following principles of
judicial review:

\newpage

\begin{enumerate}
\item Before the court will glance at the particular issue or a dispute, a definite ‘case’ or
‘controversy’ of law or inequity between bona fide adversaries under the
Constitution must exist involving the protection or enforcement of valuable legal
rights, or the punishment, prevention, or redress of wrongs directly concerning the
party or parties bringing the judicial suit.

\item The party or parties bringing a suit must have a ‘standing’.

\item The court does not render advisory opinions.

\item The court will not entertain generalities; it will deal with specific and particular
issues;

\item The party bringing a suit must be a sufferer and not a gainer by the challenged
statute;

\item All other remedies must have been exhausted before coming to the Supreme Court;

\item The question under study must be a substantial and not of a trivial nature;

\item The question of fact, as distinct from a question of law, is not normally accepted as a proper basis for the exercise of judicial review;

\item The Court may change its views from time to time;

\item The Court will not entertain political controversies; it will be concerned with legal
aspects alone;

\item  The Court will begin with the presumption that the statute under challenge is valid;

\item The Court will not ordinarily impute illegal motives to the lawmakers;

\item The Court may declare the whole law or its part as invalid;

\item The Court is not to be used as a check against inept, unwise an unrepresentative
legislation;

\item If the court finds that it must hold a law unconstitutional; it will usually try hard
to confine the holding to that particular section of the statute which is successfully
challenged on constitutional grounds;

\vspace{-.4cm}

\end{enumerate}

\vspace{-.4cm}

\subsection*{II. Principles formulated by the courts in India}

\noi
A study of the cases decided by the Supreme Court and the High Courts of India shows
that the following principles are inferable from the exercise of power by them in our
country in the context of Judicial Review of the amending power. These principles have
been invoked by the Courts even concerning matters of examining the ordinary
legislation apart from the amendments made to the Constitution and this is because of
the sound logic involved in the approach of the Court.

\noi
{\bf 1)~~Principle of Legislative Competence: }\\[.2cm] This principle is also called the Principle of Pith and Substance. This principle
implies that there is in India a written constitution that defines in detail the powers of
the Legislature; the Judiciary is therefore authorized to see whether the impugned
legislation falls within the scope of the powers of the law-making body or not. It follows
that the Courts in India have the power to pronounce upon the validity of a law on the
ground of the excess of legislative powers. While interpreting the Constitution the
Courts may look into the matter of whether such and such powers fall within the
legislative competence of the Union or state legislature either by the express or even by
that of an implied version of the Constitution. 

\noi
In \textit{Chaturbhai M. Patel v.~Union of India},\footnote{ AIR 1970 SC 424}
the Supreme Court held that the power to
legislate on any subject carries with it the power to legislate on an ancillary matter
which can be said to be reasonably included in the power given. It is within the
competence of the Parliament to provide for matters which may otherwise fall within the competence of the State legislature if they are necessarily incidental to a subject of
legislation, expressly within its powers.

\newpage

\noi
{\bf 2) The Principle of Severability:}\\[.2cm] While interpreting a law challenged before the Court, the Courts see whether the law
as a whole or any of its parts is unconstitutional. If the Court finds that the impugned
law, as a whole, is bad, it can declare it ultra vires of the Constitution; it can declare so
in the case of a part only and allow the rest as an operative. The crux of the matter
depends on whether a valid part can be separated from the invalid part. The Court will
decide such a matter on the construction of the provisions of the law in question.

\noi
from the above-mentioned cases and other such cases decided by the Hon’ble Supreme
Court of India When a law is partly valid, the valid part of the law will remain
enforceable and the part which is declared to be invalid can be severed from the other.
It is wouldn’t matter here for the implications of such rule if the invalidity of the law is
the result of the subject matter being outside legislative competence or for the reason
of breach of fundamental rights.\footnote{G. N. Joshi: Aspects of Indian Constitutional Law, p. 218}

\noi
{\bf The Principle of Progressive Interpretation: }\\[0.2cm] One of the important questions arising before the reviewing court is whether the
provisions of the Constitution should be understood in the light of the conditions that
existed at the time of the making of the Constitution or that they should be given a
broader construction from time to time to include newer circumstances arising with the
development of social and economic life of the people. Fortunately, the Judiciary in
India has been guided by the principle of progressive interpretation and therefore the
learned judges have as far as possible avoided, what Sir Maurice Gwyer said about his
Federal Court, “the spirit of formal or barren legislation”. It is due to this that the
Supreme Court does not bother much about what the members said in the Constituent
Assembly on this or that occasion. 

\noi
Shortly after its creation, the Supreme Court in the case of the \textit{State of Travancore
Cochin and others v. The Bombay Company Ltd}\footnote{1952 AIR 366,1952 SCR 1112}
observed that the speeches made in
the Constituent Assembly ‘cannot be used as aids for Constitution’. The same view was reiterated in the \textit{Bela Bennerji case}\footnote{Smt. Bella Banerjee v. the State of West Bengal, 1954 SCR 558}. Similarly in the case of the \textit{Senior Electric
Inspector v. Laxmi Narayan Chopra}\footnote{1962AIR 159,1962 SCR (3) 146} the Supreme Court observed that unless a
contrary intention appears, an interpretation shall be given to the words used taking into
view facts and situations if the words are capable of comprehending them.

\noi
{\bf Principle of Dynamism}\\[0.2cm] The principle above stated is supplemented by the principle of Dynamism, which has
the meaning that the exercise of the power of judicial review in our country is not bound
by the doctrine of stare decisis. That is, the courts are not bound by their earlier
affirmations. They may change their views or precedents from time to time while
making a dynamic interpretation of the provisions of the Constitution. In other words,
they pay heed to the judicious warning of Justice Hughes of the American Supreme
Court that one must not expect from this Court the icy stratosphere of uncertainty. The
Supreme Court of India has taken this important view that the interpretation of law
should be one in a dynamic way so that it may reconsider its previous rulings and also
depart from them if so necessary as there is nothing in the Constitution binding it to the
principle of stare decisis.\footnote{Bengal Immunity Company v. The State of Bihar, AIR 1955 SC 661.}

\noi
{\bf 5. The principle of primacy of the letter of the\\ Constitution:}\\[0.2cm]

\vspace{-.6cm}

\noi
The principle had its special connection with Article 13 (2) saying that no law can
prevail in the country that contravenes the provisions of Part III of the Constitution
dealing with our Fundamental Rights. It is very clearly laid down in Article 13 (2) that
such a law shall be void to the extent of being inconsistent with the Fundamental Rights
of our Constitution. The Fundamental Rights guaranteed by the Constitution are not of
an absolute character. They may be subjected to ‘reasonable restrictions in the interest
of morality, order, public health, national security and the like. The legislature is,
therefore empowered to lay down restrictions on the enjoyment of fundamental rights.
However, it is the function of the Courts to examine whether the restrictions imposed
by law are inconsistent with the tenor of the provisions of Part III or not. It is under this
special arrangement that the Supreme Court has invalidated several important laws
made by the Union and State legislatures concerning Right to Property, in particular, and that has since become the main source of confrontation between the executive-cumlegislative and judicial departments.\footnote{Smt.Bella Bannerji and others v. The State of West Bengal, 1954 SCR 558.}

\noi
In India, it is the Constitution that is supreme and the Parliament, as well as State
Legislature, must not only act within the limits of their respective legislative spheres as
demarcated in three lists occurring in the Seventh Schedule of the Constitution, but Part
III of the Constitution guarantees to the citizens' certain fundamental rights which the
legislative authority can on no account transgress. A statute law to be a valid must in
all cases conform with the constitutional requirements and it is for the judiciary to
decide whether any enactment is constitutional or not.

\noi
{\bf 6. Principle of Primacy of Spirit of the Constitution}\\[0.2cm] Since our Constitution is a written document it is expected that the courts will be guided
by the words contained herein and generally not try to go into the spirit. In the Gopalan
case,\footnote{A.K. Gopalan v. State of Madras AIR 1950 SC 27} Chief Justice Kania observed that “the courts are not at liberty to declare an Act
void because in their opinion it is opposed to the spirit supposed to pervade the
Constitution but not expressed in words”. Though the view of Justice Kania has been
sustained by the learned Judges in several cases it is seen that in several instances the
courts have given significance to the spirit of the Constitution. It is for this reason that
the courts have invalidated ‘colourable legislation’ they have held ‘ancillary legislation’
as good. The enunciation of the principle of the basic structure of the Constitution is a
pointer here.

\noi
The principle of ‘the spirit of the Constitution’ finds its place in the efforts of the Court
to read into and between the lines. The plausible conclusion in this regard should be
that though the Courts cannot go against the letter of the Constitution, they can renounce
this approach if the written rules of the Constitution are silent on a particular issue.

\noi
{\bf 7. Principle of Prospective Over-ruling}\\[0.2cm] 

\vspace{-.6cm}

\noi
The \textit{Golak Nath vs.~State of Punjab}\footnote{ AIR 1967 SC 1461} holding has an importance of its own in laying
down the principle of prospective over-ruling more or less on the lines of American
constitutional jurisprudence. It envisages that a law declared invalid by the Court may not necessarily affect transactions and vested rights before but may operate only for
transactions and rights arising after the judicial invalidation. It is based on the premise
that rather than disturbing the past transactions, the new view of the interpretation of
law adopted by the courts is to make the law effective as regards future transactions
only. Thus, the Court ruled that while previous constitutional amendments amounting
to the curtailment of fundamental rights were bad, they could not be made inoperative
as much work had already been done by the State. But in the future no such amendment
could be made.

\noi
\textit{“While ordinarily a Court will be reluctant to reverse its previous decisions it is its duty
in the constitutional field to correct itself as early as possible, for otherwise the future
progress of the country and happiness of the people will be at stake. As it was clear that
the decision in Shankari Prasad's case was wrong, it was pre-eminently a typical case
where this Court should overrule it. The longer it held the field the greater the scope
for erosion of fundamental rights. As it contained the seeds of destruction of the
cherished rights of the people, the sooner it was overruled the better for the country.}\footnote{I. C. Golaknath \& Ors vs State of Punjab \& Anrs 1967 AIR 1643}”
But the court also takes precaution in enforcing the judgment with retroactivity because
of the implications involved in the case.\footnote{The Superintendent and Legal Remembrancer Stale of West Bengal v.The Corporation at Calcutta,
[1967] 2 S.C.R., 170}

\noi
This is what had happened in the Golak Nath case. The Constitution (Seventeenth
Amendment) Act, 1964, since it took away or abridged fundamental rights was beyond
'the amending power of Parliament and void because of contravention of Art. 13(2).
But having regard to the history of this and earlier amendments to the Constitution,
their effect on the social and economic affairs of the country, and the chaotic situation
that might be brought about by the sudden withdrawal at this stage of the amendments
from the Constitution it was undesirable to give retroactivity to this decision. The Golak
Nath case was therefore a fit case for the application of the doctrine of "prospective.
Overruling, evolved by the courts in the United States of America.\footnote{ [805 E; 807 E, G; 808 C-D] Great Northern Railway v. Sunburst Oil \& Ref. Co. (1932) 287 U.S. 358:
77 L. Ed. 360, Chicot County Drainage v. Baxter State Bank, (1940) 308 U.S. 371, Griffin v. Illionis, (1956) 351 U.S. 12, Wolf v. Colorado, 338 U.S. 25 : 193 L. Ed. 872, Mapp v. Ohio, 367 U.S. 643 : 6 L.
Ed. (2nd Edn.) 1081 and Link letter v. Walker, (1965) 381 U.S. 618, }


\noi
The doctrine of “prospective overruling" is a modern doctrine suitable for a fast-moving
society. It does not do away with the doctrine of \textit{stare decisis} but confines it to past
transactions. While in Strict theory it may be said that the doctrine 'involves the making
of law, what the court does is to declare the law but refuses to give retroactivity to it. It
is a pragmatic solution reconciling the two conflicting doctrines, namely, that a court
finds the law and that it does make law It finds law but restricts its operation to the
future. It enables the court to bring about a smooth transition by correcting, its errors
without disturbing the impact of those errors on past transactions. By the application of
this doctrine, the past may be preserved and the future protected. 


\noi
Our Constitution does not expressly or by necessary implication speak against the
doctrine of prospective overruling. Articles 32, 141, and 142 are designedly made
comprehensive to enable the Supreme Court to declare the law and to give such
directions or pass such orders as are necessary to do complete justice. The expression
'declared' in Art. 141 are wider than the words 'found or made'. The law declared by the
Supreme Court is the law of the land. If so, there is no acceptable reason why the Court,
in declaring the law in supersession of the law declared by it earlier, could not restrict
the operation of the law as declared to the future and save the transactions whether
statutory or otherwise that were affected based on the earlier law.

\noi
{\bf 8. Principle of Empirical Adjudication}\\[0.2cm]
This principle implies that while exercising the power of judicial review; the courts do
not deal with hypothetical situations or cases. The matter brought before a court must
be ‘concrete’ so that it may not be required to indulge in abstract principles. The Court
seeks to confine its decision, as far as may be reasonably practical, within the narrow
limits of the controversy between the concerned parties in a particular case.\footnote{Sukhdev v. Bhagatram AIR 1975 SC 1231}

\noi
{\bf 9. The Principle of Indirect Judicial Review}\\[0.2cm] Sometimes a peculiar situation may arise when an impugned provision of law is capable
of two possible interpretations. In one case it may be taken as good, in another as bad.
In such a situation, the Court will take a positive constructive and not a negative or
destructive view. Here, the Court so tries to interpret the law that its validity is sustained
in a way as far as possible. In a case, instead of holding the impugned law invalid under
article 14 on the ground of lack of procedural safeguards the Supreme Court read
natural justice into the law and sustained its validity. It quashed the order made
thereunder because of the denial of natural justice for the petitioners.\footnote{ State of Mysore v. Bhat AIR 1975 SC 596}

\noi
{\bf 10. The presumption in favour of the\\ constitutionality}\\[0.2cm] When the constitutional invalidity of any law is challenged, the Court will not hold it
to be ultra vires unless the invalidity is clear beyond all doubts, for there is always a
presumption in favour of its validity. The court will begin with this presumption that
the legislature does not exceed its powers, nor does it make any law that is inconsistent
with the letter and spirit of the Constitution. While examining the constitutionality of
a statute it must be assumed that the legislature understands and appreciates the need
of the people and the laws it enacts are directed to the problems which are manifest by
the experience and that the elected representatives assembled in a legislature enact laws
which they consider to be reasonable for the purpose for which they are enacted.

\noi
In various cases like those of \textit{Chiranjitlal Chaudhary.vs Union of India}\footnote{AIR 1950 SC 41}
, \textit{F. N. Bulsara
v. the State of Bombay}\footnote{ 1951 AIR 318, 1951 SCR 682} and \textit{Hamdard Dawakhana v. the Union of India}\footnote{AIR 1960 SC 554.} the Supreme
Court has reiterated its view that the presumption is always in favour of the
constitutionality of an enactment.

\noi
{\bf 11. Non Application of Foreign Principles :}

\noi
The Courts in our country are not bound to follow foreign precedents while exercising
the power of judicial review. For instance, the functioning of the parliamentary
government at the Centre cannot be bound by the principles of English constitutional
law.24\footnote{ U. N. R. Rao v. Indira Gandhi, AIR 1971 SC 1002}

\noi
In the Gopalan Case, the Supreme Court refused to read the American concept of ‘Due
Process of Law’ into the ‘procedure established by law’ as given in Article 21 on the
ground that when the same words are not used, it will be against the ordinary canons of
construction to interpret a provision in our Constitution following the interpretation put
on a somewhat analogous provision in the Constitution of another country where not
only the language is different but the entire political conditions and constitutional set
up are dissimilar.”

\noi
{\bf 12. The Principle of Locus Standi}

\noi
This is a principle regarding which the law in India has made a lot of strides. In quite a
good number of cases involving the administrative action as well as the legislative
action of the State, the question of \textit{locus-standi} has been debated in the cases and the
general view taken by the Courts in this regard may be highlighted thus:

\noi
The principles and practices of the English, the American and the foreign courts have
had their influence on the Indian Courts, which is an important feature of the system of
Judicial Review as it exists in India.

\noi
In the famous Transfer of Judges Case certain Advocates had challenged the action of
the Government of India in the matter of the appointment of Judges and Transfer of
Judges. In this case, the question was whether the petitioners who are advocates can file
these petitions for the reliefs mentioned therein under Article 226 or Article 32 of the
Constitution has got to be considered. The contention is that members of the Bar who
are not personally affected by the circular letter of the Law Minister, by the appointment
of certain additional Judges for short-terms of three months or six months, by the nonappointment of any of the additional Judges after the expiry of the tenure fixed under
Article 224(1) or by the non-appointment of sufficient number of Judges of the High
Courts or by the transfer of some Judges have no locus-standi to file these petitions. It
is contended that neither qualitatively nor quantitatively these petitioners have
sufficient interest to prosecute these petitions the result of which would not affect them
either directly or even indirectly.

\noi
The Supreme Court in its preliminary remarks observed that: \textit{“the attitudes of the
courts on the question of locus standi do not appear to be uniform. They vary from
country to country, court to court, and case to case. Sometimes the tests applied by
courts also vary depending upon the nature of the relief sought. In some cases, courts have taken a very narrow view on this question holding that unless an applicant has
either personal or fiduciary interest in the result of the application, no relief can be
granted on his application even though it may appear that the impugned action or
omission of the administrative authority concerned is not in accordance with law. The
other extreme view is that the courts may in their discretion Issue mandamus to an
administrative authority at the instance of any member of the public. A close scrutiny
of the authorities and texts cited before the Judges showed that neither of the two
extreme views is accepted as correct in majority of the cases. It is also seen that in many
of them the courts have found some sort of special interest in the applicant who
distinguishes him from the general public before granting the relief prayed for by him.
A person who has a genuine grievance on account of an action which affects him
prejudicially is ordinarily considered to be eligible to move the Court.”}

\noi
{\bf 13. The Theory of Basic Structure}

\noi
The constitutional validity of the Twenty Fourth and Twenty-Fifth amendments was
challenged before a full bench of the Supreme Court (thirteen judges). The Apex Court
overruled its previous decision given in Golaknath by a majority of 7:6 and agreed that
Parliament can amend any part of the constitution including the fundamental rights.
But it imposed a theory known as “Basic structure theory”. According to this theory,
Parliament can amend any part of the constitution including the fundamental rights,
but it cannot destroy the basic structure of the constitution, now what would constitute
the basic structure was not defined by the Supreme Court. Nevertheless, the concept
of 'basic structure' of the Constitution gained recognition in the majority verdict. All
judges upheld the validity of the Twenty-fourth amendment saying that Parliament had
the power to amend any or all provisions of the Constitution. All signatories to the
summary held that the \textit{Golak Nath} case had been decided wrongly and that Article 368
contained both the power and the procedure for amending the constitution.

\noi
\heading{The minority view}

\noi
The minority view delivered by Justice A.N. Ray (whose appointment to the position
of Chief Justice over and above the heads of three senior judges, soon after the
pronunciation of the \textit{Kesavananda} verdict, was widely considered to be politically motivated), Justice M.H. Beg, Justice K.K. Mathew and Justice S.N. Dwivedi also
agreed that \textit{Golaknath} had been decided wrongly. They upheld the validity of all three
amendments challenged before the court. Ray, J. held that all parts of the
Constitution were essential and no distinction could be made between its essential and
non-essential parts. All of them agreed that Parliament could make fundamental
changes in the Constitution by exercising its power under Article 368.

\noi
In summary, the majority verdict in \textit{Kesavananda Bharati} recognized the power
of Parliament to amend any or all provisions of the Constitution provided such an act
did not destroy its basic structure. But there was no unanimity of opinion about what
appoints to that basic structure. Though the Supreme Court very nearly returned to the
position of \textit{Sankari Prasad VS Union of India}\footnote{1951 AIR 458,1952 SCR 89} by restoring the supremacy of
Parliament's amending power, in effect, it strengthened the power of judicial review
much more.

\noi
\heading{Critical Evaluation of Judicial review}

\noi
Judicial review since its development in the famous case of Marbury v/s Marshal is
facing one or more challenges. Judicial review was considered undemocratic on the
ground, that it is giving too much power to the judiciary to review and limits the
Parliament from its amending power. Parliament is representing the will of the people
and the judiciary is not so. Hence the will of the people cannot be curtailed at the cost
of judicial review. Even in the wordings of Rajiv Dhawan “Kesavananda pushed the
judges into open politics”. According to Sunder Raman, “the decisions in Kesavananda
and Golaknaths cases were unfortunate, as they have greatly affected the contemporary
constitutional history of India and have brought the judiciary in confrontation with the
legislature”. But on the other hand, judicial review is protecting the supremacy of the
constitution and thereby democracy. The article published in Hindu on 23$^{\rm rd}$ April 2013
specified the \textit{Kesavananda} case as \textit{“the case that saved the Indian democracy”}.

\noi
\heading{Recommendations}

\noi
My recommendations on the two themes of the research are as follows:-

\begin{enumerate}

\item The amending power should be used sparingly to formulate a new policy
concerning the system of government. Where the amendment pertains to important
matters of the State Administration the limitations as laid down in the Constitution
and the Judgments of the Supreme Court must be observed. In no case, the Basic
Structure of the Constitution should be disturbed. The theory of Basic Structure has
received the approval of society; so much so the sanctity of the theory must be
maintained. Of course, the elements of Basic Structure must be formulated by a
proper discussion of the matter.

\vspace{-.1cm}

\item As far as the power of Judicial Review is concerned it may be stated that we may
arrive at two balanced points of conclusion. First, no part of the constitution should
be defined as unamendable and the job of discovering the area of limitations,
express or implied, should not be performed by the courts as if done provocatively.
It may be added that if the Parliament cannot alter or destroy the basic structure, so
the courts cannot. Since power resides with the people it is they who should remain
vigilance and see that the deputies elected by them in a fair and free manner work
within the lines drawn by the Constitution. As Prof. P. K. Tripathi says, “It will be
some irony if a Court so severely concerned with saving the ‘essential elements of
the basic structure of the Constitution should end up with destroying the most
essential and basic principle of constitutional law, namely, that the restrictions, if
any, on the power of the amendment of a sovereign constitution can be imposed
only by the Constituent Assembly or its nominee, the amending authority, both of
whom operate upon the Constitution and not by a court which operates under the
Constitution and be subject to it.\\ 

\vspace{-.3cm} 
  
It is necessary to pay respect to the theory that the Courts are there to declare the law
and not to make the law. Judicial Legislation should be avoided to the extent it is
possible; on the contrary, the judicial interpretation for the sake of giving a proper
interpretation to the provisions of the law should be adopted.



\end{enumerate}
\end{multicols}
\label{end2016-art2}
