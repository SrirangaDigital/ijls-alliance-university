\setcounter{figure}{0}
\setcounter{table}{0}
\setcounter{footnote}{0}

\articletitle{Doctor – Patient Relationship and Medico Legal System - A
Comparative Study With USA, UK, Australia and India}\label{2016-art6}

\vspace{-.5cm}

\articleauthor{Laxmish Rai \footnote{PhD Scholar, Alliance University, Bengaluru}}
\lhead[\textit{\textsf{Laxmish Rai}}]{}
\rhead[]{\textit{\textsf{Doctor – Patient Relationship....}}}



\begin{multicols}{2}

\heading{Introduction}

\noi
All living beings have been subject to disease since the very inception of life on Earth.
The same applies to human being as well, who have been fighting disease ever since
they began living together in groups. As such, tackling disease has been an art practiced
by humans for centuries and is one of the reasons for the survival of mankind today.
With the formation of civilizations, humans began to develop cultures practiced by
entire groups of people. The culture thus practiced, gave rise to values and principles,
which formed the core of societies. These values namely integrity, morality, honesty,
civic duty and respect played a vital role in the field of medicine, on which members of
the group depended, and brought in a form of accountability, where those who practiced
medicine were required to ensure that they carried out their duties and obligations to the
best of their knowledge and abilities.

\heading{Doctor-patient relationship}

\noi
The relationship between a medical practitioner and patient is central to the regime of
the medical profession. The doctor-patient relationship constitutes one of the most
essential elements of contemporary medical ethics. According to the US Court a doctorpatient relationship can be defined as “a consensual relationship in which the patient
knowingly seeks the physician’s assistance and in which the physician knowingly
accepts the person as a patient.”\footnote{ QT, Inc. v. Mayo Clinic Jacksonville, 2006 U.S. Dist. LEXIS 33668 (N.D. Ill. May 15, 2006).} This judgment is widely accepted as a standard definition of the relationship between a doctor and patient.\footnote{Fallon Chipidza, \textit{Impact of the Doctor-Patient Relationship,} Prim Care Companion for CNS Disorders (Oct.22,2015) \url{https://pubmed.ncbi.nlm.nih.gov/26835164/.}}

\noi
In the Indian context, there is no set definition for a doctor-patient relationship. An
article from the Indian Journal of Medical Ethics defines the doctor-patient relationship
as one which is “contractual in nature and is established when the patient makes a request for medical examination, opinion, advice or treatment and the doctor undertakes to provide these”.\footnote{ Sunil K. Pandya, \textit{Doctor-Patient relationship,} 3(2)IJME (1995).} However, the biggest difference in the Indian context is perhaps the fact that the doctor-patient.

\noi
relationship not only applies to the modern doctors practicing allopathy, but also to all
physicians irrespective of the type of medicine practiced. Therefore, even the
relationship between a physician practicing Ayurveda and a patient is considered as a
doctor-patient relationship.

\heading{Elements of a doctor-patient relationship}

\noi
Doctor-patient relationships are generally voluntary in nature where the medical
practitioner and patient agree to work together in providing medical treatment to the
injury or illness.\footnote{Alllaw.com, A Doctor's Duty of Care \url{(Www.alllaw.com) https://www.alllaw.com
/articles/nolo/medical- malpractice/duty-care.html.}} As a rule, medical practitioners are required to provide sufficient
information to their patients regarding their medical conditions and options of treatment
available. The doctor-patient relationship must have a social or cultural dimension is
evidenced by its immediate effect during emergencies in which doctor and patient have
never met before.

\subsection*{Duty of care}

\noi
A duty of care generally exists from the doctor towards the patient once the relationship
has been entered into. However, in some cases, a duty of care may exist even before a
relationship has been formed. The duty of care exists amongst doctors and patients both
according to medical ethics as well as in statute. Further clarity was brought by the
English Courts in Bolam v. Friern Hospital Trust\footnote{6 Bolam v Friern Hospital Management Committee, WLR 582(1957).} where the patient was not successful
in his action against the hospital, but the court did lay down important points for
consideration. The test for determining the standard of care was set out by Lord McNair
known as the Bolam test, makes an assessment of whether the medical professional
acted in a way that was in accordance with practices accepted as correct by a
responsible body of medical professionals with knowledge and experience in that specific area of medicine. If the treatment is seen to be acceptable, the standard of care
is seen as met and the doctor is thus, not held liable.

\subsection*{Informed Consent}

\noi
Consent is a term relevant in various areas such as commercial relationships, and also
as a defense to some criminal acts. However, in healthcare, law seeks to govern the way
in which consent has to be given. In some cases, a formal acknowledgement may be
required before the conduct of certain procedures.\footnote{ALASDAIR MACLEAN, AUTONOMY, INFORMED CONSENT AND MEDICAL LAW: A RELATIONAL CHALLENGE 158-61(1d ed.2009).} A general rule of medical law, which
is recognised internationally, is that patients cannot be given medical treatment without
consent. It is the choice of the patient to undergo treatment and the doctor only plays
the role of an advisor. The doctor must respect the choice of the patient even if the
reasons given by the patient for refusing consent seem to be absurd or misplaced. The
case of Regina v Tabassum concerned a man portraying himself as a medical practitioner
with expertise in ailments related to women’s breasts. Such portrayal led to women
consenting to having their breasts examined by the man. When the truth of his
misrepresentation was revealed, the women brought legal proceedings against him. The
man pleaded that he examined the women’s breasts only after obtaining their consent.
The Court held that the agreement of the women for examination was based on
misrepresentation and thus does not qualify as ‘consent’.\footnote{Regina v Tabassum, 328 (2000).}

\subsection*{Confidentiality}

\noi
In the case of Ashworth Security Hospital v. MGN,\footnote{Ashworth Security Hospital v. MGN [2000] 1 WLR 515 (2000).} it was held that, patient visiting a hospital for the purpose of treatment holds the right to be confident that the information
pertaining to his health and medical condition remain in strict confidence between the
patient and the medical practitioner/s attending to him or her. This is not just a legal
obligation, but also part of the Hippocratic Oath which states explicitly that “whatsoever
things a doctor may see or hear concerning the life of their patients or even a part there
from, which ought not to be noised abroad, shall be treated as sacred secrets.”\footnote{Ashworth Security Hospital v. MGN [2000] 1 WLR 515 (2000).}

\subsection*{Patient Autonomy}

\noi
In the case of Montgomery, a diabetic mother was not informed by her doctors of the
risk of shoulder dystocia during the time of her labour. As a result, when the baby was
born, it suffered from shoulder dystocia due to the large size of the baby. The mother
then brought a case against the doctors for having failed to inform her of the risks and
also for not advising her of undergoing a caesarean section mode of delivery, which
was possible and advisable in her case. The Court after examining the case held the
doctors liable for failing to inform the mother of the

\noi
risks in relation to normal delivery of her child. Lady Hale, in her concurring opinion,
quoted that “it is now well recognized that the interest which the law of negligence
protects is a person’s interest in their own physical and psychiatric integrity, an
important feature of which is their autonomy, their freedom to decide what shall and
shall not be done with their body.”\footnote{Montgomery v. Lanarkshire Health Board 1430 U.K.108 ( 2015).} This case emphasizes the importance of the autonomy of the patient and how it must be observed and respected by medical professionals.

\heading{Theoretical Models of doctor – patient relationship}

\noi
The doctor-patient relationship is of great interest to many researchers not only in the
medical field, but also in the fields of sociology, economics, law etc. Different thinkers
look at the relationship between a doctor and patient with respect to the area of study.

\subsection*{Principal-Agent Relationship Model}

\noi
This model considers the doctor as an agent who acts according to the needs of his
principal, the patient. The doctor possesses relevant information about the patient health
and available medical treatment; and the patient has information about the suitability of
the treatment to his lifestyle. The doctor, as the agent, is supposed to work in the ‘best
interest’ of his principal. Another factor in such principal-agent relationships is that due
to the monetary consideration paid by the patient, he also has the power to decide on the
treatment to be undertaken.\footnote{K. Arrow, ‘Uncertainty and the welfare economics of medical care’, (1963) 53(5), AmericanEconomic Review, pp.941-973.}

\subsection*{Paternalism Model}

\noi
The doctor patient relationship in sociology is looked at from the aspect of power of
decision making. In this model, the patient plays a passive role and the doctor is the
decision maker. The doctor makes a diagnosis and provides treatment to the
patient.\footnote{C. Charles and others, ‘Decision making in the physician-patient encounter: revisiting the shared treatment decision-making model’, [1999] 49 Social Science and Medicine,pp.651-661.} This model enables the doctor to utilise his skills in order to give the most effective treatment which is likely to reinstate the patient's health. The patient is only given information which facilitates him to consent to the doctor's decisions.\footnote{\textit{Supra} at note 8.}

\subsection*{Shared Decision Model}

\noi
In this model, both parties are actively involved in the decision-making process. There
have four elements present, viz.,

\begin{itemize}
\item Both the doctor and patient have to be involved in the decision making

\item Information sharing by both parties

\item Both express treatment preferences.

\item A treatment is decided upon once both parties agree on the treatment.\footnote{Supra at note 13.}
\end{itemize}

\noi
The shared decision model treats doctors and patient as equals and enables better
communication.

\subsection*{Informed decision-making model}

\noi
In this model, there is a supply of information from the doctor to the patient but not an
exchange from both parties.\footnote{Steven H Woolf and others, ‘Promoting Informed Choice: Transforming Health Care to Dispense Knowledge for Decision Making’, [2005] 143(4) Annals of Internal Medicine.} Furthermore, the decision is taken solely by the patient after being provided all the information by the doctor.

\subsection*{Doctor – Patient relationship and medico-legal systems in the developed States}

\noi
Developed countries in the West follow a stringent regime of medico-legal ethics and adopt stricter enforcement mechanisms. Medico-legal legislation has existed in such States for decades and as a result, the legal regime around doctor-patient relationship has evolved in these countries over the years. Countries such as the US and UK have thriving systems to regulate the relationship between doctors and patients and thus, they see better healthcare being offered to their citizens.

\subsection*{Doctor – Patient relationship and medico-legal systems USA}

\noi
For decades, the US followed a system of paying medical fees to doctors on a fee-forservice (FFS) basis.\footnote{“History of health insurance benefits,” Employee Benefit Research Institute,March 2002, \textit{available at}  \url{<http://www.ebri.org/publications/facts/index.cfm?fa=0302fact>}} The relationship between family doctors and their patients was a close and confidential one. The system of medical insurance in the US has led to a surge in the number of patients visiting doctors as medical fees are paid by someone else. This has drastically changed the nature of the doctor-patient relationship in the US. The American Medical Association (AMA) is the primary medical body in the US that takes decisions and makes recommendations regarding the implementation of a code of ethics for the medical profession. Therefore, the AMA has taken several measures in the past to improve the doctor-patient relationship.

\noi
The AMA has drafted an ethical code for physicians in the US known as the Code of Medical Ethics Opinion.\footnote{AMA, \textit{Code of Medical Ethics Opinions,} (ama-assn.org)  \url{https://www.ama-assn.org/deliveringcare/ethics/code- medical-ethics-overview.}} The first chapter of the code is dedicated to the ethics of physician-patient relationships. The code stresses that a physician-patient relationship comes into existence when a physician caters to the medical needs of a patient and is consensual in nature.\footnote{\textit{Id.} 1.1.1} The US has enacted strict legislation to ensure good relationships between doctors and patients. As in the case of most other countries, doctors in the US are required to pursue and possess an approved degree and the license to practice treating patients.

\subsection*{Doctor – Patient relationship and medico-legal systems UK}

\noi
The UK is known for having one of the best medical services in the world, mostly due to its famous National Health Services (NHS). The relationship between patients who were of a wealthy class and physicians was that of a master and a servant, with the distinction that the physician was treated as a higher class of servant due to his medical
expertise.\footnote{ROY PORTER, THE GREATEST BENEFIT TO MANKIND: A MEDICAL HISTORY OF HUMANITY73648(1d ed.1997).}

\noi
In terms of regulation, like the US, doctor-patient relationships in the UK are governed
by a governmental organisation. The General Medical Council (GMC), established
under the Medical Act, 1983\footnote{Medical Act 1983, Chapter 54, 26 July 1983}, is the body that regulates the medical practice and patient safety across the UK. As part of educating doctors in the UK about medical ethics, the GMC has set out certain guidelines which are to be adhered to by all physicians in the country.\footnote{GMC, \textit{Ethical Guidance for Doctors,} GMA-UK.ORG, \url{https://www.gmc-uk.org/ethicalguidance/ethical-guidance- for-doctors#good-medical-practice.}} These guidelines advocate good medical practice, confidentiality, consent and shared decision making, raising concerns, leadership and management and maintaining professionalism among others.

\noi
The UK has implemented a strict regime of laws that address medical negligence or
clinical negligence. The NHS, which is the public provider of health services, is a
governmental organization that is bound by clinical negligence laws. Primarily, clinical
negligence in the UK is seen as an offence of tort. Doctors and physicians are seen to
owe a duty of care to their patients and are expected to act cautiously while providing
treatment. However, certain acts by medical professionals can also attract criminal
prosecution by the State. To determine negligence 3-part test is conducted to establish
whether there was indeed an act of clinical negligence or not. The procedure, therefore,
relies on establishing fault on the part of the doctor, hospital, etc. To tackle the
limitations of litigation, the UK enforced a redressal programme to address victims of
clinical negligence. This programme came into force through the NHS Redress Act 2006.\footnote{NHS Redress Act 2006, Chapter 44, 8$^{th}$ November 2006.} The Act qualifies liability in tort for acts of clinical negligence and aims to provide patients with an alternative to litigation.

\subsection*{Doctor – Patient relationship and medico-legal systems Australia}

\noi
The doctor-patient relationship in Australia is aligned with that practiced in the UK. Doctors and patients are seen to have a contractual relationship between them where the Doctor provides services and the patient provides consideration in the form of fees. \footnote{Sidaway v. Board of Governors of the Bethlem Royal Hospital and the Maudsley Hospital 1 All.ER
643 [1985]; AC 871 904 (1985).} The authority on the contractual nature of the doctor-patient relationship in Australia was established in Breen v. Williams.\footnote{Breen v. Williams 57 HCA (1996)} While examining the case, the Supreme Court, observed that the doctor-patient relationship was ordinarily a contractual one. The Court held that the relationship was contractual in origin.\footnote{\textit{ibid}} It noted that it was only under exceptional circumstances that the same could be held to be a fiduciary relationship. Therefore, Breen v Williams is considered to be the leading authority in observing that the doctor-patient relationship in Australia.

\noi
The doctor-patient relationship in Australia is guided by the codes of the Medical Board  of Australia. The MBA is empowered to register medical practitioner, develop codes and guides for the medical profession, investigate complaints about physicians, conduct panel hearings, assess international medical professionals who want to practice medicine in Australia and approve accreditation standards in medical education.\footnote{Medical Board of Australia, ‘Role of the Board’, \textit{(medicalboard.gov.au)} \url{https://www.medicalboard.gov.au/About.aspx.}}

\noi
The Code stresses core values of the medical profession such as shared decision-making,
treatment in emergencies, doctor-patient partnership, confidentiality, effective
communication, informed consent, end-of-life care and ending a doctor-patient
relationship.\footnote{Id. pp.6-22} Apart from the Code prescribed by the MBA, the Australian Medical
Association (AMA), which is a professional body for physicians in Australia, has also
established its own guidelines which Australian physicians are expected to follow.

\noi
Australia’s medical negligence law is based on the principle of torts that anybody whose
act can injure another person must perform the act with reasonable care. When a case
involving medical negligence is brought before the Court, it first examines whether and
to what extent a duty of care is owed to the patient by the physician. This is followed by
an inquiry into the causation, which establishes the connection required between the negligent act of the physician and the injury suffered by the patient for the existence of liability.

\noi
The Civil Liability Act of Australia is an important law as which addresses different torts including that of negligence. Initially passed by the State of New South Wales,\footnote{Civil Liability Act 2002, No. 22 (2002).} each State in Australia has its own version of the Civil Liability Act, but they are all based on similar principles. It puts limits on the maximum compensation that may be ordered to be paid by a physician to the patient. AUD 400,000 is the cap on compensation that may be ordered to be paid for non-pecuniary losses by patients.\footnote{\textit{Id.} Sections 16, 17 and 17A} The Act also bars Courts from imposing exemplary, punitive and aggravated damages even if negligent acts by professional’s causes grave injury or death.\footnote{\textit{Id.} Section 21} Australia’s medicolegal system is well-developed and shows considerable actions aimed at improving doctor-patient relationships as well as cases of medical negligence. However, with the Civil Liability Act, Australia presents a different picture than that presented by the US and UK, putting into focus the rights of medical professionals.

\subsection*{Doctor – Patient relationship and medico-legal systems in India}

\noi
History dates medicine in India back to the Vedic age where texts such as the Atharva
veda contains details of treatment to injuries and diseases through the use of herbs. India
has the unique distinction of having six recognized systems of medicine in this category.
They are Ayurveda, Siddha, Unani, Yoga, Naturopathy and Homoeopathy. The doctor
–patient relationship was maintained during those days in different systems of medicine
and the way medico legal disputes are handled.

\subsection*{Ayurveda}

\noi
The Indian government now considers Ayurveda as an ancient and traditional
knowledge. Several forms of Ayurvedic medicine are protected by the government
through the Traditional Knowledge Digital Library.\footnote{Government of India, Traditional knowledge digital library,  \url{<www.tkdl.res.in/>.}} There is a definite interrelation in respect of doctor patient qualities, which also determines the quality of doctor patient relationship in relation to the ultimate success of treatment. In the ancient medicine, the doctor-patient relationship was narrated pertaining to different aspects of doctor and patient. The Physician is expected to wear plain \& clean dress, close keeping of hair \& nails, cleanliness, white clothes, holding of umbrella/ hand stick, wearing of shoes \& avoid of gaudy clothes which promotes good doctor patient relationship.

\subsection*{Siddha}

\noi
Treatment under the Siddha system is based largely on herbs, diet and lifestyle. The relationship between the Siddhas and their patients was much like the one shared between Ayurvedic physicians and their patients. The Siddha system in the modern day is protected by the Government of India and promoted by the Ministry of AYUSH.\footnote{Ministry of AYUSH, Central Council for Research in Ayurvedic Sciences, \url{http://ccras.nic.in/ content/about-ccras.}} Like Ayurveda, the knowledge of preparation of medicines under Siddha are also treated as intellectual property and protected by the TKDL.\footnote{TKDL (n 71)}

\subsection*{Unani}

\noi
Unani is a form of traditional medicine practiced in India, which relies on the elements
within the human body to treat diseases. The doctor-patient relationship in the Unani
form of medicine can be considered to be similar to the relationship between Ayurvedic
or Siddha physicians and their patients. The Ministry of AYUSH in India is concerned
with the protection and promotion of Unani medicine.

\subsection*{Yoga}

\noi
The world-famous art of Yoga is acclaimed to be one of the most effective methods of
maintaining good health. Yoga is more of a set of physical, mental and spiritual practices
rather than a form of externally administered medicine. However, even today, those that
teach Yoga are rarely referred to as doctors and mostly treated as instructors or teachers.
Therefore, in modern day Yoga, there does not exist a doctor-patient relationship and
as a result, the scope for conflicts is very low.

\subsection*{Naturopathy}

\noi
Naturopathy is considered as a form of traditional medicine by the Ministry of AYUSH in India and courses in Naturopathy are offered by government institutions.\footnote{Ministry of AYUSH, ‘Courses and Study at National Institutes’, \url{http://ayush.gov.in/ education/courses-and- study>.}} Naturopathy does involve an active doctor-patient relationship were naturopaths recognize themselves as doctors or physicians. Courts from different jurisdictions have found naturopaths criminally liable for the treatments recommended that caused damage to the lives of their patients. Naturopathic medicine takes a holistic approach to patient care. Therefore, the relationship between doctor and patient plays a key role in the healing process.

\subsection*{Homeopathy}

\noi
India adopted Homeopathy as one of its national systems of medicine and it is currently governed by the Ministry of AYUSH.\footnote{Ministry of AYUSH, ‘Introduction to Homeopathy’, (ayush.gov.in) \url{http://ayush.gov.in/about-the- systems/homoeopathy/introduction-homoeopathy>.}} The Food and Drug Administration (FDA) in USA has also conducted a hearing on homeopathic product regulation in which it has decided to tighten regulations around Homeopathy in order to prevent cases of medical negligence where diseases are serious in nature.\footnote{Kendrick Frazier, \textit{FDA to Regulate some Homeopathic Products: CFI Hails Move,} 42(2)} Courts have taken note of medical negligence by Homeopathic practitioners and have punished the accused where found guilty. Therefore, while Homeopathy continues to be a popular form of medicine in India, any regulation that is concerned with medical negligence shall also make suitable provisions to include Homeopathic practitioners within its ambit.

\noi
Healthcare in India is primarily a subject under the State list in the Seventh Schedule of the Constitution.\footnote{Constitution of India, Art. 246} However, certain aspects of healthcare such as the medical profession and medical education fall with the concurrent list of the Constitution.\footnote{Constitution of India, Schedule VII, List 3.} Therefore, both State as well as Central governments in India are collectively responsible to ensure a robust healthcare system in the country. Using the ownership criterion, the health care system in India into four main categories.

\begin{itemize}
\item Public sector – This includes government hospitals, dispensaries, clinics and primary health care centres.

\item Not for profit sector – This sector includes charitable institutions, voluntary health programs, missions and trusts.

\item The organised private sector – This includes general practitioners, registered medical practitioners, private hospitals and polyclinics.

\item The informal private sector – This consists of practitioners who lack any formal qualification, priests, hakims, faith healers and vaidyas.
\end{itemize}

\noi
There is lack of clarity in India with respect to which entities are responsible for
regulating the private sector and for ensuring quality of care, as there are multiple
agencies under different ministries. There are significant inequalities exist with respect
to health care access and outcomes between Indian states, rural and urban areas,
socioeconomic groups, castes, and genders. Studies have found that most of the hospital
systems across states are inefficient. Lack of competition has made the public health
infrastructure costly. Recently, the most important reform undertaken in India is the
National Health Mission, which seeks to strengthen the healthcare systems.

\subsection*{Criminal law and medical negligence}

\noi
Indian criminal Law has placed the medical professional on a different footing as
compared to an ordinary human. Section 304A\footnote{The Indian Penal Code,1860, No. 45, Acts of Parliament,1860 (India).} of the Indian Penal Code states that “whoever causes the death of a person by a rash or negligent act not amounting to culpable homicide shall be punished with imprisonment for a term of two years, or with a fine or with both.”. Many a time the doctor will also be responsible vicariously, meaning thereby if his employee/servant rashly causes the death of a patient.

\subsection*{Civil law and medical negligence}

\noi
Under the torts law or civil law, this principle is applicable even if medical professionals
provide free services. It can be asserted that where Consumer Protection Act ends, tort
law begins. Such cases of negligence may include transfusion of blood of incorrect
blood groups, leaving a mop in patient’s abdomen after the operation, removal of organs
without consent and administering wrong medicine resulting in injury. Persons who
offer medical advice and treatment implicitly state that they have the skill and
knowledge to do so, that they have the skill to decide whether to take a case, to decide the treatment, and to administer that treatment. This is known as an “implied
undertaking” on the part of a medical professional.

\subsection*{Consumer Protection Act and medical negligence}

\noi
Since 1990’s there is a huge speculation and debate on whether medical services are
explicitly or categorically included in the definition of “Services” as enshrined under
Section 2(1)(o) of the Consumer Protection Act. Deficiency of service means any fault,
imperfection, shortcoming, or inadequacy in the quality, nature, or manner of
performance that is required to be maintained by or under any law for the time being in
force or has been undertaken to be performed by a person in pursuance of a contract or
otherwise about any service.

\subsection*{Impact of Doctor-Patient relationship in Medico Legal system}

\noi
It is not unusual for doctors and patients to have a conflict of interest over the medical condition or treatment. Doctors and patients hold different perspectives and quite likely view medical conditions differently.\footnote{Susan Wolf, Conflict Between Doctor and Patient, 16(winter) Law, Medicine \& Health Care.197-203 (1988).} Conflicts can come in various forms including reluctance by the patient in undergoing a certain type of treatment or consuming specific drugs, ignoring the advice given by doctors, getting upset over medical bills and expenses or preferring alternative forms of medicine against the doctor’s advice.\footnote{Medical economics, \textit{Managing conflict with patients} MEDICALECONOMICS.COM (April.1 2015), \url{http://www.medicaleconomics.com/modern-medicine-feature-articles/managingconflict-patients.}}

\noi
The key to avoiding conflicts and disputes between doctors and patients is to promote
better communication between the two parties. The onus of taking the lead in such an
approach is on the doctors since it is part of their routine life and profession. Doctors
must ask the right questions and try to secure the patient’s confidence during the course
of diagnosis. This can include greeting patients warmly, asking general questions about
the patient’s health and background, explaining the medical condition to the patient,
seeking consent before undertaking diagnosis and treatment and informing the patient
about the treatment being administered.

\noi
One of the leading causes of disputes between doctors and patients is the lack of communication. The patient needs to communicate details about his health conditions, lifestyle etc which are connected to the diagnosis. From the viewpoint of the doctor, communication is essential, firstly, to make the patient comfortable. The doctor also needs to communicate with the patient about the treatment being undertaken and the various technical elements related to the diagnosis.

\noi
Doctor-patient communication is a fundamental element of clinical practice. Not only the scientific and expertise in various medical specialties, but also effective doctorpatient communications is necessary for building a therapeutic doctor-patient relationship.\footnote{J.F. Ha \& N. Longnecker, \textit{Doctor-patient communication: a review,} Ochsner J, 10 (1)} In recent years, a growing emphasis on patient autonomy, patientcentered care, and consumerism has further exemplified the importance of effective doctor-patient communication.\footnote{A.K. Shukla et al. \textit{Doctor-patient communication: an important but often ignored aspect in clinical medicine}}

\noi
Good doctor patient communication allows patients to freely share vital information essential for an accurate diagnosis of their problems and enables the doctor to have a better understanding of their patient needs and potentially lead to better symptom reduction.\footnote{S Dorr Goold M Lipkin Jr, \textit{The doctor-patient relationship: challenges, opportunities, and
strategies,} 14 (Suppl 1(Ml)) J Gen Intern Med, S26-S33 (1999).}

\heading{Conclusion}

\noi
The analysis with the USA has revealed that the Indian medico-legal system lacks in terms of implementing a healthy doctor-patient relationship. As a result, India is more prone to witness medico-legal cases than the USA. The comparative analysis with the medico-legal system in the UK is important as India has inherited its legislative system from the UK. It is observed that the system in the UK is different as it involves the operation of the National Health Services. Therefore, the concept of medical expense does not become relevant in the UK in medico-legal cases. In Australia, it is noticed that a high standard of medical ethics is expected to be followed by medical practitioners as compared to India. It is also seen that the regulatory bodies for the medical profession take a strict view of medico-legal cases and actively work towards their reduction. About quality in health care, Australia is having The Australian Commission on Safety and Quality, (ACSQH) a corporate Commonwealth Government agency and part of the Health Portfolio which is the main body responsible for the safety and quality improvement in health care. The ACSQH has developed service standards that have been endorsed by health ministers. These include standards for conducting patient surveys, which must be met by hospitals and day surgery centres to ensure accreditation. In the same line in India through the accreditation of health care providers does provide a mean for achieving efficiency in the system of medical liability, the process is very much in its nascent phase. The accreditation committee of NABH, audits the legal regulations, safety guidelines and service aspects of the hospitals and health professionals during assessments. There is no continuous follow-up of the guidelines by the NABH committee which makes the accreditation meaningless. All these studies have jointly revealed that the quality of medical treatment and medicolegal system in India is weak as compared to those in developed countries in terms of prescription of norms as well as their enforcement. Therefore, any reform in the medicolegal system must take into consideration the formation of a robust mechanism to strengthen the doctor-patient relationship and reduce cases involving medical negligence.

\end{multicols}

\label{end2016-art6}
