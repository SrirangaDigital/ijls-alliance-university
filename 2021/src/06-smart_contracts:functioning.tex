\setcounter{figure}{0}
\setcounter{table}{0}
\setcounter{footnote}{0}

\articletitle{Smart Contracts: Functioning and Legal Enforceability in\\[4pt] India}
\articleauthor{Sannidhi Agrawal \footnote{Student, NMIMS School of Law, Mumbai.}}
\lhead[\textit{\textsf{Sannidhi Agrawal}}]{}
\rhead[]{\textit{\textsf{Smart Contracts: Functioning and Legal....}}}



\begin{multicols}{2}

\heading{Introduction}

\noi
In the words of Milton Friedman, the 1976 Nobel Laureate for Economics, “The three
primary functions of a government are law and order, defence, and contract enforcement.”
The last function is generally performed through deterrence, wherein penal provisions are set
for parties who violate the terms and conditions laid down in the contracts; and such
provisions are enforced through adjudication, upon their violation. However, keeping the
traditional methods of contract enforcement aside, there exists a lot of potential in technology
to revolutionise the way contracts are performed. Smart contracts provide the platform to do
exactly that.\footnote{Punit Shukla, “How India's government can build better contracts with block chain”, World Economic Forum (October 4, 2019).}

\noi
They are essentially self-executing, digitally encrypted contracts, which make use of block
chain technology to ensure due performance and execution of contracts virtually, so as to
provide a smooth and trouble-free experience.

\noi
Although so far, there is no concrete legislation which deals with smart contracts, the
Telecom Regulatory Authority of India (TRAI) released a notification in 2018 which briefly
defined the term. It stated that they work on a programmable code which can implement predetermined tasks or rules so as to check regulatory compliance in advance, in the absence of
human intervention. Further, it mentioned that such contracts are suitable for a DLT
(Distributed Ledger Technology) system to formulate a digital agreement, with certainty
(owing to cryptography) that the agreement has been executed in the ledger of every party to
the agreement.\footnote{Telecom Regulatory Authority of India, “The Telecom Commercial Communications Customer Preference Regulations, 2018”, Gazette of India (July 19, 2018). }

\noi
Through this paper, the researcher aims to study the following objectives:

\vspace{-.2cm}

\begin{enumerate}[label=$\bullet$]
\item To understand the operation and functioning of smart contracts.
\item To analyse the pros and cons of switching to smart contracts.
\item To examine whether there are any statutory provisions which could potentially govern contracts in digital form.
\item To gauge the applicability of smart contracts across various sectors. 
\item To ascertain the legal validity and enforceability of smart contracts in India.
\end{enumerate}

\vspace{-.2cm}

\heading{What are Smart Contracts?}

\noi
In 1994, it was uncovered that since cryptography is decentralized in nature, it could be used
to improve the process of execution of a contract virtually. This took the shape of ‘smart
contracts’. Block chain technology eliminates the requirement of any intermediaries (and
subsequently, unnecessary human interaction in the form of calls and emails) owing to its
decentralized nature. Thus, they operate on P2P (Peer-to-peer) technology instead of being
maintained under a central server. As a result, a lot of time is saved and it leads to avoidance
of any conflict which may arise owing to a third party.\footnote{STA Law Firm, “The Enforceability of Smart Contracts in India”, “Mondaq (December 13, 2019).”}
 They rule out room for human
intervention of any sort, thereby eliminating the risk of human error. Further, they cannot be
altered once the agreement is finally codified, even if either party wishes to modify the terms
in their favour.\footnote{Vijay Pal Dalmia, “India: Blockchain And Smart Contracts – Indian Legal Status”’, Mondaq (February 5, 2020).}
 Additionally, they help in doing away with transactional and procedural costs
associated with negotiations (paperwork) and verification (commissions); since there is no
intermediary.\footnote{\textit{Id.}}

\noi
The key feature of a smart contract is that it is self-performing in nature, i.e. the terms of the
agreement between the parties to the contract are directly incorporated into lines of the code.
The code is contained in a distributed block chain network, and it comprises all the agreement
terms. Apart from the agreements, it consists of information that enables execution of the
transactions and makes sure that these transactions are fully tracked, permanent, irreversible
and time stamped.\footnote{\textit{Supra note 3.}} Every transaction carried out by the smart contract is placed as a block on the platform, which helps in establishing a clear audit trail, and erasing or wiping- it out is
an arduous task.\footnote{Rishi A, “The Legality of Smart Contracts in India”, India Corp Law (December 10, 2017).}

\noi
The important characteristics of a smart contract are as follows:

\vspace{-.25cm}

\begin{enumerate}[label=$\bullet$]
\item Once it has been released, it is not possible for anyone, including the owner or creator to alter its terms.
\item Its performance and completion do not require submission of any physical documents.
\item Although users may be anonymous, the details of each transaction are recorded and registered.
\item Transactions under smart contracts are irreversible in nature.\footnote{\textit{Supra} note 3.}
\end{enumerate}

\vspace{-.25cm}

\noi
There can be two types of smart contracts, the first one being contracts which are entered into
in the absence of any enforceable text-based contract governing them. For instance, when two
parties agree in oral terms, the business relationship they wish to maintain and proceed to
capture that understanding into executable code; it is termed as a “code-only smart contract”.
The second type of contract can be used to execute certain clauses of a conventional textbased contract when it consists of provisions for the same. They may be termed as “ancillary
smart contracts”.\footnote{Stuart Levi and “Alex Lipton, “An Introduction to Smart Contracts and Their Potential and Inherent Limitations”, Harvard Law School Forum on Corporate Governance” (May 26, 2018).}

\noi
The main point of difference between a smart contract and a traditional contract is that the
former is a self-executing computer programme, which cannot be tampered with by parties
and works on complicated block chain technology. On the other hand, the latter relies on the
performance of the legal terms agreed upon by the parties, which can be modified at any
given parties with their mutual consent and leaves room for conflicts.\footnote{\textit{Supra} note 4} Further, the risk
factor associated with conventional contracts is very high, as there are chances of nonperformance. Whereas in case of smart contracts, since they are automated, the risk is
minimised.\footnote{Kashish Khattar, “Everything you need to know about Smart Contracts”, iPleaders (June 2, 2018).}

\heading{How do Smart Contracts Work?}

\noi
The code contains the terms of the smart contract. Thus, the contract comprehends, approves
and automatically executes any transaction which in line with the terms. The contract triggers
itself once the predetermined terms and conditions are met. Moreover, once the contract is
executed, the obligations (which are encoded) cannot be paused mid-way; making the
contract self-enforcing.\footnote{\textit{Supra} note 7.}

\noi
At present, such contracts can smoothly carry out two types of transactions that are present in
numerous contracts: ensuring the payment of funds post the occurrence of a certain event;
and imposing monetary penalties upon lack of fulfilment of certain conditions.\footnote{\textit{Supra} note 9.}

\newpage

\noi
For instance, if a contract of rent is converted into a smart contract so as to assess its
effectiveness and\break efficiency; then the tenant will pay the rent to the landlord in
cryptocurrency. Once the payment is made, the code will carry out the respective transaction
according to the terms of the contract that were entered into the code. If the said transaction is
successfully carried out, a receipt will be delivered to the landlord. Post that, they will release
the key to the house. This system is based on the ‘If-then’ principle, and everyone involved in
the block chain will observe the transaction and become witness to the contract. If the
landlord releases the key, then they will definitely receive the amount. Likewise, if the tenant
pays the rent amount, they will definitely receive the key. Therefore, one transaction cannot
be completed in the absence of the other, which ensures effectiveness and efficiency of this
mode of transaction.\footnote{\textit{Supra} note 3.}

\noi
The if-then principle can be explained best by- “If ‘x’ condition is fulfilled, ‘then’ y
obligation must be enforced”. This feature makes smart contracts extremely lucrative for the
insurance industry and the financial services sector. Moreover, creation of smart contracts is
easier when there are bare minimal to none non-operational clauses involved. Since such
clauses are ambiguous and leave room for interpretation, they are unsuitable for smart contracts\footnote{\textit{Supra} note 4.}. They are well suited for cases where the agreement has mechanical and
straightforward clauses, and well defined outcomes”.\footnote{John Ream et al., “Upgrading blockchains: Smart contract use cases in industry”, Deloitte Insights (June 8,
2016).}

\vspace{-.05cm}

\noi
The functioning can be further explained with the help of another example. Say, one A
wishes to buy a flat in a building being constructed by B, but is unable to afford the full price
of the flat. Thus, they avail the loan facility from C, which is a bank. Conventionally, A
would have to provide personal information to verify their identity, and also undergo a credit
verification process. The process would be time consuming, and it would involve multiple
people who would demand compensation in the form of commission for the services
rendered. All this would add to the overhead costs. However, with the help of block chain
technology, C would be able to download the required information from one of A’s blocks so
as to make a quick decision about their identity and credibility; thereby significantly reducing
the turnaround time for all parties involved. Post this verification, all parties to the transaction
would enter into a smart contract wherein the loan amount will be disbursed by C to B, and
ownership of the flat will be transferred to A. However, C would hold charge on the flat till
the full and final repayment of the original loan amount is made. The transfer of ownership is
automated as the transaction gets recorded on a block chain, which is visible to all the
participants on the block and its status can be viewed at any given point.\footnote{Sanmith Seth, “What’s blocking the chain?”, India Business Law Journal (July 20, 2020).}

\vspace{-.05cm}

\noi
Apart from listing the rules and penalties in relation to an arrangement (similar to a
traditional contract), smart contracts perform the function of executing these obligations
automatically. Implementation of these contracts is carried out through a platform called
“Ethereum”, which comprises two key elements: currency and contracts.\footnote{\textit{Supra} note 3.}

\heading{How do Smart Contracts Ensure Secured\\ Transactions?}

\noi
They enable the enforcement of a safe and secured transaction between the two parties to the
contract. Further, they ensure that while one party gains something of value from the other
party for some collateral, the other party is the only prioritized party to that specific collateral. Implementing the same in case of traditional would undoubtedly be difficult, as
several other factors would come into picture, such as third parties partaking in the contract. 

\noi
Further, data protection is ensured through cryptography and the operation of the distributive
ledger system. Every block consists of information and in order to modify that, each block in
a chain will have to be hacked since they are related to each other.

\newpage

\noi
This has been transformed into reality by Ethereum. Its network is very transparent and
possesses the ability to determine and formulate which party has priority over the specific
collateral. Thus, it can conveniently accept or reject that collateral, and enable faster and
more efficient implementation of contracts.\footnote{\textit{Supra} note 3.}

\vspace{-.1cm}

\heading{Smart Contracts in India: A Statutory\\ Overview}

\vspace{-.1cm}

\noi
Section 10 of the Indian Contract Act of 1872 (hereinafter referred to as the ‘ICA’),
predominantly governs contracts in India. Section 10 of the Act lays down that “all
agreements are legally binding contracts, provided they are entered into with free consent of
parties to the contract, for a lawfully accepted consideration and in order to achieve a lawful
object.”

\vspace{-.1cm}

\noi
The essential features of a traditional contract are: a legitimate offer; acceptance which is
duly communicated; consideration which is lawful and pertinent to the subject matter;
consideration; and free consent of all competent parties to the contract with regards to all
aspects of the contract.\footnote{\textit{Supra} note 3.} Thus, by definition, it would seem that a smart contract is legally
permitted under the ICA, since it fulfils the above mentioned essentials to a contract.
However, since they are not legally recognised in India yet, such a proclamation would be too
bold and immature, since several factors come into play while determining the legality and
enforceability of smart contracts. 

\vspace{-.1cm}

\noi
For instance, ‘consideration’ aspect is problematic, because if it is in the form of
cryptocurrency, then it further raises the question as to whether cryptocurrency is accepted as valid consideration under Indian law.\footnote{\textit{Supra} note 4.} The ambiguity surrounding legality of cryptocurrency
poses as one of the many challenges to the usage of smart contracts in India. The Supreme
Court, in a March 2020 judgement, lifted the ban on cryptocurrency imposed by the RBI,
which forbade banks and other financial institutions from providing banking services to those
individuals and business entities which were engaged in dealing in cryptocurrency. Prior to
that, trading was restricted to crypto-to-crypto, and not crypto-to-INR.\footnote{Dipen Pradhan, “Supreme Court Allows Trading In Cryptocurrency”, Outlook Money (March 4, 2020).} However, the Indian
Government’s stance towards cryptocurrency is not very forthcoming, owing to concerns
such as safety of consumers, market integrity and white collar crimes such as money
laundering. A lot of prominent media houses have reported in the recent past that the
Government is planning to introduce a law which bans trading in cryptocurrencies. If
implemented, it could severely hamper the functioning of smart contracts.\footnote{Archana Chaudhary and Siddhartha Singh, “India Plans to Introduce Law to Ban Cryptocurrency Trading”,
Bloomberg Quint (September 15, 2020).}

\noi
Furthermore, there is no governing authority to evaluate whether the object is lawful or not.
All these factors raise doubts about the legal validity of smart contracts.\footnote{Alok Vajpeyi and Gauri Bharti, “India: Smart Contracts: a Boon or a Bane?”, Mondaq (December 3, 2019).}

\noi
Some of the more pertinent questions with regards to enforceability are: 

\noi
1. Will an electronic signature generated through the block chain technology deemed to be
valid for authenticating an agreement under a smart contract? 

\noi
2. Can a smart contract be placed as an evidence on record in a Court of Law if a dispute
arises?\footnote{\textit{Supra} note 7.}

\noi
It is imperative to analyse the Information Technology Act, 2000 (hereinafter referred to as
the ‘IT Act’) and the Indian Evidence Act in order to arrive at suggestive answers to the
above.

\noi
Section 5 of the “IT Act permits digital signatures and holds a contract to be legitimate and
enforceable.” It states that under any law, when a document or information produced needs verification or authentication through attachment of signatures, the requirement shall be
considered to be fulfilled if it is done by means of a digital signature. Thus, a digital signature
helps in proving consent to an electronic document.\footnote{\textit{Supra} note 4.} Further, Section 65B of the Indian
Evidence Act 1872 states that electronic records, i.e. documents signed digitally shall be
admissible in the Courts.\footnote{\textit{Supra} note 3.}

\noi
However, as per Section 35 of the IT Act, an electronic signature certificate can be obtained
only though a certifying authority designated by the Government. In order for a smart
contract to commence, the generation of a hash key is required, which is done by block chain
technology. The same is used as an identifier to authenticate the contract instead of any legal
authority. Thus, the electronic signature created by block chain technology is self-generated,
and in contrast to the one authorised by the IT Act.\footnote{\textit{Supra} note 3.} Moreover, “Section 85B of the Indian
Evidence Act states that an electronic document will be deemed valid only if it is
authenticated with a digital signature.”\footnote{Anirudha Bhatnagar, “India: Smart Contract In The Indian Crucible”, Mondaq (June 19, 2018). } The problem is further compounded by “Section 88A
of the Indian Evidence Act whereby it is mentioned that the Court presumes that an electronic
record placed as evidence is genuine, but does not make any presumptions about the
originator of the message.” Thus, if a signature for authenticating the smart contract is
obtained through block chain technology, the admissibility of the document will only become
more problematic as the signature was not obtained under the IT Act. Not only does this
impair the legal validity of the method of encryption used for smart contracts, but it also
impedes the “admissibility of such contracts as evidence in a Court of law”, as uncertified
signatures do not hold much value.\footnote{\textit{Supra} note 3.}

\vspace{-.1cm}

\heading{Areas of Concern With Respect to Smart\\ Contracts}

\vspace{-.1cm}

\noi
An important question for business entities looking at the technology as a prospect is whether
the regulatory and legal compliances are being met. Thus, in order for parties to enter into a
binding contract, enforceability is a key aspect which must be looked at.\footnote{Norton Rose Fullbright, “The future of smart contracts in insurance” (2016).}

\noi
One significant legal question which subsequently arises is that of fixing the responsibility. In
case an incorrect code is entered owing to its complexity and an error of judgement, then it
will become difficult to determine the Defendant party in that case, or the person who could
be held liable for committing a negligent or wrongful act, since there is no governing body to
do the same.\footnote{\textit{Supra} note 24.}

\noi
Though automating a transaction is easy, remedies for non-performance or breach of contract
are difficult to code in a smart contract. In specific contracts such as insurance, this problem
is further compounded by “insurance-specific aspects such as pre-contractual disclosure
obligations.” Further, since it is a regulated industry, concerns of regulators must be taken
into account.\footnote{\textit{Supra} note 31.} Artificial Intelligence is infamous for the risks it poses to humankind. Since
smart contracts leave limited room for human intervention or revocation of control, there is a
high possibility of the computer wrongfully assessing a particular situation and implanting an
incorrect step. This could result in multiple ramifications for either the parties to the contract,
as once the damage is done, it cannot be reversed. Therefore, a high degree of caution is
required to be exercised.\footnote{\textit{Supra} note 24.} Furthermore, entries on the block can be tampered with by bad
actors, be it contracting parties or miners who add past transaction records to block chain
ledgers. A study discovered that close to 3.4\% of Ethereum smart contracts are vulnerable to
hacking.\footnote{Kati Suominen et al., “Top 10 big questions and myths surrounding block chain”, JSTOR (2018).}

\noi
Smart contracts enable “provision of a platform for parties who may or may not know each
other” to\break enter into a contract. Excessive caution is required to be maintained while
contracting with another party, as in case of a failed transaction, the cost must be borne by the suffering party alone. The Indian legal system provides no legal recourse with respect to such
contracts as there is a dearth of regulatory framework to govern the same. 

\noi
For ensuring consumer protection, there must be a redressal mechanism that the aggrieved
parties can resort to. However, there exists no such provision, which stirs doubts in the minds
of the potential users. Further, force majeure events might lead to frustration of a contract in
case conventional contracts; or either party, in order to maintain good business relations, may
condone the other party’s performance. But in case of smart contracts, this is not possible
since the contract is automated. Thus, cordial business relations may be compromised at the
cost of effectiveness.\footnote{J Dax Hansen, “Legal Aspects of Smart Contract Applications, Perkins Coie (2017).}

\noi
Moreover, conventional contracts provide for termination of the contract at either party’s
behest, the option is not available in case of smart contracts. If a party comes across an error
in an agreement, which leads to the counterparty gaining more rights than intended, or would
result in greater costs if fulfilled; they cannot terminate the contract.\footnote{\textit{Supra} note 9.}

\noi
Furthermore, although the ‘automatic payment’ feature of smart contracts is revered, it does
not rule out the adjudication of payment disputes in case of complex commercial contracts.
For example, the chances that the party obtaining the loan will deposit the entire loan amount
in a specified wallet linked to the contract, are low. Instead, the required repayments would
be funded on an ad hoc basis. However, if the borrowing party does not fund the wallet
regularly, a smart contract may not be able to find the necessary funds for transfer of money
from that wallet or any other source specified to be used in case a contingency arises.\footnote{\textit{Supra} note 9.}

\vspace{-.1cm}

\noi
Smart contract writers also need to be mindful of the semantics and foresee how words can be
interpreted by the smart contract, since it does not possess the human intuition to deduce the
intentions or behaviour of parties. For instance, technology may not be able to interpret open
ended terms such as “reasonable efforts”. Thus, the language of the contract needs to be
adjusted according the limited vocabulary of the code.\footnote{\textit{Supra} note 35.} This makes the application of smart contracts is restricted to standardized processes. It cannot be extended to agreements where
evaluation of the terms and conditions might be required.\footnote{Anirudha Bhatnagar, “India: Smart Contract in the IPR paradigm”, Mondaq (June 19, 2018). }

\vspace{-.1cm}

\noi
When it comes to code-only smart contracts, the execution of the code and its outcome would
be the only objective evidence of the terms of the contract, since no paperwork is involved.
E-mail exchanges between the contracting parties regarding the functions which the smart
contract should carry out, or verbal discussions would determine the code and represent their
intent. Lack of evidence could serve as another barrier.\footnote{\textit{Supra} note 9.}

\vspace{-.1cm}

\noi
All these factors combined posit a very bleak scenario for the adoption of smart contracts in
India, currently. 

\vspace{-.1cm}

\heading{Regulation of Smart Contracts in The us}

\vspace{-.1cm}

\noi
In 1999, 47 states in the United States adopted the “Uniform Electronic Transactions Act”
(UETA), which governs laws relating to e-contracts, electronic records, electronic signatures
and so on. The Act approves of the usage of an electronic signature as a valid method of
consenting to contracts. However, to keep up with the technological advancements, in 2017,
several states in the US felt the need to draft separate regulations for the adoption of smart
contracts on a larger magnitude. Consequently, Arizona passed laws recognizing digital
signatures for smart contracts via block chain technology, and granting them enforceability.
Vermont and Nevada also gave recognition to smart contracts.\footnote{\textit{Supra} note 7.} Thus, it is high time that the\break Indian Government steps up and provides some clarity on how the feasibility of operation of
smart contracts in the country, since the benefits are immaculate. 

\vspace{-.1cm}

\heading{Potential Applicability of Block Chain-Based Smart Contracts}

\vspace{-.1cm}

\noi
In 2018, what was considered as a giant technological leap forward, the SBI legitimised
sharing of KYC data among banks through block chain technology through the conglomerate
of 27 banks called ‘BankChain’.\footnote{ET Bureau, “SBI to use block chain for smart contracts and KYC by next month” “Economic Times
(November 20, 2017).”} Even the pharmaceutical industry makes use of this technology for record-keeping. Unfortunately, however, the acceptance has been restricted to
information sharing and maintenance of records, and the Government is evidently not
inclined towards cryptocurrency, which poses as an obstruction for smart contracts. There
exists a lot of untapped potential in the Indian market with respect to these contracts, and it
could change the way that household supplies are purchased and e-commerce is carried out
by streamlining the entire process and reducing costs substantially.\footnote{\textit{Supra} note 29. }

\vspace{-.1cm}

\noi
Furthermore, they could revolutionise how trading systems on the securities markets work;
by taking on the arduous function of managing approvals between market players; estimating
accurate trade settlement amounts; and ultimately transferring the funds automatically once
the verification and approval of transaction is carried out. The purpose of settlement is to
ensure irrevocable delivery of a security to a buyer from the seller, for which the latter
receives final and irreversible payment of money. The possibility of settlement failures could
be negated with the help of smart contracts, since they are irreversible in nature. In status
quo, parties engage in expensive labour-intensive methods to corroborate each other’s
performance and reconcile records. Thus, the problem of lack of trust between the parties
could also be solved by trusting the smart contract.\footnote{Adam David Long, “How Smart Contracts for Finance Will Make Stock Markets Faster, Cheaper and Less
Error-Prone”, Medium (September 17, 2018).}

\vspace{-.1cm}

\heading{Conclusion} 

\vspace{-.2cm}

\noi
A study conducted by Capgemini reported that the effective adoption and implementation of
smart contracts could help retail banking and insurance companies save around 3 to 11 billion
in USD as a result of diminished overhead costs, which in turn could help every individual
customer save up to USD 980. This proves that the execution and growth of smart contracts
could undoubtedly become the ‘next big thing’, resulting in savings of billions of overhead
costs, while making the entire procedure more efficient. Block chain can be a complete game
changer when it comes to the execution of contracts. However, the grey areas with respect to
the law, make it a less lucrative option in India. Owing to the absence of a regulatory body
that certifies the admissibility and enforceability of a smart contract under the existing
legislations, smart contracts have not gained the traction they deserve; with uncertainty looming large. A pressing question which needs to be answered is that, “Is a smart contract
purported to constitute a contract, or to simply carry out the aspect of one?” This could
possibly clarify whether the current legal provisions can govern smart contracts or not. 

\noi
Since the entire process is decentralized, and there is no single entity which handles the data
arising out of any transaction, the imposition of “reasonable security practices and
procedures” as prescribed under the IT Rules, 2011 remains a challenge. The aforementioned
rules set out various guidelines in order to protect sensitive personal data, and safeguard the
same from any potential damage by third parties through a computer resource. 

\noi
Several sectors (for instance, syndicated loans) still rely on faxes and paperwork, which
results in inefficiency. It is high time they start adopting innovative technologies such as
smart contracts. But in order for this transition to take place, the Government needs to break
its silence on this topic and come up with ways to permit businesses and individuals to make
the switch to efficient, cost-effective systems.

\newpage

\noi
Undoubtedly, smart contracts do not have a pervasive applicability, i.e. they cannot be
blindly applied to\break every industry, since they come with their own set of limitations and risks
which have been discussed in the paper. They are more suitable where calculation of risks is
relatively easier, and the limited vocabulary of the code is able to interpret the commands
given. 

\noi
Smart contracts are a part of the surging wave of technological advancements which seek to
minimise the risks and costs associated with human capital. Indeed, they do minimise the risk
of human error to a great extent but it is an undeniable fact that machines come with their
own set of biases and errors; in addition to the Herculean task of regulating their operations.
Thus, there are innumerable challenges when it comes to establishing legal accountability.

\noi
Currently, smart contracts are in a nascent stage, and they require an impetus to penetrate the
Indian market. In order to facilitate the same, lawmakers must be vocal about their stance on
the various legal questions postulated in this paper. 

\heading{Suggestions}

\noi
A major hindrance to smart contracts being viewed as a norm is that the Government and
lawmakers may not be willing to invest the requisite financial and human capital to pave way for their development; especially in developing nations. However, the potential benefits of
switching to this technology must not be ignored.

\vspace{.2cm}
\noi
Furthermore, regulatory concerns pose a major threat in the Indian market. Even if no
separate regulation is developed, the Government must amend the relevant provisions of the
Indian Evidence at, 1872 and the IT Act; so as to incorporate smart contracts into their
purview and keep up with the changing times. For smart contracts to become operational in
India, the Government will have to amend certain existing statutory provisions and confront
challenges on multiple fronts. It will be interesting to see whether the lawmakers are able to
keep up with the ever-expanding realm of block chain and smart contracts. 

\noi
The precarious status of the legality of cryptocurrencies also serves as a major roadblock to
progress in the field of smart contracts. Thus, a clear stance must be adopted by the
Government, when it comes to its legal validity.

\noi
If smart contracts do become a reality in India, on a large scale, then the legal industry would
have to keep up with changing demands of clients. This paradigm shift could potentially lead
to loss of jobs and make professionals in the legal sector rethink the way they function. But
on the bright side, it could also encourage collaborations between law firms, software firms
and start-up companies, for the greater good. Embracing technology and using it in their
favour would help legal professionals in rendering improved services. Thus, smart contracts
must not be looked at as a potential threat. 
\end{multicols}
