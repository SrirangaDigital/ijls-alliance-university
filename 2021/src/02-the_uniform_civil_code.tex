\setcounter{figure}{0}
\setcounter{table}{0}
\setcounter{footnote}{0}

\articletitle{The Uniform Civil Code: A Triple Divorce Under Muslim Personal Law}\label{2021-art2}
\articleauthor{Dr. Faisal Ali Khan\footnote{Associate Professor of Law, Galgotias University, Greater Noida}}
\lhead[\textit{\textsf{Dr. Faisal Ali Khan}}]{}
\rhead[]{\textit{\textsf{The Uniform Civil Code: A Triple Divorce....}}}



\begin{multicols}{2}

\heading{Introduction}

\noi
The Uniform Civil Code (hereinafter referred as “UCC”) is enumerated under Article 44 of
our Constitution, which says that the Legislature has to pass the law related to UCC for all
over India but Indian society always resides with their different traditional aspects related to
the religious, rituals, cultural and sociological aspects related to the personal laws of the
Country. So, it would not be advisable to implement the same in the present scenario because
we are residing in diversity of group of the people in one unity.

\vspace{-.1cm}

\noi
The discussions have been made in the Constituent Assembly that if our cultural, sociological
aspects and thinking would be similar in future because every progressive society can change
their usage and custom under these circumstances the UCC can be made. Let us take an
example that the marriages among Northern Indian Hindus cannot take place within family
but similar in South Indian Hindus Marriages can take place among Uncle and niece (Mama
\& Bhanji). Besides, marriages can take place amongst the Muslim within 1$^{\rm st}$ Cousins. So,
each \& every religious group and castes have their own traditions and values. It cannot be
bound under UCC. Although, every progressive society can make changes its customs and
traditions as per the reforms and necessity of the period. So, if there may arise such a
situation that the future of our country to be found similar in term of UCC then it can be
considered.

\vspace{-.1cm}

\noi
Normally, the triple divorce in Muslim Personal Law is related to the religious sentiments but
it is well settled inappropriate form of divorce under Muslim Personal Law and it is
irrevocable after the pronouncement of simple assertion of ‘divorce’ at a same time. Besides,
it is Talaq-e-Biddat.

\vspace{-.1cm}
\noi
It is occasionally misused by Muslim husbands according to their whims and fancies. As such
Sarla Mugdal case in which Hindu Male who tried to escape from the liability of criminal case because they had converted to Islam, but the Apex Court did not accept this sort of
pleas.

\noi
The Parliament has already passed a good piece of legislation for the protection of the victim
Muslim women i.e. The Muslim Women (Protection of Rights on Marriage) Act No. 20 of
2019; however, the problems of our Muslim sisters cannot be resolved related to Triple
Divorce. Still, they are the victim of triple divorce. 

\vspace{.1cm}

\heading{Uniform Civil Code: Debates of Constituent Assembly and Judicial Precedents}

\noi
The purposes of the Constituent Assembly have incorporated an Art. 44 of the Indian
Constitution which has made the provisions that the State can make something to take
initiative for passing appropriate law for making the UCC which can be applicable to our
entire Nation. Besides, it has opposed on the grounds that it would be infringement of the
fundamental rights enumerated in the Art. 25 of the Constitution by alleging that “right to
freedom of religious” and it may be harsh to the minority.\footnote{V.N. SHUKLA, CONSTITUTION OF INDIA 378 (Eastern Book Company 2013)}
The first issue can be raised that it
is essential to the legislature to pass appropriate law for UCC for the implementation upon all
religions and in term of Articles 245 and 14 of the Constitution is uniform law for all and it is
desirable but its enactment in one go can be counterproductive to unity of nation and Articles
25 / 26 is right to religious freedom and not denied to Hindu who are majority in population.\footnote{Pannalal Bansilal Patel v. State of Andhra Pradesh, A.I.R. 1996 SC 1023-4}
The provision of UCC is enumerated under Art 44 of our Constitution which says that State
can make such law which is capable to apply within the entire territory of our country in term
of Uniform Civil Code (UCC). Although, these sorts of provisions are enumerated under
Articles 25 which gives us fundamental right to freedom of religious and Art. 44 of the
Constitution that it may reflect that the first one guarantees religious of personal law freedom
and second one divests religion from social relations and personal law. In addition, no doubt
at all the subject matters solemnisation marriage, Law of Successions, Law related to
Maintenance, Law of Divorce and the like matters such as of a secular character and cannot
come within the ambit of Articles 25 and 26 of the Constitution. If any legislation is passed in terms of succession and other matters of secular character under Articles 25 and 26 is a
suspect legislation\footnote{John Vallmatton v. Union of India, A.I.R. 2003 SC 2906}4
. Article 44 of the Constitution that the Legislature is in the process of
passing such law in term of Uniform Civil Code in India and it is a matter of regret. Apart
from this, UCC will provide an opportunity for National Unity, Communal Harmony,
Brotherhood and Integration by avoiding the conflicts of opinion and ideologies\footnote{
\it Id.}
, The Apex
Court has dismissed a petition for seeking a direction in the nature of Writ of mandamus to
issue for the implementation of UCC by the Govt. of India. The Apex Court has held that it is
in the domain of legislature to pass appropriate law related to UCC and the Court can not
intervene into these matters\footnote{Maharishi Avadhesh v. Union of India, (1994) 1 SCC 713}
. The Supreme Court has also held the similar views that the
Courts can exercise the discretionary powers for directives under Art.142 of the Indian
Constitution. Hence, the Apex Court has directed the legislature to pass appropriate law for
“Talaq-e-Biddat” and said prospective law related to the “Talaq-e-Biddat” will
reform/advances in Muslim Law – “Shariat”. Right now, it has been reformed by other
countries’ legislations at globally even by Muslim Countries. Whereas legislation is in the
domain of the Parliament till the appropriate legislation will not pass, ultimately, we have to
issue injunction against the practice of such divorce up-to six months. Besides, if the process
for making law will begin against the Triple-Divorce before expiry of six months, then it will
continue till the legislation takes place in the form of enactment, failing which such
injunction order shall be vacated. By virtue of the decision of Apex Court held that practice
of triple divorce is set aside by a majority of 3.2.\footnote{Shayara Bano v. Union of India, (2017) 9 SCC 298}

\vspace{-.1cm}

\noi 
The {\it prima facie} of the first objection is not maintainable in the eye of law related to the Part
IV of the Constitution which says that the Directive principles of State policy as enshrined
under Article 44 of the Constitution does not violates, the fundamental rights of the “freedom
of religious” enshrined under Article 25 (2) of the Constitution which asserted that especially
the secular activities in consonances with religious practices as such guarantee of religious
freedom as enshrines under clause (1) of Article 25
.\footnote{John Vallmatton v. Union of India, A.I.R. 2003 SC 2902} Apart from this, the Apex Court has
dismissed the contention to abrogate the discrimination between men and women under Section 10 of the Indian Divorce Act, 1869 (applicable to Christians). The Judicial approach
is related to their “limit” of the Courts’ jurisdiction. The Court has categorically held that
when Indian Divorce Act, 1869 mentioned the grounds of divorce and restricts the Courts’
jurisdiction and the Court cannot re-write the law for the addition or subtraction of the certain
grounds of the divorce\footnote{Reynold Rajamani v. Union of India A.I.R. 1982 SC 1261}
. Whereas the Courts are the real interpreter of the statutes but it is not
advisable to act in such a manner to make the law in term, of “Judge Made Law”.

\noi
That Sri K.M. Munshi has asserted at the meeting of Draft Committee of the Constituent
Assembly regarding to the second objection: 

\noi
He had put forwarded an advanced argument that if the UCC would be passed, it may be
painful/ tyrannical to the minorities. Is it tyrannical? There are so many advance Muslim
countries which have also enacted a Civil Code. Let us take the example of Turkey or Egypt.
Besides, no minorities are enjoying such types of privileges of personal laws. I make a
quotation that the process for making the law took the place and as a result the Shariat Act
was passed by the Parliament in the British regime. Some sections amongst the Muslim were
also dissatisfied such as Khojas and Cutchi Memons. They want to preserves their own
usages, customs and old traditions.\footnote{SHUKLA, {\it supra} note 2, at 379.}

\noi
Because, they have reserved some usages and customs of the Hindu religions although, they
had converted to Islam religions. But they did not want to confirm to the Shariat Act,
however, other members of the Muslim community were desirous to enforce Shariat Act, but
such enactment is applicable to the entire community of Muslim to follow it. The Khojas and
Cutchi Memons did not desirous to follow the Shariat Act, but legislation has intended to
bring a uniform legislation which is applicable to the whole of community. As we take the
example of the European countries, civil code, each \& every person would follow the civil
code. Besides, it seems to be not applicable before us that the said civil code adopted by so
many countries is not harmful/painful to the minorities. The policy of our legislature to bring
a uniform and consolidate the law to unify our personal laws and such laws would be
applicable to the matter of divorce, inheritance, succession, etc. Although, what all those things to do? What radical reduction point is involved? Which I actually fail to
understand….\footnote{\it Id.} 

\noi
Let us take at the instance of the applicability Hindu Law there can be a different schools of
laws followed by different State/ parts of our country. Such as, Muyukha Law which is also
applicable in a few parts of our States and the same Mitakshara School’s law also follow in
so many parts of India. Besides, Dayabhaga school’s law is also applicable in urban parts of
India like Bengal. Although so many Hindus are also governed by separate personal law but
the Law of Mayukha is applicable in the few States of India; let us take for instance the law
related to the Mayukha, Mitakshara and Dayabhaga are different from each other. Besides,
the Personal Laws of Hindus are separated in term of separate schools like Mayukha,
Mitakshara and Dayabhaga. Right now, the question may arise that it is a good piece of
legislation on the basis of which can affect the personal laws of our country? Uniform Civil
Code does not confine to affect the rights of the minorities, but it has to affect the rights of
the majority community.\footnote{\it Id.}

\noi
It is speculated that so many Hindus are not in opinion to apply uniform civil code. India’s
ethics and culture is a multi-religious society which has followed different religious tenets,
usages \& customs by different religious. So, it is not advisable to apply uniform civil code for
religious group. If we consider that Personal Law is related to divorce, maintenance,
inheritance and so on, it would be difficult to maintain the principles of equality as enshrined
in our Constitution which says that men and women are equal, and they cannot be
discriminated on the ground of sex. To see Hindu Law; if the question arises related to issues
of exploitation, discrimination against women; if it is considered a part of Hindu religion or
practice, then we are not in a position to pass such which can be helpful for the betterment of
the Hindu women. So, we think that there is no appropriate ground why there must not be a
UCC for this whole country.\footnote{\it Id.} But the Supreme Court has observed that if the uniform law is
necessary, the legislature should make law is necessity for the unity and integrity of the
nation for which we can say that one nation is enough for one law and applicable to the entire nation. So, uniform civil code can be passed as enshrined under Article 44 of the
Constitution. 

\noi
The religions must be restricted to spheres which legitimately ascertain to religious and other
things which might be involved in our life to be recognised as an organ which can work for
the national integrity, unity, brotherhood, peace loving and law abiding citizens. It would be
better to follow the notion of unity and integrity of the Country. We think that there is a
national unity. Although so many important factors will often disturb our national
consolidation. It is essential for us to follow the secular sphere which unifies and we will be
capable to say ‘Well because our country is not simply a nation but actually our personal laws
are implemented as we are a strong and consolidated nation. Besides, if I argue to the
contrary to this view that it may not be a chance to argue with our friends that they could
never think that it can be harmful or painful attempt to the minorities.\footnote{ Pannalal Bansilal Pitti \& Ors v. State of Andhra Pradesh \& Anr., JT 1996 (1) S.C.}

\noi
Although Munshi’s observations, have not lost their relevance, no major steps were taken in
order to achieve the goal of uniform civil code except that it succeeded to the codification of
the Hindu Laws after the commencement of the Indian Constitution. But the Apex Court has
held that the divorcee Muslim lady can get maintenance under Section 125 Cr. P. C.\footnote{ Mohd Ahmad Khan v. Shah Bano Begum, A.I.R. 1985 SC 945-54 and the Muslim Women (Protection of Rights on Divorce) Act 1986; Jorden Diengdeh v. S.S. Chopra, A.I.R. 1985 SC 935}
Although, the U.P. Government has made the provisions for getting Rs. 6000/- p.a. to the
victim of triple divorcee as a maintenance amount from the Government and provide free
legal assistance to contest their cases before the Courts. So, it is an initiative step that has
been taken for the development of the separate branch of the Law of Maintenance. The
Supreme Court has also held that it is an obligation under this article and issued the directions
to take the steps into the matter.\footnote{Sarla Mudgal v. Union of India, A.I.R. 1995 SC 1531} The Apex Court has also stated not to issue direction to
pass a common civil code.\footnote{ Lily Thomas v. Union of India, (2000) 6 SCC 224; John Vallamatton v. Union of India, A.I.R. 2003 SC
2902} The Apex Court has also held that divorced Muslim lady can get maintenance after {\it iddat} period as such provisions enshrined under Section 3 of the Muslim Women (Protection of Rights on Divorce) Act No. 25 of 1986.\footnote{Danial Latifi v. Union of India, A.I.R. 2001 SC 3958}

\heading{Triple Divorce or “Talaq-e-Biddat”~:\\ a Critique}

\noi
The triple divorce is recognised under Islamic law but it has been disapproved form of
divorce. The holy Quran did not give any command and that the holy Prophet did not approve
the triple divorce. Besides, it was not a practice during the early regime of more than 2 years
of Caliph Abu Bakar. However, he had allowed the triple divorce under the peculiar situation
at later stage of his Calilafat. Because the Arabs conquered Syria, Egypt, Persia etc. then the
women of those areas were so beautiful compared to the Arabian women. Ultimately, the
Arabians was attracted to marry them. But those women had put a condition before the
Arabian to marry that they must first divorce to their existing Arabian wives in one sitting
means triple divorce. This proposal was really accepted by the Arabian just for the sake of
marriages with these beautiful women because they knew such type of triple divorce is not
recognised among the Arab. Besides, there was a provision for divorce that one divorce can
be given in first period of menstruation ({\it tuhr}) and thereafter second divorce can also be given
after second period of menstruation ({\it tuhr}) but triple divorce at one time is a un-Islamic. It is
null and void. So, the Arabs could be in position to marry these beautiful women but also to
retain their Arabian wives. The matter of divorce and marriages with Persian beautiful
women came to the notice of the Second Caliph Hazart Umar that they had taken an action
for preventive measures to avoid the misuse of the power of divorce by the husband. They
had passed an administrative order that if the husband had given a triple divorce to his wife, it
would be complete and irrevocable divorce and marriage will be annulled. It was merely an
administrative order just to tackle the emergent panic circumstances of that time but it was
never intended to be a perpetual law of divorce in nature. But later on, the Hanafi Jurists
would follow this administrative measure in the form of valid divorce and also paved
religious sanction on it.\footnote{ AQIL AHMAD ET. AL., MOHAMMEDAN LAW 165-166 (Central Law Agency 2003).} 

\noi
Besides, it seems to appear that Hazart Umar had taken punitive action against the persons
who had given triple divorce and Hazart Umar did not like the triple divorce and discouraged
the persons from doing so. The Sunni jurists are of opinion that triple divorce pronouncement
at the same time is irreconcilable.\footnote{{\it Id}. at 168.} But nowadays, it is unconstitutional and a punishable
offence. Besides, there would be differences arising between husband and wife which would
be better to refer them to the arbitration who would resolve the disputes by amicable
settlement.

\noi
Nowadays, much emphasis is laid upon a cardinal point that there would be a Uniform Civil
Code in India. As we know India is a secular State where different Communities live
together, practice different religions and follow different rituals and customs and the citizens
of India have “Fundamental Rights”, guaranteed under the Constitution, and thus, enjoy
freedom at large accordingly. So in such a pluralist State like India, would it be probable to
have an effective and unabridged UCC in India? We should also bear in mind that the
Legislature frames the law/legislation in accordance with the need and varied circumstances
of society. Consequently, some of the legislations that are passed in the Parliament even
without deliberations, which do not prove to be much effective and conducive to some
communities.

\noi
So, if the UCC is formulated, as enumerated under Article 44 of the Constitution, considering
and maintaining the spirit of respective religions, rituals, customs and culture of all the
communities of India and unifying them, appeasing all sections of the society, then, perhaps,
it may prove to be effective and achieve its objectives, which are an arduous task, indeed. 

\noi
DIVORCE under the Muslim Law has been regarded as “The worst of all the permitted
things”. A divorce under Muslim Law pronounced by a husband three times consecutively is
called Triple Divorce or Talaq-e-Biddat which is irrevocable and dissolves the marriage
straightaway. Right now, it is null and void. It can be prosecuted under the Law.

\noi
The law on the Triple Divorce, passed by the Parliament, contains provisions for sentence
and a fine for the husband, liable for the Triple Divorce, though, is apparently detrimental to him, but we still come across the cases under the said law which has not served the very
purpose it is meant for. Proper implementation of the said law and general awareness is
required in the society. NGOs can play a vital role to create awareness in our society and take
initiatives to create awareness camps among the women and educate them about their legal Right now, the Govt. of India has passed the Muslim Women (Protection of Rights on
Marriage) Act, 2019, and enumerated the provisions that if the husband would pronounce
triple divorce, then he may be prosecuted and awarded a sentence up-to 3 years and shall also
be liable for a fine as the case may be. The provisions have been made under Section 3 of the
Act for the pronouncement of talaq-e-biddat by the husband to the wife is void and illegal. It
would amount to be prosecuted under Section 4 of the Act and can be awarded sentence up-to
3 years and shall also liable to a fine. Besides, Section 5 of the Act has also made provisions
for the subsistence allowances to the divorcee and her children from her husband. Apart from
this, an offence of triple divorce is cognizable and compoundable under Section 7 of the Act;
which can be compromised by the wife with her husband.
rights. 

\heading{Triple Divorce OR Talaq-E-Biddat: A Judicial Trend}

\noi
The common phrase used by the Courts is that the talaq-e-biddat or triple divorce is good in
law though bad in theology or religious point of views.\footnote{{\it Id}. at 170.} The Privy Council has recognised
the triple divorce pronounced at one time as validly effective.\footnote{Saiyida Rashid Ahmad v. Mst. Aneesa Khatoon, A.I.R. 1932 PC 25} The Madras High Court has
held that divorce can be given in the absence of the wife by saying/pronouncing repeated
three times I divorce forever.\footnote{Aisha Bibi v. Qadir Ibrahim, (1910) 3, Mad. 22} The Family Court has exclusive jurisdiction to entertain the
petition under Section 125 Cr. P. C for maintenance of the wife and divorced wife is also
entitled to file petition under Section 125 of the Cr.P.C even after iddat period as long as she
does not remarry and the amount of maintenance to be awarded not to be restricted for iddat
period only.\footnote{Shabana Bano v. Imran Khan, A.I.R. 2010 SC 305} The Apex Court has held that the divorced wife can also be entitled to get
maintenance from her husband after the period of {\it iddaat} if she does not remarry with some other person or unable to maintain herself. Thus, being a beneficial piece of legislation, the
benefit thereof must accrue to the divorced Muslim women.\footnote{DINSHAW FARDUNJI MULLA ET. AL., PRINCIPLES OF MOHOMEDAN LAW 366 (Lexis Nexis
Publication 2014).} Merely facts narrated in the
pleadings of the Written Statement is not sufficient to pronounce a divorce but the husband
would take the plea and proof of the divorce by adduce the necessary evidence unless and
until the obligation comes to end under Section 125 of the Cr.P.C\footnote{Shamim Ara v. State of U.P. \& Anr., 2002 (7) SCALE 183}
.
\noi
The Apex Court has held that the burden of proof lies on the husband to prove that he has
pronounced the triple divorce under the Quaranic reasons. Besides, the Apex Court has held
that the practice of triple talaq or talaq-e-biddat violates fundamental rights of Muslim
women and it is unconstitutional.\footnote{Shayara Bano v. Union of India, A.I.R 2017 SC 4609} There is a lack of option of the attempt for reconciliation
and revocation as the case may arise. The triple talaq is against basic tenets of Holy Quran,
violating Shariate. Apart from this, religious freedom does not include triple talaq under
Article 25(1) of the Constitution and therefore protection under Article 25(2)(b) does not
apply.\footnote{{\it Id.} at 4612.} The Apex Court has further held that the Court is competent to entertain the petitions
under Article 32 of the Constitution to challenge the personal laws which can violate the
fundamental rights and not by the Legislature.\footnote{{\it Id}. at 4613.} The Apex Court has also observed that the
valid divorce amongst the Muslims must follow certain things to be considered that (1) There
must be a valid divorce to be based on genuine reasonable cause and must not be a ground of
husbands wish, whim and caprice (ii) It must not be a confidential act (iii) Between the
declaration of divorce and finality, there must be a gab for an amicable settlement between
the husband and wife with the aid of arbitration and conciliation (Minority view) (Per J.S.
Khehar, C.J.I. for himself and on behalf of S. Abdul Nazeer, J.)\footnote{{\it Id}. at 4614}
.

\heading{Conclusion}

\noi
To wind up, the above discussions which are the limelight to the issues and challenges of
Uniform civil code and triple divorce, it is an acute problem before us that it is not purely problem related to our law, but it is a mixed problem of law, religious, sentiments, customs
and usage. As far as the issue of uniform civil code is concerned, we will follow up the views
of the Constituent Assembly which has said that our society is multi-religious and multicultural society and uniform civil code might be applicable in such an advanced situation
whenever our cultural, customs and usages becomes similar to each other. There would be
uniformity of norms related to the issues of divorce and marriage etc. Besides, the issue of
triple divorce is concerned as the aforesaid origin at the regime of Hazrat Umar how it came
in practice since long but it is well recognised law related to divorce in the present era to the
followers of the Hanafi Law but Shia and others did not recognise it. It is occasionally
misused by male Muslims and it is painful rule to the female. Besides, Our NGOs and the
Islamic Scholars will propagate the massages of our Quran and Prophet Mohammad who
says that it is most disliked and disapproved form of divorce. Hence, it should be avoided.
Ultimately, our sisters may not be compelled to come to the Court for their redressal of the
grievance. Disputes may be settled amicably out of Court. It may be a milestone to the cheap
and speedy justice to our mothers and sisters. Besides, the Apex Court recently has rightly
banned the practice of triple divorce at a time. It was an arbitrary exercise of powers by the
husbands and our sisters have faced acute problems under these circumstances which is
resolved by the Apex Court and our legislature has also passed the legislation for prohibition
of the Triple Divorce and for prosecution for the same practice of talaq-e-biddat.

\heading{Suggestions:}

\noi
\begin{enumerate}
\item To Establish Rehabilitation Centre for the victims of triple divorcee/ victim of acid attack.
\item To give free medical facility to the victims of triple divorcee/victim of acid attack.

\newpage

\item To give them free vocational training just to earn their livelihood.
\item To give scholarship and free education to the children of triple divorcee/victims of acid attack.
\item To provide free legal aid and advocate to contest their cases and provide legal assistance.
\item To provide a loan at concessional rate of interest with subsidies to initiate their own cottage industries like swigging and tailoring centre, food and craft centre.
\item To provide them accommodation, in case they do not have any shelter.
\item To provide free home under different Govt. schemes on priority bases.
\item To provide employment under different Govt. Schemes on priority bases. 
\item To establishment counselling centres to reduce their mental agony.
\item . To provide free education to the children of the triple divorce/acid attack victims. 
\end{enumerate}

\end{multicols}
\label{end2021-art2}
