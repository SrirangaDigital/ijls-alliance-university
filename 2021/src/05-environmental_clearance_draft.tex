\setcounter{figure}{0}
\setcounter{table}{0}
\setcounter{footnote}{0}

\articletitle{Environmental Clearance Draft Notification, 2020: A Case of\\[4pt] Obscure Visibility?}\label{2021-art5}
\articleauthor{Dr. Parna Mukherjee \footnote{Assistant Professor, GLS Law College, Ahmedabad, India} and Rhishika Srivastava\footnote{Student, 4th Year, B.A. LL.B., GLS Law College, Ahmedabad, India}}
\lhead[\textit{\textsf{Dr. Parna Mukherjee and Rhishika Srivastava}}]{}
\rhead[]{\textit{\textsf{Environmental Clearance Draft....}}}



\begin{multicols}{2}

\heading{Introduction}

\noi
The year 1987, remains to be a significant milestone when the ‘Brundtland Report” was
officially released, curving out the gist of the concept “Sustainable Development.”\footnote{Our Common Future [Report], \url{https://sustainabledevelopment.un.org/content/documents/5987our-commonfuture.pdf}}
 The
United Nations in its General Assembly urgently appealed to the World Commission to
prepare a vision document for \textit{"A global agenda for change"}.\footnote{Report of the World Commission on Environment and Development: Our Common Future, Chairman’s Foreword, \url{https://sustainabledevelopment.un.org/content/documents/5987our-common-future.pdf}}
 This action UNGA was to
create a global alternative to the existing conflicting dynamics of environment and
development. The concept of sustainable development was designed to create a synthesis
amongst the conflicting dimension of human development and natural environment. The
conference at Stockholm in 1972 was partially successful in creating the awareness towards
the environmental obligations of mankind.\footnote{Stockholm Conference 1972 [e-book], \url{https://www.un.org/en/conferences/environment/stockholm1972}}

\noi
Thereafter in 1992, at Rio de Janeiro the more detailed blueprint map of sustainability was
curved out in the next global conference held on the theme integrating the environment and
human development.\footnote{Sustainable Development vis a vis International Cooperation, \url{https://sustainabledevelopment.un.org/milestones/unced}}
 The principles of Rio Conference along with its two binding
instruments are considered as the essential part of the international environmental
jurisprudence till date. A few important principles of the Rio Convention such as principle 4,\footnote{Principles 4: In order to achieve Sustainable Development, environmental protection shall constitute an integral part of the development process and cannot be considered in isolation from it.}
highlights the need for and importance of integration of the developmental growth with environmental protection. Further, in principle 10 it lays down the stress for people’s
participation for environmental decision making and the need for developing the adequate
environmental redressal mechanism.\footnote{UNEP Principle 10 and the Bali Guideline, \url{https://www.unenvironment.org/civil-societyengagement/partnerships/principle10}\#:~:text=Principle\%2010\%20was\%20adopted\%201992,citizens\%2C\%20at\%20the\%20\\relevant\%20level.\&te xt=States\%20shall\%20facilitate\%20and\%20encourage,by\%20making\%20information\%20widely\\\%20available.}
 Lastly, the Principle 17 encourages for the adoption of
the domestic instrument for guidance in the decision making in cases with potential adverse
decision making.\footnote{Environment Impact Assessment, Principle 17 UNEP,\\
\url{https://www.jus.uio.no/lm/environmental.development.rio.declaration.1992/17.html}}
 Thus, to sum up conceptually, we can say that EIA:

\noi
\textit{“Environment Impact Assessment (EIA) is a planning tool to integrate the environmental
concerns into developmental process right at the initial stage of planning and suggest
necessary mitigation measures. EIA essentially refers to the assessment of environmental
impacts likely to arise from a project.”}\footnote{As defined in PARIVESH, FAQ, by the Ministry of Forest and Environment and Climate Change,
\url{http://parivesh.nic.in/writereaddata/Draft_EIA_2020.pdf}}

\noi
According to the aforesaid principles of the Rio Convention, most of the state nations
developed environmental legal framework for environmental decision making within their
own jurisdiction. India in the early 1980’s had no specific legal mechanism for scrutinizing
the developed projects for environmental safety. The department of Science and technology
was the sole authority to grant environmental clearances having a narrow jurisdiction mostly
focused on river-valley projects. Further, it was only post Rio Conference, in 1994 the
MoEF(Ministry of Environment, Forest and Climate Change) passed the first exclusively
devoted legislative instrument for environmental decision making in India based on the
precautionary approach of sustainability popularly known as Environmental Impact
Assessment Notification, 1994. The said notification prescribed both the substantive and
procedural guidelines for the granting environmental clearances in diverse fields of
developmental projects in India. 

\heading{Journey of EIA Law Till Date}


\noi
The EIA Notification of 1994 seems to have paved the way for better environmental
governance in India. This legislative initiative soon met with several dilution and successive
amendments to suit the need of the other socio-economic domains, i.e., trade- commerce,
industries, tourisms and other developmental lobbies.\footnote{EIA Notification 1994, \url{http://dest.hp.gov.in/?q=eia-notification-1994}} Hereafter, in 1997 a welcome change
was initiated to make this decision-making tool more effective, and goal oriented by
introducing the process of “Public Hearing” \textit{vide} amendment.\footnote{EIA Notification No. \textit{SO 319} (E) \textit{dated} 10th May, 1997 issued by the MoEF, India.}

\noi
Further, in September 2006 a major amendment \textit{vide} S.O.1533 (E) was issued to replace the
original EIA Notification of 1994 and to substitute with a new notification with a legislative
intent to restore the spirit and objective of law.\footnote{\textit{Id.}} This new notification introduced several
welcome changes by prescribing for seeking the prior environmental clearance at the
planning of the project itself, launching of online system to bring about more transparency in
the decision making, increase the scope for delegation by bifurcating the projects into
category “A”- whereby the Central government is the regulator and category “B”- whereby
the respective State is the regulator and establishing the standardization in EIA evaluation
process etc. Several amendments were also introduced to the existing 2006 EIA Notification
to accommodate the directions of NGT for decentralization and better implementation.\footnote{Key amendment to Environment Impact Assessment (EIA) Notification 2006, to ramp up
availability/production of bulk drugs within short span of time., Press Information Bureau
Government of India, Ministry of Environment, Forest and Climate Change.,
\url{https://pib.gov.in/newsite/PrintRelease.aspx?relid=202284}}
Thus, we can say that with successive amendments the letter of the law changed significantly,
and its spirit was diluted to defeat the legislative intent.


\heading{Proposed Draft, Its Legislative Intent and Challenging Issues and Pitfalls}

\noi
In November 2014, the High Court of Jharkhand passed an order in respect to the writ
petition filed before it in the matter between Hindustan Copper Limited and Union of India.\footnote{Hindustan Copper Limited Versus Union of India, W.P. (C) No. 2364 of 2014, in the High Court of
Jharkhand}
The said High court laid an important precedent stating that all project proposals for
environmental clearance must be thoroughly examined solely on the basis of its merits. Any
collateral issue of alleged environmental violations must be looked into separately rather than
clubbing it with clearance issue.\footnote{Ministry of Forest and Environment and Climate Change, PARIVESH,
\url{http://parivesh.nic.in/writereaddata/Draft_EIA_2020.pdf}} Similarly, the National Green Tribunal in 2018 in the case
of Sandeep Mittal v/s MoEFCC\footnote{Original Application Number 837/2018, NGT, \url{http://parivesh.nic.in/writereaddata/Draft_EIA_2020.pdf}} directed that the MoEF would make the compliance
monitoring system more robust prior to granting any environmental clearance. So, there has
been on and off several such judicial pronouncements highlighting the grey areas of the EIA
domain and exposing the loopholes of the existing system.

\noi
Further, in March 2020, a new draft EIA Notification\footnote{MoEFCC issued the EIA Draft Notification, 2020 under section 3 (2) r/w S 23 of EPA, 1986 r/w EPA rule5 to
suppress the existing EIA Notification of 2006.} was proposed to substitute the exiting
EIA Notification of 2006. The said draft notification was published in April 2020 in the
official gazette to invite the objectives and comments but later due to the prevailing pandemic
the deadline for same was extended up to the month of June 2020. Soon, the said EIA
Notification caught the eye of the activists and environmental experts for wrong reasons. By
virtue of a comparative analysis, it was put forth that these newly proposed draft of 2020, i.e.
EIA Draft 2020 is weaker even than those of 2006. The question arises can we substitute a
legislation with a weaker legislation?\footnote{Abhijit Mohanty, \textit{“Why draft EIA 2020 needs a evaluation?} Published by Down to Earth, CSE, Delhi, July
2020 , \url{https://www.downtoearth.org.in/blog/environment/why-draft-eia-2020-needs-a-revaluation-72148}}

\noi
Concept of \textit{“Post-facto Clearance”} is completely opposite of the basic philosophy of the
underlying Precautionary principle on which the very EIA process is designed. The concept
of granting Post-facto clearance will encourage the overlooking of the potential and adverse impacts of any developmental project. However, inspite of this fallacy the environmental
clearance will granted as proposed under this new draft notification. Those who may argue
against this logic and favour the idea of post-factor clearance, they need to consider the recent
industrial disasters of Oil India Ltd.

\heading{Other Adversities With The Proposed Draft}

\noi
{\large\it\bfseries a.~It puts in vain, the very purpose of Public Consultation}

\noi
The draft gives a deadline to raise feedbacks, inputs and objections by the people within 60
days of the distribution of the notice. These 60 days were inclusive of the Nationwide
Lockdown due to the COVID 19 Pandemic. It was not feasible for general society to send in
their remarks as the majority of the postal administrations were suspended during these
uncommon conditions.

\noi
Also, Principle 10 of the Rio Declaration unmistakably expresses that, \textit{"States will encourage
a lot of open mindfulness and investment by making data generally accessible"}.\footnote{\textit{Id.}}

\noi
Ironically no regard has been made to the above principle in the present instance. The Central
Government has merely circulated the Notice and has not made practical for any State
Governments or Organizations to cast their inputs on the same. Owing to the action of
Ministry of Forest, Environment and Climate change, no external evaluation whatsoever has
been propounded by other authorities.

\noi
Subsequently, it is proposed that either as far as possible for submitting remarks be
broadened or the draft be circled, even more generally this time, since the lockdown has been
lifted, so successful public conference can occur.  

\noi
{\large\it\bfseries b.~Reducing duration for written responses by public is a step backwards}

\noi
By virtue of Rule 3.1. of the draft, 20 days have been provided to the public members for
submitting their responses for Public Hearing. This has further reduced the deadline from 30
days under the 2006 Regulations to 20 days.\footnote{Draft EIA 2020, MINISTRY OF ENVIRONMENT, FOREST AND CLIMATE CHANGE,
\url{http://parivesh.nic.in/writereaddata/Draft_EIA_2020.pdf}}

\noi
Principle 10 of the Rio Declaration reads as: \textit{“Environmental issues are best handled with the
participation of all concerned citizens”}.\footnote{Principle 10, Rio Declaration on Environment and Development, 2020.} Even Brian Clark has emphasized that, \textit{“the input
of the public reflects a better understanding of the choices involved than the vote of an
elected official who does not have the time to study each issue in depth”}.\footnote{Brian D. Clark, \textit{Improving Public Participation in Environmental Impact Assessment}, 20 (4), Built
Environment 294, 307, (1994).}

\vspace{-.05cm}

\noi
If reducing the time is the sole key towards achieving the goal, the time limit for
advertisements, endorsements, approvals, etc. could have been reduced. Shifting all the
veracities to public shoulders in order to submit a hurried response serves no use.
Reducing the response time given to the general population for presenting their composed
reactions is really a step-in reverse. 

\vspace{-.05cm}

\noi
The people at large should be provided with sufficient amount of time in order to make a
rational choice and put forth their assumptions and inferences by submitting a detailed
composite response. Eventually, it is they who will get straightforwardly influenced by the
task and along these lines they would need to endure the worst part of taking a rushed choice. 

\vspace{-.05cm}

\noi
Hence, the Ministry should ponder over increasing the deadline for response with respect to
the general population and strengthen their ability to present a composite reaction. 

\vspace{-.02cm}

\noi
{\large\it\bfseries c.~The provision for Post-Facto Approvals is contrary to established principles of law}

\vspace{-.05cm}

\noi
As per Rule 22 of the proposed draft Notification, the Appraisal Committee is constituted to
analyze any cases of violation. 

\vspace{-.05cm}

\noi
If Appraisal committee grants approval to a project, it then becomes eligible to receive the
formulation of a remediation plan, assessment of resource damage and resource and
community augmentation plan.\footnote{Rule 22 (2), Draft EIA Notification, 2020.} Sub-rules 8 and 9 also provide for submission of late fee by
the project proponent.

\vspace{-.05cm}

\noi
These provisions diverted from being strict in nature and contradicts the very purpose of their
presence as against the earlier rules that provided stringent penalties and no possibility of
rectification whatsoever. These rules do nothing but grant provisions for post-facto approvals
which is a blatant violation of the precautionary principle and defeats the very purpose of an
EIA framework. This even runs against the expert committee constituted by UNEP in. 1987
which provided that any framework for EIA must operate “prior” to the beginning of the
project.\footnote{Rule 9 (2), Draft EIA Notification, 2020.}

\vspace{-.05cm}

\noi
The Supreme Court in the latest 2020 case of \textit{Alembic Pharmaceuticals Limited v. Rohit
Prajapati \& Ors}.,\footnote{Alembic Pharmaceuticals Limited v. Rohit Prajapati \& Ors., MANU/SC/0353/2020.} has pointed out in specific terms that that no retrospective approvals can
be given in matters of environmental clearance. Reliance was placed on \textit{Common Cause v.
Union of India}.\footnote{Common Cause v. Union of India, (2017) 9 SCC 499.} 

%~ \vspace{-.07cm}
\vspace{.5cm}
\noi
In 2014, the NGT held that, it is not lawfully acceptable to consider post commencement
examination upon the completion of project. \textit{The importance of conducting an exhaustive EIA
before any project is granted Environmental Clearance has been acknowledged
internationally}.\footnote{S.P. Muthuraman v. Union of India, MANU/GT/0116/2015.}

\vspace{-.02cm}

\noi
{\large\it\bfseries d.~Increase in land for projects which do not need EIA approval from 20,000 to 1,50,000 square meter a retrograde step}

\vspace{-.05cm}

\noi
By virtue of Rule 42 of the draft 2020 Notifications, the territorial extent of foregoing the
EIA requirement has been increased to 1,50,000 sq mts as against 20,000 sq mts under 2006
rules.\footnote{\textit{9 Supra Note 21.}} This leads to the practice that more such construction activities shall now be outside
the domain of the notice.

\vspace{-.02cm}
\noi
{\large\it\bfseries e.~The Number of compliances has been reduced that may lead to incompetency of framework }

\vspace{-.05cm}

\noi
Rule 20 (4) of the draft EIA Notification, 2020 now limits the submission of compliances
reports to annually, i.e., once a year.\footnote{Rule 24, Draft EIA Notification, 2020.} This is a shift from the present law that provides for
half- yearly compliances, i.e., two compliances per year on a gap of six months. 

\vspace{-.02cm}
\noi
{\large\it\bfseries f.~Empowering the Central Government to declare certain projects as “Strategic” could have severe adverse implications}

\vspace{-.05cm}

\noi
Rule 5 (7) of the draft EIA Notification, 2020 further puts that in case the central government
decides any area to be of strategic importance, any information regarding to such projects
will not be placed in the public domain. This is a regression against transparency of actions.

\heading{EIA-Public Hearing: Karnataka High Court Pil W.R.T. Peripheral Ring Road Project in Bengaluru}

\vspace{-.1cm}

\noi
This EIA Draft Notification not only raised a series on debates and controversies but also
gave rise to a number of judicial interventions. For example, in August 2020 hearing a PIL
filed by local NGO namely United Conservation Movement Charitable and Welfare Trust,
the high court of Karnataka ordered a stay on publication of the said notification till its
further hearing31.\footnote{EIA Notifications to remain non- implemented, \url{https://www.downtoearth.org.in/news/environment/eianotification-2020-delayed-till-september-7-72673}}

\vspace{-.1cm}

\noi
Similarly, another collateral petition was filed in the said high court at Bengaluru to highlight
a case of procedural and substantive violation of EIA norms of 2006. This PIL was flied by
students from NLSIU and JGLS. The matter highlights the call taken by the local regulator
KSPCB, directed to only conduct an online appraisal for granting the environmental
clearance for a proposed eight lane peripheral ring road project. The high court admitted the
matter and ordered an interim stay.\footnote{\url{2https://www.livelaw.in/news-updates/bengaluru-peripheral-ring-road-project-karnataka-hc-stops-appraisal-ofeia-report-on-law-students-plea-163407}}

\vspace{-.1cm}

\noi
The HC gave notification to Karnataka State Pollution Control Board (KSPCB) and the
Bangalore Development Authority (BDA), finding out if virtual hearings were adequate and
legitimate under the Environment Impact Assessment notice (EIA), 2006.\footnote{Highlights of EIA Draft Notification 2020, \url{https://www.drishtiias.com/daily-updates/daily-newseditorials/draft-eia-notification-2020}} It requested that
they show how a particular hearing could guarantee the soul of the EIA, 2006. Three
understudies of law appealed to the court September 20, provoking the choice to hold formal
proceedings by means of Zoom — a private, web-based application. The HC just as the
public virtual hearings occurred the exact day; the HC one was before recorded for
September 22 yet was deferred.\footnote{Public Hearing under EIA in times of COVID- 19 Pandemic, \url{https://www.mondaq.com/india/clean-airpollution/990100/public-hearing-amidst-the-covid-19-pandemic}}

\vspace{-.2cm}

\noi
As per the request, the fast EIA study led for the undertaking found that 63 towns, settlements
and homes would be straightforwardly affected by the venture.\footnote{EIA Notification 2020 and its impact on the Environment on the Environment and Society,
\url{https://blog.ipleaders.in/draft-eia-notification-2020-and-its-impact-on-the-environment-and-society/}} It added that 14.32 percent
of the influenced populace had a place with the networks from planned clans and standings;
12.67 percent of influenced family units were going by ladies.\footnote{Environment Clearance Report, Press Trust of India, \url{https://environmentclearance.nic.in/}} The eight-path project,
traversing 65.5 kilometers, was proposed twenty years prior to decongest the city. The BDA
had before conceded that the development would require the expulsion of 33,838 trees.\footnote{Experts slam EIA 2020 Draft for Lack of Public Participation,
\url{https://timesofindia.indiatimes.com/city/nagpur/experts-slam-eia-2020-draft-for-lack-of-public-participation-invetting-projects/articleshow/75543124.cms}}


\heading{What can be the Potential harms?}

\noi
The project has been in news due to various fears, the major of them include –

\noi
non-accessibility of the Detailed Project Report (DPR);

\noi
the legitimacy of the Environmental Impact Assessment (EIA) report;

\noi
the quantity of trees which will be felled for it. 

\noi
However, the major issue that the researchers would like to address here is the futility of the Outer Ring Road Project.

\noi
A 'ring road', by definition, is a detour street that encompasses a town. Its responsibility is to
permit free-streaming traffic that lessens the traffic in the center of the city.\footnote{Ring Road meaning, Cambridge, \url{https://dictionary.cambridge.org/dictionary/english/ring-road}} At present
instance, the city is provided with two ring roads already- The Inner Ring Road and the Outer
Ring Road.\footnote{Bengaluru has Rain Hangover: RIngroads to blame?, \url{https://www.thehindu.com/news/cities/bangalore/Citystill-has-rain-hangover/article16301668.ece}}

\vspace{.2cm}
\noi
In 2015, an investigation demonstrated that The ORR was intended to oblige 5,400 PCUs
(Passenger Car Units) each hour, yet is teared up with only 4,000 PCUs each hour and has in
no substantial way, reduced the load it was meant to lessen. \footnote{Peripheral Ring road zoning, \url{https://bengaluru.citizenmatters.in/prr-peripheral-ring-road-orr-traffic-land-usezoning-rmp-2031-52237}}

\noi
Now, after the construction of the ORR, rather than decongesting, the city, its unchecked
commercialization merely multiplied the traffic congestions and issues. An interesting
expression, if there ever was one.\footnote{Ejipura Hosur Flyover, \url{https://lbb.in/bangalore/ejipura-hosur-road-corridor-flyover/}}

\noi
PRR may witness the same problem. This is so because it has a similar issue of permitting
huge scope business activities in huge lots of land.\footnote{Revision of the Revised Master Plan,\\ \url{https://www.thehindu.com/news/cities/bangalore/bda-withdraws-draftrevised-master-plan-2031-to-review-and-rewrite/article31929610.ece}} This shall serve no purpose but to add the thickness of traffic or vehicles and shall serve as a classic example of the Devil and the Deep
Blue Sea.\footnote{\textit{Id.}}

%~ \vspace{-.1cm}

\heading{Conclusion and Suggestions}

\vspace{-.1cm}

\noi
The COVID-19 pandemic has taught humans the lesson in keeping up the perplexing and
fragile connection between the nature and development. Human advancement ought not to
happen at the expense of harming the environment as this well lead to a total breakdown of
the environmental equilibrium. Remembering this, the Central Government explicitly the
Ministry of Forest, Environment and Climate Change, need to pay notice to the worries of the
different gatherings and work towards fortifying the EIA cycle such that the security and
preservation of the climate turns into the point of convergence while additionally pursuing
reasonable turn of events.

\vspace{-.2cm}

\noi
It appears to be that for the legislators, the driving force behind these modifications have been
the ability to get a geared-up approval from the people in the shadow of the Nationwide
Lockdown. This works by coupling the fact of scarcity of opposition from the general people
in these uncommon pandemics of COVID-19.

\noi
Apart from the aforementioned, problems or tricks such as the nitty gritty meaning of each
term, presentation of the Technical Expert Committees, presentation of Accredited EIA
Consultant Organization (ACO), presentation of online method of entries, diminishing the
timeframe for award or dismissal of EC, presentation for arrangement for offer etc have been
playing a major role in turning these notifications as a regressive departure from the existing
law. 

\noi
Notwithstanding these arrangements, some other recommendations which are at present
lacking under the draft EIA Notification 2020 and might be viewed as added before
conclusive inconvenience are as under:

{\it\bfseries - Strict Timelines for Getting EC and Penalty provisions in case of delay}

\noi
The council may consider consolidating an arrangement for considered endorsement, is that
the candidate may consider the endorsement to have been given on the off chance that the
specialists postpone the equivalent past a specific time limit. Administrative postponement in
allowing ecological freedom should be cut down. This would guarantee opportune removal
on applications for natural leeway.

{\it\bfseries - Mandate on States for wider circulation of the Notification}

\noi
The draft EIA legislation does not presently provide for any provision for adding officials
from the State Governments into the decision-making process. This, in turn directly infringes
Principle 10 of Rio Declaration. 

{\it\bfseries - More Clarity on New Concepts}

\noi
The draft EIA Notification proposes to present a few new ideas which are strange to the
current structure, for instance, an Environment Permission, people group asset expansion
plan, Accredited EIA Consultant Organization and so forth. It is along with these lines
proposed before the last warning, the Ministry to give some further lucidity.

\end{multicols}
\label{end2021-art5}
