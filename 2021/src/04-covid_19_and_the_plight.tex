\setcounter{figure}{0}
\setcounter{table}{0}
\setcounter{footnote}{0}

\articletitle{Covid-19 and The Plight of Animals in India: Safety and\\[4pt] Prevention Approaches}
\articleauthor{Nimita Aksa Pradeep\footnote{Student, Symbiosis Law School, Hyderabad } and Noureen Siddique\footnote{Student, Symbiosis Law School, Hyderabad}}
\lhead[\textit{\textsf{Nimita Aksa Pradeep and Noureen Siddique}}]{}
\rhead[]{\textit{\textsf{Covid-19 and The Plight of....}}}



\begin{multicols}{2}

\heading{Introduction}

\noi
COVID-19, commonly known as coronavirus, has brought numerous changes in the world as
a whole as well as on the individual lives of human beings. From a decrease in economic
growth to a reduction in human interaction and psychological fear and panic, coronavirus has
covered it all through its rapidly spreading nature.\footnote{Matias Carvalho Aguiar Melo \& Douglas de Sousa Soares, \textit{Impact of Social Distancing on Mental Health
During the COVID-19 Pandemic: An Urgent Discussion}, 1 International Journal of Social Psychiatry 1 (2020).}
 Along with the daily lives and well-being
of human beings it has affected on a large-scale. Animals also have their share of hardships
and changes in the atmosphere. As there are several instances recorded that showed cruelty
towards pet animals, stray animals, etc. 

\noi
Pet animals are abandoned in large numbers in many areas. Misconceptions had spread
throughout the world regarding the spread or transmission of the virus by animals.\footnote{Coronavirus Disease (COVID-19) Advice for the Public: Myth Busters, World Health Organisation (May 31,
2020, 12:28 am), \url{https://www.who.int/emergencies/diseases/novel-coronavirus-2019/advice-for-public/mythbusters}. }
 With very
few incidents of animals being affected by the virus and due to the impact of false news
spread through different media sources, cruelty towards animals witnessed a steep increase
during this difficult period. Hunger and starvation of animals remain another issue that needs
action along with the improper treatment of animals by pet shop owners by locking them
without food, etc.

\noi
In the case of India, though the Indian Constitution stresses the importance of compassion
towards animals and the existence of the Prevention of Cruelty to Animals Act, 1960, the
condition of animals during such a pandemic situation is getting worse. This gives rise to the need for intensive research and better formulation of laws considering different scenarios.\footnote{Abha Nandkarni \& Adrija Ghosh, \textit{Broadening the Scope of Liabilities for Cruelty Against Animals: Gauging
the Legal Adequacy of Penal Sanctions Imposed}, 10 NUJS Law Review 1 (2017).}
The fine structure and punishment for such offenses need to be reconsidered taking into
consideration the present situation.\footnote{Id.}
 Lack of proper treaties and guidelines also pave way for
the increase in such cruelty. The involvement of the Judiciary and Animal Welfare
Authorities and the steps taken to reduce these atrocities along with hunger of stray animals.,
are worth appreciating.

\noi
Cases such as {\it {\bfseries Sangeeta Dogra v. Central Zoo Authority of India}}\footnote{Sangeeta Dogra v. Central Zoo Authority of India, Writ Petition (Civil) No. 10856/2020.}
 have been of great
importance in promoting the welfare of animals by reducing their hunger. More initiatives of
similar nature along with the active involvement of people in protecting their pets and other
animals during such a difficult period shall be the base for bringing into existence a proper
environment for them to live in. This should be accompanied by stricter laws to punish those
in violation of the provisions. 

\heading{Background}

\noi
\textit{“Ahimsa”} or “non-violence towards all living beings” is one of the \textit{“Pancha maha vratas”} of
Jainism as well as one of the “five precepts” of Buddhism and thereby form an integral part
of both religions.\footnote{What Each Major Religion Says About Animal Rights, Sentient Media, Nov. 15, 2019.}
\textit{“Ahimsa”} also finds mention in several ancient Hindu texts such as the
Chandogya Upanishad, Sandilya Upanishad, Rig Veda, Yajur Veda, etc.\footnote{Louis Caruana S. J., Different Religions, Different Animal Ethics?, 10 Animal Frontiers 8 (2020). }
 Furthermore, some
of these Upanishads and Vedas explicitly prescribe \textit{“pashu ahimsa”} or “non-violence
towards animals”.\footnote{E. Szucs, R. Geers, T. Jezierski, E. N. Sossidou \& D. M. Broom, \textit{Animal Welfare in Different Human
Cultures, Traditions and Human Faiths}, 25 Asian-Australasian Journal of Animal Sciences 499 (2012).} The concept of \textit{“ahimsa”} also finds mention in the Indian epics such as
the Mahabharata and Ramayana. In furtherance to this, the Mahabharata reiterates the phrase
\textit{“ahimsa paramo dharma”} or “non-violence is the highest moral virtue” multiple times.

\noi
Furthermore, different animals were considered to be avatars of various Hindu deities or
associated with them in some manner or the other under the Sanatana Dharma.\footnote{Aaron S. Gross, \textit{Internal Diversity of Animals in Religion, Religion and Animals} (May. 31, 2020), \url{https://www.oxfordhandbooks.com/view/10.1093/oxfordhb/9780199935420.001.0001/oxfordhb9780199935420-e-10}. } These
animals were hence not only treated well, but they were also oftentimes worshipped and
treated at par with the associated deity.\footnote{Id.} Thus, the protection of animals and animal rights
have both religious and cultural backing in India. Although animal sacrifice was prevalent to
a considerable degree during the ancient period, kings and rulers considered the abovementioned religious and cultural aspects which were duly reflected in their policies. 

\noi
However, with the continuance of British colonial rule in India, the condition of animals in
the country went from bad to worse. Subsequently, after observing the pitiable and deplorable
state of animals in the country, English animal rights activist Colesworthey Grant established
an SPCA (Society for the Prevention of Cruelty to Animals) in the Bengal Presidency in the
year 1861. Later on, several other SPCAs also came up in different parts of the country. They
advocated the prevention of cruelty to animals and campaigned for the enactment of laws and
legislation in this regard. 

\noi
At around the same time, a branch of the Humanitarian League, an English organization
advocating the rights of sentient beings, i.e., both humans and animals, was established in
India. However, their campaigns were influenced to a considerable degree by religion and
religious preferences, and they confined their advocacy to topics such as vegetarianism and
cow protection, ignoring most other aspects of animal protection.\footnote{S. A. A. Tirmizi, \textit{The Cow Protection Movement and Mass Mobilization in Northern India 1882-93}, 40 Proceedings of the Indian History Congress 575 (1979).} More or less, during the
same period, cow protection movements became extremely popular in the northern parts of
the country.\footnote{Id.}

\noi
Post-independence, the constitution-makers incorporated “to have compassion for living
creatures” in the Directive Principles of State Policy. In furtherance to this, the Prevention of Cruelty to Animals Act, 1960 and the Wildlife Protection Act, 1972 were enforced.\footnote{Gilles Tarabout, Compassion for Living Creatures in Indian Law Courts, 10 Religions 383 (2019).} There
also exist various provisions for punishment of certain forms of cruelty to animals in the
Indian Penal Code, 1860.\footnote{Id.}


\heading{Covid-19 and The Condition of Animals in India: Current Scenario}

\noi
The advent of coronavirus or COVID-19 has led to some major changes in the protection and
welfare of animals throughout the world. There are many cases of harassment and cruelty
reported in India against animals of different nature in recent times. The major ways in which
animals are affected during the spread of the virus are starvation and cruelty against stray
animals, abandonment of pets due to misconceptions of people, animals being left in pet
shops without requirements, lack of veterinary facilities and pet food.\footnote{Pragya Tiwari, \textit{Amid COVID-19, Are We Really Looking Out for Stray Animals and Pets?}, The Quint, Apr. 12, 2020.}

\noi
Abandonment of pets is a major concern that has arisen during this pandemic. It is observed
by the People for Animals (PFA) that about 15-20 pets are being abandoned daily in Delhi
alone. Even though it is made clear that animals do not spread coronavirus, people are
ruthlessly abandoning pets in different parts of India and abroad. Also, there are some
instances which include dog, cat, and tiger, having tested positive for the virus. The situation
is becoming worse due to fear and negligence on the part of pet owners. The elders in the
house are tempted to throw away their pets due to the fear of being infected.\footnote{Neha Kirpal, \textit{Friendicoes Society for the Eradication of Cruelty to Animals Rescues Dogs Amid COVID-19}, The New Indian Express, Apr. 26, 2020. }

\noi
There are other incidents of cruelty where animals are brutally beaten up with hard objects
such as rods, etc. Even though these offenses are punishable under Section 11(1) of the
Prevention of Cruelty to Animals Act, 1960, many incidents have been reported of similar
nature.\footnote{9Manka Behl, \textit{Strays and Pets Do Not Spread COVID-19, Treat them Well, Times of India}, Apr. 30, 2020.} People for the Ethical Treatment of Animals (PETA) have raised a plea to take necessary action regarding this issue.\footnote{Special Correspondent, \textit{COVID-19: Assam Police to Book Pet Deserters After PETA India Plea}, The Hindu, May 12, 2020.} Misleading and misinformative advertisements that
led to people abandoning their pets have been made to strike down by authorities such as
Brihanmumbai Municipal Corporation, etc.\footnote{Vijay Singh, \textit{BMC, Other Civic Bodies to Delete False Info Regarding Animals in COVID-19 Advertisements}, Times of India, Mar. 20, 2020.}

\noi
Another area of concern is, animals being locked up without providing due care and
protection in pet shops that remain locked due to the nationwide lockdown. NGOs working
for animal welfare have stated that the situation is horrific wherein the animals are famished,
thirsty, and afraid. There is no proper ventilation in these shops. Most of the dogs, fish, cats,
and birds remain dehydrated and the condition is so worse that they require hospitalisation
immediately. Due to the unavailability of transport facilities and doctors, many of these fishes
and birds lay dead. Various organisations are on a mission to rescue them.22\footnote{Aditi Chattopadhyay, \textit{COVID-19: Animals Left to Die as Owners Abandon Pet Shops in Bengaluru}, The Logical Indian, Apr. 7, 2020.}

\noi
Stray animals have also been targeted to a great extent during this situation of fear and panic.
There have been instances of stray dogs being shot with airguns, etc. Cruelty against stray
animals increases daily due to the misconceptions of people. All this started due to the
headlines in several newspapers stating that stray animals may have played a role in
transmitting the coronavirus disease and other misinformation that spread through social
media and other platforms.\footnote{Mittur N Jagadish, \textit{COVID-19: Do Not Target Stray Dogs}, Deccan Herald, Apr. 24, 2020.}

\noi
Even though Former Union Minister for Women and Child Welfare and Chairperson of
People for Animals, Smt. Maneka Gandhi had issued a statement bursting the myths
regarding the spread of SARS-Cov-2 by animals, such instances continue.\footnote{Theja Ram, \textit{Pet Abandonment, Cruelty Against Strays Rise as COVID-19 Rumours Trigger Fear}, The News Minute, Apr. 13, 2020. } But one good
news is that some people and organisations are putting tremendous efforts into reducing the hunger of stray animals by feeding them at regular intervals which also has great involvement
from the part of the Animal Welfare Board of India (AWBI) and other authorities.\footnote{Feeding Stray Dogs and Other Animals During COVID-19 Lockdown: Delhi HC Seeks Centre, AAP Govt. Stand, The Indian Express, Apr. 27, 2020.}


\heading{Legal Provisions for The Protection of Animals in India}

\noi
Though the Government of India is “of the people, by the people, and for the people”,
animals and animal rights should not be forgotten. The Parliament of India as well as the
State Legislative Assemblies of the country have power under Article 246 of the Indian
Constitution to enact laws and legislations for the “prevention of \underline{cruelty to animals}” as it falls
under Entry 17 of the Concurrent List.\footnote{Bhavya Srivastava, \textit{Laws Relating to Animals}, 3 Journal of Law and Public Policy 1 (2018).} In addition to that, in the case of {\it {\bfseries Animal Welfare
Board of India v. A. Nagaraja}}\footnote{Animal Welfare Board of India v. A. Nagaraja, (2014) 7 SCC 547.}, the Supreme Court stated that the right to life guaranteed
under Article 21 of the Constitution applies to animals as well.28\footnote{Jessamine Therese Mathew \& Ira Chadha Sridhar, \textit{Granting Animal Rights Under the Constitution: A Misplaced Approach? An Analysis in Light of Animal Welfare Board of India v. A. Nagaraja}, 7 NUJS Law Review 349 (2014).} Article 51A(g) of the
Constitution also confers the fundamental duty “to have compassion for living creatures” on
all citizens of the country.\footnote{Rhyddhi Chakraborty, \textit{Animal Ethics and India: Understanding the Connection the Capabilities Approach}, 8
Bangladesh Journal of Bioethics 33 (2017).} In furtherance to this, the Prevention of Cruelty to Animals Act,
1960 was introduced by the Parliament.\footnote{Supra 3.}

\noi
Section 3 of the said Act prescribes that all owners of animals in the country have the duty
“to prevent the infliction of unnecessary pain or suffering” on animals in their custody as well
as “to take all reasonable measures to ensure the well-being” of animals under their care.\footnote{Id.}
Dumping household pets on the street or abandoning them in front of animal shelters during a
global pandemic such as COVID-19 can hence be considered as a violation of the abovementioned duty. Furthermore, Section 11 of the Act stipulates that abandoning an animal “in
circumstances which render it likely to suffer starvation or thirst” is punishable with a fine
ranging from Rs. 10 to Rs. 50 in case of first-time offenders and fine ranging from Rs. 25 to Rs. 100, imprisonment up to 3 months or both together in case of subsequent offenders; if the
subsequent offense is committed within 3 years of the first one.\footnote{Supra 27.}

\noi
The above-mentioned Section also makes the act of “beating, kicking, torturing or causing
unnecessary pain or suffering” to an animal, pet, or otherwise punishable.\footnote{Id.} Furthermore,
“rendering useless, killing, poisoning or maiming” an animal is punishable with fine,
imprisonment up to 2 years or 5 years or both together as per Sections 428 and 429 of the
Indian Penal Code, 1860.\footnote{Supra 24.} Owners who resort to throwing their pets, onto the street or in
front of animal shelters, from moving cars, etc. can hence be brought under the ambit of the
above-mentioned Sections. Despite several countries across the globe having animal
protection laws in place, there is currently no international law or legislation in this regard
though Universal Declaration on Animal Welfare (UDAW) as well as other similar
international treaties have been proposed in recent years.\footnote{Sabine Brels, \textit{A Global Approach to Animal Protection}, 20 Journal of International Wildlife Law and Policy 105 (2017).}

\heading{Legal Provisions for The Protection of Animals in Other Jurisdictions}

\noi
In the {\bf United Kingdom}, the laws regarding cruelty towards animals and protection are very
strict. The legislation speaks about both cruelties committed towards animals and negligence.
In case if such acts are done, the wrongdoers can obtain a lifelong ban on ownership of pets, a
maximum prison term of 51 weeks, and a fine which can extend up to £20,000.36\footnote{Iyan Offor, Animal Welfare, Bilateral Trade Agreements, and Sustainable Development Goal Two, 3 The UK Journal of Animal Law 3 (2019).}

\noi
In {\bf Germany}, the constitution itself states that it is the responsibility of the state to protect the
interests of future generations by conserving natural beings and animals. Germany is the very
first country to protect animals under the Constitution.37\footnote{Erin Evans, \textit{Constitutional Inclusion of Animal Rights in Germany and Switzerland: How Did Animal Protection Become an Issue of National Importance?}, 18 Society and Animals 231 (2010). }

\noi
In the {\bf Netherlands}, The Animal Welfare Act has provisions that deal with cruelty against
animals and its punishments. Also, the duty to care for animals is enumerated in the Act. The
protection also extends to the ban on testing cosmetics on animals.38\footnote{B. K. Boogaard, \textit{Elements of Societal Perception of Farm Animal Welfare: A Quantitative Study in the Netherlands}, 104 Livestock Science 13 (2006).} The law prohibits animal
suffering and recognizes the sentience of animals. 


\noi
In {\bf Austria}, the Animal Welfare Act, 2004 states that the welfare and protection of animals
are equally important to that of humans. The country hosts one of the harshest anti-cruelty
laws in Europe. The law bans owners of pets from cropping their dogs’ ears, tails, etc., has a
provision that states farmers to not put their chicken in cages and ensures that small pet
animals do not swelter in pet shops.39\footnote{Martina Pluda, \textit{Important Novelties in Austrian Animal Welfare Legislation as of 1/4/16}, 8 Derecho Animal Forum of Animal Law Studies 1 (2017).}

\noi
{\bf Switzerland}, has recognized the rights of animals through its constitution and separately
states the importance of protecting the dignity of animals. Any activity that is violative of the
right to dignity of animals is banned in the country.40\footnote{Stefanie Schindler, \textit{The Animal’s Dignity in Swiss Animal Welfare Legislation: Challenges and Opportunities}, 84 European Journal of Pharmaceutics and Biopharmaceutics 2 (2013).}

\heading{Critical Analysis}

\noi
Social distancing and strict sanitation are becoming an essential part of our lives due to the
effect that COVID-19 has brought. Our constant efforts are being channelled for protecting
ourselves and ensuring our safety. During this hustle, there are many instances where
animals, who also have an equal share in safe and protected lives, are being targeted due to
misconceptions and the spread of false information. Due to fear of transmission of the disease
by pets, many of them are being abandoned on the roads and animal welfare homes. This has
led to emotional trauma and hardships for the pet animals.\footnote{Akhil Kadidal, \textit{Owners Throw Pets on Streets Amid Coronavirus Scare, Deccan Herald}, Apr. 9, 2020. } The situation is also such that the
treatment of stray animals has become ruthless and cruel. Animals on the street are being
pelted with stones, beaten up with sticks, etc. and there is also widespread hunger among these animals due to the lockdown period.\footnote{ COVID-19: Stray Animals Feel the Bite as Pandemic Spreads Across World, Indian Express, Apr. 9, 2020.} Pets are locked up in pet shops without
ventilation and food.

\noi
The Supreme Court of India has directed the Central Government to look into the matter of
animal welfare during COVID-19 which includes improving the condition of wild animals,
pets, strays, zoo animals, etc. Food shall be served to these animals, rescue centres shall be
set up, medical facilities be provided in zoos, etc. Also, the police shall have all due right to
take strict action against those who show cruelty towards animals during this period.\footnote{Badri Chatterjee, \textit{SC Directs Centre to Look into Animal Welfare Concerns During Lockdown}, Hindustan Times, Apr. 22, 2020.} The
Animal Welfare Board is also taking measures to ensure the protection of animals during the
current situation. These measures are appreciable to a great extent. But the fact that until and
unless the mentality of human beings changes and more importance is given to the welfare of
all types of living beings, the situation will persist and similar incidents will continue to
occur. The fact that animals also go through their share of physical and mental difficulties
due to acts of human beings should be kept in mind.

\noi
To ensure animal welfare to its truest meaning, instead of fearing the least probability of
spread of the virus by animals, reduction in human interaction should be focused on which is
a sure transmission area.\footnote{Nicola M. A Parry, \textit{COVID-19 and Pets: When Pandemic Meets Panic}, 2 Forensic Science International
Reports 5 (2020).} A multidimensional approach must be brought into force which
involves the government, protection authorities, and laymen for animal protection. This must
include methods to reduce animal hunger, educating people about the spread of incorrect
information, taking steps to take better care of pets, etc. We should also not forget the fact
that the advent of coronavirus started from the rampant mistreatment and improper fostering
of animals in sweat markets.\footnote{David Benatar, \textit{Our Cruel Treatment of Animals Led to the Coronavirus}, New York Times, Apr. 13, 2020.}

\noi
Considering the situation around the world, it is also noticed that to fight loneliness and
boredom during the lockdown, many people have adopted pets and brought them home. The
adoption rate has increased to a great extent during this pandemic. Though it sounds positive at first sight, there are high chances of the pets feeling mentally frustrated once the lockdown
gets over as there shall be instances of lack of care due to a change in the routine of the pet
owners. Also, there are chances of abandonment and taking back pets to animal welfare
centres.\footnote{Heather Fraser, \textit{Abuse and Abandonment: Why Pets are at Risk During this Pandemic}, The Conversation, Apr.
15, 2020.} To avoid such a situation, adopting pets only when one can commit to them after
this period shall be highly recommended.

\noi
Though sufficient legal provisions to protect pets and strays do exist in India, several of them
are obsolete and irrelevant in the present context.47\footnote{Aarefa Johari, \textit{\#NoMore50: Activists are Rising to Demand Harsher Punishments for Cruelty to Animals}, Scroll, May. 16, 2016. } Section 11 of the Prevention of Cruelty to
Animals Act, 1960 prescribes a fine ranging from Rs. 10 to Rs. 50 in case of first offenders
and Rs. 25 to Rs. 100 in case of subsequent offenders is a prime example of such a
provision.\footnote{8Supra 24.} The amount of fine that has to be paid as punishment for violation of the Section
has not changed since the time of the Act’s inception, thereby effectively ignoring the fall in
the value of money and rise in income of people that has occurred to date.\footnote{Bill in Parliament Advocates Huge Punitive Measures for Animal Cruelty, India Legal, May. 27, 2017.} Though Rs. 50
and Rs. 100, the maximum fine prescribed by the Section for first and subsequent offenders
respectively, was a huge amount at the time the Act was enacted; the same has close to no
value now and is therefore no longer a deterrent to crimes committed against animals.\footnote{Sanya Dhingra, \textit{Modi Govt. to Hike Penalty for Cruelty to Animals}, 120 Times, The Print, Dec. 17, 2018.}

\noi
Though amendments to the Act have been proposed in the Parliament more than once, none
of these Bills ever became Acts.\footnote{Bhumika Sharma \& Priyanka Sharma, \textit{Rights of Animals at Practice in India}, 3 Journal on Contemporary
Issues of Law 1 (2017).} It is hence highly advisable that the fine amount be
enhanced. The said Section also alternatively provides for imprisonment up to 3 months.\footnote{\textit{Supra 3.}} In
this regard, increasing the quantum of imprisonment and thereby making the Section more
stringent is advisable to truly deter violation of animal rights in the country. This would also
be in adherence to the judgment in the case of {\it {\bfseries Animal Welfare Board of India v. A.
Nagaraja}}\footnote{\underline{{\it Supra 25}}.} in which the Supreme Court held that the Parliament would do well to amend the Prevention of Cruelty to Animals Act, 1960 such that the “object and purpose” of the Act are
met.\footnote{Vishrut Kansal, \textit{“The Curious Case of Nagaraja in India: Are Animals Still Regarded with No Claim Rights?”},
19 Journal of International Wildlife Law and Policy 256 (2016).} In this case, the Supreme Court also further opined that sufficient “penalty and
punishment” for violation of the provisions of the Act, especially Section 11, should be
imposed.\footnote{Id.}

\noi
Another point for consideration is that despite the existence of the Disaster Management Act,
2005 exclusively for the “effective management of disasters and for matters connected
therewith or incidental thereto”, there is no legal provision concerning the protection of
animals during pandemics such as COVID-19. The fact that the Act does not provide for the
same despite containing provisions to prevent “false warning, false claim, misappropriation
of money or materials, etc.” During disasters it is truly disheartening as this indirectly
connotes that animal rights are not given due weightage in India. Incorporation of such a
provision in the Act is hence highly advisable. 


\heading{Conclusion}

\noi
“To have compassion for living creatures” is our fundamental duty as per Article 51A(g) of
the Constitution of India. Furthermore, Section 3 of the Prevention of Cruelty to Animals
Act, 1960 stipulates the duty that all owners have towards the animals in their custody or
care. Despite this, incidents of dumping of pets on the street as well as abandoning pets in
front of animal shelters are a common feature in newspapers today. Incidents of dogs and cats
with their collars intact being mercilessly pushed onto the pavement in front of animal
shelters and veterinary clinics from moving vehicles: half-dead or gravely injured, though not
as common, also feature in newspapers now and then.

\noi
This is truly heart-wrenching as India was once a country where animals were revered and
worshipped. Furthermore, the concept of \textit{“ahimsa”} or “non-violence towards all living
beings” can be said to be the foundational pillar of several Indian religions such as Hinduism,
Jainism, and Buddhism. Religions such as Islam and Christianity also advocate love and
kindness towards animals. Furthermore, Indian epics such as Mahabharata also propagate \textit{“ahimsa paramo drama”} or “non-violence is the highest moral virtue”. Despite such religious and cultural roots, panic-stricken owners are deserting their pets left and right due to
the false information that is being circulated through newspapers as well as other sources. 

\noi
What we seem to forget, however, is that animals also have the right to life under Article 21
of the Constitution of India. Section 11 of the Prevention of Cruelty to Animals Act, 1960 as
well as Section 428 and 429 of the Indian Penal Code-1860, are the legal provisions that
punish those who abuse animals and violate their rights. However, despite the existence of
such provisions, people continue to make use of the loopholes in the law and escape
punishment despite the gross violation of animals’ rights that they had committed.
Addressing the fallacies in the animal protection laws in the country is hence an absolute
necessity and the need of the hour. 

\heading{Suggestions}

\begin{enumerate}[label=$\bullet$]

\item The amount of fine and quantum of imprisonment prescribed as punishment for violation
of the provisions of the Prevention of Cruelty to Animals Act, 1960, particularly in
Section 11 of the Act, needs to be amended.

\item Provisions in other statutes and legislations in India like the Indian Penal Code, 1860
about the protection of animals such as Sections 428 and 429 need to be made stricter and
more stringent.

\item A legal provision about the protection of animals in the country during pandemics such as
COVID-19 needs to be inserted in the Disaster Management Act, 2005.

\item The misconception amongst the common masses concerning animals being carriers of the
coronavirus needs to be corrected. The general public also needs to be made aware of the
importance of protecting animals and their rights.

\end{enumerate}


\end{multicols}
