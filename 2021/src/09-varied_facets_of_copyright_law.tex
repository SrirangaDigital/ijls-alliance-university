\setcounter{figure}{0}
\setcounter{table}{0}
\setcounter{footnote}{0}

\articletitle{Varied Facets of Copyright Law: Special Reference to\\[4pt] Performer’s and Celebrity Rights}
\articleauthor{Urvashi Sharma \footnote{Assistant Professor, GLS Law College, Ahmedabad }}
\lhead[\textit{\textsf{Urvashi Sharma}}]{}
\rhead[]{\textit{\textsf{Varied Facets of Copyright Law: Special....}}}



\begin{multicols}{2}


\heading{Introduction}

\noi
The powerful human mind is an epicentre of lot many creations of various forms of
properties which refer to as Intellectual property rights. These properties take various forms
which are classified as varied intellectual property like copyright, patent, trademark etc.
These are to be protected in the same manner as any physical form of property for the value
attached to the same. These creations can generate a fortune for some people and be one of
the best ideas of business which can make a country grow. The power is endless, but it
requires protection from the countries and Governments to ensure that there is no dearth of
the innovation and creation in the society for the reason of not enough protection from and
widespread infringement. Thus, the countries have two main reasons to protect the IPs.
Firstly, to keep the reward theory in place and not let the economic justification face any setback by way of infringement. Secondly, to keep the innovation and creativity intact in the
society that leads to the developed society.

\noi
Intellectual Property is divided into two broad categories i.e., the Industrial property and
Copyright. When we refer to Copyright, we are specifically referring to author’s rights. The
rights in the form of creations like books, music, painting, sculptures etc. The main feature of
copyright is the originality and the form of expression. Unless something created is original
and expressed in a tangible form and not just an idea there will be no protection granted for
the same. The ideas cannot always be original, but the form of expression should always be
original to gain the protection. Thus, these things make the features to create a copyright that
can be protected by law. The copyrights are considered as bundle of rights and not just mere
single form of right. Copyright in itself has two important rights economic and moral. The
moral form of right is not available in any other form of intellectual property; this is a unique
feature which remains attached to the copyright no matter it gets transferred to any other person through license or assignment. Thus, copyright is called the right given to the author
and importantly referred as author’s rights. Even if the author dies the moral rights in the
form of right of paternity and right of integrity. It’s not always the economic advantage that
these properties create but it is also the direct link to the author and the name and fame that is
to be protected in these forms of copyrights. The etymological origin of the word “moral
rights,” was taken from the French term “droid moral”, which denotes the non-commercial or
intrinsically attached rights to the author. This concept reduces the original economic
approach to copyright with indefinite character rights.\footnote{PAUL GOLDSTEIN, INTERNATIONAL INTELLECTUAL PROPERTY LAW: CASES AND
MATERIALS, 295 (University Casebook Series, 2001).}

\noi
The basic difference when it comes to invention and literary or copyright is the form it takes
the creation of mind transferred in a form of book, music, sound, painting etc and an
invention would take a different form of technology that is utilised in the industries and in the
market by individuals to ease the problems of the humans. 

\heading{Performer’s Rights Under Indian Copyright Act}

\noi
The Copyright Act, 1957 that was one of the first legislation drafted in by Indian framers as
per the needs but the need of changing it and altering it came very soon and thus the
alteration happened in 2012. A very important amendment was related to the Performers’
right in India. The effect came in the Chapter 8 of the Copyright Act and was also in
consonance with Article 14 of the TRIPS agreement and the similarity can also be seen in
Article 5 to 10 of WPPT Treaty.\footnote{SELVAM \& SELVAM(Dec 10, 2020), \url{https://selvams.com/blog/judicial-recognition-of-performers-rights/}}

\noi
When we specifically talk about the rights attached to the cinematograph movies and the
cinema, all the bundle of rights attached like the music, acting, singing, dramatics everything
becomes the sole economic right of the producer of the cinematography film and the original
authors are just left with minimum rights mainly the moral right not to use the work in a
manner that derogates the original author. The commercial exploitation of the creative
process is useful for the original authors, and it was discussed at various world forum meets
but they were not recognised until the Rome Convention 1961.

\noi
The international treaty had an intention of providing the protection against any usage that is
without permission from the authorized author and thus the broadcasting of the performances
which are outside the purview of use or permissions not granted needed to be protected so
that the performers are the content creators have their performing rights intact. These rights
are essentially very important for the copyright owners.\footnote{Nithin V. Kumar, \textit{Performers Rights under Indian Copyright Act}, (Dec 9, 2020) \url{https://www.bananaip.com/ipnews-center/ip-blog-a-thon-performers-right-under-indian-copyright-law-part-i/}}

\noi
Indian intellectual property laws are mostly derived from the international legislations which
are either the British laws or the international treaties that India becomes signatory to. The
actual laws in India related to the performer’s right were nonexistent before the amendments
made in the IP laws specifically the Section 38 of the Copyright Act. After this amendment
the royalty or the imbursement for the performance was being recognised in the Indian laws
as well.\footnote{Swetha, \textit{Delusion Over Indian Performance Rights Society Being A Part Of Copyright Society}, (Dec 11, 2020),
\url{https://www.mondaq.com/india/copyright/709542/delusion-over-indian-performance-rights-society-being-apart-of-copyright-society}}

\noi
If we look at the specific Section of the Act related to the performer, section 2(qq) of the Act
defines performer which includes wide variety of people showcasing their talent like an
acrobat, a music composer, a singer, actor etc just to name a few. There are the societies also
that have now taken the role to highlight these rights like the PRS i.e. Performing Rights
Society which acts as an intermediary in supporting the rights and managing the royalty
collections for the copyright holders.\footnote{Id.}

\noi
To understand the outlook, the importance of the performers is very remarkable in the whole
entertainment industry. Nothing works if it is not performed and thus the economic gains are
to be protected for the ulterior importance of proper functioning of the same. The performers
have lost considerable amount till date for non-recognition of these rights but now protecting
and recognising these rights is an important step towards the apt protection of IPs in India.
These form a part of not the original form of copyright but the related or the neighbouring
rights that are the ancillary rights attached to the main original rights.\footnote{Aashita Khandelwal, \textit{Performers Rights in India}, Fastforward Justice law Journal. Vol. 2 (1) 2018.}


\noi
Related or neighbouring rights form an ancillary part of the copyright, and they add to the
overall enjoyment of the rights. Intellectual property law generally adheres to the theory that
performers shall be provided the rights in relation to their performances. Performers shall
have the right to manage not only the performing episode but also any further form of
utilization of such performances. These kinds of occurrences need to accord noteworthy
protection to the rights of performers came about with the expansion of technology that
allowed performances to be recorded and broadcast.\footnote{Michael Seadle, \textit{Copyright in the Networked World: Moral Rights}, Vol. 20(1) 124-127 HI-TECH LIBRARY.
2017.}
 The early judgments that came in India
never recognised the performer’s rights and favoured the producer rather than the original
author of the work. Like in the case of \textit{Fortune Films International v Dev Anand}\footnote{AIR 1979 Bom 7}
 it was said
by the court that the actors in a movie do not possess the performers’ rights and it is the
producer that enjoys the ultimate rights. But with the international recognition of the same
and gradual acceptance of the same in the Indian Copyright Act changed the view of the
judiciary as well which led to the wider interpretation and the addition to Section 38 and 39
of the Indian Copyright Act. \textit{The Super Cassettes Industries v. Bathla Cassettes Industries}\footnote{107 (2003) DLT 91}
this case and decision in it marked an important step towards the recognition of the
performer’s rights which in a way stated that the re-recording of any song cannot be done
without the proper approval and consent of the author would amount to violation of the
copyright and the related right attached to it. Original performer has all the rights to consent
for any usage that is done of the copyright, and it was duly recognised in this particular case
and it became a precedent to then recognise and protect the performer’s rights under the
Indian Copyright Act 1957 as well.

\noi
Another important case in the same line is \textit{Neha Bhasin v. Anand Raj Anand}\footnote{2006 (32) PTC 779 Del.} the main issue
here was of what would constitute “live-performance”, it was held in this case that every
performance has to be live at the first instance if it is live or if it is recorded in the studio,
both have to be performed by the performer in the first instance. If the copyright created in such a manner is exploited without the prior consent of the performer than in such a case, it
would be infringement of the performer’s right. 

\heading{Celebrity Rights: Global Perspective}

\noi
In our country we do not have specific rights identified as celebrity rights, but it is an issue
which requires some attention from the public at large. The celebrity life is considered as
public life in lay parlance and there is widespread infringement of privacy rights attached to
their copyrights. The wide misuse of photographs and morphed images is a day-to-day
activity which a celebrity has to be immune with to survive the industry. Tabloids and news
reports without prior permission of the celebrity is also massive infringement. The cost of
being famous is paid with losing the privacy rights and complete denial of privacy in certain
matters. The celebrity at home or walking out in public is always at the risk of losing some
repute and information which in normal circumstances would not be used unless a proper
consent is sorted. In the famous case of \textit{Martin Luther King Jr Center for Social Change v
American Heritage Products Inc}\footnote{250 Ga. 135, 296 S.E.2d 697, 1982 Ga.8 Media L. Rep. 2377} in the present case the facial sculpture of the famous
leader was being sold by a company without the official and prior consent of the family
members. The use of such art for financial gain without permission does infringe the IP right
gained in the form of personality rights being infringed. The damages in such cases are to be
calculated as per the fame the celebrity enjoys. The case also identified the right of publicity
which was considered as distinct from the privacy. The right of publicity identifies that the
name and likeness of the celebrity cannot be used for economic purpose without consent.\footnote{Tabrez Ahmed \& Satya Ranjan Swain, Celebrity Rights: Protection under Copyright laws, Vol. 16
JOURNAL OF INTELLECTUAL PROPERTY RIGHTS 7-16 (2011). }
Looking at the current Copyright Act that we have, the celebrity rights are not identified but
the rights of performers are recognised which can be considered as similar to the above ones.
The contributions of a few well-known intellects cannot be neglected like Kant and Hegel
who have understood the concept of private ownership and property and the theory the not
only the labour but the will of a person can also be instrumental in owning the property
rights.\footnote{G.W.F. HEGEL, PHILOSOPHY OF RIGHT (Oxford Univ. Press 1952).}

\noi
This view leads us to the consideration of personality linked to the ownership of any privately
owned object or the IP rights. The artist leaves a specific personality in making an art form
which can be any form like a song, painting etc and thus creating that kind of extension of
personality needs to be protected and even paid adequately so that these authors are
motivated to create more such art forms and are not de - motivated by the unauthorized usage
of such work.\footnote{Neil Weinstock Netanel, \textit{Copyright Alienability Restrictions and the Enhancement of Author Autonomy:A
Normative Evaluation,(Dec 1, 2020)}, \url{https://www.mondaq.com/india/trademark/777368/celebrity-rights-is-itimportant-in-india#_ftn25}}

\noi
In the case of Sonu Nigam v. Amrik Singh\footnote{MANU/MH/0517/2014.} the case arose when the pictures of famous
Indian singer were used to promote an upcoming music award function without the consent.
The official award pictures used were consented for but the various billboards that were
placed across the city for the promotion were not consented to by the singer and thus violated
the publicity rights attached to the fame of the singer. Bombay High Court recognised and
identified the existence of the personality and publicity rights attached with the fame of
celebrities and thus ordered removal of the hoardings.

\noi
In yet another such case \textit{D.M. Entertainment v Baby Gift House}\footnote{MANU/DE/2043/2010} the case arose when the
defendants in the present case started selling dolls that were the miniature imitations of
likeness of Daler Mehendi. In addition to resemblance of the personality of the singer the doll
can also sing few lines of the famous composition of the singer. Aggrieved from the conduct,
the singer had filed for a case with the help of his company, the Delhi High Court held that
this was the clear infringement of the personality rights that singer Daler Mehendi has gained
because of the celebrity status achieved and thus the consent and approval of the singer is
pre-requisite before using the personality features for commercial activities.

\noi
In the case of \textit{Shivaji Rao Gaikwad v. Varsha Productions}\footnote{(2015) 2 MLJ 548.} the superstar of the south,
Rajnikant had filed for this case in which he alleged that his personality/ dialogue delivery /
persona and his intrinsic celebrity features are to be mocked and copied in a cinematography
filmed named “Main hoon Rajnikant”, which he objected and also stated that the story line and depiction of his name and character in the movie would be derogatory. The court here
held that there is no doubt in the level of fame that the celebrity has gained and the rights
attached to it and thus the facts in the present case and material evidences are easy to prove
the infringement and thus the same were being stopped.

\noi
In \footnote{Rajat Sharma and Another v. Ashok Venkatramani and another}\footnote{CS(COMM) 15/2019} in the present case the
Zee Media house in promotion of their new anchor-less news channel “Zee Hindustan” has
made an advertisement where they used the name of the famous news anchor “Rajat Sharma”
with a few others and stating that who would want to listen to the famous shows that they
telecast once this channel is on Air. This was being alleged as an infringement of the
personality rights that are attached to the famous news anchors. The court relied on the earlier
decisions passed on the same line and principles and identified the celebrity rights that are a
part and parcel of the people who acquire the fame and the use of their name and personality
traits or famous things attached to them have to be done only with the prior consent. 

\heading{Conclusion and Suggestions}

\noi
The Copyright, considered as bundle of rights, provides varies facets to the intellectual
property. The recent trend and addition to these bundles have been the performers or the
recognition of the celebrity rights. This facet is unique in its sense as it would focus on the
neglected relates to the neighbouring rights which are very important and form a very
important form of IP. It is the right time looking at the blatant breach of privacy, performance
and celebrity rights in our country that a separate legislation or much improved form of the
current laws need to be in place to protect the innovative creations and stop the misuse that is
done of the rights which these performers/celebrities acquire by the skill and art that they
have.

\noi
The secondary set of rights in the form of the performer and celebrity right and its protection
is the need of the hour and the growing litigation in this matter shows the importance and the awareness in this relation which the celebrities have gained, and the courts have also given
decisions in the line to protect the same which can be seen as a positive step towards the full
realisation and protection of such rights.

\end{multicols}
