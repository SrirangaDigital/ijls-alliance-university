\setcounter{figure}{0}
\setcounter{table}{0}
\setcounter{footnote}{0}

\articletitle{COVID-19 Lockdown: A Refuge from the Pandemic or the\\[4pt] Harbinger of a Woman’s Agony}\label{2021-art7}
\articleauthor{Shreyaa Mohanty \footnote{2nd year BBALLB student at National Law University Odisha} and Swikruti Mohanty\footnote{2nd year BBALLB student at National Law University Odisha}}
\lhead[\textit{\textsf{Shreyaa Mohanty and Swikruti Mohanty}}]{}
\rhead[]{\textit{\textsf{COVID-19 Lockdown: A Refuge From....}}}



\begin{multicols}{2}

\textit{“Domestic violence is a burden on numerous sectors of the social system and quietly, yet
dramatically, affects the development of a nation… batterers cost nations fortunes in terms of
law enforcement, health care, lost labour and general progress in development. These costs
do not only affect the present generation; what begins as an assault by one person on
another, reverberates through the family and the community into the future.”}

\vspace{-.2cm}

\hfill{{\it\bfseries-Cathy Zimmerman}}

\heading{Introduction}

\noi
Gender-based violence expounds violence that is inflicted upon a woman on the sole basis of
her sex. It entails physical, mental or sexual harm, coercion and denial of a basic liberty. Such
violence includes domestic violence (DV), non-consensual sex and other forms of sexual
violence, trafficking in women, female genital mutilation and dowry-related deaths. Not only
does this have an adverse effect on health status of a woman but also affects her degree of
productivity, the belief of self-sufficiency, confidence and overall quality of life.

\noi
In 1983, domestic violence was finally recognized as a criminal offence in India. However,
until the enactment of the PWDVA\footnote{Protection of Women from Domestic Violence Act, 2005, No. 43 of 2005, India Code [Hereinafter
“PWDVA”]. }
, which came into effect in 2006, there wasn’t any
specific civil law as such to discuss the complexities of domestic abuse, including the
underlying existence of violence within family networks, the urgency for protection and
maintenance of the victims of abuse. The mere punishment and imprisonment of the abuser
does not entail the fact that justice has been served through and through. The in-Toto
recovery of the victim should be the main goal.

\noi
As per the Crime in India Report of 2018 released by the National Crime Records Bureau
(NCRB), every 1.7 minutes in India, there is a crime against women and every 4.4 minutes, a
woman is subjected to domestic violence.\footnote{\textit{Crime In India 2018 Statistics}, 1 NCRB (2018),\\\url{https://ncrb.gov.in/sites/default/files/Crime}\%20in\%20India\%202018\%20-\%20Volume\%201.pdf.}
 The COVID-19 outbreak has only worsened the
case. The National Commission for Women (NCW) registered an increase of 94 per cent in
complaint cases where women were assaulted in their homes during the lockdown. Another
factor that has not received coverage is the growing number of cases where migrant women
walked hundreds of miles with men, some with their children in their advanced stage of
pregnancy, without basic amenities like food. Therefore, owing to the pandemic, nearly half a
billion women are at risk in India. Yet, no formal policy or any detailed COVID response
plan has been proposed by the government to overcome these issues.
 
\heading{Rise in Domestic Violence During Lockdown} 
 
 \noi
In a study, conducted by an American organization, on the aftermath of Hurricane Harvey, it
was observed that stress borne out of disasters increased the occurrence of violence in
families during and post the disaster\footnote{Ashley Abramson, \textit{How COVID-19 may increase domestic violence and child abuse}, AMERICAN
PSYCHOLOGICAL ASSOCIATION (September 12, 2020, 3:44 PM), \url{https://www.apa.org/topics/covid-19/domesticviolence-child-abuse.}}
. A WHO report, upon comparing the violence rates
before and after any disaster, suggests that the shortage of basic amenities, lack of social
support, disruptions to the economy, feelings of helplessness, powerlessness and paucity in
access to basic means of livelihood could have both sudden and indelible effects on violence
in society. The report showed a general trend of increase in sexual violence and intimate
partner violence rates whenever there was an occurrence of disaster.\footnote{Krug EG et al., eds. \textit{World report on violence and health}, World Health Organisation (2002) [Hereinafter
“WHO Report”].} 
 
 \noi
Conditions during COVID-19 appear strikingly similar to those during disasters- loss of jobs,
alienation from social support and economic strain, to name a few. It wouldn’t be ludicrous to
say that such analogous circumstances following the pandemic could actually harbour the
chances of violence in families. Several startling figures were revealed from the first quarter
by the NALSA reports which also revealed that a total of 144 domestic violence cases were reported in Uttarakhand. Moreover, Haryana reported 79 cases while a total of 69 cases were
reported in Delhi.\footnote{\textit{Domestic violence cases in India on the rise during lockdown, says reports}, TIMES OF INDIA (May 18, 2020,
2:00 PM),\\ \url{https://timesofindia.indiatimes.com/life-style/relationships/love-sex/domestic-violence-cases-inindia-on-the-rise-during-lockdown-says-report/articleshow/75801752.cms.}}


\heading{Economic Stress and Domestic Violence:\\ A Reciprocal Relationship}

\noi
Researchers have found out a direct link between financial stress and domestic violence. The
rates of domestic violence seemingly rise with an increase in financial stress. Research also
shows that the repeat victimization of women is seen to be more frequent in cases where the
family is under some sort of financial strain.\footnote{Traci Pederson, \textit{Is Financial Stress a factor in domestic violence}, PSYCHCENTRAL, (2016),
\url{https://psychcentral.com/news/2016/04/26/is-financial-stress-a-factor-in-domestic-violence/102318.html.}}

\noi
Now, the cause-and-effect relationship could be applied the other way round as well.
Domestic violence may even bring distress and poverty to women, entrapping them in a
vicious cycle of poverty and abuse. The decision of a woman to leave her abusive partner,
more of often than not, depends on her economic backing or condition. For example, a
woman with not much resources and means to support herself economically could experience
severe financial stress while deciding to leave her abusive partner. Sometimes, women just
succumb to these financial ambiguities, coupled with the lack of empowerment and selfworth, and decide to stay with their abusers for the rest of their lives.

\noi
On March 24$^{\rm th}$, India entered into the first phase of lockdown together with the lockout of all
public transport systems, restaurants, offices, factories and educational institutions. In what
we envisaged would last for only 21 days but continued for months-long and disrupted the
Indian economy to the core. By the first week of April, economists had stated that there had
been a job loss of approximately 40 million people in the unorganised sector.\footnote{Tanisha Mukherjee and Nilanjan Ray and Sudin Bag, \textit{Opinion: Impact of Covid-19 on the Indian Economy},
ET GOVERNMENT, (Oct. 20, 2020), \url{https://government.economictimes.indiatimes.com/news/economy/opinionimpact-of-covid-19-on-the-indian-economy/75021731.}}
 Stringent
lockdown rules restricted most of the economic activities, causing millions to lose their
source of income. By June, more or less 84\% of the Indian households had seen a decrease in their income.\footnote{How the COVID-19 \textit{Lockdown Is Affecting India’s Households}, KNOWLEDGE@WHARTON, (Jun. 09, 2020),\\
\url{https://knowledge.wharton.upenn.edu/article/covid-19-lockdown-affecting-indiashouseholds/}\\\#::text=Rural\%20households\%20have\%20seen\%20disproportionately,resilience\%E2\%80\%9D\%2
0than\%20their\%20rural\%20\\ counterparts.} Such economic distress stemmed anxiety and feelings of helplessness among
people. The abusers projected their frustration by inflicting a higher level of abuse on women. 
 
\noi 
By the end of May, the numbers of domestic violence complaints were on a ten-year high.
India generally has an underreporting problem when it comes to domestic violence, where
90\% of the victims seek help from their friends and immediate family members.\footnote{Vignesh Radhakrishnan and Sumant Sen and Naresh Singaravelu, \textit{Data | Domestic violence complaints at a
10-year high during COVID-19 lockdown}, THE HINDU, (Jun. 22, 2020, 12:04 AM),\\
\url{https://www.thehindu.com/data/data-domestic-violence-complaints-at-a-10-year-high-during-covid-19-
lockdown/article31885001.ece.}} However,
given the lockdown, the victims faced a dearth of social support, where otherwise they could
have sought shelter and help. Generally, victims could flee and find shelter elsewhere, which
isn’t possible during lockdown. Lack of social support is one of the major factors that foster
domestic violence. This situation is not something exclusive to India; many women across the
globe are facing similar problems of being locked up with their abusers. 

%~ \vspace{.05cm}

\heading{More Complaints From Red Zone Areas}

\noi
The Central Government mandated the nationwide lockdown, dividing all the districts in the
country into three zones: Red, Orange and green, with 130 districts being listed under ‘Red
Zone’ with the most stringent lockdown provisions.\footnote{Bindu Shajan Perappadan, Corona virus | \textit{Health Ministry identifies 130 districts as red zones}, THE HINDU,
(May 01, 2020, 11:07 AM), \url{https://www.thehindu.com/news/national/coronavirus-india-lists-red-zonesas-it-extends-lockdown-till-may-17/article31478592.ece.}} Several researchers from the University
of California, Los Angeles and UCLA studied the variations in the number of reported crimes
between 2018 and 2020, categorising the crimes into four types- domestic violence,
harassment, cyber-crimes and rape and sexual assault. After scrutinizing the reports, they
found out that the districts with more stringent lockdown measures (Red Zones) recorded a
131\% increase in domestic violence complaints compared to that in the green zones where
there were less stringent lockdown provisions.\footnote{\textit{Domestic violence complaints peaked in red zones during lockdown: Study}, DECCAN HERALD, (Jul. 24, 2020,
11:41 AM),\\ \url{https://www.deccanherald.com/national/domestic-violence-complaints-peaked-in-red-zones-duringlockdown-study-864965.html.}} The same report also showed the cases of
rape, and sexual assault fell down by almost 66\% in the red zone areas. This could be attributed to the fact that people barely came out of their homes which meant less mobility in
offices and public places. The researchers also gathered Google Trends data and Google
Community Mobility Reports which indicated that the frequency of search of the terms
“domestic abuse” and “domestic abuse help-lines” had been increasing since mid-March.
Perhaps, this report made one thing certain - violence against women remained constant.
While the lockdown took the edge of certain crimes like rapes and sexual assaults, it
definitely aggravated the prevalence of other forms of violence like cybercrimes and
domestic violence.

\heading{The Psychological Ramification Post Abuse}

\vspace{.08cm}

\noi
The deleterious effect of Lockdown is no secret, and the plight of the victims, being locked in
with their abusers, coupled with the general state of uncertainty has led to a surge in the
number of domestic abuse complaints in the past few months. The victims are subjected to a
portfolio of abuse, both physical and psychological. The general emphasis is given on
physical sufferings; however, not much is anticipated or discussed about the psychological
trauma that these victims go through.

\vspace{.08cm}

\noi
Economic and psychological stress, followed by isolation due to being locked-in had
disrupted the natural and social environments of a lot of people, making them feel helpless
and vulnerable. The abusers now compensate for their lack of control by exerting mental and
physical violence on the victims.\footnote{Shelly M. Wagers, \textit{Domestic violence growing in wake of corona virus outbreak}, THE CONVERSATION, (Apr.
08, 2020, 10:11 PM), \url{https://theconversation.com/domestic-violence-growing-in-wake-of-coronavirus-outbreak135598.}} The physical injuries endured by the victims are no secret:
bite marks, cuts, bruises, loss of vision and hearing, knife cuts, and sexually transmitted
diseases which sometimes even lead to death.\footnote{Kavita Alejo, \textit{Long-Term Physical and Mental Health Effects of Domestic Violence}, 2 Themis: Res. J.J.S.F.S.
(2014), \url{https://scholarworks.sjsu.edu/cgi/viewcontent.cgi?article=1016\&context=themis.}} A study conducted on domestic violence
victims shows that among the women that have reported being abused, nearly 50\% of them
were found to be malnourished.\footnote{Leland K. Ackerson and S.V. Subramanian, \textit{Domestic Violence and Chronic malnutrition among women and
children}, 167 (10) Amr. J. Epi. (2008), \url{https://academic.oup.com/aje/article/167/10/1188/232214.}}

\vspace{.08cm}

\noi
Women are also subjected to psychological abuse like demeaning, belittlement and insults,
threats of being abandoned, threats of hurting someone they care about or general infidelity
of the husband. The psychological effects of abuse are more deep-seated and unrealized.
More often than not, domestic abuse victims get diagnosed with depression and PTSD. It has
been found out that depression in abused women tends to be chronic and has a life-long effect
even in the absence of abuse for a long time.\footnote{Dr R.C. Ahuja, et al., \textit{Domestic Violence: A Summary Report of a Multi-Site Household Survey}, International
Center for Research on Women and The Centre for Development and Population Activities, (2000),\\
\url{https://www.icrw.org/wp-content/uploads/2016/10/Domestic-Violence-in-India-3-A-Summary-Report-of-aMulti-Site-Household-Survey.pdf.}}

\vspace{-.07cm}

\noi
The rate of PTSD among women who have a history of domestic violence ranges from 30\%
to 81\%, which is way more than the rate of PTSD among the general community of women.\footnote{Mindy B. Mechanic and Terri L. Weaver and Patricia A. Resick, \textit{Mental Health Consequences of Intimate
Partner Abuse: A Multidimensional Assessment of Four Different Forms of Abuse}, NCBI (2010),
\url{https://www.ncbi.nlm.nih.gov/pmc/articles/PMC2967430/#R31.}}
The victims have frequent episodes of anxiety attacks. Research shows that women that have
been sexually abused have faced more severe physical abuse than the women that were only
battered. When it comes to psychological disorders, there is no stark difference between the
victims of marital rape and stranger rape.\footnote{Jennifer A. Bennice and others, \textit{The Relative Effects of Intimate Partner Physical and Sexual Violence on
Post-Traumatic Stress Disorder Symptomatology}, NCBI (2010),
\url{https://www.ncbi.nlm.nih.gov/pmc/articles/PMC2981038/#R16.}} Feelings of unworthiness and hopelessness,
coupled with a lack of self-esteem and apprehension of future abuse often culminate into
suicidal thoughts in the minds of the victims. Needless to say, the lack of social and
emotional support often leaves the victims feeling isolated and alone in their battles. 

\vspace{-.07cm}

\noi
Several women have lost their jobs and sources of earning, owing to the pandemic. This has
led to the loss of some level of empowerment that these women had. Several researches back
“economic dependence” of women on their husbands as a predominant reason for women to
stay in abusive relationships. With the loss of empowerment, women have now accepted their
fates, being juggled in the hands of their abusers. Being trapped with their perpetrators,
women are left with little to no hope to ask for help or find shelter, and in the meantime, they
have to fight their abusers, being isolated from any support. 

\vspace{-.1cm}

\heading{Laws Dealing With Domestic Violence in India}

\vspace{-.1cm}

\noi
There are several laws which protect a married woman from being abused by her husband or
any in-laws for that matter.

\vspace{-.15cm}

\heading{Section 498A of India Penal Code}

\vspace{-.15cm}

\noi
It states that if a woman’s husband or his relatives subject her to harassment or any act of
cruelty, they’ll be liable for imprisonment that might extend up to three years as well as fine.
The term cruelty has been defined under the same section as any act that amounts to coercion
for dowry demands from the woman or her family members or any activity that abets the
woman to commit suicide or inflict grievous injury upon herself (mental or physical). But,
since marital rape for women above 15 years of age has not been explicitly recognised under
the ambit of “cruelty”, victims often have to rely on PWDVA to seek justice.

\vspace{-.15cm}

\heading{Protection of Women From Domestic Violence Act 2005}

\vspace{-.15cm}

\noi
It prohibits a broad range of physical, mental, sexual and economic violence against women,
and all of these are exhaustively described under the Act. The ambit of the Act includes
women who are in a live-in relationship as well. Under this Act, a woman has the right to be
free from abuse and can choose from different recourses. She has the right to get a restraining
order against her husband and his relatives, to continue living in the same house, i.e. even
after reporting her abusers, to claim maintenance, to have custody of her children and to
claim compensation and to not be thrown out of her marital home.

\vspace{-.15cm}

\heading{Family Court Act, 1984}

\vspace{-.15cm}

\noi
The 59$^{\rm th}$ report of the Law Commission laid emphasis on the institution of distinct courts to
deal with matrimonial, personal and family issues. Upon establishment, such courts would
help in speedy disposal of matrimonial and personal issues that are long pending in civil and
criminal courts. The Supreme Court in \textit{K.A Abdul Jaleel v T.A. Shahida}\footnote{K.A. Abdul Jaleel v. T.A. Shahida, Appeal (civil) 3322 of 2003 (India).} had said ,"The
Family Court was set up for settlement of family disputes and the reason for the enactment of
the said Act was to set up a court that would deal with disputes concerning the family
by adopting an approach radically different from that adopted in ordinary civil proceedings.” 

\heading{Delay in Criminal Justice: The Loopholes}

\noi
As collected by the National Crime Records Bureau (NCRB), data under PWDVA covers
entirely the criminal violation of the orders of the courts under PWDVA, including the cases
of the contempt of a courts’ protection order passed when the case is still in trial. According
to NCRB data, cases reported under the PWDVA breach rose by 8 percent, from 426 in 2014
to 461 in 2015.\footnote{Manisha Chachra, \textit{Ten Years Of Domestic Violence Act: Dearth Of Data, Delayed Justice}, HINDUSTAN
TIMES (2017).} This implies that the data does not include the actual data recorded under
Section 306, 304B and 498A addressing dowry deaths, torture and cruelty by the husband and
in-laws, and abetment to commit suicide respectively. However, under PWDVA, which is a
civil law, cases are directly listed with the court with respect to matters dealing with
maintenance in domestic abuse cases, and protection of the victim from the in-laws and the
husband and not recorded by the NCRB. There have been numerous attempts made by
women's rights organisations, however, this data on court proceedings have remained
unavailable. 

\noi
What makes the process more tedious is the involvement of different authorities like
protection officers (PO), service providers, lawyers, magistrates to ensure justice and
rehabilitation of the victim. Each level of implementation comes with its own sets of
complexities. POs are frequently overwhelmed with various cases simultaneously and are not
adequately guided and trained in enforcing such provisions of the law. More than 50\% of PO
still consider domestic violence as more of a family affair and urge the complainant to sort it
amongst the family.\footnote{\textit{Id.}}

\noi
Another drawback in the law's implementation is that service providers do not have a
standardised protocol. In such scenarios, service providers are primarily NGOs, and usually
have no contact with the POs or the police. They are mostly not qualified under the Domestic
Violence Act or instructed on how to deal with domestic violence cases. While some of the
aggrieved women are directed to shelter homes, they are mostly overcrowded and in poor
conditions with no means for women to be self-sufficient. So, the victims have no option but
to relocate or be homeless. 

\noi
A lot of lawyers are not acquainted with the notion of service providers (SP) and hence fall
back in coordinating with the domestic violence victim in the timely delivery of proper legal
services. Moreover, the judiciary is not even remotely aware of the SP’s roles in counselling
or even the filing of the Domestic Incident Report (DIR). Neither do the magistrates follow
the procedures provided for speedy trials. 

\heading{Under Reporting Problem in India}

\noi
The NFHS report that is published every 10 years shows that a majority of the victims of
domestic violence don’t register any formal complaint or follow the institutional routes. Both
personal (humiliation, fear of retribution, financial dependency) and social influences
(distorted power dynamics of men and women in society, family privacy, victim-blaming
tendencies) are the reasons too many cases go unreported. But we still need to know whether
all those unreported incidents are still invisible to the victim's social community or not. The
silence and repression of those who know and victim blaming behaviours lead to the creation
of an environment of tolerance that decreases inhibitions towards abuse, making it more
difficult for women to come forward, and encourages social defeatism.\footnote{Elizabeth Shrader and Monserrat Sagot, \textit{Domestic violence: women’s way out}, PAN AMR. H. ORG. (2000),\\
\url{https://www.paho.org/hq/dmdocuments/2011/GDR-Domestic-Violance-Way-Out-EN.pdf.}} Among the social
aspects that influenced violence rates are those that create an environment conducive for
violence.\footnote{WHO Report, \textit{supra} note 4.}

\noi
Healthcare providers hardly ever search for signs of abuse or ask women about abuse
experiences, though most women prefer regular questioning about domestic violence by their
doctors. India appears to lack mandated mechanisms such as regular screening and
monitoring by hospitals when women visit with suspected injuries.\footnote{Fiona Bradley, et al., \textit{Reported frequency of domestic violence: cross sectional survey of women attending
general practice}, NCBI (2002), \url{https://www.ncbi.nlm.nih.gov/pmc/articles/PMC65059/>}.}

\heading{Fundamental Barriers to Seeking Remedy}

\noi
Like any other case, cases under domestic violence also require the investment of time,
energy, courage and money. Although women, under the Indian Constitution, are entitled to
free legal aid\footnote{INDIA CONST. art. 39A. } and although the statute clearly states the duties of disclosing to the woman, who has been subjected to domestic violence, that they are entitled to free legal aid,\footnote{ PWDVA, § 5(d).} we
don’t see such provisions being disclosed to the victim. Thus, they end up appointing
advocates and paying for the costly process. 

\noi
The PWDVA talks about the first hearing happening within three days\footnote{PWDVA, § 12(4).} and thereafter the
application to be disposed of by the court in 60 days.\footnote{PWDVA, § 12(5).} Nevertheless, it seldom happens in
reality. That is because, to a certain extent, the Act itself provides for a provision of appeal.\footnote{PWDVA, § 29.}
So, the moment one order is passed by the court, and there is a slimmest of chance to appeal
against it, immediately an appeal is filed. Thereafter, the matter keeps dragging, ultimately
moving up to the High Court. The cumbersome process ultimately bears down on the victim.

\noi
There is a provision under section 23 in the PWDVA\footnote{\textit{Id.}} for granting ex parte interim relief.
Most of the time, when a complaint is filed, we hardly see the court exercising this power.
Rather the court adopts the other option, serves notice, lets the other side appear, lets them
file an objection and then takes up interim hearing. This oftentimes acts as a disadvantage for
the victim whose safety is at imminent risk. 

\noi
Home is a site of violence, where violence is normalised. The victim and the perpetrator are
in the same place, and this happens because of power imbalance. The Act definitely changed
the general regime of laws dealing with violence against women in India; however, the
Judiciary needs to approach the cases in a prompt manner. Apart from the mandatory
physical and sexual medical exams post abuse, a psychological evaluation should also be
done because the moment psychological trauma enters in to play, it strengthens the victim’s
case and helps in getting relief. Ex parte interim orders should be passed in cases where there
is imminent danger to the victim, and not just in exceptional cases. Also, there should be an
adequate number of protection officers present at any given time. The \textit{Gujarat High Court in
Suo Moto v State of Gujarat \&Ors},\footnote{Suo Motu v. State of Gujarat and Ors., MANU 2013 GJ 0302 (India).} while deciding on the matter of an inadequate number
of protection, the officers present in Ahmadabad, held that “need of the hour was that Government assesses needs of each District and accordingly, appoint an adequate number of
Protection Officers in each District to receive and attend complaints in time”.

\heading{The Response of The Judicial System Based on Deep Rooted Prejudices}

\noi
While Section 498(A)\footnote{Indian Penal Code, 1860, § 498A.} provides a massive list of activities that are considered aggressive,
judgement analysis indicates that adjudicators frequently let their perception of
progressivism, patriarchal hegemony and family responsibilities affect their evaluation of the
degree of crime and violence involved in these cases. Besides, many judges order detailed
medical and forensic evidence before convicting suspects, a few judges convict on the basis
of testimony from eyewitnesses, dying statements and precedents, and the absence of medical
proof demonstrating previous violence is not considered necessary for convictions to be
obtained. Therefore, what qualifies as facts in the court of one judge might not be scrutinised
by another judge, highlighting contradictions inherent in the evidentiary procedures and the
arbitrary handling of domestic abuse proceedings by judges.

\noi
While the concept of a ‘Dynamic Court’ tends to suggest that courts are capable of
establishing systemic change and there are instances of such revolutionary jurisprudence in
India, there are also various ways in which the criminal justice system fails victims of
domestic violence like the innate prejudices of police and lack of efficiency in investigation,
to name a few. Some defence interventions, with the implicit or outright collusion of the
judicial system, take a shape which is often obscure and invisible. These are intentionally
used in cases that have a gender aspect; for example, the systemic negation of victims’
statements of abuse by the use of particular discursive devices such as the passive voice to
minimise the perpetrators’ accountability, deflecting the blame from perpetrators to victims
and making assumptions about the mental health of victims without any legitimate medical
reason. 

\vspace{-.1cm}

\heading{Measures Taken in India to Combat Increase\\ in Domestic Violence Post COVID-19\\ Lockdown}

\vspace{-.1cm}

\noi
Lockdown 1.0 saw a rapid upsurge in the complaints that were being reported at the National
Commission for Women (NCW). In between the first week of March and the start of April,
the number of domestic violence cases increased by twice the previous rates in India.\footnote{Jagriti Chandra, \textit{National Commission For Women Records A Rise In Complaints Since The Start Of
Lockdown}, THE HINDU (Apr. 03, 2020, 2:22 AM), \url{https://www.thehindu.com/news/national/nationalcommission-for-women-records-a-rise-in-complaints-since-the-start-of-lockdown/article31241492.ece.}} While
most of the countries had rolled out a lot of safety measures in anticipation of a rise in
domestic violence cases, it wasn’t until April 10 that the NCW announced a special
WhatsApp helpline number. Various other helpline numbers by different state government
and the central government followed suit. In the initial lockdown period placed as of March
25th 2020, the Courts constrained their operation to dealing with demanding and necessary
matters via video-conferencing. The primary statute that deals with domestic violence
matters, Protection of Women from Domestic Violence Act 2005, falls within the ambit of
civil laws, and the Court during this period, did not hear matters and cases falling under this
category. Additionally, Solicitor General in \textit{“All India Council of Human Rights, Liberties
and Social Justice v Union of India311”} said that, a complaint portal has been started by the
National Commission of Women as well as a WhatsApp number has also been released to
help women facing violence.\footnote{Gananath Pattnaik v. State of Orissa, (2002) 2 SCC 619 (India).} Further, Information and Broadcasting Ministry has called on
all the radio channels and private satellite TV channels to assess information on the
emergency response support system (121) operating for safety of women and women in
difficult situation. 

\noi
The High Court of Kashmir, in matter of \textit{Court on Its Own Motion v Union Territories of
Jammu \& Kashmir and Ladakh},\footnote{Court on Its Own Motion v. Union Territories of Jammu \& Kashmir and Ladakh, WP (C) PIL No. 14/2020
(Through Video Conferencing) (India).} took suo motu cognizance of the rise in the intimate partner
violence cases in the state and issued a verdict recommending various measures like
increased call-in service availability to encourage anonymous and protected reporting of
violence, establishment of dedicated funding to resolve issues of violence against women and girls in relation to the Jammu and Kashmir Union Territories and Ladakh Territories'
response to COVID-19, providing immediate media coverage with regard to all the abovementioned steps, as well as the provision of services for finding relief and redress against
domestic abuse urgent designation of safe spaces as shelters for women obliged to flee their
household situation. It is important to treat these shelters as open and accessible shelters. The
court also ordered the assigning of informal safe zones for women, such as convenience
stores, local pharmacies, where domestic violence or harassment can be reported without
alerting the offenders. Lastly increased legal and counselling support for girls and women
through an online medium was also suggested.\footnote{\textit{Id.}}

\vspace{-.1cm}

\noi
While these measures seem achievable in theory, the same cannot be said for the practical
implementations. A complete transition to therapy across phones and online media exposed
the disparities in women's access to communication networks as it left women with little
opportunity to reach out from underprivileged communities. It was further pointed out by the
women advocacy groups that NCW received grievances only (and no longer by post) through
emails and WhatsApp. Women from only a few sections of the vast demographic section of
women have access and are literate to use these technologies. The NCW president, too,
noticed that most complaints were generally received by the commission (NCW) not by mail,
but by post.\footnote{Nomfundo Ramalekana and Aradhana Cherupara Vadekkethil and Meghan Campbell, \textit{Increase Of Domestic
Violence}, OX. H. R. HUB (2020), \url{https://ohrh.law.ox.ac.uk/wordpress/wp-content/uploads/2020/06/OxHRHSubmission-to-UNSR-on-Violence.pdf.}}

\vspace{-.1cm}

\noi
While making all kinds of stringent policies to ensure an effective lockdown, the government
certainly didn’t take the downside of lockdown into consideration, and hence no explicit
exemption measures as such were provided for the victims of domestic abuse. 

\vspace{-.1cm}

\heading{Curious Case of South Africa and Sweden}

\vspace{-.1cm}

\noi
While the world was preparing to go under lockdown amidst the chaos of the pandemic
outbreak, South Africa was dealing with another despairing crisis of its own: the genderbased violence, which was anticipated to hit an all-time high with the abusers and the victims being locked away in the same house for months to come. What the country wasn’t expecting
was a drop in the cases being reported by 69.4\% between March and April.\footnote{Elizabeth Dartnall and Angelica Pino and AnikGevers, \textit{Domestic Violence During COVID-19: Are We Asking
The Right Questions?} (Jun. 26, 2020), \url{https://reliefweb.int/report/south-africa/domestic-violence-during-covid19-are-we-asking-right-questions.}}

\noi
In several provinces, this rising concern regarding the intimate partner abuse had a
constructive effect. In such scenarios, the Social Welfare Department had partnered actively
with NGO-run shelters to ensure resources and help are open and available to the domestic
abuse victims.\footnote{\textit{Gender-Based Violence During Lockdown: Looking For Answers - ISS Africa}, ISS AFRICA (May 11, 2020),
\url{https://issafrica.org/iss-today/gender-based-violence-during-lockdown-looking-for-answers.}} It also helped NGOs like Rape Crisis to adapt rapidly by providing accessible
online and supplementary resources for telephone reporting and therapy. Thuthuzela Care
Centres remained open (one-stop facilities for victims of sexual offence at state hospitals).
And there had been extensive pollicisation of a nationwide gender-based abuse hotline.
Neither of these programmes had, however, seen a major rise in incidents.\footnote{\textit{Id.}} At three
Thuthuzela centres in the Cape Town metro, the National Prosecuting Authority closely
worked with the Rape Crisis Cape Town Trust, the police and the Social Development
Department. In rape and sexual harassment cases against women during this time, Director
Kathleen Dey reported an approximately 50 percent decrease.\footnote{\textit{Id.}}

\noi
This change in trend might seem shady from afar, but there are a few possibilities that can
substantiate the same, one of them being the ban on alcohol sales. The ban on sale of alcohol
reduced the prevalence of cases of domestic abuse and the risk of stranger rape. Sober
partners, despite still having the tendency to be aggressive and oppressive, would be less
inclined to device excess physical abuse.

\noi
Refuge, a British domestic violence NGO, recorded a 700 percent rise in victims' calls and a
25 percent rise in men's calls to change their actions.\footnote{Jamie Grierson, \textit{UK Domestic Abuse Helplines Report Surge In Calls During Lockdown}, THE GUARDIAN
(Apr. 09, 2020, 10:30 AM), \url{https://www.theguardian.com/society/2020/apr/09/uk-domestic-abuse-helplinesreport-surge-in-calls-during-lockdown.}} This means that the ban on sale of
alcohol and the mandated lockdown might be the reason behind the fall in the number of
recorded cases and the decrease in cases of serious abuse and injuries in South. 

\noi
On the flip side, there was an increase in the cases of police brutality (with a 12 percent
increase in the cases being reported to the Independent Police Investigative Directorate) and
the tight lockout may also have kept women stuck in their homes. This might deter them from
venturing out since they do not have any legitimate excuse to offer to their perpetrator, or
they might be afraid of fear questioned by the police.\footnote{Katie Trippe, \textit{Pandemic Policing: South Africa’s Most Vulnerable Face A Sharp Increase In Police-Related
Brutality} (Jun. 18, 2020), \url{https://www.atlanticcouncil.org/blogs/africasource/pandemic-policing-south-africasmost-vulnerable-face-a-sharp-increase-in-police-related-brutality/.}} This might lead to women not being
able to report the violence since they might not be able to call one of the helpline numbers
without the fear of being caught by their perpetrators or they simply can't get to a police
station. This will contribute to the number of recorded cases declining. Under the lockdown
law, people are authorised to travel in order to provide access to essential resources. The
Lockdown Regulation describes essential services that include social work, gender-based
abuse and recovery therapy services. In principle, therefore, domestic violence victims are
being able to leave their homes and seek help and support, getting access to relief homes and
shelters for domestic abuse. However, in actuality, reports of military and police brutality
against individuals deemed to be contravening the lockdown regulations, especially in poor
low-income areas, are likely to dissuade many women from leaving their homes and seeking
support.

\noi
Another possibility – one that we may have disregarded – is that the shutdown and the
situation brought about by the COVID-19 pandemic has altered habits of violence, for good,
and there has indeed been a drop in incidence of violence. Notwithstanding how strange this
scenario might be, it would be an error on the part of researchers and activists not to consider
every possible alternative.

\noi
A topsy turvy trend was seen in Sweden, which is deemed as one of the most progressive
countries on the gender-based aspects. Since the beginning of the Covid-19 crisis, there has
been a strong soar in domestic violence incidents in Sweden, despite the fact that the Swedish
policy response was comparably lax and that an absolute lockdown had so far been avoided.
One argument that might possibly explain the increased domestic violence during the outbreak and lockdowns elsewhere could be the decreased availability of sex services. A
conjecture is sure to be checked in future work.\footnote{Giancarlo Spagnolo, et al., \textit{The Role of Prostitution Markets In The Surge of Domestic Violence During Covid-19}, VOX CEPR (September 13, 2020), \url{https://voxeu.org/article/role-prostitution-markets-domestic-violenceduring-covid-19.}}

\noi
Domestic violence seems to be the systemic adverse effect of limiting the availability of sex
services. A likely reduction in prostitution offering during the Covid-19 pandemic may also
have encouraged domestic violence. As the first anti-symmetric criminalization of
prostitution was introduced in Sweden in 1999, punishing buyers, but not sellers of sexual
services, a third path between criminalization and legalisation seemed to have been
identified.\footnote{Arthur Gould, \textit{The Criminalisation Of Buying Sex: The Politics Of Prostitution In Sweden}, 30 J. SOC. POL.
(Aug. 06, 2001),\\ \url{https://www.cambridge.org/core/journals/journal-of-social-policy/article/criminalisation-ofbuying-sex-the-politics-of-prostitution-in-sweden/4349F5BC49487E902AD4ECD95AA22753.}} Therefore, it is important to recognise the implications and counterproductive
consequences, that such policies can have, while contemplating different types of
criminalization or prohibitions on conduct, such as those introduced during the pandemic.

\heading{International Organizations on The\\ Impending State of Affairs}

\noi
Amid the struggles of battling the virus, the rise in domestic violence post the onset of the
pandemic has become a global issue. Several International Organizations have raised their
concerns towards the deteriorating conditions of women around the globe and have urged the
governments of all countries to pay heed towards the safety of women and make it their
utmost priority.

\heading{United Nations}

\noi
The United Nations reported a steep rise in the number of calls made to domestic helpline
numbers in countries like Malaysia, Australia, China and Lebanon. The UN Secretary
General Antonio Guterres appealed to various nations to make sure that women don’t have to
face violence in their homes, where they should be safest.\footnote{\textit{UN chief calls for domestic violence ‘ceasefire’ amid ‘horrifying global surge’}, UN NEWS (Apr. 06, 2020,
\url{https://news.un.org/en/story/2020/04/1061052.}}

\noi
The United Nations had passed two resolutions: one in 1993- DEVAW\footnote{\textit{Declaration on the Elimination of Violence against Women}, G.A. Res. 48/104, (Dec. 20, 1993), [Hereinafter
DEVAW].} and the other in
2004- Resolution 58/147.\footnote{\textit{Elimination of domestic violence against women}, A/RES/58/147, (Dec. 22, 2003).} The former addressed the violence against women in general and
puts forth comprehensive guidelines and standards to protect women from all forms of
violence while the later specifically addressed domestic violence against women,
condemning it and addressed the different forms of domestic violence. During times like
these, where all governments have to fight battles on two fronts: one against the virus and one
against the violence against women, these resolutions and international obligations should be
given utmost importance.


\heading{The United Nations People Fund}

\noi
The UNFP issued a public warning that the continuing lockdown is estimated to cause nearly
31 million extra gender-based violence all across the globe. The pandemic resulted in the
delay of various programmes targeted to end violence against women, which would lead to
nearly 2 million more cases of female genital mutilation, child marriage and domestic
abuse.\footnote{\textit{Millions more cases of violence, child marriage, female genital mutilation, unintended pregnancy expected
due to the COVID-19 pandemic}, UNPF (Apr. 28, 2020), \url{https://www.unfpa.org/news/millions-more-casesviolence-child-marriage-female-genital-mutilation-unintended-pregnancies.}} The UNFP asked the countries to formulate measures to curb the violence against
women that is on the rise and to expressly make protection of women a priority.

\heading{The European Union}

\noi
Joseph Borrell, High Representative, speaking on behalf of the European Union, addressed
the challenge that faced the countries globally and reaffirmed that role of civil liberties and
human rights defenders be given utmost importance and solicited extensive support and
solidarity to the women battling with violence all over the globe stating that human rights
cannot be forgotten during a global crisis.\footnote{\textit{Declaration by the High Representative Josep Borrell, on behalf of the European Union, on human rights in
the times of the coronavirus pandemic}, EC-CEU (May 5, 2020, 3:20 PM),
\url{https://www.consilium.europa.eu/en/press/press-releases/2020/05/05/declaration-by-the-high-representativejosep-borrell-on-behalf-of-eu-on-human-rights-in-the-times-of-the-coronavirus-pandemic/.}}

\newpage

\heading{The Committee on The Elimination of\\ Discrimination Against Women (CEDAW)}

\noi
The Committee expressed its serious concerns regarding the heightened disadvantages and
the risks of violence that women face globally due to the pandemic and the subsequent
lockdown measures adopted by various countries to curb the spread of the virus. The
committee stated that all the signatory states of the ‘Convention on the Elimination of All
Forms of Discrimination against Women’ should ensure that the lockdown measures do not
put the women at a disadvantage, abridging them from seeking any form of shelter, health
care and economic life. The Committee asked the state signatories of the convention to be
accountable for the economic, social and overall wellbeing of women and to do so by
ensuring their participation and involvement in various decision and policies that need to be
formulated for various preventive and recovery measures.\footnote{Alyssa Cannizzaro and Eduarda Lague, \textit{International Women’s Human Rights: COVID-19’s impact on
domestic violence and reproductive rights}, PENN LAW (Apr. 20, 2020),
\url{https://www.law.upenn.edu/live/news/9986-international-womens-human-rights-covid-19s-impact.}}

\heading{Possible Makeshift Approach Towards\\ Addressing The Issue}

\noi
The lockdown severely curtailed the mobility of people, thus restricting the victims to register
complaints with the police. Even though a complaint could also be lodged via an online portal
or via a helpline number, it would be rather ludicrous to assume that all women are aware,
have access and are literate to lodge a complaint online, given the wide demographic sections
that women belong to, with only 45\% of women in India owning cell-phones. This calls for
law enforcement agencies to come up with pertinent awareness measures.


\begin{enumerate}
\itemsep=0pt
\item It has become imperative for the law enforcement agencies to come up with the
innovative policies which could come in aid of the victims. In several countries, such
policies have helped temporarily in backtracking of the gender-based violence issue at
hand.

The Supermarkets and pharmacies in Columbia were declared as safe spaces for
victims to report for abuse and were given training on how to aptly respond if women
approached them seeking help for domestic abuse. In Spain and France, victims are
seeking help in pharmacies by using code-words. In France, grocery stores have turned into temporary counselling homes for women and the French Government has
reserved thousands of hotel rooms to serve as shelter for the victims, in light of the
social distancing norms.\footnote{\textit{How Egypt, France and Other Countries Took Measures to Support Women During COVID-19 Crisis},
EGYPTIAN STREETS (May 12, 2020), \url{https://egyptianstreets.com/2020/05/12/how-egypt-france-and-othercountries-took-measures-to-support-women-during-covid-19-crisis/.}}

Several media sources reported that the Police in Gardai, Ireland, have launched
‘Operation Faoiseamh’, so as to contact every domestic abuse victim proactively,
those who had previously contacted the police about any domestic abuse, with an
immediate arrest policy.\footnote{\textit{Operation Faoiseamh}, AN GARDA SIOCHANA (Jun. 09, 2020),\\ \url{https://www.garda.ie/en/about-us/ourdepartments/office-of-corporate-communications/press-releases/2020/june/operation}\%20faoiseamh\%20-
\%20domestic\%20abuse\%209th\%20june\%202020\%20.html.} Similar measures have been undertaken by the police in
Odisha and Tamil Nadu. The Canada government is lending cell-phones and free
services to vulnerable people. Civil societies have tied up with Uber to provide free
emergency rides for victims.\footnote{Meg Black, \textit{Uber Offers Free Rides for People Fleeing Domestic Violence During COVID-19 Pandemic},
GLOBAL CITIZEN (Apr. 29, 2020), \url{https://www.globalcitizen.org/en/content/uber-offers-free-ries-for-peoplefleeing-domestic/.}}

In India, government authorities can learn from the same and adopt such innovative
initiatives as well as take the support from the private sector to scale up the initiatives. 

\item Fifty-two helpline numbers have been made operational throughout India, some being
national while some being state-specific. However, steps must be taken to make
women across the country aware of such helpline numbers. Further, free and
immediate counselling should also be provided to victims over calls, should a victim
seek for help, regarding possible escape plans and child care during abuse, to name a
few. 

\item More number of NGOs, feminist organisations and domestic abuse shelters should be
added in the list of essential services and be allowed to operate, even in strict
containment areas. In Quebec and Ontario, similar measures have been implemented;
along with the inclusion of shelters for domestic violence as ‘essential services’
during the period of lockdown.\footnote{Cillian O’Brien, Essential services: \textit{What's staying open when shutdowns expand in Ontario, Quebec, CTV}
NEWS (Mar. 24, 2020, 8:24 AM),\\ \url{https://www.ctvnews.ca/health/coronavirus/essential-services-what-sstaying-open-when-shutdowns-expand-in-ontario-quebec-1.4865643.}}

\item Frontline workers working for the National Commission for women for physical
rescue of women should be provided with adequate PPE suits and training to follow
appropriate social distancing norms while rescuing victims.

\item Vigorous nationwide and state-wide awareness campaigns must be launched to spread
awareness about domestic violence along with other safety and hygiene measures
relating to COVID-19. 

\item An app called ‘one love’ which enables the users to answer certain questions and post
the assessment of the answers, tells the users if they are being abused by their partners
or if their partners could potentially turn aggressive in future. It also suggests a proper
response or a safe road plan to escape in case an emergency situation should arise.
Similarly, an app called ‘Aspire news’ disguises as a regular news app, but caches a
‘help’ option which stores suggestive measures for domestic abuse victims.\footnote{Kevin McCarthy, \textit{3 apps that can help those experiencing domestic violence}, NUEMD (Nov. 13, 2016),\\
\url{https://nuemd.com/news/2016/11/30/3-apps-can-help-those-experiencing-domestic-violence.}} An app
called Victims-Voice allows the victims to document their experience of abuse, store
pictures of cuts or injuries and medical reports after the abuse. All the information is
encrypted and kept off-device so that the abuser may not destroy them and the victims
could use the data, if in future they wish to pursue any legal action.

In India, even though we have numerous apps that target domestic violence, we need
an app, developed on the similar lines to those mentioned above, which could perhaps
help us with the demanding state of gender-based violence.

\end{enumerate}

\vspace{-.3cm}

\heading{Conclusion}

\noi
{\large\it\bfseries O’ Cruelty, why thy victim is generally a woman!}

\noi
While another hindrance to the prevention of domestic violence has been generated by the
COVID-19 outbreak and isolation, studies on this problem have already suggested the
prevalence of the crisis before the pandemic. The pandemic has only amplified the limitations
of current domestic violence preventative initiatives. Conditions during COVID-19 appeared
strikingly similar to those during disasters- loss of jobs, alienation from social support and
economic strain and it wouldn’t be farfetched to say that such analogous circumstances
following the pandemic could actually harbour the chances of violence in families. Women have also been subjected to psychological abuse like demeaning, belittlement and insults,
threats of being abandoned, threats of hurting someone they care about or general infidelity
of the husband. Despite there being several laws to prevent domestic violence such as the
PWDVA and the penal provisions under IPC, these laws hardly came to the aid of the
domestic violence victims during the lockdown. The reason for the same seems to be the
major under-reporting issue in India caused due to general unawareness, victim blaming
mentality, casual attitude of police authorities, and even the unbalanced power dynamics in
families has deterred the victims from reporting their cases. Sometimes, even the
cumbersome process of investigation, and proceedings under the PWDVA have acted as
major barriers for recourse. Further, the response of the judicial system is also plagued with
deep rooted prejudices. We need to encourage and reinforce policies of action to tackle
domestic violence in every region. This virus outbreak has exposed the limits of existing
domestic violence prevention measures and has shown that there is still a lot of work that
must be done. Several International Organisations such as the UN, UNFP, and the EU have
strongly called on the nations to take steps towards mitigating the rising violence against
women amidst the severe lockdown restrictions. India could learn from the countries
implementing several make-shift approaches towards alleviating the plight of the women.
The COVID-19 pandemic may be seen as a window to rethink our future and to take longlasting steps. Homes should not be relegated to torture chambers for women. In the so-called
'private' spaces, the protection of women should not be violated, further reducing women to
second-class citizens who rob them of their identity and their beings. In order to build a better
society, smashing patriarchal hegemony and gender bias at home and in public spaces is
imperative. It is vital that actions are taken to stop abuse in order to ensure that homes remain
as 'safe spaces' for everyone. The corona outbreak is new and scientists are seeking a solution
to deal with it, but the patriarchal virus is ancient and there is still no remedy for it. To build a
better world, addressing patriarchy and all other types of injustice and prejudice is necessary.
In addition to taking these suggestions into account, the government has to take other
important measures to combat domestic abuse. Only then can it alleviate the brunt of violence
that women face. 
\end{multicols}
\label{end2021-art7}
