\setcounter{figure}{0}
\setcounter{table}{0}

\articletitle{Regulation Of The Competition Commission On Foreign Investment Combination\\[4pt] Agreements with Special Reference to Vodafone Case}\label{2021-art1}
\articleauthor{B. Priya\footnote{Assistant Professor, Government Law College, Namakkal, Tamil Nadu} and Dr.\ R. Haritha Devi\footnote{Associate Professor, The Tamil Nadu Dr.\ Ambedkar Law University, Chennai}}
\lhead[\textit{\textsf{B. Priya and Dr.\ R. Haritha Devi}}]{}
\rhead[]{\textit{\textsf{Regulation Of The Competition Commission On Foreign Investment...}}}

\begin{multicols}{2}

\heading{Introduction}

\noi
According to the World Investment Report 2020, India stands in the status as 9th largest
Foreign Direct Investment (FDI) holder in 2019, with 51 billion dollars of FDI inflows in this
year.\footnote{World Investment Report, 2020, \textit{CHAPTER I GLOBAL INVESTMENT TRENDS AND PROSPECTS}, 12
\url{https://unctad.org/system/file}} India’s embrace of the free‐market paradigm in 1991 considerably expanded the
economic policymaking autonomy of subnational governments.\footnote{Loraine Kennedy, \textit{The Politics of Economic Restructuring in India: Economic Governance and State Spatial
Rescaling}, 20 (Routledge, 2$^{\rm nd}$ ed.\ 2013)} The rich resources of India
attract many foreign investors to pour investment in India. India is also in need of FDI for the
fulfilment of investment, technology, managerial skills and market access.\footnote{FDI India, (Jul.,25,2020,03.00PM), \url{https://fdiindia.in/about-us},} Foreign
investment bringing in goods and services helps not only in the increase of GDP rate but it is
also accused of anti-competitive impacts on domestic companies.\footnote{Badri Narayan Rath and Debi Prasad Bal, \textit{Do FDI and public investment crowd in or crowd out private
domestic investment in India}’, 48 J Dev.\ Areas, 269, 269 (2014)} In 1999, the conservative
and restrictive Foreign Exchange Regulation Act (FERA) of 1973 was repealed and replaced
with the flexible Foreign Exchange Management Act (FEMA) of 1999.\footnote{Chanchal Kumar Sharma, \textit{Federalism and Foreign Direct Investment: How Political Affiliation Determines the
Spatial Distribution of FDI–Evidence from India} 1--7(GIGA Working Papers No.\ 307, 2017)} Competition is the
life force of markets that create the best incentives for businesses to increase efficiency, drive
their productivity and fuels innovation.\footnote{CCI Annual Report (2018--19), (Nov.\ 22, 2020, 9.00PM),
\url{https://www.cci.gov.in/sites/default/files/annual reports/}} The Competition Commission of India (CCI)
safeguards the corporate market from anti-competitive impacts. This paper analyses the
regulations of the Competition Commission on FDI combination agreements through its anticompetitive
impacts for understanding its consequences on the consumers’ adhesion
agreements and impacts on crowding in or out of domestic investors. The main objective is to
study on the indispensable need of Foreign Direct Investment flow in India and to make a
study on the monitoring role of the Competition Commission in the approval of FDI
Combination agreement.


\heading{Foreign Direct Investment}

\noi
Foreign Direct Investment (FDI) means investment by non-resident entity/person residing
outside India in the capital of the Indian company under Schedule 1 of FEMA (Transfer or
Issue of Security by a Person Resident Outside India) Regulations 2000.\footnote{Department of Industrial Policy and Promotion (DIPP)Ministry of Commerce and Industry Government of
India, Circular Consolidated FDI Policy,2015, \textit{Dipp manual definition} Chapter 2-1-12, p.4, (Nov 05, 2020, 
9.00AM), \url{https://dipp.gov.in/sites/default/files/FD}} In the Foreign
investment process, the resident of the home country acquires assets to control the activities
of the firm in the host country.\footnote{B.K. Lokesha, and D.S. Leelavathy, \textit{Determinants of Foreign Direct Investment: A Macro Perspective}, 47(3)
IJIR 459, 460 (2012)} Trade Liberalization as a result of GATT-WTO agreements
and Competition policy share the common objective of eliminating the\break barriers of foreign
investment in trade and commerce.\footnote{UNCTAD, Report by UNCTAD Secretariat, UN Conference on Trade \& Development, (Empirical Evidence
of the Benefits from applying Competition Law and Policy Principles to Economic Development to attain
greater efficiency in International Trade \& Development((TD/B/COM.2/EM/10).) pp.1--22, (1998)} FDI benefits the host country if the domestic firms avail
similar sophisticated technologies.\footnote{12Christian Fons-Rosen et al., \textit{Foreign investment and domestic productivity: Identifying knowledge spill overs
and competition effects}, 2(National Bureau of Economic Research, 2017), w23643.pdf (nber.org),} On April 17, 2020, the Department for Promotion of
Industry and Internal Trade (DPIIT), India altered its Foreign Investment policy to protect
Indian companies from ``opportunistic takeovers/acquisitions of Indian companies due to the
current COVID-19 Pandemic".\footnote{FDI Policy Section Press Note No. 3(2020 Series) Government of India Ministry of Commerce \& Industry
Department for Promotion of Industry and Internal Trade, ch5.2.27.3 Para3.1.1}
%~ \smallskip

\vspace{-.15cm}

\heading{Types Of Combinations}

\vspace{-.15cm}

\noi
The Foreign Direct Investment can be made by Horizontal, Vertical and Conglomerate
Combination agreements executed between investors and acceptors. According to Section 5
of the Competition Act, 2002, the Combination of Enterprises is defined as the acquisition of
one or more enterprises by one or more persons or merger or amalgamation of enterprises.\footnote{14The Competition Act,2002, Act No.12 of 2003, Ministry of Law and Justice, \S5.}
Horizontal merger or Combination related to FDI refers to the merger of foreign and
domestic companies at the same level of production or distribution in the relevant market. In
the acquisition of Monsanto by Bayer\footnote{Competition Commission of India, Combination Registration No. C-2017/08/523 dated June 14, 2018
\url{Order_14.06.2018.pdf} (\url{cci.gov.in})}, the Commission recognised that, as both the parties
were active in the downstream market for commercialization of Bt. Cotton seeds in India, it
resulted in horizontal merger. The global acquisition by the German chemical and pharma
major Bayer AG on US-based biotech major Monsanto of \$63million was justified on the
grounds of innovation in the agricultural field.\footnote{16PTI. Leverkusen, \textit{Bayer completes acquisition of Monsanto}, The Hindu Business Line.com, June 07, 2018,
\url{https://www.thehindubusinessline.com/} (Lastly Visited Dec.2, 2020).} A Vertical Merger is a combination of a firm
in the upstream market with a firm in the downstream market.\footnote{Steven C. Salop, \& Daniel. P. Culley, \textit{Potential Competitive Effects of Vertical Merger, A How-To Guide for
Practitioners}’,4(2014) \url{https://scholarship.law.georgetown.edu/cgi/}} According to the SVS
Raghavan Committee, the horizontal merger proves to be anti-competitive than vertical
mergers.\footnote{Justice SVS Raghavan, High-Level Committee of Competition Law, (May 22,2000) Government of India,
Report of the High-Level Committee on Competition Policy and Law (May 2000),
\url{https://theindiancompetitionlaw.files.wordpress.com/2013/02/report_of_high_level_committee_on_competition
_policy_law_svs_raghavan_committee.pdf}.} However, the Vertical mergers can be anti-competitive if it results in foreclosure
or enhanced co-ordination.\footnote{Jeffrey Church, \textit{Vertical Mergers, Issues in Competition Law and Policy}, 2 AA Section of Antitrust Law,
1455, 1461--63 (2008)} The Conglomerate merger is a merger between businesses that
operate in different product markets happening between diversified companies. According to
the SVS Raghavan Committee the Conglomerate Mergers do not prove to be anticompetitive.
The Corporate mergers and acquisitions are aimed at enhancing competitive advantage and
amplifying efficiency of firms.\footnote{Onyanko Mark Omondi, \textit{Effect of Merger and Acquisition Strategy on Competitive Advantage of ICEA and
Lion Group, Kenya}, 1, 53--55 (Research project submitted to School of Business, the University of Nairobi,2016)
\url{https://pdfs.semanticscholar.org}}
%~ \smallskip

\heading{FDI’s Impact On Consumers And Domestic Investors}

\noi
The Indian consumer market creates major opportunities and challenges for both the Indian
and multinational companies in a similar way.\footnote{Jonathan Ablett, Aadarsh Baijal, Eric Beinhocker, Anupam Bose, Diana Farrell, Ulrich Gersch, Ezra
Greenberg, Shishir Gupta, and Sumit Gupta, ‘The Bird of Gold - The Rise of India's Consumer Market’,
McKinsey and Company, (2007).} The Foreign investment occupies from
Energy sectors to Retail sectors and the Government invest or investigate to invest only for
improving its image in global competition. The advent of foreign investment in Retail sectors
such as Wal-Mart displays altogether numerous products in single and convenient locations,
thereby decreasing the transaction costs of consumers.\footnote{Leonardo Iacovone, Beata Javorcik, Wolfgang Keller \& James Tybout \textit{Walmart in Mexico: The impact of
FDI on innovation and industry productivity}, 1, 14--17 (The University of Colorado, 2009)} It has been found that Wal-Mart is
largely responsible for the demise of small discount retailers.\footnote{Panle Jia, What \textit{Happens when Wal-Mart comes to Town? An Empirical Analysis of the Discount Retailing
Industry}, 76(6) \textit{Econometrica} 1263, 1302(2008).} The elevated competitive
pressures by foreign investors through logistics discounts on the competitive products may
pave way for the exit of some marginally profitable domestic firms out of the corporate
market absolutely.\footnote{E.A.S. Sarma, \textit{Need for caution in retail FDI}, 40(46) Economic and Political Weekly, 4795, 4798 (2005).} These consequences may lead to the monopoly establishment of foreign
investors and may prove to be anti-competitive. In these circumstances, the monitoring role
of CCI may prove to be indispensable to maintain the competitive efficiency of India.
The entry of foreign investors to an existing industry may provide sizeable impact on the
structure and level of competition by giving valuable innovative updated knowledge to
domestic firms thereby increasing export propensity paving way for economic
development.\footnote{P. Enderwick, Attracting \textit{desirable FDI: theory and evidence}, 14(2) Transnational Corporations, 93, 106 (2005)} According to ASSOCHAM report 2012, the investment of FDI in
Automobile, Telecommunication and IT has brought them global market recognition.
According to ICRIER (Indian Council for Research on International Economic Relations)
survey on FDI retailing in India, mostly 85\% of consumer durables manufacturers favoured
for FDI in retailing in India.\footnote{M. Arpita \& P.\ Nitisha, \textit{FDI in retail sector, India}. 104 (Academic Foundation in association with ICRIER and
Department of Consumer Affairs, New Delhi, 2005)} Hence, foreign investment has both pros and cons which has to
be balanced for economic development. The two main types of efficiency promoted by
competition are “static efficiency” (optimum utilization of existing resources at least cost)
and dynamic efficiency (optimal introduction of new products).\footnote{UNCTAD, Report by UNCTAD Secretariat, UN Conference on Trade \& Development, (Empirical Evidence
of the Benefits from applying Competition Law and Policy Principles to Economic Development to attain
greater efficiency in International Trade \& Development((TD/B/COM.2/EM/10).) pp.1, 6(1998)} The competition in market
is necessary to maintain both efficiencies. The Role of Competition Commission's broad tent
approach is imperative and indispensable to balance the efficient foreign direct investment
and also the protection of consumers and domestic investors.

%~ \vspace{.5cm}

\heading{CCI’s Control On Fdi Before Combination}

\vspace{-.15cm}

\noi
The CCI’s control on FDI before Combination can be analysed based upon the Combination
thresholds, CCI approval procedure, CCI analysis on main clauses of Combination
agreements and factors on CCI approval of Combinations.
%~ \smallskip

\vspace{-.15cm}

\heading{Combination Thresholds In India}

\vspace{-.15cm}

\noi
The Competition Act, presently in the transitional stage is fairly recent legislation vital to the
economic growth of the country.\footnote{T.\ RAMAPPA, (2014) \textit{Competition Law in India, Policy, Issues and Development}, 30 (3rd ed., Oxford
University Press, Oxford.2014)} According to Section 5 of the Competition Act, 2002, the
foreign investors have to get CCI approval for the Combinations with domestic investors
based on the thresholds on the value of assets and turnover for Enterprises or Group Level.
The Competition Commission also provides the De Minimus exemption for boosting FDI
investments. The De Minimis Exemption is provided to the enterprises with control, shares,
voting rights or assets of value not exceeding Rs. 350 Crore in India or turnover not greater
than Rs.\ 1000 crore in India from the application of Section 5 of the Competition Act for a
duration of 5 years and for the group exercising less than 50\% voting power, an extension of
the exemption for further five years is granted.\footnote{Central Government Notification, \textit{Revised Thresh holds in Section 5 of the Competition Act, 2002}, Ministry of
Corporate Affairs MCA \textit{through} Notification No.\ S.O.\ 674(E) \& 673(E) dated March 04, 2016, Notification |
Competition Commission of India, Government of India (\url{cci.gov.in}) SO \url{673(E)-674(E)-675(E).pdf} (\url{cci.gov.in})} The Competition Commission regulates
foreign investment through combination thresholds with De Minimis relaxation.
%~ \smallskip

\vspace{-.15cm}

\heading{CCI’s Approval On FDI Combination}

\vspace{-.15cm}

\noi
According to Section 6\footnote{The Competition Act,2002, Act No.12 of 2003, Ministry of Law and Justice, at 6.} in the Competition Act, 2002, the combination which causes or
likely to cause an appreciable adverse effect on competition are held to be void. The CCI
approval can be obtained through the Green channel procedure or Normal procedure. The
combinations that are not likely to have an Appreciable Adverse Effect on competition in
India are exempted from filing of an application with the Competition Commission through
Green Channel Approval. Green Channel procedure automatically approves the mergers
reducing transaction costs. If the declaration is found to be deceptive, approval will be
declared void-ab-initio after a reasonable opportunity to the companies.\footnote{M.P. Ram Mohan, and R. Vishaka, \textit{Merger control for IRPs: Do acquisitions of distressed firms warrant
competition scrutiny?} 2, 4(Indian Institute of Management Ahmedabad, WP No.2020-05 2020)} Out of the total of
33 filings received from 1st January 2020 to June 2020, eight (8) filings are under the green
channel as not likely to cause adverse impacts in competition. The reducing of approval stand
still period of 210 days by Green Channel Procedure improves the foreign investment in
India.\footnote{CCI, \textit{Fair Play}, 33 The Quarterly Newsletter of Competition, 10 (2020).} The normal approval procedure involves the inquiry or cooling period of 210 days
after notification and regulation 19(2) provides for approval after modification by parties of
the combination agreements.\footnote{CCI (Procedure regarding the transaction of business relating to combinations) Amendment Regulations,
2018 (October 9, 2018)} Thus, CCI has every right to revoke combination approval
orders if the combination results in anti-competitive effects.

\vspace{.2cm}

\heading{CCI’s Analysis On Main Clauses In\\ Combination Agreements}

\noi
The Competition Commission analyses the clauses in Combination agreements on deciding
the approval of Mergers and Acquisitions. As per the Material Adverse Event clause, the risk
of loss on events has to be borne by any one of the parties if it incurs in the interim period
between the signing and execution of merger agreements.\footnote{R.T. Miller, \textit{The Economics of Deal Risk: Allocating Risk through Mac Clauses in Business Combination}
50 Wm. \& Mary L. Rev. 4 (2008) \url{https://www.osborneclarke.com/insights/mac-clauses-in-ma-agreements/}} These clauses empower the
acquirer, investor or buyer to terminate the transaction on occurring of any material adverse
change.\footnote{R.J. Gilson \& A. Schwartz. \textit{Understanding MACs: Moral hazard in acquisitions}, 21(2) J.L. Econ. \& Org.
330, 331 (2005)} The non-competition clause imposes restrictions on both buyer and seller. As noncompete
clauses create an artificial barrier to trade, they also fall outside the basic principles
of free trade.\footnote{N. Hanni, \textit{Exclusive Distribution and Non-Compete Clause in Trade: Transnational Agreements in the
European Union and the United States}’, 3(2) Udayana Journal of Law and Culture 141,145(2019)} In\footnote{Competition Commission of India, Combination of Advent International Corporation and MacRitchie
Investments Limited, Combination of Registration No. C-2015/05/270 (June 12, 2015), Order under Section 31(1) of the Competition Act, 2002. \url{C-2015-05-270_0.pdf} (\url{cci.gov.in})} notice under Section 6(2) for the proposed combination of Advent
International Corporation and MacRitchie Investments Limited, Singapore incorporated
company (acquirer), CCI directed for modification u/s 31(1) for the reduction of non-compete
clause period from 5 to 3 years. The Competition Commission also permitted Rescue mergers
as failing firm defence to prevent the closure of companies in Covid-19 situation. Hence, CCI
also proposed to omit Non-compete restrictions in paragraph 5.7 in Form-I of Combination
agreements for Foreign Investments flexibility. The Umbrella Clause commonly refers to the
parties abiding by the obligations of state and giving additional protection to investors
concerning the contractual obligations in the provisions of the state.\footnote{OECD, \textit{Interpretation of the Umbrella Clause in Investment Agreements}, International Investment Law:
Understanding Concepts and Tracking Innovations: A Companion Volume to International Investment
Perspectives, OECD Publishing, Paris, \url{https://doi.org/10.1787/9789264042032-3-en}.(2008)} The domination of
parties through any of these clauses can be scrutinised by Competition Commission and the
agreements may be approved after the inclusion of suggested modifications and rectifications.
%~ \smallskip

%~ \vspace{.2cm}

\heading{Factors In CCI Approval}

\noi
As two sides of a coin, the performance of FDI has been impressive on some fronts,
satisfactory on several other fronts, and inadequate in certain respects.\footnote{S. Chandrachud \& N. Gajalakshmi, \textit{The economic impact of FDI in India}, 2(2) Int. J Humanit. Soc Sci.
Inv., .47, 52 (2013)} According to
Section 20(4) of the Competition Act, 2002,
\begin{enumerate}
\itemsep=0pt
\item[i)] Adverse factors are barriers to new entrants and increase in price
\item[ii)] Determining factors are sustainability and elimination of competition based on market share and alternatives.
\item[iii)] Defence factors are Combinations outweighing adverse impacts and Economic
development.
To meet the demands of the competitive market forces, Mergers and Acquisitions are crucial
growth catalysts to sustain in the business world.\footnote{A. Mishra, A. Pradhan, \& O. Bisht, \textit{The impact of trust on leadership during mergers and acquisitions: case
studies from the Indian telecom sector}’. 4(2) People: Int.J. of Soc.Sci., 1035,1036(2018)} The CCI's analysis and determination of
appreciable adverse effect can be explained through Vodafone-Idea merger case.\footnote{Vodafone-Idea Merger Approval, Combination Registration No.\ C-2017/04/502 (CCI, Oct 03, 2017)
\url{Order_C-2017-04-502.pdf} (\url{cci.gov.in})} On
17.04.2017, Vodafone India Limited (VIL) the Indian subsidiary of UK-based Vodafone
Group, Vodafone Mobile Services Limited and Idea Cellular jointly filed a Notice under
Section 6(2) of the Competition Act,2002 for merger and amalgamation of
telecommunication business of VIL. The Commission analysed the following factors under
Section 20(4) of the Competition Act, 2002. The analysed factors are;
\end{enumerate}

\vspace{-.1cm}

\begin{enumerate}
\itemsep=0pt
\item[a)] Concentration Analysis\footnote{Id. At 4-6}- As the Vodafone Idea Ltd. holds 25\% of the assigned spectrum
and 50\% of specific band spectrum exceeding 50\% share the Commission observed that the
Proposed Combination is likely to result in significant market shares and change in
concentration in 14 telecom circles
\item[b)] Competitive constraints post the Proposed Combination- On deciding closeness of
competition, out of 14 telecom circles, the Parties appear to be close competitors in 10
telecom circles namely Andhra Pradesh, Mumbai, Punjab, Uttar Pradesh (East), Uttar
Pradesh (W), West Bengal, Gujarat, Haryana, Kerala and Maharashtra.
\item[c)] Buyer Power\footnote{Vodafone-Idea Merger Approval, Combination Registration No. C-2017/04/502 (CCI, Oct 03, 2017)
\url{Order_C-2017-04-502.pdf} (\url{cci.gov.in}) at p.7}- As per Mobile Number Portability Regulations, 2009,
Telecommunications Service Priority, the near-zero switching cost ensures that there is price
competition amongst the TSPs (Telecom Service Providers) to retain customers.
\item[d)] Competition Extent after the Proposed Combination\footnote{Id At 8-10}- The Commission on examining the
potential of the competitors like Bharti Airtel, RCOM and Aircel, Jio, Tata and BSNL/MTNL
opined that they are in a position to exercise adequate competitive constraints on the Merged
Entity and to eliminate any likelihood of adverse effect on competition resulting from the
Proposed Combination.
\item[e)] Level of combination in the market- The increase in the number of subscribers requires
significant investments by operators to build coverage, data capacities and quality. The
Commission analysed that out of 220 countries 213 countries have 4 or fewer operators, 6
countries have 5 TSPs and only India will have more than 5 TSPs. Hence, the Commission is
of the opinion that a reduction in the number of competitors at this stage is not likely to have
any adverse effect on competition in mobile telephony markets. After analysing these adverse
and determining factors, CCI approved the Vodafone-Idea merger as it does not result in
adverse impacts on competition. The analysis of the factors affecting competition in
corporate sectors before and after the Combination helps to maintain a healthy competition
for economic elevation. On analysing with Michael Porter’s five forces model, the reasons
for the merger is given to be Threat of entrants \& substitutes, bargaining powers of buyers \&
suppliers and the competition rivalry.\footnote{Saurav Kumar, \textit{Restructuring of Indian Telecom Industry: Emergence of Reliance Jio and M\&A Cases
Involving Airtel-Telenor And Vodafone-Idea}, 14(1) \textit{Management Insight}, 42, 45 (2018)} These reasons make a need for larger investment
through mergers.

\vspace{-.3cm}
\end{enumerate}

%~ \newpage

%~ \vspace{-.2cm}

\heading{CCI’S Control After Combination}

\noi
The CCI also has the power to order a demerger under Section 28 of the Competition Act
2002, if the merged entity is abusing its dominant position. This means that if the merged
entity engages in any form of exploitative or exclusionary practice, the CCI can take
appropriate action inclusive of the order to segregate the merged firm.\footnote{M. Tewari, what \textit{does India think}?’,(GODEMENT F., ED.). European Council on Foreign Relations, 79,83,
(2015)} In Mohit Manglani v.\ Flipkart India (P) Ltd,\footnote{Mohit Manglani v.\ Flipkart India (P) Ltd., Case No. 80 of 2014,
\url{http://www.cci.gov.in/sites/default/files/802014.pdf}} the informant alleged exclusionary agreements by e-portals with
the sellers. The Commission analysed and opined that e-commerce accounts for less than one
portion of total retail market. Hence, ordered that there is no dominant position creating
adverse effects in competition. The Competition Commission also has powers to impose
penalty extending 1\% of turnover or assets value whichever is higher for non-furnishing of
the information under Section 43A of the Act and penalty of fifty lakhs to one crore for
furnishing of false information under Section 44 of the Act. CCI can also monitor the cartels
and vertical agreements executed by the company u/s 3 and also abuse of dominant position
u/s 4 of the Competition Act after combination. The monitoring power of the CCI after
Combination helps to regulate FDI investments even after the approval of combination
%~ \medskip

\vspace{-.15cm}

\heading{Result Analysis And Implications}

\noi
Foreign investment brings about innovations and improvements in Indian industry.\footnote{Enderwick, P., supra.20 at 204--207} The
retailers have the fear of their eviction from market due to foreign investment but the
manufacturers are gearing themselves to face the global investors.\footnote{ARPITA MUKHERJEE, NITISHA PATEL, AND ARVIND VIRMANI, \textit{FDI in retail sector, India}, 69
(Academic Foundation, 2005)} The Competition
Commission has powers to reject or suggest modifications if foreign investment
combinations prove to be anti-competitive.\footnote{Ram Mohan, M. P., and Vishaka, R., supra. 27 at 28.} The study on the regulation of the Competition
Commission on Foreign investment before and after the Combinations clearly portrays the
indispensable regulating and monitoring position of CCI and it also shows that the
administrative and executive powers of CCI have to be strengthened for efficient functioning.
Liberalisation and the elimination of distortions within an economy do not automatically lead
to growth in the absence of the supply capabilities to take advantage of new opportunities,
and the prevalence of competition is only one factor determine countries' growth rates.\footnote{Report by UNCTAD Secretariat, UN Conference on Trade \& Development, (Empirical Evidence of the
Benefits from applying Competition Law and Policy Principles to Economic Development to attain greater
efficiency in International Trade \& Development((TD/B/COM.2/EM/10).)1--22, 13(1998)} The
Competition Laws usually allow the competition authorities to assess the trade-off between
the injury to consumers on permitting a monopoly versus potential benefits.\footnote{R. Gupta, \& P. Malik, \textit{FDI in Indian retail Sector: Analysis of competition in Agri-food sector}, 40 (Internship
Project Report, Competition Commission of India,2012)} The study on
the impact of FDI on domestic investors and consumers portrays the need for effective
participation of CCI for balancing the protection of domestic investors and FDI investment
much needed for the economic development of India. Though Foreign Investment is often
perceived as a channel of progress and development; it is also criticised as an instrument
employed by rich countries to control resources in developing economies.\footnote{M. PANT \& S. DEEPIKA, \textit{FDI in India: History, policy and the Asian perspective}, (Orient Black swan Private
Limited, New Delhi,2015)} In this stage,
Foreign Direct Investment and Competition has to be balanced to compete with Globalized
countries in the world market. The main limitation of this study is the lack of analysis on the
efficiency of CCI and there is further scope of study on analysis of efficiency of CCI based
on its current functioning.

\heading{Conclusion and Suggestions}

\noi
The Foreign investment liberalisation by removing formal barriers to the entry of FDI has
increased the competition among the national markets. The weighing balance of economic
growth has to be balanced between Foreign Direct Investment and Competition in the
Corporate sectors. The control of CCI on FDI through fixation of Combination thresholds,
approval procedures, factor analysis in Combination agreements with reference to Vodafone-
Idea merger brings forth the regulation of FDI in India. The monitoring role of CCI after
Combinations also paves way for the protection of domestic investors. In light of the
above considerations, a welfare‐oriented competition regime could adopt a preventative
approach to abuses of buyer power, and the remedies proposed should be prophylactic in
nature. The Provisional powers of Competition Commission have to be strengthened to
segregate the beneficial effects of Foreign Investment from the bad effects affecting
consumers and domestic investors.

\noi
In the Vodafone case discussed above the role of CCI in approving FDI is commendable. The
CCI while approving any combinations takes into consideration as provided in the
Competition Act. But, it is also seen that combinations are analysed only in terms of the net
worth of companies and whether there is any appreciable adverse effect on combination and
the consumer perspective is missed in most cases. In Vodafone case by this merger the
reduction in the number of telecom companies in the market before and after the said
combination has come down. This results in the reduction of choice for customers in
accessing the service. Reduction in choice may lead to monopoly and also reduce the quality
of services provided. In any combination the CCI has to give importance to customer’s
choices which will in turn uphold the objectives of the Competition Act. Bringing in more
FDI is important to the economy of the nation but at the same time it is equally important that
customers rights are safeguarded. Hence a customer-oriented approach is needed in analysing
combinations in the future. The upcoming changes in the Indian consumer market will create
major opportunities and challenges for both the Indian and multinational companies and the
strengthening of Competition Commission with consumer- oriented approach will pave way
for better economic development.

\end{multicols}
\label{end2021-art1}
