\setcounter{figure}{0}
\setcounter{table}{0}
\setcounter{footnote}{0}

\articletitle{Whether Artificial Intelligence is a Person or Machine:\\[4pt] an Analysis Under Patent Law}
\articleauthor{Janist Dhanol\footnote{LL.M. (IPR Specialization), Legal Researcher}}
\lhead[\textit{\textsf{Janist Dhanol}}]{}
\rhead[]{\textit{\textsf{Whether Artificial Intelligence is....}}}



\begin{multicols}{2}

\noi
For the first time in 1956, the word ‘Artificial intelligence’ was mentioned in the Dartmouth
Summer Research Project on Artificial Intelligence. Since then, it has taken a remarkable
journey with more than 1.6 million AI-related scientific publications and nearly 3,40,000
patent applications for AI-related inventions.2\footnote{WIPO, \textit{The Story of AI in Patents}, \url{https://www.wipo.int/tech_trends/en/artificial_intelligence/story.html} (last
visited Nov.01, 2020).}
However, it was not before the AI DABUS
case, that many people considered the prospects of Artificial intelligence as an inventor. It
would be justified to state that inventors in patent applications mentioned AI’s role in the
invention but before the AI DABUS case, as a rule, inventors after discussing the matter
with their lawyers, used to show themselves as the main inventor and AI in supportive role,
which was a smart decision, considering the existing law of Patent. With the change in
technology and its application, it is necessary to address the issue. Many countries in the
world have already started their journey toward legal implication of AI as an inventor but, in
India our legislature has failed to address the issue on an earlier date and thus India started
its journey in 2017, when India’s Ministry of Commerce and Industry set up the 18-member
Task Force on Artificial intelligence, with the mandate to advise inventors on the creation of
a framework to promote and deploy Artificial intelligence, after taking all social and
technological factors into account. After the Ministry of Commerce and Industry’s first step,
the NITI Aayog, an Indian government think tank, published a discussion paper called,
‘India’s National Strategy for Artificial Intelligence’3\footnote{NITI Aayog, \url{http://niti.gov.in/sites/default/files/2019-01/NationalStrategy-for-AI-Discussion-Paper.pdf} (Last
visited March 3, 2021)}
which discussed global developments
in Artificial intelligence and discussed the role of Artificial intelligence in the Indian society
along with the challenges faced by it. However, it was highly concerning that such detailed
discussion failed to address an essential aspect of AI, i.e., the legal implication of Artificial intelligence. In this research paper we will discuss the legal implication of Artificial
intelligence with special reference to the AI DABUS case.

\heading{Different Aspects of Ai as Personality}

\noi
The History of Intellectual property laws can be traced back to the 19$^{\rm th}$ century and since
then suitable changes have been made according to the dynamic field of science and art.
However, one thing which remained constant is that rights are always granted to human
beings or in legal terms, to a natural person. The reason behind the concept was that, at that
time in history, technology was not as advanced compared to the current times and at that
time only a human being was capable enough to invent or create. But in the past 50 years,
the entire scenario has changed; now with the development of artificial intelligence,
possibilities for new inventors have increased. This leads us to the most important question,
whether artificial intelligent should be considered as a legal entity with capabilities to hold
property right or not. For that purpose, we need to understand the nature and characteristic
of artificial intelligence. 

\noi
According to Salmond, to be a legal person is to be the subject of rights and duties. To
confer legal rights or impose legal duties, therefore, is to confer legal personality.\footnote{Salmond, \textit{Jurisprudence} (5$^{\rm th}$ edition)(London: Stevens and Haynes,1916)}
In 1956
John McCarthy, invited the leading researches from different fields to discuss a new topic of
‘thinking machine’ and the topic was so novel that he had to coin his own term for reference
and it was termed as artificial intelligence.5\footnote{J. McCarthy, M. L. Minsky, N. Rochester, C.E. Shannon, A Proposal for the Dartmouth Summer Research
Project on Artificial Intelligence, \url{www-formal.stanford.ed} (Aug. 31,1955), \url{http://wwwformal.stanford.edu/jmc/history/dartmouth/dartmouth.html.}}
Over time, the definition of artificial intelligence
evolved with according to the functions of artificial intelligence and the goals which humans
tried to achieve. According to English Oxford Living Dictionary, artificial intelligence
means “The theory and development of computer systems able to perform tasks normally
requiring human intelligence, such as visual perception, speech recognition, decisionmaking, and translation between languages.”6\footnote{\textit{Artificial Intelligence}, LEXICO, \url{https://www.lexico.com/definition/artificial_intelligence} (Nov. 4, 2020). }
 From this definition we can say that, artificial
intelligence possess as the ability to communicate, to achieve the identified aims, creativity, and to develop itself by cognitive process. Even though it has human replication, artificial
intelligence fails to be sufficient to consider it as a legal personality. After going through the
details of legal personality it would not be difficult to realize that there are entities other than
human beings which are considered as legal personality, like corporations. So, the question
here is, what makes artificial intelligence different? One explanation to this which could be
given is that, in case or corporations, (even though the corporation has a separate legal
identity from its creators) in cases of criminal liability, humans are considered to be
responsible for the act because a corporation doesn’t have its own mind or consciousness
hence, it works as per the sweet will of its controller i.e. a human being. However, the
situation is entirely different in case of artificial intelligence, because it has its own decisionmaking ability. But then again, the entire artificial intelligence system works on codes,
which were made by human beings. It is also questioned whether it is safe to say that, just
like in case of a corporation a human being can be held liable in case of an act of criminal
nature of an artificial intelligence? The answer to this question is not a straightforward yes
or no, because it is mentioned that artificial intelligence can develop itself from cognitive
process, as it happened in Facebook AI bot,7\footnote{Roman Kucera, \textit{The truth behind Facebook AI inventing a new language}, TOWARDS DATA SCIENCE,
(Nov. 4, 2020, 9:29 PM), \url{https://towardsdatascience.com/the-truth-behind-facebook-ai-inventing-a-newlanguage-37c5d680e5a7.} }
 so even though the correct codes are made, it is
possible for an artificial intelligence system to develop its own rationality and take a wrong
decision. Thus, a new question arises, that who would be liable for that act and hence before
granting a status of legal personality to artificial intelligence all this issue must be taken into
consideration. 

\noi
Another issue with the granting Intellectual Property (IP) rights to an artificial intelligence is
that it will not justify the Incentive Theory related to IP rights. When it comes to justifying
the granting of IP rights, Incentive Theory makes a strong point in favour it. According to
incentive theory, IP rights should be granted to the creator so that it would help him to earn
monetary benefit and providing an incentive to create more. As we all know the basic
psychology of the human mind, that a person’s needs motivation to complete a task, and if
there is no incentive in creation eventually the rate of invention will drop which would
hinder the progress of our society. When it comes to artificial intelligence, there is no need to provide incentive since they are not human beings and therefore does not require
motivation to perform tasks. 

\noi
Furthermore, if we work on hypothesis that AI can be granted IP rights, it brings up another
question of how AI would exercise the rights granted to it as an IP owner. Rights like
licensing and franchising require parties to enter into a contract and AI, not being given
human entity recognition, cannot enter into a contract. Hence it cannot exercise the right
given to it as an IP owner therefore it would defeat the purpose of granting IP rights. It is
also worth mentioning that a similar situation would occur in case of the violation of IP
rights of an AI, as till date AI has not granted the right to sue, therefore the purpose of
granting IP rights to inventor will be defeated. 

\noi
After considering the points above mentioned, it would be safe to assume that the answer to
the question whether AI is eligible to hold IP rights or not, is not straight forward and the
problem lies in the jurisprudence of IP laws. 

\heading{Artificial Intelligence and Indian Patent Law}

\noi
When it comes to legal implications of Artificial intelligence with respect to patent law, it
would be safe to say that it is an uncharted territory, and we are still trying to figure out our
first step. When we look at the Indian Patent law, it is clear that there is no provision to
address the issue. To discuss the possibility of patentability of Artificial intelligence related
invention, we need to understand the essence of AI related invention. AI related invention is
a combination of several other inventions and not a single invention, which includes
algorithms, mathematical formulae or methods and calculation or combination of both. With
this clarification another thing comes into equation i.e. Section 3(k) of Indian Patent Law.
According to Indian Patent Act, mathematical or business methods or a computer programs
or algorithms are not eligible for patent protection.\footnote{The Patents Act,1970, § 3(k), No. 39, Acts of Parliament, 1970 (India)}
Since the law is not clear, the next
reliable source of information is the guidelines issued by Indian Patent Office. Till this date,
Patent Office approach is to react to the situation rather than being proactive. It suggests that
decisions of Indian Patent Office are heavily influenced by outside pressure and opinions. In recent years, the Indian Patent Office issued a few guidelines with respect to Computer
related invention but rather than solving the problem they left applicants with more doubts
and speculations regarding future guidelines. The reason behind this is that it is the Patent
Office’s habit to react to a situation rather than being proactively solving the issue because
with every guideline there were new rules and requirement to be followed. 

\noi
Recently in 2019 the Delhi High court took different approach in Ferid Allani\footnote{FeridAllani V. Union of India, W.P.(C) 7/2014 \& Cm Appl. 40736/2019}
 case. In this
case the Delhi High court, while emphasizing on the principles discussed in draft guideline2013, relating to Computer related invention, held that Computer related inventions are not
barred from patentability, subject to investigation assessing the “technical effect” and
“technical contribution “of the underlying invention. In this judgment the court rightly held
that Section 3(k) put bar on computer programs per se and not on all invention based on
computer program. 

\noi
The Judgment of the Delhi High court has shown to us the opening towards the uncharted
territories. 

\heading{Approach Till Date and Ai Dabus Case}

\noi
In recent years, with the increase in developments of the AI field, the debate related to AI as
IP rights holder has also increased. But, the IP law has failed to match the speed of these
developments; hence uncertainty among the AI inventors has also increased. When we look
at the precedents set, it is clear that the judgments are not in favour of granting IP rights to a
AI. These judgments revolve around the possibility of considering AI as a legal personality. 

\noi
AI DABUS case was the latest development in the debate. In this case, the problem revolved
around legal personality of AI. According to Stephen Thaler, owner of AI DABUS,
inventorship should be granted to AI since his contribution in invention in zero and it would
be correct to recognize AI DABUS as an inventor and he be made himself an assignee and
successor in title, hence he will perform all the rights and duties attached to a person as an IP
rights holder. From a simple look at this statement, the solution to the entire problem can be assessed, but it is not simple enough. Under the patent law, there are specific methods
through which ownership can be assigned. These methods are, the inventor is the employee
of the company or working as a contractor of the company. In both these cases, ownership
can be assigned at the time of applying for the patent. These methods take us back to the real
problem i.e., legal personality. In both the cases the parties are considered as a legal
personality, but AI is not a legal personality hence it cannot perform the above-mentioned
functions. Similar opinion was given in the famous Monkey selfie case.\footnote{Naruto v. Slater, 2018 WL 1902414 (9$^{\rm th}$ Cir. April 23, 2018)} In the judgment,
judges of the 9$^{\rm th}$ circuit held that animals are not human, lacks statutory standing under the
Copyright Act hence denied Copyright. If we apply same statement in case of AI, it is
evident that AI cannot claim IP rights.

\noi
Moreover, since AI does not have any legal personality and entity, it cannot be an employee
of a company. AI can be owned by the company or an individual, but it cannot be hired for
employment.

\noi
In the AI DABUS case, another argument was given that in case of a minor or incapacitated
person; their rights are exercised by their legal guardian, so the same can be done in case of
AI. For that, the judges held that the situation in case of a minor and incapacitated person is
different from AI, in the former case they have a legal personality and legal right which they
can transfer but in case of AI, it cannot be done since an AI doesn’t have any legal
personality.

\heading{Conclusion and Suggestions}

\noi
Granting AI any IP right would be a challenge to our legislators. Since the problem is related
to jurisprudence of legal personality, it is essential to categories AI as a legal personality, in
2017, the European Union parliament discussed the matter of AI and suggested that selflearning robots such as AI can be granted a legal personality and they categorized them as
‘electronic personalities. The Motive behind this is that it will allow robots to be personally
liable if they go rogue and start hurting humans and damaging property. Nathalie Navejans,
a French law professor at the Universitéd'Artois, while giving her opinion on the motion said, “By adopting legal personhood, we are going to erase the responsibility of
manufacturers.” In a sense it could be true, for example if a person develops an AI and teach
it, through cognitive process, to do some illegal act, it would be helpful for the manufacturer
to escape from liability since the AI has legal personality and can be held liable for action.

\noi
There is another side of the problem, and it is correctly put by Ryan Abbott is AI DABUS
case. According to him if AI cannot be registered as an inventor due to the lack of legal
personality and humans cannot be registered since they are not directly involved, then the
invention may not be patentable at all and it would be wrong because it will prevent
inventors from investing money and time in AI technologies and as a result, it will prevent
major breakthrough in important areas of science. It is also true, since the purpose of IP right
is to promote development in society and by refusing AI as inventor, we are hindering the
development.

\noi
In the end, it could be concluded that there is a huge gap between the current requirements
and existing legal framework and with each passing day Artificial Intelligence creates new
challenges for our legislatures. Since Artificial Intelligence is inevitable in future, it is
necessary to match the pace to promote and limit the Artificial Intelligence. Till then it is the
duty of stakeholder like us to do our bit by voicing concerns at different stages. 

\noi
After discussing various issues related to the patentability of AI invention the following
suggestions are made to help to curb the issues: 

\renewcommand{\theenumi}{\alph{enumi}}
\begin{enumerate}[label=\alph*.]
\item There should be a uniform treatment of AI related invention at the global level, for
that all the member nations of TRIPs must come together to bring a suitable
amendment in the agreement. 

\item At the national level, government should start research work from an academic level
and should establish research department related to the subject and should provide
financial help to these institutions for conducting research and development of the
subject.

\item Furthermore, legislatures should commence work towards developing a law with
respect to AI in connection with IPR laws, while keeping in mind all the issues
discussed in the paper and while developing AI related law, legislatures can borrow
principle from other laws, for instance, while deciding accountability for
consequences of AI invention, a well-established principle of ‘Lifting of Corporate Veil’ in company law can be included to hold actual people working behind the AI
accountable for the consequences. 
\end{enumerate}

\end{multicols}
