\setcounter{figure}{0}
\setcounter{table}{0}
\setcounter{footnote}{0}

\articletitle{The Uniform Civil Code: A Triple Divorce Under Muslim Personal Law}
\articleauthor{Dr. Faisal Ali Khan\footnote{Associate Professor of Law, Galgotias University, Greater Noida}}
\lhead[\textit{\textsf{Dr. Faisal Ali Khan}}]{}
\rhead[]{\textit{\textsf{The Uniform Civil Code: A Triple Divorce....}}}

\begin{multicols}{2}

\heading{Introduction}

\noi
The Uniform Civil Code (hereinafter referred as “UCC”) is enumerated under Article 44 of
our Constitution, which says that the Legislature has to pass the law related to UCC for all
over India but Indian society. Always resides with their different traditional aspects related to
the religious, rituals, cultural and sociological aspects related to the personal laws of the
Country. So, it would not be advisable to implement the same in the present scenario because
we are residing in diversity of group of the people in one unity.


The discussions have been made in the Constituent Assembly that if our cultural, sociological
aspects and thinking would be similar in future because every progressive society can change
their usage and custom under these circumstances the UCC can be made. Let us take an
example that the marriages among Northern Indian Hindus cannot take place within family
but similar in South Indian Hindus Marriages can take place among Uncle and niece (Mama
\& Bhanji). Besides, marriages can take place amongst the Muslim within 1$^{\rm st}$ Cousins. So,
each \& every religious group and castes have their own traditions and values. It cannot be
bound under UCC. Although, every progressive society can make changes its customs and
traditions as per the reforms and necessity of the period. So, if there may arise such a
situation that the future of our country to be found similar in term of UCC then it can be
considered.

Normally, the triple divorce in Muslim Personal Law is related to the religious sentiments but
it is well settled inappropriate form of divorce under Muslim Personal Law and it is
irrevocable after the pronouncement of simple assertion of ‘divorce’ at a same time. Besides,
it is Talaq-e-Biddat.

It is occasionally misused by Muslim husbands according to their whims and fancies. As such
Sarla Mugdal case in which Hindu Male who tried to escape from the liability of criminal case because they had converted to Islam, but the Apex Court did not accept this sort of
pleas.

The Parliament has already passed a good piece of legislation for the protection of the victim
Muslim women i.e. The Muslim Women (Protection of Rights on Marriage) Act No. 20 of
2019; however, the problems of our Muslim sisters cannot be resolved related to Triple
Divorce. Still, they are the victim of triple divorce. 

\heading{Uniform Civil Code: Debates of Constituent Assembly and Judicial Precedents}

\end{multicols}
