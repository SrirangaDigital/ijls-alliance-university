\setcounter{figure}{0}
\setcounter{table}{0}
\setcounter{footnote}{0}

\articletitle{India’s Migrant Issue During COVID-19: a Crisis Within a Crisis}
\articleauthor{Tusharika Vig\footnote{Student, BA.LLB. 3$^{\rm rd}$ year, Christ (Deemed to be University), Bengaluru}}
\lhead[\textit{\textsf{Shreyaa Mohanty and Swikruti Mohanty}}]{}
\rhead[]{\textit{\textsf{India’s Migrant Issue During....}}}



\begin{multicols}{2}

\vspace{-.1cm}

\heading{Introduction}

\vspace{-.1cm}

\noi
The situation of migrant labourers in India has been characterized with insufficient wages,
irregular and unorganized employment conditions, lack of financial security and a constant
struggle to make ends meet for survival and livelihood. According to the figures of Census
2011, India witnessed movement of around 4.5 crore migrant workers in the year 2011.\footnote{Madhunika Iyer, \textit{Migration in India and the impact of the lockdown on migrants}, (10 June 2020, 10:00am),
\url{https://www.prsindia.org/theprsblog/migration-india-and-impact-lockdown-migrants}}
 In
spite of the massive and indeed continuous contribution of migrant workers in the growth and
development of small-scale businesses, enterprises and industries ultimately adding a
significant percentage to the national economy, their socio-economic conditions remain
precarious and has worsened even more in the past year of crisis.

\vspace{-.1cm}

\noi
The Corona Virus officially called as the COVID-19 virus started as an epidemic affecting
only a few parts of the world in small communities or regions unexpectedly took a dramatic
turn and spread across the globe in no time forcing the governments across nations to shut
down schools, colleges, workplaces and every other activity part of our daily normal life. The
first case of COVID-19 in India was reported in January 2020 and soon after the world shut
down, the Prime Minister of India announced a 21-day national lockdown under the Disaster
Management Act, 2005 which gives power to the Central Government to impose restrictions
and pass specific policies with the aim of managing the disaster and thereby preventing its
risks.\footnote{The Disaster Management Act, No.53 of 2005, INDIA CODE (2008), Vol. 12}
 Globally, the World Health Organisation officially announced on 4$^{\rm th}$ April 2020 that
the world had crossed 1 million cases with emphasis on the tenfold increase in the previous
month i.e. March 2020.~\footnote{WHO Coronavirus Disease (COVID-19) Dashboard, WHO, (4 June 2020, 6:14 A.M), \url{https://covid19.who.int/}}

\vspace{-.1cm}

\noi
At the time of such a chaotic health crisis faced in the recent years of modern history, it was
obvious that the misery would be unequally directed towards the most disadvantaged and socio-economically backward sections of society. Moreover, the disadvantaged class were
rendered even more helpless without any safety net or government support when the
announcement of lockdown came without any prior intimation to the public. While a huge
chunk of the privileged population of the country faced the pandemic inside the safe walls of
their comfortable homes with all facilities, the migrant labourers on the flipside suffered and
fought the pandemic in the scorching summer heat while struggling to meet the basic needs
of food, clothing and shelter. Being a democratic state that chooses its own representatives
through the process of Universal Adult Franchise, it is bare minimum to expect a sense of
responsibility and full disclosure of information from the authorities especially such
information that has potential to immediately disrupt the survival and daily functioning of a
large section of society. Mere prior notice could also aid them in the process of preparation
and prevent any additional damage to their health and livelihood. 

\heading{Evolution of Labour Movement and Labour Laws in India}

\noi
The history of labour exploitation in India can be dated back to the colonial period and
recalled from well-known events that unfolded during Gandhi’s Satyagraha movements in
Champaran district, Kheda and in form of Ahmedabad Mill strike in the pre-independent era.\footnote{ BIPIN CHANDRA \& ORS, India's Struggle for Independence 1857-1947 335 (8$^{\rm th}$. Ed. Penguin Book House,
1988)}
 The events highlighted the marginalization and economic suffering of the class of
agricultural labourers, peasants and mill workers. Despite several instances of labour
exploitation in the Indian territory including the Chargola Exodus where large number of
strikes were witnessed in the rich tea gardens of Assam, labour struggles and their problems
were not in the limelight or covered in literatures by academicians and scholars for a long
time. The Labour movement in India started taking an uproar only after the Royal
Commission on Labour came about in 1931 further triggering publication and research on
labour history. A clear parallel can be drawn between the attitude of colonial government and
the current authorities pertaining to lack of enough emphasis on labour issues and problems
especially the migrant workers. It was not until the 1960s when study of labour and their
issues became a part of curriculum for students and academicians. The Commission also led to the fulfillment of some of the then existing demands of the labour class by passing
historical labour legislations like Trade Unions Act, 1926, Workmen’s Compensation Act,
1923 etc.\footnote{Sanat Bose, \textit{Indian Labour and Its Historiography in Pre-Independence Period}, 13 Social Scientist 3, 3-5
(1985).}

\noi
The establishment of the International Labour Organization which came as a result of the
World War I led to widespread demands pertaining to labour problems such as reforms in
wages, social security, general welfare and social justice for the working class across
borders.\footnote{Nobel Prize History, (15 June, 2020, 7:15 P.M), \url{https://www.nobelprize.org/prizes/peace/1969/labour/history/}}
 With the development of the International Labor standards and labour conventions
and treaties, India’s labour movement also started taking a drastic yet progressive turn.
Moreover, with the enactment of the Constitution of India after independence, new goals and
aims emphasizing on social welfare were established for new India. Adhering to the goals
and desires of the founding fathers behind the Constitution, the concepts of Right to Work,
Dignity of Labour and Social Equality became the source and reasons for several legislations
enacted in the post-independence era.\footnote{INDIA CONST. art 38-39 }

\noi
The Industrial Disputes Act, 1947 was one of the first legislations of newly independent
India. It aimed to prevent disputes between workers within the industry and help in
establishing healthy and harmonious working environment for labour. The Act also attempted
to balance the interests of the workers as well as the employers by providing for dispute
resolution which would prevent any unnecessary halt in the production process as well as
safeguards for the employees pertaining to the conditions of work.\footnote{Industrial Disputes Act, No. 14 of 1947, INDIA CODE (1993), VOL. 15 }
 Payment of Bonus Act,
Equal remuneration Act, Payment of Wages Act, Payment of Gratuity Act, Unorganised
Sectors Social Security Act etc are other post-colonial legislations that were enacted to line
with the social justice aims of the constitution. With the changing circumstances,
globalization and increasing complexities in markets and trade, the Industrial Disputes Act
and other labour laws have been amended and reformed in theory multiple times to keep
them up to date and promote better working conditions for labourers. 

\noi
However, the recent migrant issue rendered all these detailed legislations covering almost all
aspects of the lives of labourers redundant and ineffective owing to the poor application of
the laws and ineffective implementation on the target population at the ground level.\footnote{S.I.A. Muhammed Yasir, \textit{Labour Legislation in India – A Historical Study}, 6 I.J.A.R 34, 35 (2016)}

\heading{A Crisis Within a Crisis}

\noi
With the close down of all shops, collapse of businesses, production and manufacture units
during the lockdown, it was obvious to predict the devastating impact on the economy of the
nation and the world at large leading to grave effects for the common man such as salary cuts
and loss of jobs to millions. The biggest concern of the crisis was that most migrant workers
are daily wage labourers who were left with nowhere to go when they were rendered jobless
overnight especially without the creation of any social support mechanism mandated by the
government. The nature of work of the migrant labourers is informal as they are unskilled in
comparison to the educated population thus, their migration from their rural
hometown/villages to cities in search of big money often leaves them in an unsecure often
low waged occupation. The sad reality of poverty traps them in the same poverty ridden cycle
because of the inability to afford good education or gain skill for a formal, secure job. The
living conditions of migrant workers who come from rural households have been deplorable
since years, even before the pandemic with the urban poverty level getting strained without
any improvement in the lifestyle of migrant workers.\footnote{Sengupta, A, \textit{Migration, Poverty and Vulnerability in the Informal Labour Market in India}, 4 TBDS, 99, 110-116 (2013)}

\noi
The announcement of the lockdown led to a heightened sense of fear and panic for these
individuals who had already been living a hand-to-mouth existence further causing confusion
and distress to the mass of daily wage earners, seasonal migrant workers, contract labourers
etc who were most often the bread-earners of their respective families. On top of that, with
the abrupt suspension of all means of transport mandated by Clause 6 mentioned in the 2020
Guidelines of Essential Commodities Act, 1955 without the creation and establishment of any
kind of alternative arrangements for commuting worsened the situation even more for this section of society. The guidelines strictly restricted the use of transport only for the purpose
of essential goods and case of fire, law, order and emergency services.\footnote{The Essential Commodities Act, No. 10 of 1955, INDIA CODE (1988), Vol. 27}

\noi
At a time when directions, policies and new guidelines were passed on a daily basis to handle
the crisis at hand, it is a thing to ponder why no central or even state authority/institution
consider the return of thousands of migrants as an unprecedented ‘emergency’ for the ones
who were walking miles on road risking their lives and of the fellow others during a global
health crisis. Even after amendment of the guidelines after relaxation of lockdown rules,
operation of flights, waterways and railways was kept mainly limited to the movement of
cargo. The railway stations and bus terminals became overcrowded during this time because
of the sudden emergency outflow of migrant labourers.\footnote{Amlegals, \textit{COVID-19 Outbreak: How About Essential Commodities}, (5 May 2020, 10:20 P.M),
\url{https://www.mondaq.com/india/operational-impacts-and-strategy/928382/covid-19-outbreak-how-aboutessential-commodities}} Due to this situation, safety
measures such as social distancing that was mandated by the orders and directions of the
government were violated. The migrant workers from unprivileged section of society were
simply left trapped without any safe shelter wandering around the inter-state boundaries
struggling to fend for themselves. Further, the state of affairs was aggravated due to the
absence of coordination between the central and state governments with the ineffective
implementation of existing policies and orders put in place in an attempt to handle this
unfolding humanitarian crisis. 

\heading{The Continuing Loss and Suffering}

\noi
The uncertainty in the time of the pandemic has also led to a lack of financial security for the
working poor, as about 89\% of all workers in India fall under the category of informal
workers. Of these workers about two-thirds are not covered by any minimum wage laws. This
is especially the case with inter-state migrants who constitute the “footloose laborers”\footnote{Sunanda Sen, \textit{Rethinking Migration and the Informal Indian Economy in the Time of a Pandemic}, THE
WIRE, (1 June 2020, 8:14 P.M), \url{https://thewire.in/economy/rethinking-migration-and-the-informal-indianeconomy-in-the-time-of-a-pandemic}} of the
country. Between 2011 and 2016, the magnitude of inter-state migration was estimated to be approximately nine million over a period of year in India. According to the \textit{Economic Survey},
2017, workers mostly from the states of Uttar Pradesh, Bihar, Madhya Pradesh, and
Rajasthan migrate to states like Delhi, Kerala, Maharashtra, Gujarat, and Tamil Nadu in
search of jobs. In cities, they are usually employed in menial jobs leading a precarious
existence working long hours for low wages, often in poor working conditions and living in
squalid surroundings. Among these workers include agricultural laborers, coolies, street
vendors, domestic servants, rickshaw pullers, garbage pickers, auto-rickshaw and taxi drivers,
construction workers, brick kiln workers, workers in a small way-side hotels and restaurants,
watchmen, lift operators, delivery boys, etc that we often see and interact with in our day-today lives.

\noi
A closer look at the history and situation of informal labour sector in India produces the
immediate inference that most existing laws, policies and institutions which are in place to
advance the goals of the labour class merely emphasize on or are accessible to the formal
sector only whereas the bulk of labour class in India is filled with unorganised, informal and
seasonal migrant workers. With the ease of doing business which has been promoted in the
recent years by the current government, small businesses, enterprises and industries have
become common. This expansion of freedom and opportunity for the employers comes at a
cost to the workers of the informal sector. The small businesses cannot afford the extra cost
that comes with providing better employment conditions and social security benefits to their
employees which is relatively easier for large-scale businesses that are capable of balancing
risks and costs equitably between both the employer and employees. Unfortunately, in the
informal sector, the entire brunt of suffering and costs falls on the labour.\footnote{Meghnad Desai, \textit{Informal Work}, 48(3) IJIR 387, 387-389 (2013) } The results of
which, one can witness with the Migrant Crisis of 2020.

\noi
The difficulty in organisation of the informal sector due to problems of irregularity, nonregistration, unincorporated business worsens the situation even more and renders even the
law-making bodies helpless. At such a difficult crossroad, the only question that the scholars
and law-makers need to address is whether there is a solution to provide the informal sector the benefits of the formal sector with the existing irregularities or is it more feasible to
attempt to transform the entire labour class of India to formal sector?\footnote{\textit{Bose, Supra} note 5 at 6}

\heading{An Issue of International Importance}

\noi
During a year where everyone was confined to their homes, social media became the main
source and platform where people interacted with one another and also got their share of
information. The social media avenues and \textit{whatsapp groups} were consistently flooded with
tips, tricks and myths about the novel pandemic where everyone tried to give their expert
opinions on the future and present. Amidst the chaos of fake chaos and a dilemma about what
to believe and what not, the citizens became aware about the instances of police brutality
across the globe. Unfortunately, India was not left behind in this race and this is aimed at
bringing the same into broad daylight. There have been recent instances of Police Brutality
that took place and came to the forefront through the circulation of inhumane videos on
media and sadly the instances have continued ever since then for some reason or the other. 

\noi
India being one of the founding members of the International Labour Organisation (ILO)
which has ratified a total of 47 conventions and 1 protocol of the same brings into the
spotlight various lapses in the sphere of labour law and the alarming state of rural workers at
large which needs to be improved.\footnote{Ministry of Labour and Employment,(June 19$^{\rm th}$ 2020, 8:34 P.M), \url{https://labour.gov.in/lcandilasdivision/indiailo}} The situation of migrant labourers is not merely an issue
of national importance but an international humanitarian crisis by the virtue of India’s
ratification of various international treaties and conventions concerning human rights and
rights of labourers, children and women.

\noi
Though Guidelines issued by the Ministry of Home Affairs as per the Essential Commodities
Act, 1955 categorically exempt “essential services” from the lockdown, lack of coordination
and miscommunication resulted in mayhem and fear among the general masses even after the
existence of enough authorities in place in theory as per the legislations. In this unexpected
situation, a lot of statutes were reviewed and directions, notices and orders were passed every
day by State Governments in pursuance of safety measures directed by the Central Government. Section 188 of the Indian Penal Code\footnote{Indian Penal Code, No. 45 of 1860, INDIA CODE (2002), Vol. 18} was a recurring of these orders. Since
the announcement of the lockdown, the police machinery had adopted various techniques,
ranging from non-violent method of clicking a picture of the violators holding placards
saying how they are ‘samaj ke dushman’ (enemy of society) to causing direct physical
violence against public ranging from vicious lathi charges to dragging ‘violators’ on the
roads. A live example would be the events that unfolded in several parts of Gujarat where
tear gas and lathis were used to tame helpless migrant workers. More than 90 workers were
immediately detained for violating the orders imposed under the Epidemic Diseases Act,
1897.\footnote{Mahesh Langa, \textit{Migrant Workers, police clash in Ahmedabad}, THE HINDU, (May 18 2020, 9:43 P.M),
\url{https://www.thehindu.com/news/national/other-states/migrant-workers-police-clash-inahmedabad/article31613118.ece}} It is important to note as previously stated that these workers usually had no safe
housing in the city/town of work thus thousands of them walked miles in groups to reach
home struggling to make ends meet for them and their families. Furthermore, there is the case
of a man who died after being beaten up by the police in West Bengal. The ‘violator’ had
stepped out of his house to buy milk. Imagine the lack of humanity and the plight of the
family whose breadwinner stepped out to buy milk and come back home in a coffin. Even the
Disaster Management Act, 2005 which was the primary central law in force to contain the
spread of coronavirus provides for a maximum arrest for a period of one year for proven noncompliance of orders by the concerned authorities.\footnote{The Disaster Management Act, No.53 of 2005, INDIA CODE (2005), Vol 12} However, the police taking the law in
their hands with the blatant use of force and violence is a clear violation of Human Rights to
say the least but to look at it legally, one needs a closer look into relevant provisions. 

\heading{Legal Aspects and Provisions}

\noi
{\large\it\bfseries{Constitutional Provisions}}

\noi
The COVID-19 crisis not only saw loss and suffering every single day but also an uneven and
inequitable treatment of the persons suffering from the same crisis. The divide between the
privileged and unprivileged section started becoming clearer when one side had the ability to
stock up necessities, food supplies and toiletries lasting for over a month while on the other hand, some persons on the road could not even procure one meal a day. Right to Equality
enshrined in Article 14 of Constitution of India\footnote{INDIA CONST. art 14 } was unfortunately denied to more than half
of the Indian population when the government failed to treat everyone facing the same health
crisis equally. The effects of the money and muscle power is evident from the time and effort
used by the government to organize flights for students and other residents mostly from good
income families who were stranded abroad while thousands of migrant workers often from
marginalised sections were denied even train tickets from one part of the country to their
homes. The migrant crisis also brings focus on the blatant breach of the principle of ‘equal
pay for equal work’ enshrined under Article 14. The same principle was elaborated and
argued in State of \textit{Orissa v. Balaram Sahu}\footnote{State of Orissa v. Balaram Sahu, AIR 2008 SC 5165} where there was a contention on the entitlement
of equal pay by the daily wage, casual workers in the State of Orissa for undertaking the same
responsibilities and duties as the permanent employees. The Supreme Court elaborated on
this point of law by upholding the dissimilarity between the duties and responsibilities of
casual workers and permanent employees. The Apex further emphasized on the qualitative
difference under Article 14 of the Indian Constitution. However, the Court made it clear
through the judgement that the State must ensure that minimum wages are prescribed and
paid to workers in an attempt to bring socio-economic parity.\footnote{State of Orissa v. Balaram Sahu, AIR 2008 SC 5165}

\noi
Article 19 of the Indian Constitution is an umbrella provision constituting six freedoms
within its purview.\footnote{INDIA CONST. art 19 } The ones applicable to the migrant crisis are Article 19 (1) (e) that
specifically provides for the right to reside and settle in any part of the territory of India read
with Article 19 (1) (g) which allows for any person to practise any profession, or to carry on
any occupation, trade or business.\footnote{INDIA CONST. art 19} True economic growth and realization of these
fundamental rights in its true sense without compromising on the lifestyle would occur only
with even and equitable territorial development all over the country. In the present situation,
the workers are often forced by their dire circumstances to migrate in order to run their
households often doing menial jobs.

\noi
The current issue at hand where there are instances of denial of basic necessities like food,
clothing, shelter with safe and healthy living conditions for the migrant workers can be
termed as a clear violation of Article 21 provided by the Constitution of India.\footnote{INDIA CONST. art 21} Over the
course of legal development, the scope of Article 21 has been evolved and made more
inclusive through various judicial pronouncements across the world. In \textit{Munn v. Illinois},\footnote{Munn v. Illinois, 94 U.S. 113 (1877)}
the meaning of life which is recognised as a human right which became a fundamental right
in most civilised nations was transformed to include quality of life. The same principle was
laid down in the Indian landmark judgement of \textit{Kharak Singh v. State of Uttar Pradesh}\footnote{Kharak Singh v. State of Uttar Pradesh (1963) AIR 1295, 1964 SCR (1) 332 (India)}
expanded the phrase personal liberty to include more than mere animal existence. 

\noi
In \textit{Delhi Development Horticulture Employee's Union v. Delhi Administration},\footnote{Delhi Development Horticulture Employee’s Union v. Delhi Administration, AIR 2010 SC 6645} the
petitioners who were daily wager employees in the Jawhar Rozgar Yojna filed a petition
contending their right to life. They claimed that their right to life under Article 21 included
the right to livelihood and thus, they were entitled to the right to work. The Apex Court
through its verdict dictated at that time that the Indian State had not found a uniform effective
mechanism for the implementation of right to livelihood as part of fundamental rights under
Part III of the Constitution. It instead laid down the onus of the Directive Principles of State
Policy prescribed under Part IV which directs the State through Article\footnote{INDIA CONST. art 41} to make effective
provision for securing the same according to the capacity of its economic resources within the
limits of development.\footnote{Delhi Development Horticulture Employee's Union v. Delhi Administration, AIR 2010 SC 6645}

\vspace{-.05cm}

\noi
In another instrumental judgement of \textit{D.K. Yadav v. J.M.A. Industries}, the Hon’ble Apex
Court while deciding on the matter of termination of service of a worker held that Article 21
of the Constitution that lays down the fundamental right to life also includes the right to
livelihood and therefore no worker could be terminated from their service in an unfair, arbitrary and unlawful manner as that would deprive the person of their means to earn their
living.\footnote{D.K. Yadav v. J.M.A. Industries, AIR 2015 SC 5564}

\vspace{-.05cm}

\noi
The Constitution of India also provides for Directive Principles of State Policy under Part IV
(Article 38-51) to aid in achieving the goals of a welfare state and social, economic and
political justice as well as equality of opportunity and status as mandated by the Indian
Preamble. Article 39 of the Constitution aims to secure to the citizens the right to adequate
means of livelihood and puts a responsibility on the State to also ensure equal pay for both
men and women for the same job or activity.\footnote{INDIA CONST. art 38-39} Right to work and secure wages with a good
standard of living is provided for in Article 41 and 43.\footnote{Id at art 41-43} Even though Directive Principles of
State Policy are not enforceable in the Court of law and citizens cannot approach the Court to
get protection against the violation of these rights, the judgement of the landmark case of
\textit{Minerva Mills Ltd. \& Ors v. Union of India}, DPSPs were given the equal weightage as the
Fundamental Rights and they were placed on an equal footing.\footnote{Minerva Mills Ltd \& Ors v. Union of India (1980) AIR 1980 SC 1789} The principles enshrined in
Articles 39(a) and 41 must be given enough weightage for the true realization of enforceable
fundamental rights. The obligation upon the State to secure adequate means of livelihood and
the right to work to each and every citizen comes with the unspoken inclusion of the right to
livelihood within the broad umbrella of right to life. Any denial of the right to livelihood
which is outside the exception of procedure established by law is a downright offensive to the
principles of Right to life under Article 21. 

\vspace{-.05cm}

\noi
{\large\it\bfseries Legislative Provisions}

\vspace{-.05cm}

\noi
On account of containing the spread of COVID-19, the Prime Minister of India, Narendra
Modi announced a nationwide curfew for a period of 21 days on March 24, 2020 invoking
the strict implementation of The Disaster Management Act, 2005. The impact of the same
could be seen on everyday basis of almost every citizen of this democratic nation. The
legislation has a wide interpretation of the word ‘disaster’ which is not limited by its name and takes within its purview all man-made as well as natural disasters which have caused or
have the potential of causing human loss and suffering and the control of which is beyond the
control of society in general.\footnote{Supra note 5} The Act also establishes Disaster Management Authority at
National, State and District levels. Section 51 of the Disaster Management Act, 2005
prescribes penal punishments for persons non-complying with the directions and orders of
Central or State Government or any other authorities established under the Act.

\noi
Another legislation which was in operation during the COVID-19 crisis was the Epidemic
Diseases Act, 1897.\footnote{The Epidemic Disease Act, No. 3 of 1897, INDIA CODE (2013), Vol. 12} The legislation was amended by the recent Epidemic Diseases
(Amendment) Ordinance, 2020.\footnote{Epidemic Diseases (Amendment) Ordinance, 2020, INDIA CODE (2020), Vol. 6} The original Act which aimed to contain and prevent the
spread of dangerous diseases has been modified by the ordinance to include specific
provisions for healthcare personnel who are involved at the frontline to fight the deadly virus.
The key features of the Act include safeguards and protection against any act of violence
committed against the healthcare service workers. It also makes provisions for quarantine
facility owing to the specific contagious nature of the disease at hand.\footnote{Epidemic Diseases (Amendment) Ordinance, 2020, INDIA CODE (2020), Vol. 6} Although the
legislation claims to cover all dangerous diseases, the lack of any comprehensive definition of
a dangerous disease puts blot on the effectiveness of the Act and its application on potential
diseases and health catastrophes of future. Moreover, the Act places immense powers in the
hands of the government authorities without any provisions prescribing punishments or
compensation against their illegal actions which creates a lot of scope for misuse and
arbitrary use of power by the State and its agencies. 

\noi
The recent instances of police brutality that came into the forefront in 2020 forces one to take
a closer look at the penal provisions. According to the language of Section 188 of the IPC,
the provision mainly deals with any kind of disobedience of orders, summons, directions
issued by the State which includes all public servants. The sanction mandated by this
provision however in no interpretation implies violence on the violator. Even if the
disobedience or omission to do an act required by the State relates to human life, health and
safety which was definitely the case with the COVID-19 virus, the punishment cannot be more than imprisonment of 6 months plus fine\footnote{Supra note 10, Section 188} and that too if the step by step procedure is
followed as per the Criminal Procedure Code.\footnote{Code of Criminal Procedure, No. 2 of 1974, INDIA CODE (2002), Vol. 13}

\noi
{\it\bfseries This further takes us to the notion that whether the police officers get an authority to
initialize the attempt of lathi charge in case of the irregularity of violators?}

\noi
The answer to this rhetorical question is no. At most, the police have authority to detain an
individual for not adhering to the conditions of lockdown and gives the offender right to be
released on bail on account of offence being bailable.

\noi
The outcome of lathi charges by police authority and mercilessly beating the people who
went out to accommodate essential goods for their mere survival are an example of the major
misuse of powers done by the state authority in order to mislead a riot and where the people
of the state who chose the government ultimately becomes the victim in such cases. In these
instances, what most of the common public can do is to educate one another and make them
understand the rights and obligations which the Constitution of India has given to the citizens
so that they can understand what rights cannot be infringed if such cases of curfew arise. The
curfew is justified on account of public welfare but the disproportionate use of force is not
justified. If the curfew is seen from the perspective of public welfare then it can be somehow
justified on the grounds of social, secular and republic but when there is use of force in such
circumstances then it is not justified.

\heading{Affirmative Action}

\noi
After much chaos and suffering, it was the Apex Court that took \textit{suo moto} cognizance of the
daily distress caused to the migrant labourers by recognizing their problems and deciding to
pass certain directions in their favour. A group of 20 senior advocates wrote a letter dated 25$^{\rm th}$
May 2020 emphasizing on the issue as a violation of fundamental rights and the situation of
migrant workers as a massive human rights crisis.\footnote{Kruthika R, Jai Brunner, \textit{COVID Coverage: Migrant Labourers} (9$^{\rm th}$ June 2020, 6:30 P.M),
\url{https://www.scobserver.in/the-desk/migrant-labourers-crisis-and-the-supreme-court}} Following this initiative, a three-judge bench of the Hon’ble Supreme Court directed the Centre and State Governments to facilitate
the free and safe return of all the migrant workers who were stranded away from their homes
or villages without any means to go back.\footnote{The Print, \textit{Supreme Court asks Centre, States to send migrant workers home free of charge}, (19 June 2020,
8:44 P.M), \url{https://theprint.in/india/supreme-court-asks-centre-states-to-send-migrant-workers-home-free-ofcharge/445173/}} By virtue of good media coverage, several
people in the country became aware of this ongoing crisis. As a result of which, many Public
Interest Litigation (PILs) were filed seeking fulfilment and redressal of specific demands
from the Court of Law. The case \textit{Alah Alok Srivastava v. Union of India}\footnote{Alah Alok Srivastava v. Union of India, AIR 2012 SC 4435 } is one such petition
demanding a dignified treatment to the migrant labourers by the local administration and
concerned police authorities.\footnote{Madhunika Iyer, \textit{Migration in India and the impact of the lockdown on migrants}, (10 June 2020, 10:00 A.M),
\url{https://www.prsindia.org/theprsblog/migration-india-and-impact-lockdown-migrants}} It also aims to secure provision of food, water, medical
treatment and temporary shelter home/accommodation to the needy labourers till the situation
gets better.\footnote{Supra note 1} At the territorial level, governments, particularly that of Kerala and Tamil
Nadu, spared no idle opportunity to establish instruments, for example, direct benefit transfer
and transfer plans to aid the helpless handle the emergency as much as possible. While an
economic relief package of \rupee 1.7 lakh crore was declared by the central government, this
amount would not be adequate to cater to the gravity of the issue. Moreover, the accessibility
and availability of this sum to the un-registered migrant workers who usually do not possess
bank account or ration cards also raises a crucial question and points to the loopholes in the
framework.\footnote{Pranab R. Choudhury, \textit{The Lockdown Revealed the Extent of Poverty and Misery Faced by Migrant Workers},
THE WIRE, (July 16, 2020,8:34 P.M), \url{https://thewire.in/labour/covid-19-poverty-migrant-workers}}

\noi
Seeing the aftermath of the migrant crisis and its continuing financial suffering, the Lok
Sabha initiated three new labour bills in the month of September 2020 in an attempt to
consolidate, update and simplify laws relating to the matters of trade and commerce.\footnote{Aayushi Kiran, \textit{A Critical Analysis the three Labour Bills 2020}, (20$^{\rm th}$ October 2020, 6:40 P.M),
\url{https://www.latestlaws.com/articles/a-critical-analysis-the-three-labour-bills-2020/}} Three
instrumental codes pertaining to the present matter came into existence after this major policy change namely The Code on Social Security, 2020,\footnote{The Code on Social Security, No. 36 of 2020, INDIA CODE (2020), Vol 13} The Occupational Safety, Health and
Working Conditions Code, 2020\footnote{The Occupational Safety, Health and Working Conditions Code, No. 37 of 2020, INDIA CODE (2020), Vol 1} and the Code on Wages (Central Advisory Board) Rules,
2021.\footnote{Madhunika Iyer, \textit{Migration in India and the impact of the lockdown on migrants}, (10 June 2020, 10:00 A.M),
\url{https://www.prsindia.org/theprsblog/migration-india-and-impact-lockdown-migrants}} With proper coordination and sound implementation at the ground level, the Indian
legal machinery can avoid the devastating after effects of an emergency situation such as the
migrant crisis of 2020 in future. 


\heading{Conclusion and Suggestions}

\noi
Since most of these workers are employed in a very small and medium enterprises in the
informal sector, in enterprises that are teetering at the edge of collapse, given the deepening
economic slowdown. Preventing shutdowns of these enterprises in the informal sector in
order to ensure their survival is also essential in testing times. These enterprises should be
provided with special assistance, such as a special economic package from the government to
stay afloat in drastic circumstances. Most importantly, the need of the hour is to ensure that
adequate measures are taken to cushion the impacts of the pandemic on the working poor in
the midst of a deepening economic slowdown. This has to be done by both the central and
state governments working in tandem not only to ensure adequate resources but also to
implement schemes suited to the unfolding situation at the ground level. Unemployment and
money problem, along with public goods and transportation shut down, hundreds and
thousands of migrant workers who do not have a job security or protection were forced to
trek hundreds of miles back to their home towns and villages along with some deaths on their
journey. 

\noi
The crisis which had affected the life and property of thousands of people across different
states brought into the limelight the inefficiency of our existing laws and policies concerning
this section of population. One silver lining of this situation is that these instances and the
plight of the workers were widely reported by the media channels. Without the imposition of
lockdown as a side effect of COVID-19, the plight and inefficiency of policies probably
wouldn’t have been recognised as an early stage. The time is now for the statesmen, legislators and policy makers to take action on this newly spread information and awareness
of this human rights issue. 

\noi
The labour laws have evolved all across the globe since the Industrial Revolution in order to
make the workplace and living conditions more favourable for the working class. However,
the past transformation does not imply that the present policies are perfect without flaws. The
laws are dynamic and policies must continue changing and becoming more inclusive
according to changing contemporary social-economic scenarios. The unexpected turn in our
normal lives caused by the corona virus has given us time to reflect, become more aware and
most importantly be prepared as much as possible for any potential damage to the social
structure so as to preserve the dignity of the lives of the citizenry.

\noi
Despite the doctrine of sovereign immunity and the fact that the virus could not have been
stopped at its origin as it was beyond the control of the Communist-Party led nation, the
nation still needs to be made liable for the costs which other great nations have to bear just
because of its negligent conduct of not being able to duly inform and give cautions or
warning about the transmission of the virus among humans, which could have prevented this
massacre or at least reduced the suffering to a certain limit. The pandemic has cost the world
nearly 26,40,349 deaths as of 14th March 2021 all across the globe\footnote{World Health Organisation, Coronavirus disease (COVID-19) Pandemic (14 March 2021, 12:00PM),
\url{https://www.who.int/emergencies/diseases/novel-coronavirus-2019}} while major economies
and developed nations have also been broken. China caused a humongous aftermath which
needs to be significantly realised by any way possible. If there would be any case where
justice should be done, then this should be it. In the sectorial sector, the Micro Small and
Medium Enterprises (MSMEs)\footnote{Ministry of Micro, Small \& Medium Enterprises, \url{https://msme.gov.in/}, last accessed on 20 November 2020 } which owes 30\% contribution to economic development, is
now totally out of action. But the government has regarded \rupee 20,000 crores for this sector
including other sectors like real estate, aviation, tourism and automobiles which is a view that
can pull up the economy back to its development process to a certain level. The status of the
health sector in India is in a crucial state and the impact of the pandemic in such a populous
nation digs a deep hole in its already suffering state. The private health sector is facing a lot of challenges whether it is for the provision of ventilators, manpower, hospital beds, testing,
types of equipment, pharmaceuticals, or other consumables.

\noi
The impact of the migrant crisis during the COVID-19 pandemic must teach and serve as a
wake-up call to the concerned authorities to prepare for more efficient laws and policies for
the migrant labourers along with organised implementation at the ground level so that the
nation is better equipped to handle and contain any potential crisis that may arise in future.

\noi
To look on the brighter side, India is in its 12$^{\rm th}$ month since the virus took over and the
journey has been a rollercoaster ride. With the great cooperation, dedication and selfless
efforts of doctors, government officials and manufacturers/workers ensuring us basic
necessities, the nation has witnessed an incredible recovery rate from the virus. According to
the live data released by the Union Ministry of Health and Family Welfare, the recovery
percentage of India as of 23$^{\rm rd}$ December 2020 is 95.65\%. However, it is important to note that
the crisis is not yet over as the workers are still facing the financial brunt of their
displacement even months after the lockdown. With the recognition and detection of a new
variant of SARS-CoV-2 virus by the United Kingdom government, there might be a
possibility of imposition of lockdowns or restrictions in future with grave impact and further
worsening of the situation of these migrant labourers. The adequate precautionary measures
need to be taken at this preliminary stage of a potential novel virus to avoid the fatal impacts
on the health and situations of people.\footnote{Ministry of Health and Family Welfare, \url{https://www.mohfw.gov.in/}, last accessed on 30 November 2020}


\end{multicols}
