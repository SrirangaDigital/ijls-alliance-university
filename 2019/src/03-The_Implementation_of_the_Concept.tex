\setcounter{figure}{0}
\setcounter{table}{0}
\setcounter{footnote}{0}

\articletitle{The Implementation of the Concept and Provisions of Copyright Legislation in the Indian Film Industry}
\articleauthor{Paramita Choudhury\footnote{Assistant Professor and Research Scholar, School of Law, Galgotias University.}}
\lhead[\textit{\textsf{Paramita Choudhury}}]{}
\rhead[]{\textit{\textsf{The Implementation of the Concept...}}}

\begin{multicols}{2}

\heading{Introduction}

%~ \vspace{-.1cm}

\noi
When consumers enter “digital version”, “free download”, “Torrent” or “Pirated Install” on
their web browsers, malware assaults and cyber vulnerabilities are two of the top result that
appear. As a result of digitalization, traffic to pirated websites has surged, and so has online
video consumption. According to many, digital piracy authority, piracy in India increased by
62 percent due to this development of internet and cyberspace. Stremio, Popcorn Time,
Solarmovies, 123Movies, and Tamil Rockers are some of the most popular pirated sites.

%~ \vspace{-.1cm}

\noi
The cyberspace is amongst the most crucial innovations that humanity has produced until
time. The mode of communication that is increasing at the highest rate is conversational. The
volume of internet traffic continues to increase by one-hundred-fold every 100 days. There is
little doubt that its influence on the spread of knowledge is significant. The introduction of
the internet has transformed the mode, quality, and speed of information transfer. Information
is distributed globally via this medium, and it is possible to assess, read, print, and download
information from all around the world. No single individual or entity is able to exercise
absolute control over the internet. This is referred to as the information technology
communications anarchy since there is no centralised authority for information technology
communications.

%~ \vspace{-.1cm}

\noi
While the internet and the expansion of knowledge-based enterprises have resulted in the
creation of a new type of property, this new property is the result of human intellect and
effort. This newly devised type of property is known as intellectual property. This type of
property is derived from the product of the intellect of human, such as fictional works,
paintings, artistic designs, and other forms of creativity in the fields of practical or fine arts.
The set of rights includes the right to buy, sell, or lease. This class of things also includes
copyrights, patents, trademarks, and designs. Intellectual property differs from traditional of
asset as its worth and applications of such asset are very unclear. Because it is easily and freely available to the general public, it is also more likely to be stolen. Intellect is a primary
contributor to the making of these goods. It is tough to preserve, which makes it an important
area to safeguard.

\vspace{.1cm}

\heading{Origin of Copyright Law}

\vspace{.1cm}

\noi
Although it was originally implemented during colonial occupation, copyright legislation in
India has gone a long road ahead ever since. One of the very first laws regarding copyright
was passed in India in 1847 by the then Governor General of the country. When the
Copyright Act of 1911 came into effect in England, it become inherently enforceable to India
because India was then an intrinsic part of the British Raj. The copyright act (the Act of
1957) was in effect for the entire country from when it was enacted until independence in
1958, after a fresh copyright law (the Act of 1957) went into effect. Following the enactment
of the Act, several revisions have been made to it.\footnote{\textit{Study on Copyright Piracy in India}, sponsored by Ministry of Human Resource Development Government of India, (Jan.29, 2018, 10:04 AM)\\ \url{http://copyright.gov.in/Documents/STUDY}\%20ON\%20COPYRIGHT\%20PIRACY\%20IN\%20INDIA.pdf.}

%~ \vspace{-.1cm}

\noi
According to the 2012 changes, Indian Copyright Law is now in accordance with the Internet
Treaties and accords namely the WIPO Copyright Treaty (WCT) and the WIPO
Performances and Phonograms Treaty (WPPT). Additionally, by including technological
protection mechanisms in the legislation, the new law assures that fair use of copyrighted
materials does not suffer in the digital era because of specific fair use rules. A large number
of revisions have been added to the bill to facilitate accessibility for handicapped individuals,
making the bill friendlier to authors, special provisions for people with disabilities, and
streamlining copyright management.

\noi
To categorise revisions made via The Copyright (Amendment) Act 2012, one can think of
them as

\vspace{-.1cm}

\begin{enumerate}[label=$\bullet$]
\itemsep=0pt
\item Including changes to privileges in works of art, cinematograph movies, and audio
recordings.

\item Right related amendments included in the WCT and WPPT.

\item Additional author-friendly modifications, about the manner of Assignment and
Licenses.

\item Measures to improve access to copyrighted works.

\item Strengthening enforcement of online piracy while safeguarding against it.

\item Minor adjustments, like the Reform of the Copyright Board, should be made to our
copyright system.\footnote{\textit{Inside Views: Development In Indian IP Law: The Copyright (Amendment) Act 2012,} (Jan. 29,2018, 10:20 AM)\\  \url{https://www.ip-watch.org/2013/01/22/development-in-indian-ip-law-the-copyright-amendment-act-2012/.}}
\end{enumerate}

\vspace{-.1cm}

\noi
The Indian Copyright Act provides protection to original literary, dramatic, musical, and
artistic works, cinematic films, and sound recordings. It is important to recognise that
when you say anything is unique, you imply it is not a rip-off of someone else's work or
idea. This Act grants copyright holders the power to carry out a range of actions or
authorise others to do so. Prominent amongst these are:

\vspace{-.2cm}

\begin{enumerate}[label=$\bullet$]
\itemsep=0pt

\item to reproduce the work in material form;

\item to publish the work;

\item to perform the work in public or communicate it to the public;

\item to produce, reproduce, perform or publish any translation of the work;

\item to create a cinematic film or a recording in connection with the work;

\item to make any necessary changes to the work; and

\item to perform any of the acts in regard to a translation or adaptation of the work
listed in sub clauses to (a) to (f))\footnote{Chapter-III Legislative Provisions in India, (Jan. 29, 2018, 11:23 AM)\\
 \url{https://shodhganga.inflibnet.ac.in/bitstream/10603/128961/16/09_chapter}\%203.pdf.}
\end{enumerate}

\vspace{-.4cm}

\heading{Cinematography in the Realm of Copyright Law}

\vspace{-.1cm}

\noi
The copyright sector, throughout the globe, in general and film business in particular supports
not just to cash creation for its rightful owners but also safeguards the labour that is dedicated
to it. Aside from this, the government's exchequer also obtains cash through a revenue
collected by the entertainment tax. India is home to one of the biggest movie industries in the
world, with around one thousand movies made every year.

\vspace{-.1cm}

\heading{Cinematograph Film}

\vspace{-.1cm}

\noi
In the case of a cinematograph film, the soundtrack is present, even if it is silent. In addition,
it encompasses any cinematography-style work done by any procedure similar to
cinematography. A film believed to be an activity undertaken by a methodology akin to photography is defined as a video film. An unedited performance of a live event such as a
sporting event, or a theatrical or musical performance, may be filmed for use in a movie.

\vspace{-.1cm}

\noi
In association with the film, the music used in the film involves coordination of the film's
cinematographic film and is therefore protected by copyright. In \textit{Balwinder Singh v. Delhi
Administration,}\footnote{AIR 1984 Delhi 379.} and \textit{Tulsidas v. Vasantha Kumari},\footnote{(1991) 1 LW (Mad) 220 (229)} Video and television is also
cinematographic works, according to the argument in both the judgments.

\vspace{-.1cm}

\noi
Copyright relates to the right to authorise most filmed performances in cinematography and
especially in the following events:

\begin{enumerate}[label=$\bullet$]
\itemsep=0pt
\item to make a duplicate of the movie;

\item to permit the film to be seen and heard in public in the case of visual pictures, and in
the case of audio, to be audible in public in the case of sound.;

\item to use such sound track to produce any record containing the audio in the segment of
the musical score connected with the film;

\item to disseminate the film via television.
\end{enumerate}

\vspace{-.2cm}

\noi
In general, copyright protects two types of rights: exploitable and moral rights. Exploitable
rights (sometimes known as "economic rights") are those that the work's owner can use to
make money. The sole freedom to make copies, adaptations, or images of copyrighted
content, as well as the right to licence these rights to others, belongs to the copyright owner.\footnote{Copyright Act of 1957.}
In addition to his ownership of his own creation, the author of a work is always guaranteed to
maintain his or her moral rights. Moral rights are rights attached to an individual's identity as
an author. Authors have the right to select when and if their work will be published, as well
as the right to retain authorship and the responsibility to protect their reputation. The fact that
many instances both India and the U.K. preserve the author's unique entitlement to derivative
works supports the author's distinct claim to derivative works. The idea itself (form, style,
and arrangement), but not the representation of that idea, is copyrightable. This means that
even though two authors separately conceive the same idea, they are not blocked from
publishing their work as long as they use distinct materials.

\noi
A revolution in the motion picture industry has occurred across the world during the last
several years. A new trend has emerged in the world of entertainment, particularly in the area
of digital media. Thanks to the broadband networks made possible by broadband networks,
users can now freely download unauthorised copies of pre-recorded media files, referred to as
"pre-cords," over P2P networks. Consumers who have downloaded a song or other file are
able to transmit it to other users in digital format using the P2P software. The precipitous
decrease in the growth of the recording industry over the last few years is attributable to file
sharing. While online piracy could lead to the loss of intellectual property rights in digital
goods, that doesn't mean all piracy leads to copyright infringement. With the introduction of
the internet, new methods of copyright infringement have emerged, which makes it much
more difficult to combat the damage caused to copyright-based companies. In today's music
market, both online piracy and illicit downloads have diminished sales of genuine CDs and
lawful digital distribution are becoming a replacement for legitimate CD purchases.

\noi
Digitalization has posed a serious threat to entertainment industry. It is manifested from
scheme of Digital Literacy Scheme for rural India. The widespread adoption of these
approaches has the potential to create more sustainable and successful business models across
several media sectors. Cable television officially became digital in 2012, marking the
beginning of a long process of digitization. In various phases, initiatives have been
undertaken.

\noi
A substantial amount of progress was made during the first phase of implementation in the
four metros. Industry is now working to realise short-term benefits that include the potential
to commercialise content, greater transparency, and fairer revenue distribution throughout the
value chain. These short-term benefits are achieved by a reduction in the costs of content
delivery and by making more money available for investment in distinguishable and classy
content.

\noi
Even when projected timelines are accounted for, it is likely that the implementation of Phase
2 digitization will happen on a similar timeframe to what has been envisaged so far, but with a delay. The entertainment industry is almost 77\% digitized. Indian digital industry is expected to cross 3100 crores by 2020.\footnote{Report published by Delliote International, April 2015, (Jan. 30, 2018, 11:00 AM)\\  \url{https://www2.deloitte.com/content/dam/Deloitte/in/Documents/technology-mediatelecommunications/IMI}\%20report\_singlePage.pdf.}

\noi
The growth of India in 2014 ranked it as the world's fastest-growing smart phone market.\footnote{E Marketer newsletter, 29 December 2014,KPMG Report 2014, (Jan. 30,2018, 11:35 AM)   \url{https://assets.kpmg/content/dam/kpmg/pdf/2014/03/FICCI-Frames-2014-The-stage-is-set-Report-2014.pdf.}}
This growth was achieved due to the implementation of the Digital India project. India was
the first country in 2014 to be placed on the United States' International pirate watch list,
which included countries like Nigeria, Bangladesh, Somalia, and others where piracy is a
problem.\footnote{The Hollywood Reporter \textit{“India join China, Russia, Switzerland on Piracy watch list”,} 24 June 2014, (Feb.
12, 2018, 11:50 AM)\\ \url{https://www.hollywoodreporter.com/news/general-news/india-joins-china-russiaswitzerland-714572/.}} India has negotiated co-production agreements with China and Canada.\footnote{The Hollywood reporter, \textit{“India, china sign film co production,”} 18th September 2014, (Jan. 29, 2018, 12:30 PM)  \url{https://assets.kpmg/content/dam/kpmg/pdf/2015/03/FICCI-KPMG_2015.pdf.}} Such
Treaties not only allow Indian filmmakers to benefit from tax breaks, but also from reduced
visa requirements in partner nations.

\vspace{-.1cm}

\noi
While India is appreciated for its efforts in the past decades to fight piracy in terms of the
impacts of it was, the raid on criminal camcorder pirate cartels Yamraj and NiCkkk DON has
been certainly one of the main effects that came from that.\footnote{The Hollywood Reporter, \textit{“China Asia-India the problem areas in camcorder piracy cases”} 8th December 2014, page 10, (Jan.30,2018, 01:05 PM)  \url{https://www.hollywoodreporter.com/movies/movie-news/cineasiaindia-china-problem-areas-755349/.}} While the multiplex tickets of a legitimate recently released movies cost INR 150-200, the pirated DVD costs INR 30-40.\footnote{\textit{Id.}}

\vspace{-.1cm}

\heading{Meaning of the Term Piracy}

\vspace{-.1cm}

\noi
Piracy is an unlawful duplication of content and is then offered on the open market at a
significantly cheaper price. It is one of the prevailing threats for entertainment. Facilitated
usage of technology has become a cause for wild piracy. Piracy is now a straightforward
business. At a relatively modest price, CD authors are available on their own. Wherever there
is a doubt about penalising wealthy countries, the penalty of pirating connected issues is quite
serious, whereas the government has not paid adequate attention in Asiatic countries and
particularly in India owing to more interesting issues.

\vspace{-.1cm}

\noi
After being overtaken by China, India became the second-largest country with respect to the
number of internet users.\footnote{AMAI-IMRB Internet in India Report,2014, (Jan. 30,2018, 01:45 PM)\\
 \url{https://cms.iamai.in/Content/ResearchPapers/e7cb87e7-74b3-4c2f-8bfc-09ccfd7fb265.pdf.}} In India more than millions of mobile internet users have been identified by January 2016.\footnote{Economic Times, Sunday, January 10,2016.} Wireless modems, notably 4G, sustained robust increase in the number of 3G subscribers, while at the same time, significant amounts of 2G coverage in
rural India as well as through other digital ecosystem participants in support of Digital India
Program had contributed to making this possible. The availability of inexpensive Smartphone
and tablet devices has fuelled the growth of mobile screen sizes. the number of Smartphone
users in India is estimated to be 10\%.\footnote{KPCB Internet Trend Report,2014, (Jan. 30,2018, 03:10 PM), \url{https://www.kleinerperkins.com/perspectives/2014-internet-trends/.}}

\heading{Types of Piracy}

\noi
{\large \bfseries 1.~\textit{Internet Piracy:}}  Illegal downloading is the downloading and dissemination of unauthorised
copies of intellectual property, such as films, shows, songs, games, and software
programmes, via the online file sharing network, rogue server, websites, and hacked
computers. Black market pirates are also known to utilise the online platform to sell
unlawfully replicated DVDs through online auctions.\footnote{LuigieProserpio, Severino Salvemini and Valerio Ghirnghelli, \textit{Management Entertainment Pirates Determinants of Piracy in the software, music and movie Industries,} p 34-36, 2015.}

\noi
{\large \bfseries 2.~\textit{Peer-To-Peer Piracy :}} One of the greatest threats to the current revenue model of the media
industry is the illicit sharing of files over peer-to-peer networks. This section serves to
provide a guide for media industry experts on the construction of P2P database networks,
particularly with respect to their capacity to assess responsibility and the abilities of media
companies to litigate and press charges against P2P software designers and consumers. When
everyone on the network has equal participation in the resources and direct peer-to-peer
communication does not require a centralised server or gateman, then this is considered a P2P
model. In typical client-server systems, the IP of the server is fixed. However, in P2P
systems, a search mechanism is activated that can locate the right node in real-time. Several
different approaches to peer-to-peer systems have evolved, some differing greatly with
respect to their search and storage strategies.

\noi
While Napster, as one of the young crops of P2P file-sharing websites, initially employed a
centralised index to keep and browse for items on the network, the development of another
network, Bit-Torrent, introduced decentralised searching and content distribution. The
Napster approach has participants connecting to a central database, where they publicly stated
the content, they intend to distribute. The user seeks certain material in the database, and so
acquires IP addresses of the servers where the content is located. Once they have this
information, they are able to immediately download the content from one of the nodes they
have discovered. Discovery and tracking of developers becomes easier with this centralised
system since it enables easy identification of all the files, and it is therefore straightforward to
trace who is providing and who is receiving data.\footnote{Sanjay Goel, Paul Miesing \& Uday Chandra, \textit{The Impact of Illegal Peer-to-Peer File Sharing on the Media Industry,} (Jan. 30, 2018, 03:45 PM)  \url{https://www.researchgate.net/publication/259729302_The_Impact_of_Illegal_Peer-to-Peer_File_Sharing_on_the_Media_Industry/link/5c701aa9299bf1268d1df998/download.}}

\noi
{\large \bfseries 3.~\textit{Theatrical camcorder piracy:}}  Filmmaker Cam cording or infringement copies of new
release films that emanate from cinema halls are quickly posted online following the
premiere of the film.\footnote{FICCI-KPMG REPORT 2015, (Jan. 30, 2018, 04:30 PM),\\ \url{https://ficci.in/spdocument/20723/Executivesummary-FICCI_KPMG-report-2016.pdf.}} This has a significant impact on distribution cycle, performance, and jobs. When someone arrives to the cinema hall through any form of recording device, such as
a camcorder, a voice recorder, or any other kind of equipment, they are known as camcording.

\noi
{\large \bfseries 4.~\textit{Cable piracy :}} The term "cable piracy" denotes to the illicit broadcasting of movies over
cable network. It is not unusual for films, specifically the most recent releases, to be shown
on cable with no permission from the copyright holder. Piracy is an uncommon occurrence
with satellite channels due to the fact that these are usually organised and are largely used for
distributing films without having paid for the necessary copyright permissions.

\noi
{\large \bfseries 5.~\textit{Software Piracy :}} The software piracy consists of first of all, using the software illegally,
and then disseminating it without authorization. While both small and large businesses are
plagued by software piracy, the degree to which this issue plagues them varies greatly.

\noi
{\large \bfseries 6.~\textit{Optical Disc Piracy :}} When optical disc piracy takes place, this activity refers to illegal
manufacturing, selling, distribution, or trading of discs in optical disc formats with motion
pictures on them, and the illegal fabrication and transmission of feature films.

\noi
{\large \bfseries 7.~\textit{Internet and Mobile Piracy :}} The further advancement of technology is definitely going to
lead to an increase in piracy, as the enforcement methods are presently so feeble. There are
very high hopes for the growth of the Indian mobile phone market, and it's one of the fastest
growing in the globe. There are many other methods of getting pirated materials, such as
accessing the Internet, downloading material from peer-to-peer websites, and torrenting. Files
like these can be significantly compressed and then transferred to a Smartphone or obtained
from the internet without further processing.\footnote{\textit{Id.}}

\heading{2016 National Policy on Intellectual Property Rights}

\noi
A huge step in the country’s strong protections and promotion of the Intellectual Property
Rights (IPRs) was taken with the launch of the National IPR Policy in 2016, which entailed
significant and revolutionary reforms on the board. To aid in fostering creativity and
innovation while also acknowledging the importance of intellectual property in the
development of the economy, the GOI approved a new policy. The Action plan for India's
quest to increase creativity and stimulate innovation details the various aspects involved in
the pursuit of such goals, including a well-informed public about intellectual property, an
expanding ecosystem for creating, commercialising, and enforcing IP, and policies and
practises that encourage innovation.

\vspace{-.1cm}

\noi
The first objective of the policy is to increase public knowledge of the fiscal, societal, and
cultural values of intellectual property rights. In order to facilitate this goal, additional IPR
educational initiatives will be implemented, such as the incorporation of IPR lessons in the
education system. Although school-age children are a substantial portion of the piracy
market, both in terms of films and of downloading copyrighted music, the students’ demand
for stolen materials isn't strong. It is intended that adding IPR studies in their coursework will
assist students comprehend the value of Intellectual Property rights as well as how
infringement of those interests results in financial loss to not just the rights holder, but also
seems to have a large impact on the country's economy. Additionally, IPR studies would
emphasize to at minimum a significant portion of the population who are not aware that they
may be inadvertently aiding piracy that the intermediaries for illicit items are not equal to
legitimate ones.

\vspace{-.1cm}

\noi
The second objective of the policy suggested that relevant revisions be made to the
Cinematograph Act, 1952 in order to include criminal penalties for the unlawful duplicating
of films and the fourth objective in the policy emphasises the importance of media and public
consciousness along with strict regulation procedures to prevent both physical and virtual
piracy.

\vspace{-.1cm}

\noi
The process of implementing guidelines and procedures as well as augmented cooperation
between many organisations and giving vision and direction on enforcing anti-piracy metrics;
integration and communicating of competence and best practises at the global level; study of
the scope of IP infringements in different sectors; analysis of regulatory issues and obstacles
in enforcing anti-piracy regulations; and introduction of suitable innovation remedies for
suppressing online piracy.

\vspace{-.1cm}

\noi
By working with partners to conduct fact-finding investigations to determine the degree of
piracy and also the causes for it as well as ways to counteract it, we will set the stage for
initiatives to prevent piracy.

\heading{Strategies for Combating Piracy}

\noi
{\large \bfseries 1.~\textit{The Legal Framework :}} The first step in combating piracy is the implementation of
legislation to protect copyrights, as described in the start of this essay. The Statute of
Anne, which is most commonly known as the beginning of modern copyright law, has
led to the formation of a multitude of domestic and international law and agreements
that help to keep copyright protection and pirate prevention simple worldwide. One
way to think of Berne Convention for the Protection of Literary and Artistic Works
(1886), Universal Copyright Convention (1952) and World Intellectual Property
Rights Copyright Treaty (1996) is as accords designed to protect copyright.
Regardless of the fact that the online is a worldwide instrument, the value of these
accords cannot be overestimated.

\noi
{\large \bfseries 2.~\textit{Network Administration :}} Additionally, with regard to the legal aspects, the question
of network administration is significant. Additionally, if a user cannot be induced to
refrain from downloading by the prospect of punishment, another alternative is to
restrict the authorized users from accessing the resources that they would use to
download the application. In the most prevalent circumstances, publishing companies
obtain ISPs (also known as Internet Service Providers, or ISPs) to block access to 
certain sites and Web addresses to restrict customers from accessing certain sites.
Some governments undertake these activities by creating their own country-wide
internet filters. As of April 2013, 29,000 websites are inaccessible in Turkey due to a
blockade imposed by the Turkish government. The filtration is used to ban a list of
websites, such as news sources, which contain pornography, other contentious
material, and websites that are known to have illegally copied another site's content.

\noi
{\large \bfseries 3.~\textit{Anti-Counterfeiting and Piracy Initiative (APEC) :}} APEC was enacted in 2005 and
dealt with minimising the spread of fake goods, as well as cracking down on the
selling of fraudulent products over the internet, while also raising awareness on IPR
security and border enforcement measures.\footnote{See; \textit{APEC Anti-Counterfeiting and Piracy Initiative,} (Feb. 10, 2018, 11:15 AM), \url{http://www.wcl.american.edu/pijip/go/research-and-advocacy/enforcement/anti-counterfeiting.}}

\noi
{\large \bfseries 4.~\textit{Anton Pillar Injunction :}} To receive the order, the claimant must establish the
satisfaction of the Court that the following requirements are fulfilled:

\noi
a) the harm is extremely serious.

\noi
b) evidence exists to prove that this individual has evidence in his possession that will
most likely be demolished before any application can be made; and

\noi
c) there is a probability that incriminating evidence will be wrecked before an
application can be made.

\noi
In Anton Pillar v. Manufacturing processes,\footnote{(1976) Ch. 55; R.P.C. 719.} The court agreed to implement a
mechanism of substantial relevance to some intellectual property rights, one that had
previously been submitted for approval. In these kinds of proceedings, the plaintiff
appeared before the High Court or Patents County court in Camera unaccompanied by
any notice to the defendant for an order that the defendant allow him with his solicitor
to inspect the defendant's premises and to seize copy of photograph material related to
the infringement. It is also possible for the defendant to be forced to provide the
infringed items, keep infringing stock, or reveal incriminating documentation. He may
also be asked to provide information, for instance regarding his supplier or the
whereabouts of the infringing items that have transited through his hands.

\noi
{\large \bfseries 5.~\textit{John Doe Order :}} Before a defendant's infringing activities result in harm to the
intellectual property protection of the artist, who created artistic works such as movies, songs, and so on, the rights of the inventor are preserved by granting a John
Doe Order. Also, it is likely that many people will identify John Doe's identity as
Rolling Anton Pillar, Anton Pillar, or Ashok Kumar. The long-established and wellrespected British Queen's Bench created the concept of a John Doe Order, a complex
equitable remedy that grants the plaintiff the right to seek and obtain an injunction
order from a hypothetical defendant; thus, it gives the plaintiff the opportunity to
prevent any evidence of their wrongdoing from being destroyed.\footnote{Ajay Sharma, \textit{John Doe Orders in Indian Context} (Feb. 12, 2018, 03:05 PM), \url{https://rmlnlulawreview.com/2017/10/25/john-doe-orders-in-indian-context/.}}

\noi
John Doe orders were first passed in Indian Courts in a case by the Delhi High Court,\footnote{Taj Television Ltd. \& Anr. vs. Rajan Mandal \& Ors, [2003] F.S.R. 22} which relied on the judicial systems of developed nations like Canada, the United States, England, and Australia.

\noi
As detailed processes for checking websites' listings are implemented with standardised
norms, judicial clarity and predictability to copyright holders and website users are provided.

\noi
Some good practises could be adopted from jurisdictions like Singapore, where courts have
also come to rely on testimony that the internet sites that were targeted were obstructed in
other jurisdictions, or that a large amount of traffic was generated, or that the web pages did
not agree with takedown notices that were served by the plaintiffs, or that the websites
included instructions for circumventing measures to disable access.\footnote{Disney Enterprises Inc, and Others vs. M1 Ltd and Others, [2018] SGHC 206.}

\noi
Besides the aforementioned ways to prevent online piracy of movies and additional tactics,
such soft legislation and technological techniques, are also accessible in this high-tech era. As
technology like watermarking and block-chain technology continue to evolve, new
developing technologies to combat unlawful material consumption, such as digital media, are
becoming possible. Additionally, the extent to which other soft legal enforcement measures,
such as public-private partnerships and infringing website listings, may be made public will
be revealed.

\heading{Best Practices}

\noi
All copyright infringement deterrence strategies and regulations must be built on an in-depth
analysis of the reasons for copyright infringement.

\noi
In India, laws governing copyright infringement do not equate to piracy, even if the legal
system is much weaker and the literacy and affluence in the country is lower. The same is
true of locations with varied histories and greater regulatory authority.

\noi
Furthermore, the new National IPR Policy aims to fuel additional alterations to India's IPR
framework, especially with regard to the streamlining of laws and regulations in order to
improve their efficiency and the stimulation of additional research in order to guide
legislative policymaking. Additionally, it ought to help make alterations to the Copyright Act,
which is often pushed to fit the pace of media convergence, the constantly changing content
landscape, and the advancement of technology. In addition, another area of reform involves
making the adjudicatory and redress mechanisms function more effectively, with the aim of
putting these mechanisms to the test to date. With regards to this, there is the further
challenge of reform for the functioning of the adjudication and redressal mechanisms, and
how they may perform and more effective which led the Delhi High Court releasing notice to
the Department for Promotion of Industry and Internal Trade (DIPP) and requesting a status
report.

\noi
It is also possible that India might examine additional strategies to pool resources and deal
with the pending cases by, for example, creating up expert IP forums and other bureaucratic
frameworks with protections to help the courts. In order to safeguard civil liberties, it is
necessary to guarantee that any administrative framework has been clearly defined; features
clear legal authority and are subject to scrutiny by the courts. In addition to serving as an IP
arbitrator, an IP ombudsman might also be established to aid injunctions, and additionally
verify the data made by both the litigants in such matters.

\end{multicols}
