\setcounter{figure}{0}
\setcounter{table}{0}
\setcounter{footnote}{0}

\articletitle{The Arbitration Amendment Act, 2019 and the Changing Arbitration Eco-System in India: Needs a Re-Look?}\label{2019-art5}
\articleauthor{Dr. Rohit Moonka\footnote{Assistant Professor of Law, Campus Law Centre, Faculty of Law, University of Delhi} and Dr. Silky Mukherjee\footnote{Assistant Professor of Law, Campus Law Centre, Faculty of Law, University of Delhi}}
\lhead[\textit{\textsf{Dr. Prashna Samaddar and Victor Nayak}}]{}
\rhead[]{\textit{\textsf{The Arbitration Amendment Act...}}}

\begin{multicols}{2}

\heading{Introduction}

\noi
Arbitration has always been a preferred mode of dispute resolution especially in commercial
matters. For long, India has been striving to become a preferred seat of arbitration not only
for domestic arbitration but also for international commercial arbitration since the enactment
of Arbitration and Conciliation Act, 1996.\footnote{The Arbitration and Conciliation Act, 1996, Act No.26 of 1996} In this endeavour, there have been several ups and downs noticed by the observers which were either due to certain judicial pronouncements or fallacies in the drafting of the Arbitration and Conciliation Act, 1996 itself.\footnote{Sumeet Kachwaha, The Indian Arbitration Law: Towards a New Jurisprudence, 10 INT. A.L.R. 13, 17 (2007)}  Soon it was
realized both by the judiciary as well as by the legislature that they need to change the
approach, if India has to become a preferred seat of arbitration. In this context, after various
failed attempt to amend the Arbitration and Conciliation Act, 1996, major changes were
introduced though the Arbitration and Conciliation Amendment Act, 2015. Immediately after
this amendment, new problems were faced by the parties and need was felt to bring another
amendment in the existing law.

\noi
In this backdrop, in order to overcome the existing lacunae and to boost the confidence of
commercial entities to make India an international hub of arbitration, an expert committee
headed by the Supreme Court judge (Retd.), Justice B. N. Srikrishna, which was assigned the
charge to suggest improvement in the existing arbitration law. The committee submitted its
report in July 2017\footnote{Report of the High Level Committee to Review the Institutionalisation of Arbitration Mechanism in India (October 28, 2019 10:04 AM), \url{http://legalaffairs.gov.in/sites/default/files/Report-HLC.pdf}}  suggestive of numerous actions for revamping the arbitration law in
India. Its suggestion were mainly focused on facilitating the working of the institutional
arbitration in India and removing few ambiguities in the Arbitration Amendment Act 2015. It
is largely on the basis of Justice B. N. Srikrishna Committee report, the Central Government
brought the Arbitration and Conciliation (Amendment) Act Bill, 2018 which was regarded as
a noteworthy attempt by the Central Government to facilitate and streamline the working of  
Institutional Arbitration in India and also to make India a preferred seat of arbitration both for
domestic as well as international commercial arbitration. This Bill subsequently received the
assent of the President on 9$^{\rm th}$ August 2019 and became the part of the statue. The Central
Government exercising its powers provided under section 1(2) of the Arbitration Amendment
Act, 2019, appointed 30$^{\rm th}$ August 2019 for the enforcement of different sections under the
Arbitration Amendment Act 2019.

\heading{Key Features}

\noi
There is no doubt that the Amendment Act, 2019 has been brought with the intention to
refine and strengthen the existing Arbitration Law. In this regard, there are many new and
innovative features that have been added to this which was demanded by several quarters. A
few of the key changes are discussed below.

\noi
{\large \bfseries $\bullet$ Arbitration Council of India (ACI):}

\noi
ACI is a statutory body sought to be established by the Amendment Act, 2019 which is given
the mandate to grade arbitrary institutions and also to accredit Arbitrators by laying down the
guidelines and rules in this regard. ACI is also entrusted with the task to recognise
professional institutes providing accreditation of arbitrators.\footnote{Section 43 D (2) b} 
 In addition to this, ACI is entrusted with the task to conduct training, workshops and different courses in the field of
arbitration in association with law institutions/universities, law firms and arbitrary institutes.\footnote{Section 43 D (2) d}
It is also mandated to establish and maintain a depository of arbitral awards made both in
India and abroad.\footnote{Section 43 D (2) j}
 It will also work towards promotion and encouragement of arbitration and
other ADR mechanisms in India.\footnote{Section 43D (1)}
 Since there was no such statutory body to regulate the
conduct of arbitrary institutions and also for their accreditation, establishment of ACI is
indeed a step in right direction.

\noi
{\large \bfseries $\bullet$ Application of Arbitration Amendment Act 2015:}

\noi
To clarify on the application of the Arbitration Amendment Act 2015, a new provision by
way of section 87 is incorporated to clarify that the Arbitration Amendment Act 2015 will
apply to such arbitrary proceedings which were commenced after 23rd October 2015 or if
parties explicitly approve the same. This section specifies that the courts can apply the
provisions of the Arbitration Amendment Act 2015 only if the two situations as mentioned 
are attracted. However, this newly introduced section 87 did not take into account the latest
judgment of the Hon’ble Supreme Court of India on this very point i.e. BCCI v. Kochi
Cricket Pvt. Ltd\footnote{2018 (4) SCALE 502} which decided otherwise on the point which is discussed in the latter part
of this paper. This particular section has created more ambiguity then clarity on the point.

\noi
{\large \bfseries $\bullet$ High Court and Supreme Court recognized Arbitral Institutions:}

\noi
The 2019 Amendment Act has provided that the arbitration institutions duly recognized by
the High Court/Supreme Court can be approached by the parties directly for appointment of
arbitrator and in such case, parties are not required to file petition in the High Court/Supreme
Court under section 11 of the Arbitration Act 1996 as it was required earlier.\footnote{Section 11 \textit{(3A)}} It is mandated
that those Institution Arbitration houses will have the power to appoint the arbitrator in
international commercial arbitration if they are duly recognized by the Supreme court and for
domestic arbitration, if they are duly recognized by the respective High Court.\footnote{Id} This
provision is indeed a welcoming step as it will not only decrease the avoidable burden of the
courts but will also be expedient for the parties. This amendment, however, is yet to be
notified.

\noi
{\large \bfseries $\bullet$ Changes to the timelines provided under\\ Section 29A:}

\noi
The 2015 Amendment Act had brought in the timeline of twelve months for the completion
of arbitration proceedings and declaration of an award.\footnote{Section 29A(1)} It had also provided that the
Arbitrator’s mandate will come to an end automatically after twelve months are completed if
the parties don’t consent for the extension of arbitrator’s mandate by another six months.\footnote{Section 29A(4)}
Further, after completion of eighteen months, if the time limit is not extended by the court,
arbitrator’s mandate will abruptly end. However, through the Arbitration Amendment Act
2018, it is provided that the authorization of the arbitrator will be extended even after
eighteen months till the petition filed before the court in this regard is decided.\footnote{Id} This
provision is going to benefit all those arbitration proceedings which are going to be halted 
due to completion of eighteen months as now they can continue their proceedings during the
pendency of the decision of the court in this regard.

\noi
In addition to this, The 2019 Amendment Act further provides for calculating the time's line
of twelve months not from the date of appointment of the arbitrators (unlike the 2015
Amendment Act) but from the date of end of the pleadings.\footnote{Supra note 12} The Arbitration Amendment
Act 2019 provides for six months for completion of the pleadings and twelve months for the
delivery of the arbitral award in case of domestic Arbitrations. These timelines as introduced
by the Arbitration Amendment Act 2019 are however, not made applicable to the
international commercial arbitration in India.

\noi
{\large \bfseries $\bullet$ Confidentiality of the Arbitration Proceedings:}

\noi
Confidentiality is one of the unique features of arbitration. Because of the fact that the
arbitration proceedings are private and confidential, many parties resort to arbitration.
However, the Arbitration and Conciliation Act, 1996 did not have adequate provision on this.
The 2019 Amendment Act has introduced a new Section 42A, which will ensure the
confidentiality of the arbitration proceedings except for arbitral award.\footnote{Section 42A} This step will
strengthen the confidence of parties and encourage them to opt India as a seat of arbitration.
This provision has safeguarded the arbitrator for any of his/her act done in good faith.\footnote{Section 42B} This
was a much needed provision and has been rightly acknowledged and incorporated in the
law.

\noi
{\large \bfseries $\bullet$ Reduced Scope of Section 17 interim\\ injunction:}

\noi
As per the provisions of Arbitration Amendment Act, 2015 a party can move to the arbitrary
tribunal at any time during the pendency of the arbitrary proceeding or at any time after the
delivery of the arbitral award but before it is enforced.\footnote{Section 17(1)} Through the Arbitration Amendment
Act 2019, this power of arbitrary tribunal to entertain an application under section 17 of the
Act has been confined to the date of the delivery of the final arbitral award.\footnote{Id} As the mandate
of the arbitrator automatically ends after the pronouncement of final award, through the 
Arbitration Amendment Act 2019, it is clarified that the arbitrary tribunal will not have any
powers after the pronouncement of the award to grant any interim relief by making necessary
amendments in section 17(1) of the Act. In such situation, after the award is delivered by the
arbitrary tribunal, only the courts will have the powers to grant interim orders under Section 9
of the Arbitration Act.

\noi
{\large \bfseries $\bullet$ Qualifications and Experience of Arbitrator:}

\noi
The 2019 Amendment Act has introduced a new schedule namely “Eighth Schedule” to the
Arbitration Act, which provides exhaustive detail of qualifications required to become an
arbitrator.\footnote{Section 43J} This Eighth Schedule provides for 10 years’ experience as an advocate or an
Officer of Indian Legal Service or a CA or an engineer who are eligible to become an
arbitrator. It also provides the general standard applicable to the arbitrators. It however, failed
to provide for law professor with considerable years of experience who could also be made
eligible for becoming arbitrators. This would have increased a wide pool of professionals
with varied experiences to be eligible for becoming arbitrator.

\heading{Problems with the Proposed Amendments}

\noi
{\large \bfseries $\bullet$ Arbitration Council of India}

\noi
With regard to the Arbitration Council of India, Justice B.N. Srikrishna Committee had
provided for a statutory institution which will have its members nominated by the CJI, the
Central Government and also a well-regarded foreign professional. But, the Central
Government did not take into account this recommendation of the Justice B.N. Srikrishna
Committee and provided that the Arbitration Council of India will be a body only of
members nominated by the Central Government only. It will have the Secretary to the
Central Government’s two departments as ex officio members.\footnote{Section 43C(1)} Since the role of the ACI is
very wide and its powers includes accreditation of the arbitrary institutions, it would have
been better, had the involvement of the government official kept limited in the composition
of ACI. This is all the more important when in a large volume of arbitration cases,
government is a party to such matters.

\noi
{\large \bfseries $\bullet$ Applicability of the amendments}

\noi
The question of applicability of the Arbitration Amendment Act, 2015 was subject to lots of
deliberation amongst stakeholders and also several inconsistent views taken by various High
Courts on this. But when the Supreme Court of India was seized with the issue of
applicability of Arbitration Amendment Act, 2015 in the case of BCCI v. Kochi Cricket Pvt.
Ltd\footnote{Supra note 9} it held that the Arbitration Amendment Act 2015 is prospective in its nature. The
implication of this judgment of the Supreme Court was that the 2015 amendments were
applicable to arbitrary tribunal and court proceedings started subsequent after the Arbitration
Amendment Act 2015 came into force. It was also held in this case by the Supreme Court
that the Arbitration Amendment Act, 2015 will apply to pending proceedings that might have
been instituted before the Arbitration Amendment Act, 2015 came into existence but were
pending on the date of the amendments came into force i.e. 23$^{\rm rd}$ October, 2015. This case also
provided that section 36 of the Arbitration Act, 1996 after its amendment will apply to
pending applications for setting aside of arbitral awards under section 34 of the Arbitration
Act 1996. This was to remove the problem of automatic stay on enforcement of arbitral
award upon the filing of a setting aside application under section 34 of the Arbitration Act,
1996

\noi
However, through the Arbitration Amendment Act 2019 it is provided that the Arbitration
Amendment Act 2015 will apply only to ‘arbitration proceedings started on or after the
commencing of the Arbitration Amendment Act 2015 and to the court proceedings occurring
out of such arbitration proceedings.’\footnote{Section 87(b)} It would have been wise on the part of the legislature
that the law as stated in the above judgment of the Supreme Court is taken into account and
the Arbitration Amendment Act 2019 should be amended in a manner that the judgment of
the Supreme Court of India on this point is not negated and is retained.

\noi
{\large \bfseries $\bullet$ The issue with Confidentiality requirement}

\noi
Through the Arbitration Amendment Act 2019, confidentiality has been made an essential
feature of the Arbitration Act, 1996.\footnote{Section 42A} However, in other jurisdictions having provisions on
confidentiality in their arbitration laws mostly provide for many exceptions in this regard.
This aspect of the Arbitration Amendment Act 2019 should be re-looked as it should not bind 
the parties to arbitration to such broad confidentiality requirements where they can not
disclose anything. It should be subjected to party autonomy wherein parties should have the
autonomy to determine their limits of confidentiality. Party autonomy as explained by the
Redfern and Hunter in the following words:

\noi
“Party autonomy is the guiding principle in determining the procedure to be followed in an
international commercial arbitration. It is a principle that has been endorsed not only in
national laws, but by international arbitrary institutions and organisations. The legislative
history of the Model Law shows that the principle was adopted without opposition...”\footnote{Redfern and Hunter, Law and Practice of International Commercial Arbitration, 315 (4th ed., 2004)}
In the backdrop of the seminal principle of party autonomy, it is also for the legislature
important to understand that such provision of confidentiality should not be made a
mandatory provision.

\heading{Problem with the timeline under section 29A}

\noi
In the Arbitration Amendment Act 2019, newly inserted provision of timeline for completion
of pleadings within six months without any scope for an extension creates unnecessary
limitation on the parties and the arbitrator to frame the arbitration proceedings as per their
convenience. This is also against the seminal principle of party autonomy.\footnote{Id} Further, it is also
not clear from the provision that if the respondent who is also required to file a counter claim
has to file it within these six months. Therefore, it is not clear as what could be the intention
for providing six months’ time for filing both statement of claim and its defence. It is
therefore submitted that the pragmatic approach should have been in not dividing the time
limits into different parts for different facets of the arbitrary proceeding. It would be
pragmatic if simply an eighteen-month timeline for the completion of arbitrary proceeding is
provided.

\noi
{\large \bfseries $-$ Qualification of Arbitrators}
The Arbitration Amendment Act 2019 is not very clear as to the implication of the Eighth
Schedule through which the qualifications of the Arbitrators have been introduced. It
indicates that only such persons who satisfy those requirements as specified in the schedule
are qualified to act as arbitrators. But, when the Eighth schedule is read along with Section
43D of the Arbitration Act, 1996 it implies that the requirements are pertinent at the phase of accreditation only.\footnote{Section 43D(2)} This vagueness can make the arbitral award susceptible to challenge
under section 34 of the Act which may be delivered by the arbitrators who do not meet these
qualification as specified in the schedule. Therefore, this entire provision requires to be relooked and needs to be amended at the earliest to remove any complications to the parties in future.

\noi
{\large \bfseries $-$ Missed opportunity}

\noi
The Arbitration Amendment Act 2019 missed out to provide any provision for emergency
arbitration. It is all the more important that despite an unambiguous recommendation in this
regard by Justice B. N. Srikrishna Committee, the Central Government did not understand the
veracity of its requirement under the Indian Arbitration Law. Provision for emergency
arbitration is provided by almost all prominent jurisdictions abroad by introducing necessary
amendments in their respective arbitration law.\footnote{\textit{See e.g.,} Singapore International Arbitration Center (SIAC) included the emergency arbitration provision in July 2010 only. The International Chamber of Commerce (ICC) included emergency arbitration provisions in the 2012 Rules through Article 29 and Appendix V. In the same manner, London Court of International
Arbitration (LCIA) amended its Rules of 1988 in 2014 to provide provision for emergency arbitration through
Article 9.} It is therefore suggested, that if adequate
provision with regard to emergency arbitration is incorporated in the Arbitration Act, 1996 it
will raise its status and bring it at par with the International standards.

\heading{Concluding Remarks}

\noi
Unquestionably, the Arbitration Amendment Act 2015 followed by the Arbitration
Amendment Act 2019 aim at a more pragmatic and robust arbitration mechanism in India by
overcoming the lacunae in the existing arbitration law. It is indeed laudable step towards
achieving international standards of arbitration mechanism in India for domestic as well as
international commercial arbitration and to make India a hub of international arbitration. It is
also reflective of the willingness of the present government to rationalize the arbitration
mechanism in India and make it at par with its other counterparts. However, to streamline the
existing arbitration law and to minimize inconvenience to the future instigation which will be
subjected to arbitration, certain loopholes as pointed out in this paper must be looked into to
make the arbitration process in line with the international best practices.

\noi
In its existing form, the Arbitration Amendment Act 2019 creates more confusion then
ironing out the discrepancies which is detrimental to the image of Indian arbitration law which already had a bad past no so long ago. By doing the necessary amendments as
suggested in this paper, the government will further boost the confidence of the disputing
parties to choose India as a preferred seat for arbitration. Any change made in the Arbitration
Act 1996 is expected to build up the arbitration framework for the holistic development of the
arbitration ecosystem in India and if the necessary changes as suggested in this paper are
made, it will further strengthen the existing arbitration law in India.
\end{multicols}
\label{end2019-art5}
