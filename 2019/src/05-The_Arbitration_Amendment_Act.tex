\setcounter{figure}{0}
\setcounter{table}{0}
\setcounter{footnote}{0}

\articletitle{The Arbitration Amendment Act, 2019 and the Changing Arbitration Eco-System in India: Needs a Re-Look?}
\articleauthor{Dr. Rohit Moonka\footnote{Assistant Professor of Law, Campus Law Centre, Faculty of Law, University of Delhi} and Dr. Silky Mukherjee\footnote{Assistant Professor of Law, Campus Law Centre, Faculty of Law, University of Delhi}}
\lhead[\textit{\textsf{Dr. Prashna Samaddar and Victor Nayak}}]{}
\rhead[]{\textit{\textsf{The Arbitration Amendment Act...}}}

\begin{multicols}{2}

\heading{Introduction}

\noi
Arbitration has always been a preferred mode of dispute resolution especially in commercial
matters. For long, India has been striving to become a preferred seat of arbitration not only
for domestic arbitration but also for international commercial arbitration since the enactment
of Arbitration and Conciliation Act, 1996.\footnote{The Arbitration and Conciliation Act, 1996, Act No.26 of 1996} In this endeavour, there have been several ups and downs noticed by the observers which were either due to certain judicial pronouncements or fallacies in the drafting of the Arbitration and Conciliation Act, 1996 itself.\footnote{Sumeet Kachwaha, The Indian Arbitration Law: Towards a New Jurisprudence, 10 INT. A.L.R. 13, 17 (2007)}  Soon it was
realized both by the judiciary as well as by the legislature that they need to change the
approach, if India has to become a preferred seat of arbitration. In this context, after various
failed attempt to amend the Arbitration and Conciliation Act, 1996, major changes were
introduced though the Arbitration and Conciliation Amendment Act, 2015. Immediately after
this amendment, new problems were faced by the parties and need was felt to bring another
amendment in the existing law.

\noi
In this backdrop, in order to overcome the existing lacunae and to boost the confidence of
commercial entities to make India an international hub of arbitration, an expert committee
headed by the Supreme Court judge (Retd.), Justice B. N. Srikrishna, which was assigned the
charge to suggest improvement in the existing arbitration law. The committee submitted its
report in July 2017\footnote{Report of the High Level Committee to Review the Institutionalisation of Arbitration Mechanism in India (October 28, 2019 10:04 AM), \url{http://legalaffairs.gov.in/sites/default/files/Report-HLC.pdf}}  suggestive of numerous actions for revamping the arbitration law in
India. Its suggestion were mainly focused on facilitating the working of the institutional
arbitration in India and removing few ambiguities in the Arbitration Amendment Act 2015. It
is largely on the basis of Justice B. N. Srikrishna Committee report, the Central Government
brought the Arbitration and Conciliation (Amendment) Act Bill, 2018 which was regarded as
a noteworthy attempt by the Central Government to facilitate and streamline the working of  
Institutional Arbitration in India and also to make India a preferred seat of arbitration both for
domestic as well as international commercial arbitration. This Bill subsequently received the
assent of the President on 9$^{\rm th}$ August 2019 and became the part of the statue. The Central
Government exercising its powers provided under section 1(2) of the Arbitration Amendment
Act, 2019, appointed 30$^{\rm th}$ August 2019 for the enforcement of different sections under the
Arbitration Amendment Act 2019.

\heading{Key Features}

\noi
There is no doubt that the Amendment Act, 2019 has been brought with the intention to
refine and strengthen the existing Arbitration Law. In this regard, there are many new and
innovative features that have been added to this which was demanded by several quarters. A
few of the key changes are discussed below.

\noi
{\large \bfseries $\bullet$ Arbitration Council of India (ACI):}

\noi
ACI is a statutory body sought to be established by the Amendment Act, 2019 which is given
the mandate to grade arbitrary institutions and also to accredit Arbitrators by laying down the
guidelines and rules in this regard. ACI is also entrusted with the task to recognise
professional institutes providing accreditation of arbitrators.\footnote{Section 43 D (2) b} 
 In addition to this, ACI is entrusted with the task to conduct training, workshops and different courses in the field of
arbitration in association with law institutions/universities, law firms and arbitrary institutes.\footnote{Section 43 D (2) d}
It is also mandated to establish and maintain a depository of arbitral awards made both in
India and abroad.\footnote{Section 43 D (2) j}
 It will also work towards promotion and encouragement of arbitration and
other ADR mechanisms in India.\footnote{Section 43D (1)}
 Since there was no such statutory body to regulate the
conduct of arbitrary institutions and also for their accreditation, establishment of ACI is
indeed a step in right direction.

\noi
{\large \bfseries $\bullet$ Application of Arbitration Amendment Act 2015:}

\noi
To clarify on the application of the Arbitration Amendment Act 2015, a new provision by
way of section 87 is incorporated to clarify that the Arbitration Amendment Act 2015 will
apply to such arbitrary proceedings which were commenced after 23rd October 2015 or if
parties explicitly approve the same. This section specifies that the courts can apply the
provisions of the Arbitration Amendment Act 2015 only if the two situations as mentioned 
are attracted. However, this newly introduced section 87 did not take into account the latest
judgment of the Hon’ble Supreme Court of India on this very point i.e. BCCI v. Kochi
Cricket Pvt. Ltd\footnote{2018 (4) SCALE 502} which decided otherwise on the point which is discussed in the latter part
of this paper. This particular section has created more ambiguity then clarity on the point.

\noi
{\large \bfseries $\bullet$ High Court and Supreme Court recognized Arbitrary Institutions:}

\noi
The 2019 Amendment Act has provided that the arbitration institutions duly recognized by
the High Court/Supreme Court can be approached by the parties directly for appointment of
arbitrator and in such case, parties are not required to file petition in the High Court/Supreme
Court under section 11 of the Arbitration Act 1996 as it was required earlier.\footnote{Section 11 \textit{(3A)}} It is mandated
that those Institution Arbitration houses will have the power to appoint the arbitrator in
international commercial arbitration if they are duly recognized by the Supreme court and for
domestic arbitration, if they are duly recognized by the respective High Court.\footnote{Id} This
provision is indeed a welcoming step as it will not only decrease the avoidable burden of the
courts but will also be expedient for the parties. This amendment, however, is yet to be
notified.

\noi
{\large \bfseries $\bullet$ Changes to the timelines provided under Section 29A:}

\noi
The 2015 Amendment Act had brought in the timeline of twelve months for the completion
of arbitration proceedings and declaration of an award.\footnote{Section 29A(1)} It had also provided that the
Arbitrator’s mandate will come to an end automatically after twelve months are completed if
the parties don’t consent for the extension of arbitrator’s mandate by another six months.\footnote{Section 29A(4)}
Further, after completion of eighteen months, if the time limit is not extended by the court,
arbitrator’s mandate will abruptly end. However, through the Arbitration Amendment Act
2018, it is provided that the authorization of the arbitrator will be extended even after
eighteen months till the petition filed before the court in this regard is decided.\footnote{Id} This
provision is going to benefit all those arbitration proceedings which are going to be halted 
due to completion of eighteen months as now they can continue their proceedings during the
pendency of the decision of the court in this regard.


\end{multicols}
