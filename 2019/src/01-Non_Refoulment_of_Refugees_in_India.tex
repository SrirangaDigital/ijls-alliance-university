\setcounter{figure}{0}
\setcounter{table}{0}

\articletitle{Non-Refoulment of Refugees in India- A Human Rights Perspective}
\articleauthor{Abhilash Arun Sapre\footnote{Assistant Professor of Law, Kalinga University, Raipur }}
\lhead[\textit{\textsf{Abhilash Arun Sapre}}]{}
\rhead[]{\textit{\textsf{Non-Refoulment of Refugees in India...}}}

\begin{multicols}{2}

\heading{Introduction}

\vspace{-.2cm}

\noi
Delving into the words of Augustus Comte, \textit{'demography is destiny',} invokes an
understanding that the \textit{‘associated strands’} of political discourse (political framework of
actions \& inactions respectively), people and migration remain inherently inclusive in
perspective.\footnote{Bill Bonner, \textit{Destiny is Demography,} Business Insider(17$^{\rm th}$ Sept,2011,12.46 am)\\
\url{https://www.businessinsider.com/destiny-is-demography-2011-9?IR=T}}  This means that, with the very existence of population within a nation-state
comes the inevitable interplay of policy framework as a natural denouement, along with the
added vigour of cross-border migration standards. India has remained susceptible to varied
perspective of cross border movements within South Asia for the longest time. For instance,
the inter-relations between India-Pakistan, India-Afghanistan, India-Sri Lanka, IndiaBangladesh, among the others, are trapped in a quagmire of cross-border movements vis-a-vis
policy framework (legal stipulations). Therefore, it would be pertinent to infer that the Indian
foreign policy in particular, succumbs to the politics of cross-national ethnic issues pertaining
to the concerned nation-state. On the other hand, when the issue of granting refugee status
come to the fore, India’s position remains prone to objective interpretation. Objective in the
sense that, India has always scrutinised the same (granting refugee status) on an ad hoc basis.
The reasons for it germane from the lack of any substantial legal mandate in delineating the
issues of refugees’ or migrants’ or asylum seekers on one side and the absence of a
concrete/recognised refugee policy in consonance with international instruments on the other.
Refugees and migrants are legally distinct. On one hand, owing to the 1951 Convention
Relating to the Status of Refugees (‘Refugee Convention’) and the 1967 Protocol Relating to
the Status of Refugees (‘Protocol’), a refugee is a person \textit{“who flees across an international
border because of a well-founded fear of being persecuted in her country of origin on account
of her race, religion, nationality, membership of a particular social group, or political
opinion”.\footnote{Convention Related to Status of Refugees,1951, Art.1}}

\noi
On the other hand, Migrants are a much wider group of people who move away from their
usual residence to live somewhere else. Since it is an umbrella term, there is no legal
definition of a migrant; The phrases refers to high-wage labour travelling between developed
economies, people fleeing destitute countries, and those fleeing persecution. As a result,
while all refugees are migrants in the sense that they leave their usual home, not all migrants
are refugees. The extensive powers granted exclusively to the Centre to act with
unconstrained discretion in regard to foreigners enable India's ad hoc refugee system. The
Foreigners Act of 1864, passed by India's colonial government in the nineteenth century, was
the first law to restrict, arrest, and expel foreigners. The law was harsh since it was meant to
consolidate imperial dominance and retain social control. The colonial government
considered the 1864 legislation to be too lenient for the absolute powers it requested, thus the
Foreigners Act of 1940 was enacted to replace it. The 1940 wartime legislation was further
consolidated as the Foreigners Act, 1946 (‘Foreigners Act') following the war's end and the
ensuing large-scale displacement.

\heading{Delineating Refugees}

\vspace{-.2cm}

\noi
{\normalsize\bfseries {Refugees}}

\vspace{-.2cm}

\noi
At its Thirtieth Plenary Meeting on 12 February 1946, the United Nations General Assembly
adopted Resolution A/8(l)\footnote{United Nations Resolutions, I (Dusan J. Djanovich ed),8 (1946).} that recognized the problem of refugees and displaced persons \textit{of
all categories was one of immediate urgency.} Importantly, the resolution was specific to
delineate between \textit{genuine refugees} and \textit{displaced persons}, on the one hand, and \textit{war
criminals} and \textit{other criminals} on the other side who may claim to be refugees.
Recognizing that the problem of refugees and displaced persons was an international one, the
General Assembly recommended to the \textit{UN Economic and Social Council} (ECOSOC) that the
Council recognizes the principle that \textit{“no refugees or displaced persons who have finally and
definitely, in complete freedom, and after receiving full knowledge and facts, including
adequate information from the governments of their countries of origin, expressed those who
have valid objections to returning to their countries of origin will be forced to do so”.}\footnote{\textit{Id.}}
Through this recommendations, resolution A/8(l) thereby became the first official
international recognition that a genuine refugee could not be compelled to return to his or her
country of origin. 

\noi
An analysis of the determinants of refugee flows by Myron Weiner reveals those internal
conflicts rather than wars between states are the principal generators of population
flight.\footnote{Weiner, M. \textit{“Bad neighbours, Bad neighbourhoods: an inquiry in the causes of refugee flows”} 21
International Security, 36-39(1996).} These conflicts include wars of secession by territorially based ethnic groups or wars
of central or local governments against such groups\footnote{\textit{Id.} For instance, such groups against which there was wars of central or local government may include notably, Chechens, Kurds, Eritreans, Southern Sudanese, Sri Lankan Tamils, Serbs, Croats, Bosnian Muslims. } and political persecution by authoritarian regimes such as Iran, Iraq, China among the others.

\noi
However, problem arises the moment when the distinction in identifying case studies wherein,
an individual or mass population facing persecution not 'on the basis of one of the protected
grounds. In this situation, he may not be counted as a \textit{refugee}. Further, they would also not be
counted as refugees, when they face persecution on the basis of a protected ground, but who
are not outside their country of citizenship. Interestingly, there may arise situations wherein
they would be classified as Internally Displaced Persons (IDPs) involving individuals/groups
who have not crossed international borders, albeit having all the characteristics of a refugee
\textit{except that they have not crossed an international border.}\footnote{Lister, M, \textit{who are Refugees?} 32 LAW AND PHILOSOPHY, 645-667 (2013).}

\noi
The office of the UNHCR, therefore, should adapt itself and reinvigorate its legal mandates as
well as administrative procedures to understand the problem of refugees as a facet of the
human displacement within the broader framework of security concerns, involving the
security of refugees and humanitarian workers as well as states.\footnote{S. Ogata, \textit{“Statement, UNHCR, to the Third Committee of the General Assembly of the United Nations,”} November 12, (1999).}

\heading{India’s Position with Respect to Refugees}

\noi
India is a non-signatory to both, the 1951 Refugee Convention and the 1967 Protocol on
status of Refugees. However, it has acceded to other international instruments whose
provisions are relevant to the rights of refugees. For instance, in 1979 India acceded to the
\textit{“1966 International Covenant on Civil and Political Rights” (ICCPR)}\footnote{Article 13 ofthe ICCPR deals with the expulsion of a person lawfully present in the territory of the state. India hasreserved itsright under this Article to applyits municipal lawrelating to aliens} respectively. Further, in 1992 India acceded to the \textit{“1989 Conventions on the Rights of the Child”} (CRC), wherein, the incorporation of Article 22 dealing with refugee children and refugee family reunification\footnote{Article 10 dealing generally with family reunification and Article 38 dealing with children in situations of armed conflict are also relevant.} remain vital.

\heading{The Factor of \textit{“Well-Founded Fear of\\ Persecution”}}

\noi
The refugee framework in India is not well equipped to attend to situations of mass influx.
While the framework sometimes succeeds in dealing with \textit{individual claimants}, but it would
still succumb to failure when large numbers of people flee persecution. Further, the refugee
system affords substantially more protection to people who have crossed an international
border than to those who have not. While this distinction makes sense in some cases, in
others, particularly in situations of mass influx as a result of persecution by the state, it creates
a large gap in the international protections.\footnote{Arulanantham, T., \textit{ Restructured Safe Havens: A Proposal for Reform of the Refugee Protection System,} HUMAN RIGHTS QUARTERLY 22 ,1-56 (2000).}

\noi
In incorporating the Convention definition into a domestic statute, \textit{“nation-states decided to
recognize refugee status when one is outside the country of origin because of persecution or a
well-founded fear of persecution”.}\footnote{\textit{Id.}} The central question in the Convention definition and in
most countries' refugee law, however, relates to the fear of future persecution. Decision
makers thus focus on trying to determine what is likely to happen to the individual in the
future if she returns to the home country.

\noi
In India, the general laws that regulate outsiders apply to refugees. These rules, however, do
not ensure that refugees receive the treatment that they are entitled to.\footnote{RAJEEV DHAVAN, “REFUGEE LAW, POLICY AND PRACTICE IN INDIA” 81-83 (2004).} More Importantly,
both the Supreme Court and High Courts have on several occasions, provided a liberal
interpretation of rights of refugees in specific cases dealing with \textit{specific refugees.}\footnote{Dr. Malvika Karlekar v Union of India Crl. W.P. No.243 of 1988 (unreported, available on file with PILSARC), wherein, the Supreme Court stopped deportation of twenty-one Burmese refugees from the Andaman Islands whose applications of refugee status were pending and gave them the right to
have their refugee status determined.} It is Important to note that, the Madras High Court has on various occasions prevented \textit{forced repatriations} and upheld \textit{non-refoulement.}\footnote{Gurunathan v. Government of India, W.P. 6708 and 7196 of 1992 (unreported, available on file with PILSARC).}

\noi
Therefore, it could be said that the Indian Judiciary’s stand on refugees are far from uniform.
The reason being not the conflicting judicial decisions rather, \textit{“the lack of legal recognition of
rights and a separate framework for refugees versus Ordinary Foreigners”.}\footnote{\textit{Supra note 12.}}

\vspace{-.2cm}

\heading{Judicial Expansion of Rights to Refugees}

\vspace{-.2cm}

\noi
The Constitution of India is dogged of notable provisions\footnote{INDIAN CONST. art. 13, 14, 15, 20, 21, 22, 23, 24, 25, 27, 32 and 51.} wherein, the refugees are given protections. In general, the rights granted to refugees in India are the same as those granted to
all foreigners under the Indian Constitution, that is, under Art.14: the right to equality before
the law; under Art. 21: the right to have unrestricted access to the courts for the protection of
one's life and personal liberty, which may not be taken away except in procedure established
by law; Art 25: the right to freedom to practice and propagate one’s own religion. Further,
Art.51 states that– \textit{“The State [India] shall endeavour to foster respect for international law
and treaty obligations in the dealings of organized peoples with one another”}\footnote{INDIAN CONST.art.51}

\noi
However, it is important to ask whether the Indian Courts are empowered enough to intervene
\& enforce any international instruments or not, especially when the same has not been
incorporated into the Municipal law.

\noi
On one side, it could be argued that \textit{“the Indian courts do not have the authority to enforce
the provisions of the above international human rights instruments unless these provisions are
incorporated into municipal law by legislation”.}\footnote{State of Gujarat v. Vora Fiddali A.I.R. 1964, SC 1043. Here, the Court observed that, the well-established position that \textit{“the making of a treaty is an executive act, while the performance of its obligations, if they entail alteration of the existing domestic law, requires legislative action”}} The reasons for the same may be attributed
to the fact that the Indian Parliament is under no responsibility to establish laws to give effect
to a treaty unless there are compelling reasons to do so, and the judiciary is not competent to
enforce the Executive's compliance with treaty commitments in the absence of such
enactment.\footnote{CHANDRASEKHARA RAO, THE INDIAN CONSTITUTION AND INTERNATIONAL LAW, 130
(Taxman Publication) (1993).}  Therefore, It might be claimed that each nation-state has a responsibility to
observe its respective duties arising from international law, and that they cannot blame their
failure to do so on their legislative or executive apparatus.\footnote{Reparation for Injuries Suffered in the Service of the United Nations, Advisory Opinion, ICJ Reports 1949, p.174, at p. 180.}  Moreover, In the event of failure
of a state to bring its municipal law in line with its international obligations, \textit{“International Law does not render such conflicting municipal law null and void and which”}\footnote{“Standards Dealing with Specific Human Rights of Refugees adumbrated in Declaration on Territorial Asylum” 18 (1967).}  On the other hand, various court decisions in the absence of a concrete legislative structure \textit{“have tried to provide humane solutions to the problems of refugees, primarily with regard to the principles on non-refoulement, right to seek asylum, and voluntary repatriation”.}\footnote{Ahmad.N., \textit{The Constitution-Based Approach of Indian Judiciary to The Refugee Rights and Global Standards of the UN Convention,} 8 THE KING’S STUDENT LAW REVIEW, 30-55 (2018).} By implementing the same, the Indian courts gave their own interpretation by-passing a deliberation on the
principles of international refugee law.\footnote{T. Ananthachari, \textit{Refugees in India: Legal Framework, Law Enforcement and Security,} ISIL Year Book of International Humanitarian and Refugee Law, \url{wvw.worldii.org/int/journals IS1LYB1HRL7 2001.}} Meanwhile, it is observed that in certain
circumstances the courts \textit{can} take the treaty provisions into account. In the case of Vishaka,\footnote{Vishaka Vs. State of Rajasthan (1997) 6 SCC 241.} the Supreme Court of India stated that the \textit{“contents of international conventions and norms
consistent with the fundamental rights must be reflected in safeguarding gender equality and
right to work with human dignity that lacks in municipal law”.}

\noi
The Indian perspective is often deluged with the debate concerning as to whether the
constitutional protection afforded to the refugees \& asylum seekers protects them from
refoulement or not.\footnote{\textit{Supra} note at 35.}  \textit{“Majority is of the opinion that the right to non-refoulement has neither
been read into Indian constitutional jurisprudence, nor can be extrapolated”.}\footnote{The term \textit{‘foreigner’s issue’} was first used in the “Memorandum of Understanding signed between the Central Government and the All-Assam Students Union.” (Assam Accord,1985).\\ \url{http://www.assam.gov.in/documents/1631171/0/Annexure_10.pdf?version=1.0.} Subsequently, the same term found its way into the Statement of Object and Reasons of The Citizenship (Amendment) Act 1986.} In the case of \textit{Ktaer Abbas Habib Al Qutaifi v. Union of India},\footnote{Ktaer Abbas Habib Al Qutaifi v. Union of India, 1999 Cri LJ 919. The hon’ble court here observed that “Article 21 of the Constitution encompasses the principle of non-refoulement that is subject to \textit{“law and order and security of India”.}} the issue pertaining to whether Article 21 of the Constitution encompasses non-refoulement or not was raised. Here, the learned justice
did not rule out refoulement, but instead ordered the government to reconsider its deportation
order based on humanitarian concerns. Habib. J., categorically invalidated the principle of
non-refoulement by reasoning that:

\noi
\textit{“It expressly permits deportations on the basis of public order and national security, and it is
powerless against the Supreme Court’s confirmation of the Centre’s unrestricted right to
expel”.}\footnote{Hans Muller v. Supt., Presidency Jail, AIR 1955 SC 367.}

\heading{Understanding National Security and\\ Individual Liberty with Respect to Right of Non-Refoulement}

\noi
India though not bound by any of the major international agreements protecting the rights of
refugees, has largely followed international norms.\footnote{\textit{Supra} note 22} But, in cases involving matters of
\textit{'national security',} India preferred to deviate from its conventional mode of hospitality. For
instance, asylum seekers from Burma were jailed and approximately 5,000 Burmese refugees
were pushed back home from 1995 to 1997.\footnote{TAPAN K BOSE, PROTECTION OF REFUGEES IN SOUTH ASIA: THE NEED FOR A LEGAL
FRAMEWORK (2000).} Therefore, it would be inferred that for a nation-state with no refugee laws in position till date, its general practice as regards to refugee remains a case study to pursue.

\noi
The right of non-refoulement in the Indian context remains a debatable topic to ponder. A
facet of the perceptive observers tends to affiliate the contention that “the right of \textit{non-refoulement} is very much operational in India even without having signed any related international agreements.”\footnote{The two (2) main proponents of this theory include of Saxena and of Veerabhadran Vijayakumar respectively. Importantly, Saxena's argument is summarised in Tapan K Bose, Protection of Refugees in South Asia: The Need for a Legal Framework (2000), while Vijayakumar's argument can be found in Veerabhadran Vijayakumar, 'Judicial Responses to Refugee Protection in India', 12 International Journal of Refugee Law 238 (2000).}  Moreover, the prevention of refoulement generally, includes \textit{“both the rejection of refugees at the border as well as the deportation of refugees from inside India”}.\footnote{The principle of non-refoulement constitutes a fundamental aspect of the 1951 Refugee
Convention. Article 33 states that \textit{"[n]o Contracting State shall expel or return ('refouler') a refugee in any manner whatsoever to the frontiers of territories.”}} More importantly, \textit{non-refoulement} prevents nation-states from returning a refugee
to persecution in one’s country of origin. \textit{Non-refoulement} operates as the first and most basic
right of the refugee. Moreover, there exists several rights in the host country which are
generally called as \textit{“secondary rights”} in form of the right to education, the right to hold
property among the others. However, it could be argued that these secondary rights exist,
\textit{generally} relative to citizens in the host country.

\noi
Although, the very existence of international as well as national in form of current legal
institutions like UNHCR and the NHRC prevent the return of valid refugees to their country
of origin.\footnote{Ahmad. N., 23 \textit{Refugee Constitutionalism In India: Measuring Supremacy of Judicial Sovereignty Against Global Human Rights Standards,} RELIGION AND LAW REVIEW, 37-118(2014)} They contend that, \textit{non-refoulement} is granted as part of a broader constitutional
and statutory structure. To substantiate the same, two (2) main arguments emerge in support
of \textit{non-refoulement.}

\noi
$\Box$ First, the very enshrinement under Article 21 of the Constitution that promises \textit{“non-refoulement} as a fundamental, substantive right.”

\noi
$\Box$ Second, the very incorporation of the “international rule of \textit{non-refoulement} into India's
domestic laws” owing to Article 51 of the constitution.

\noi
Moreover, the difference between the legal and literal/colloquial definitions of 'refugee' is
causing impediments in evaluating refugee outcomes and its subsequent considerations.\footnote{Ranee K L Panabi, \textit{International Politics in the 1990s: Some Implications for Human Rights and the Refugee Crisis,} DICKINSON JOURNAL OF INTERNATIONAL LAW, 11 (1991)} The fact remains that, while many of those fleeing danger in the developed world would fit the legal definition, the same cannot be said for the least developing states and majority of the
developing nation-states. Say for instance, the areas of \textit{Latin American} and \textit{sub-Saharan
African} regions are particularly prone to mass-migrations situations due to famine, natural
disaster or economic collapse.\footnote{\textit{Id.}} On the other hand, although the Convention implies an
\textit{individualised determination of refugee status} (RSD) in a court of law, the same procedure for
status determination may borne futile when considerations of mass influx situations remain at
stake at the border of the receiving nation-state. Interestingly, albeit these, the existing
predicaments, the advent of the 21$^{\rm st}$ century bears testimony to mass influxes of people that
fit the legal definition of refugee.

\heading{Non-Refoulement Under Article 21}

\noi
Art 21 of the Constitution of India aims at placing a striking balance between the competing
interests of governmental power and individual right. Article 21's right to life \textit{"is the most
fundamental of all ... [but] is also the most difficult to define."}\footnote{Technically speaking, non-refoulement is a duty of the host country, not a right of the refugee. However, the \textit{"duty of non-refoulement,"} for all practical purposes, creates in the refugee a right to prevent return}

\noi
Delving through the initial vision of Article 21, the state could deprive someone of life or
personal liberty if the state has followed a \textit{valid procedure} established by
parliament.\footnote{The term \textit{‘foreigner’s issue’} was first used in the “Memorandum of Understanding signed between the Central Government and the All Assam Students Union.” (Assam Accord, 1985). Available at\\ \url{http://www.assam.gov.in/documents/1631171/0/Annexure_10.pdf?version=1.0.} Subsequently, the same term
found its way into the Statement of Object and Reasons of The Citizenship (Amendment) Act 1986.} Importantly, in case of deprivation of one's life or personal liberty, the test for
compliance with Article 21 encompass broadly of three (3) steps.

\noi
$\Box$ First, there had to be a law justifying interference with the person's life or personal liberty.

\noi
$\Box$ Second, the law had to be a \textit{valid law} in consideratio

\noi
$\Box$  Third, the procedure laid down by the law should have strictly adhered to. Here, the state
could justify serious infringements on life or personal liberty. For instance, the state could
create a law allowing indefinite detention of suspected murderers, so long as the three
procedural steps were followed in its creation and enforcement.

\noi
Further, as jurisprudence developed in the area of fundamental rights, the test for compliance
with Article 21 \textit{became entangled} with the standards for Article 14. The reason being, though
the two articles comprise 'fundamental rights' of the Constitution that apply to non-citizens.
The ensuing effect was that the Indian courts started implementing and adopting \textit{similar tests}
to gauge whether or not a particular law/(s) have complied to the pre-requisites of the said
articles. The Indian courts have always maintained a position wherein, Article 14 is being
viewed from a 'reasonableness' analysis, that is, if a law discriminated between two groups,
that discrimination would have to be \textit{'reasonable in character.'}

\noi
Most importantly, the scope of Articles 14 and 21 remain predominantly \textit{procedural} in
character. That is, while Article 14 protects people from disparate treatment by the courts and
police. 'Reasonableness' therefore referred to the validity of a given procedure for both
Articles 14 and 21. Here it would be pertinent to note that article 21 adopted substantive
connotations when the 'reasonableness' test for Articles 21 and 14 were further entangled with
the test for Article 19 compliance.

\noi
Interestingly, the co-relation of these rights (with the rights of Articles 14, 19 and 21 were
seen as overlapping) act differently in its operation pertaining to citizens \& non-citizens
respectively. For instance, Articles 14 and 21 apply to citizens and non-citizens alike, whereas, Article 19 applies only to citizens. Further, while the former provisions were
intended to confer procedural rights and the latter attempts to confer substantive rights.
The Indian courts since \textit{early stages} maintained that the \textit{'personal liberties'} described in
Article 21 were mutually exclusive from those described in Article 19. However, the courts
took a broader interpretation of Article 21, due to its all-encompassing interpretation as
inferred from various notable judgments. Although Article 21 was never seen as completely
subsuming the rights of Article 19, Amidst such developments, it would be garnered that
since the interpretation of Article 19 extends solely to the citizens, then in that circumstance,
any influence it had on Article 21 should probably be limited to citizens. Interestingly,
delving into the Indian context there is no express mandate till date that \textit{unwaveringly}
confirms nor denies the hypothesis that only citizens receive substantive rights under Article
21.

\noi
In a notable case of \textit{Railway Board V. Das,}40\footnote{Railway Board V. Das, 2 SCC 465 (2002). The case involved a Bangladeshi woman visiting India. Several employees of the Indian railways raped her at a station and subsequently the hon’ble court upheld her claim that the state-run Railway Board breached her fundamental rights.} involving a tourist's (non-citizen's) right to 'life and personal liberty', the Supreme Court of India categorically stated the following:

\vspace{-.1cm}

\noi
\textit{“The primacy of the interest of the nation and the security of the State will have to be read
into every article dealing with Fundamental Rights including Article 21 of the Indian
Constitution”.}

\vspace{-.1cm}

\noi
Here, the present case involved a situation wherein, the individual's right is being deprived,
and since the state has no interest, the individual's right to personal liberty should therefore
prevail. It would therefore be analysed that when the state's interest is weighed against the
interest in individual rights, the interests of the nation and security assume the highest
priorities. Therefore, it can be argued that the interests of refugees have rarely overcome the
primacy of state interest in refoulement in India till date.

\vspace{-.1cm}

\noi
Further, with respect to the concerns of sovereignty that implies broad control over
immigration and other matters of foreign affairs, the Indian courts have always opted for a
perspective wherein, the same must be read into the constitution without an express
constitutional provision. As stated by the hon’ble court in \textit{Railway Board vs Das}.

\vspace{-.1cm}
\noi
\textit{“The primacy of the nation's interest and security must be 'read into' other parts of the
Constitution. Other nations, including the US, have based their broad immigration powers on
the sovereignty of the nation. Those countries argue, like India, that the very essence of statehood involves federal control over immigration, which must be implied throughout one's
constitution”.}\footnote{\textit{Id.}}

\noi
More importance must be given where matters pertaining to citizenship \& foreign affairs
come to the fore, the Constitution grants broad powers to parliament to develop laws and
regulate thereby. In particular, art. 11 of the Constitution read as follows: \textit{“Nothing in the
foregoing provisions of this Part shall derogate from the power of Parliament to make any
provision with respect to the acquisition and termination of citizenship and all other matters
relating to citizenship”}\footnote{INDIAN CONST. Art.11.} Further, the Indian Parliament also retains full control over India's
international obligations, with the sole authority to have a discretion to \textit{create} and \textit{maintain}
international treaties on behalf of the nation.\footnote{INDIAN CONST, Art. 253}

\noi
However, on the other hand, the arguments put forth by \textit{Saxena} and \textit{Vijayakumar} tend to
imply a perspective that rely heavily upon a 1996 court case in the form of NHRC \textit{vs State of
Arunachal Pradesh.}\footnote{(1996) 1 SCC 742} They argue that through the judgment the hon’ble court establishes a
more assertive right of non-refoulement for refugees.

\noi
Albeit aforesaid perspectives, the insights that would be gleaned remains that the principle of
\textit{non-refoulement} retains a rather arcane status under the Indian legal paraphernalia. The
present case is dogged of a dispute involving between Chakma refugees residing in
Arunachal Pradesh and a group of hostile locals, the All-Arunachal Pradesh Students Union
(AAPSU). Delving into the historical context, the Chakma people were subjected mass
displacement in 1964 from the erstwhile East Pakistan (now Bangladesh) and subsequently
migrating from Assam to the current state of Arunachal Pradesh. However, the problem
ensued when majority of them applied for citizenship and got rejected by the orders of the
local officials from reaching the federal government.

\noi
However, as the Chakma population proliferated, the AAPSU issued 'quit orders', demanding
that the Chakma leave or suffer severe harm. Meanwhile, the Arunachal Pradesh government
formulated plans to move the Chakma’s to another state. On the other hand, the Ministry of
Home Affairs (MHA) was furthering its attempt in according \textit{“blanket citizenship”} and
ordered the state government to provide security. Finally, the NHRC filed a writ petition in
the court demanding the state government to stop the Chakma’s' forced migration from the
state.

\noi
Importantly, the Supreme Court upheld that notices that were given to the Chakma’s to quit
the state as amounting to a violation of \textit{“Article 21 under the Indian Constitution, and
categorically observed that no person can be deprived of his/her right to life and liberty
except in accordance to the procedure established by law”.}\footnote{Importantly, the hon’ble court in its interim order on November 2, 1995 directed the state government to ensure that the “Chakmas situated in its territory are not ousted by any coercive action not in accordance with the law. The court further directed the state government to ensure that the life and personal liberty of each and every Chakma residing within the state should be protected.”} Emphasising the role of the state government herewith, the hon’ble court was decisive in observing that it remains the
imperative of the state government to protect the Chakma’s from such threats accruing to
their respective lives and liberty. The hon’ble Court was upright in holding the observation
that the Union government by not forwarding the Chakma’s citizenship applications to the
concerned department is flagrantly impinging upon the rights of the Chakma’s, especially of
their constitutional and statutory right in form of citizenship.

\noi
The temporal impact of the said judgment bore nuanced interventions that had telling effect
with respect to the Chakmas in particular and refugees/asylum seekers in general. For the
reason being, aftermath the judgment \textit{periodic} interventions through the NGOs, the NHRC
and the Supreme Court of India, enabled around 65,000 Chakmas (approx.) residing in
Arunachal to the citizenship status by the Government of India.\footnote{Mahanirban Calcutta Research Group, \textit{The State of being Stateless: A Case Study on Chakmas in Arunachal Pradesh} (2009). Retrieved from \url{http://www.mcrg.ac.in/Statelessness/Statelessness_Concept.asp.}}

\noi
Pervading through the interpretation forwarded by the hon’ble court, Saxena and
Vijayakumar herewith opines the case to be a  \textit{"landmark decision by the Supreme Court with
regard to refugee protection"} that in turn be read into the court's decision a right of nonrefoulement under Article 21 of the Constitution.

\heading{Non-Refoulement Under Article 51}

\noi
Article 51 enshrined in the Constitution of India talks about the prospect of the
\textit{imperativeness} of national and international law respectively by mandating the Union
government to \textit{'maintain respect'} for international law. It reads as follows:

\noi
\textit{“The State shall endeavour to ... foster respect for international and treaty obligations in
the dealings of organised people with one another.”}

\noi
The Indian Courts have interpreted Article 51 to demand adherence to international law when
there is no clear conflict with domestic law. While recognising the supremacy of domestic
over international law, the Constitution's drafters thus realised the importance of fulfilling
international obligation. On the other hand, when India has a clear international obligation
but an unclear domestic obligation in a particular area, it should follow international law.
Article 51 refers to both 'treaty obligations' and 'international law', some have interpreted the
latter term as referring to \textit{'customary international law'.}\footnote{Statute of the International Court of Justice, Article 38(1) (b).}

\heading{Non-Refoulement as International Customary Law}

\noi
\textit{Non-refoulement,} as a principle invokes a rather feeble version of customary international
law, for it to qualify as customary international law, the same must be widely incorporated to
be morphed into \textit{"an international custom, as evidence of a general practice accepted as
law”.}\footnote{\textit{Id.}} However, the normative practice prevalent in most of the least developing nationstates including India remains that though they instil a framework protecting the refugees, the
same remains subject to discretionary mechanisms outmanoeuvring the protectionary
standard at any point.\footnote{Kannan. A., \& Supratim Guha. S., 3 \textit{Humanising the Indian Refugee Policy: A case for the Refugees’ Right to Work,} NLUJ LAW REVIEW 150 (2015)} Against this background, it would be pertinent to insinuate an observation that \textit{"[i]insofar as there is legal consensus on an expanded conceptualisation of
refugee status based on custom, it sure is... 'at a relatively low level of commitment.”}\footnote{\textit{Id.}}

\heading{Inapplicability of Customary International Law to Nonrefoulement}

\noi
There remain concerns that even if customary international law become acceptable and being
incorporated into the Indian domestic law, the same would remain a futile effort when
interpreted in the refugee context. There remain three (3) notable problems for the same.

\begin{enumerate}[label=$\bullet$]
\item First, the existing statutes pertaining to immigration aspect 'already occupy the field'
of refugee law.

\item Second, the existing statutes offers an interpretation that sits directly in contrast with
principle of non-refoulement.

\item Third, India’s insistence to be not be legally governed by any the Refugee
Convention.
\end{enumerate}

\noi
Here, it would be inferred that the domestic legislation in India \textit{'occupies the field'} of
immigration and refugee law completely, thus leaving out any hope of incorporating new
rules into the domestic sphere. The incorporation of customary international law is not
permitted when parliament 'occupies the field' of a given area.\footnote{Vishaka vs State of Rajasthan, AIR SC 3011 (1997).} Parliament is said to 'occupy the field' when it legislates in such a broad and comprehensive manner.

\noi
Importantly, India has effectively passed several laws governing immigration, notably, the
\textit{Passport (Entry into India) Act,1920, the Foreigners Act, 1946, the Foreigners Order, 1948
and the Citizenship Act,1957.}\footnote{P. CHANDRASEKHARA RAO, THE INDIAN CONSTITUTION AND INTERNATIONAL LAW, 130 (Taxman publication) (1993).} Further, the Foreigners Act and the Foreigners Order
respectively empowers the Union government to restrict movement of aliens inside the
territory of India, subsequently mandating the likes of \textit{medical examinations,} limiting
employment opportunities thereby, and limiting opportunity to associate, among others.
Therefore, the broad ambit of India's immigration stipulations indicates that the legislation
\textit{'occupies the field'}.

\noi
However, it remains to be analysed that the Indian courts only incorporate international law
through Article 51 only when the ambiguity in the domestic law \& policy framework
ensues.\footnote{Chadrahasan. N., 16 \textit{Access to Justice and Aliens: Some Insight into Refugee Groups in India,} Windsor Y B Access to Justice 135, 139 (1998).} Therefore, it would be safe to conclude that the Indian courts powers remain \textit{circumscribed} rather, inhibited to enforce a customary international norm that clearly
impinges the policy framework. Finally, it would be argued that Article 51 is enshrined as a
'Directive Policy' and likewise cannot be enforced in the court of law. This remains a
pertinent reason as to why the Indian courts remain hesitant to interpret the wordings
enshrined under Article 51 of the Constitution of India.

\heading{Conclusion}

\noi
The Indian history is usually replete with instances where human rights jurisprudence has
always occupied a place of \textit{primacy} amidst other considerations at stake.\footnote{RATIN BANDYOPADHYAY, HUMAN RIGHTS OF THE NON-CITIZEN: LAW AND REALITY, 217 (Deep \& Deep Publications Pvt. Ltd., New Delhi).} Looking closely, we cannot underestimate the affinity between the refugee problem and broader issues of human rights. The ensuing realities of these flagrant violations of human rights may tend to
proliferate instances of \textit{mass exodus \& non-voluntary repatriation} among the others that has
grappled the imaginations of most of the nation-states of the 21$^{\rm st}$ century.\footnote{S.N. Bhargava, “The Relationship Between National Human Rights Institutions and the Judiciary in Protecting Refugees”, Report on Judicial Symposium on Refugee Protection, 92 (13-14 Nov.1999, New Delhi).} The introductory
part primarily concerns itself with the position of the refugee within the Indian periphery and
also identifies the problems, research questions and the methodology to tackle those issues.
So, differentiating the intrinsically associated features of the said category of persons through
the help of varied international instruments and though the perspective of various political
theorists. Therefore, it would be argued that delineation of \textit{refugees} from the \textit{other categories
of foreigners} remains an important step in ameliorating the present situation of mass influxes
in India.

\noi
Further, It with the scope of constitutional provisions (right to equality before law, protection
of life and liberty and the right to fair trial respectively. In this context, several judicial
precedents, involving the decisions of Hon’ble High Courts and Hon’ble Supreme Court
respectively. Here, it would be important to state that moreover, there is no specific provision
in the Indian Constitution that requires the state to enforce or implement treaties and
agreements. A joint reading of all the provisions as well as an analysis of the case law on the
subject shows international treaties, covenants, conventions and agreements Only if they are
specifically incorporated in the law of the country can they become part of Indian domestic
law. Typical understanding persists that the Indian Courts \textit{generally} apply norms of Only
when there is a clear agreement between international law and domestic law should
international law be applied, and the principles of international law should not be in conflict
with domestic legislation.

\noi
Perceptive observers state that there is a blessing in disguise for not having in place a legal
framework that may be prone to aspects involving \textit{political character.} For they believe it
would impose certain liabilities and obligations which may have political dimensions. But the
same may suffer from misconception, as absence of laws pertaining to the refugees may not
only affect the refugees, rather the broader framework of mass influx situations involving
illegal migrants \& asylum seekers respectively. Gradually, it diminishes the accountability
factor of the Union government in terms of \textit{affording} refugee protection and \textit{observing} human rights. Therefore, by referring to the United Nations Charter and the Universal Declaration of
Human Rights in its preamble, as well as efforts such as ensuring that refugees have the
broadest possible exercise of these fundamental rights and freedoms, the document must be
evaluated and given a rights-based approach to the issue.

\heading{Recommendations}

\noi
Therefore, the following points must be considered in any future administrative, executive
and legislative exercise by the national governments in this part of the globe, especially in
case of India are as follows-

\begin{enumerate}[label=$\bullet$]
\item In the case of countries hosting large refugee populations, states should also provide
bilateral assistance both financial and technical support, depending on the host
country’s needs to enable the host state to provide support to refugees and asylumseekers, including ensuring access to adequate shelter, food, health care and education.
Further, the extent of such bilateral assistance should also be published annually.
However, such aids \& support (financial) should not be considered as a substitute for,
or come at the expense of, programmes to accept people in need of protection.

\item Proper stratification of refugees on terms of economic, climate, humanitarian and
political factors must be carved out while re-defining the \textit{‘refugee’.}

\item The principles of \textit{non-refoulement} must be enforced effectively into the administrative
as well as legislative practice.

\item In India's legal system, the principle of non-refoulement must be made a nonnegotiable human right for refugees.

\item Importantly, carving out necessary provisions to individuals who fail to refugee status,
but whose return would be in breach of international human rights obligations. The
instances of the same are replete in the Indian scenario. Therefore, the Union
government should embark upon facilitating \textit{appropriate status} in consonance with
their fundamental human rights.
\end{enumerate}

\end{multicols}
