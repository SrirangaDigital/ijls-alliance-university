\setcounter{figure}{0}
\setcounter{table}{0}
\setcounter{footnote}{0}

\articletitle{Right to Education in India - A Dream or a Reality?}\label{2019-art2}
\articleauthor{Francis Assisi Almeida\footnote{Ph.D. Research Scholar, Alliance School of Law, Alliance University, Bengaluru}}
\lhead[\textit{\textsf{Francis Assisi Almeida}}]{}
\rhead[]{\textit{\textsf{Right to Education in India...}}}

\begin{multicols}{2}

\heading{Introduction}

\vspace{-.1cm}

\noi
The Annual Status of The Education Report of 2018,\footnote{Annual Status of Education Report (Rural) 2018, \textit{ASER Centre,} 41, New Delhi, (2019)\\ \url{http://img.asercentre.org/docs/ASER}\%202018/Release\%20Material/aserreport2018.pdf} having conducted a survey on 546,527 children aged between 3 to 16, who were chosen from 354,944 households of 596 districts in
rural India, presents a positive picture of the situation. The statistics of the report unravel that
a proportion of children between the age group of 3 to 16, not enrolled in the school, had
fallen from 3\% to 2.8\% in 2018. The survey also showed that girls between the age group of
11 to 14 who were absent from the school has fallen to 4.1\% from 10.3\% in the year 2006.
However, an upward trajectory reflected in the status report of 2018 purports to be a positive
growth but practically failed to reach the expected growth both in enrolment ratio and quality
education to all. Universal quality education is the primordial importance of the state
machinery. Inadequate ratio between teacher and pupil, low budget allocation by the
Government towards strengthening the education, poor implementation of the Education
Policies, high taboo system towards female education in the remote areas, lack of skill and
job-oriented education in the schools, illiteracy and lack of motivation by the parents towards
education etc. are some of the reasons for slow progress of right to education in India.
Gradual downward move in the public expenditure by converting education to be a private
sector could also be one of the reasons for its downfall. The governments, ever since the
introduction of Liberalization, Globalization and Privatization (LPG) in 1991, have failed to
accomplish the goal of right to education with suitable provisions of free education under its
fiscal policies. In this research article, while elucidating the system of education developed
through the centuries beginning from ancient times till date, analysis will be done on how far
the right to education to all has fulfilled its dream.

\vspace{-.1cm}

\heading{Historical Development of Education System in India}

\vspace{-.1cm}

\noi
Ever since its inception, the Indian education system is centred around culture and traditions
of man and not merely on a technical education, so called a ‘Liberal or Modern Education’.
Since from the Early Vedic period, there was a tradition of learning and generally the aim of 
ancient education system was the formation of character, personality, preservation of ancient
culture and training in the spheres of social and religious duties.\footnote{Altekar, A. S. Ideals, Merits and defects of Ancient Indian Educational System, (15), \textit{Annals of the Bhandarkar Oriental Research Institute,} 137,138 (1933), JSTOR, \url{www.jstor.org/stable/41694847.}} In ancient India, the training
centres were normally ‘Gurukuls’, where all were considered to be equal and Guru (teacher)
and Shishya (students) lived in the same house or nearby places. The subjects of study
centred on techniques of worship and sacrifice and teachers were mostly Brahmins who
succeeded from the priestly class.\footnote{Choudhary, Sujit Kumar., Higher Education in India: A Socio-Historical Journey from Ancient Period to 2006-07, (8) \textit{Journal of Educational Enquiry,} 50, 52 (2008)} It is noteworthy that education, in ancient India, was considered not merely a mechanical process of imparting information from outside, rather it was considered a biological process, a process of growth from within and considered
something which depends on caryã - practice or realization, and not on mere theory or
intellectual apprehension of truth.\footnote{Mookerji, Radha Kumud. Glimpses of Education in Ancient India, 25, \textit{Annals of the Bhandarkar Oriental Research Institute} 63, 66 (1944) \url{http://www.jstor.org/stable/41688549.}} The ultimate end of the highest education is thus
expressed in Manu's Book of Laws; “To learn and to understand the Vedas, to practise pious
mortifications, to acquire divine knowledge of the Law and of Philosophy, to treat with
veneration his natural and his spiritual Father (The Priest) - these are the chief duties by
means of which endless felicity is attained”.\footnote{Laurie, S.S. The History of Early Education. III. The Aryan or Indo-European Races, 1, \textit{The School Review,} 668, 676 (1893) \url{http://www.jstor.org/stable/1074070,}} The Laws which were collected and written under the name of Manu were of great antiquity, but their formulation does not date back
prior to 600 B.C.\footnote{\textit{Id,} at 676.} The oldest and famous university among existing universities in India was
the Banaras Hindu University\footnote{GUPTA, AMITA, ‘EARLY CHILDHOOD EDUCATION, POSTCOLONIAL THEORY, AND TEACHING PRACTICES IN INDIA’ BALANCING VYGOTSKY AND THE VEDA’, 41 (Palgrave Macmillan, New York, and Hound mills, Basingstoke, Hampshire, England, 2006)} and Taxila, the Buddhist learning centre was prominent for the
Medicine, Law and Military Science which attracted scholars from distant parts of India.\footnote{S. R. DONGERKERY, UNIVERSITY EDUCATION IN INDIA, 1-2 (Manaktalas, Mumbai. 1997)}
The Sarnath Monastery under Ashok’s reign became famous as a centre of learning
captivating large number of Buddhist monks.\footnote{Choudhary, \textit{supra note} 4, at 53}

\vspace{-.2cm}

\noi
During the medieval period, Sanskrit language was prominent, and the literature was mostly
in the form of metaphor, imagery, adjectives and adverbs. Since it was difficult for simple
people to understand, it widened the gap between the landlords and the peasants. The
literature became the asset of elitist class which led to have an authoritarian trend in intellectual life.\footnote{SHARMA, R.S., INDIA’S ANCIENT PAST, 293-294 (Oxford University Press, 2005)} During the Mughal period education was considered to be an important
asset and under Akbar’s rule the mode of education (madrasas) was brought under state
purview. A comprehensive education, stressing on branches of sciences along with religious
learning, was prevalent. During the reign of Humayun, some schools were constructed.
References also can be found for the existence of medical sciences, even though not like
those of Aleppo, Egypt or Iran, but some students were sent to other countries for further
studies.\footnote{Rezavi, Syed Ali Nadeem, The organization of Education in Mughal India, 69, \textit{Proceedings of the India History Congress,} 389-397 (2007), \url{https://www.jstor.org/stable/i40173377}}

\noi
British period saw a rapid progress in western science and literature through the medium of
English. Various Christian missionaries along with religious teachings made remarkable
contributions in the spheres of education for all, women education, and adult literacy etc. The
British education reformation unified all the states and regional kingdoms removing gender
and caste bias despite virtually banishing the traditional gurukul system and other religious
holistic ancient schools that were prevalent in the country.\footnote{Krishnamoorthy A, Srimathi H. Education of India In Pre-Independent Yore , 9, \textit{International Journal of Scientific \& Technology Research}, 2251 (2020)} After the conquest of Bengal in 1757, the history unravels, the education became a profitable venture and schools came up
like mushrooms. Advertisements were given on the school curriculum, facilities and fee
structure. Under these schemes, prominence was given for girls’ education including new
subjects like dancing skills, embroidery, stitching, etc., and boys were offered with special
subjects like accountancy, mathematics along with traditional subjects.\footnote{Manzar, Nishat, European Education in Indian Environment: Early History of Western Educational Institutions in India (17$^{\rm th}$ and 18$^{\rm th}$ Century), 75, \textit{Proceedings of Indian History Congress,} 1389 (2014), \url{www.jstor.org/stable/44158542}}

\noi
The Charter Act of 1813 rejuvenated the education system by providing financial assistance
to revive and improve the literature among the intellectuals of native Indians and employed
various methods to promote sciences among the inhabitants of British territories.\footnote{H. Sharp, Ed., Selections from Educational Records, Part I (1781- 1839), 22 (Bureau of Education, Superintendent Government Printing, Calcutta, 1920. (Reprint) Delhi: National Archives of India, (1965)} Despite methodological drawbacks in implementation of given objects, the Charter Act of 1813 was considered to be a remarkable legislature and a great initiative of British rulers in the field of education.\footnote{S. NURURLLAH \& J. P. NAIK, HISTORY OF EDUCATION IN INDIA DURING THE BRITISH
PERIOD, 68 (The MacMillan Company, New York, 1943)} Subsequent developments in education reforms came from Lord Macaulay (Minutes of 1835), Wood‘s despatch 1854 famously known as Magna Carta of English
Education in India, The Indian Education Commission of 1882, The Shimla Conference in
September 1901, The Government Resolution on Educational Policy of 1913, and finally The
Calcutta University commission of 1917-19 are some of the important developments in
education under British India. The Montague - Chelmsford Reforms of 1919 has transferred
the education department under provincial control.

\vspace{-.1cm}

\heading{Constitutional Provisions on Education}

\vspace{-.1cm}

\noi
The issue of education was primordial in the minds of the framers of the Constitution and it
was even discussed in the Constituent Assembly Debates during the deliberations on Part III
and IV of the Constitution of India. Right to education did not find a place under the
fundamental rights (Part III) of the Constitution despite strong discussions on other related
issues of education like language, age of the child, minority rights etc. It was placed under
article 45 (Directive Principles of State Policy) of the Indian Constitution which provided,
"the State shall endeavour to provide, within a period of ten years from the commencement of
the Constitution, for free and compulsory education for all children until they complete the
age of 14 years”.\footnote{INDIA CONST. art. 45} Even though Article 29 of the Constitution provides, “no citizen shall be
denied admission into any educational institutions maintained by the State or receiving aid
out of State funds on grounds only of religion, race, caste or language”,\footnote{INDIA CONST. art. 29} but it failed to place the right to education under the fundamental rights of the Constitution and provide guarantee
to the foundational or elementary education to all irrespective of their social and economic
background. Finally, the amendment to the Constitution was made to amalgamate the
educational right as a fundamental right at the instance of Supreme Court’s decisions. The
most significant amendment to the constitution for the inclusion of education as a
fundamental right was 86th amendment in the year 2002. The said amendment included
Article 21A within Part III of the Constitution ‘considering right to Education as a
fundamental right and children aging from 6 to 14 are provided with free and compulsory
education’.\footnote{INDIA CONST. art. 21A}

\vspace{-.1cm}

\heading{National Education Polices}

\vspace{-.1cm}

\noi
The extensive debates on free and compulsory education up to the age of 14 years in the
Constituent Assembly ultimately took shelter under the Directive Principles of State Policy.
The goal set for the country’s educational policy was to work out a system of universal
elementary education by 1960. Necessary changes were also affected in the system of
secondary and higher education in keeping with the needs of the country.\footnote{SAIKIA, S. HISTORY OF EDUCATION IN INDIA (Mani Manik Prakash Publishers, 1998)} Much deliberation
on the topic of education and reporting has also taken place. The country observed several
commissions like the Radha Krishnan Commission on university education (1948-49), the
Kher Committee on primary education (1948), the Mudaliar Commission on secondary
education (1953), and the last and the most comprehensive effort on education came in the
form of the Kothari Commission (1964-66). These commissions have considered almost
every aspect of the education.\footnote{Kamat, A. R. Educational Policy in India: Critical Issues, 29, \textit{Sociological Bulletin}, 189, (1980)}

\vspace{-.15cm}

\noi
Radhakrishnan Commission of 1948 saw the academic issue in a larger perspective as
universities come under its preview.\footnote{Government of India, Report of University Education Commission. (1950)} Subsequently, the Kothari Commission of 1968, while
reiterating the constitutional injunctions about free and compulsory education up to the age of
14, laid emphasis on advanced outlays for education and advised the curtailment of higher
education. The Gajendragadkar Commission that followed on the path of the Kothari
Commission dwelt on the governance of universities that gave the State greater control on
higher education. Such authoritarian structure took a great importance during the emergency
when education was transferred from the State to the Concurrent list.\footnote{ Raina, Badri. Education Old and New: A Perspective, 17, \textit{Social Scientist}, 7-8 (1989)}

\vspace{-.15cm}

\noi
In response to the Kothari Commission, the Government of India formulated the ‘National
Policy on Education 1968’ (NPE).\footnote{National Policy on Education, 1968, Government of India (1968)} Compulsory education for the children between the age
group of 6-14 years, adequate emoluments, opportunities of training and freedom to improve
the teaching capacities of teachers; three language formula; equal opportunities of education
irrespective of physical, social and economic background; prominence to tribal and backward
areas; inclusion of local community to build national integration; importance to science,
mathematics, agriculture, and technical education to improve the economy; assessment to
identify the talents of students; availability of books at the reasonable rate; comprehensive examination pattern; extension of secondary education to the remote areas and importance to
technical and vocational training at the secondary education level; considerable importance to
university education; establishment of part time courses in large scale; special emphasis on
adult education; incorporation of sports and games within the education curriculum;
protection of the minority educational interests; uniform educational system throughout the
country on 10+2+3 pattern; and review of policy on every 5 years interval are some of the
important provisions of NPE 1968.

\noi
The NPE, 1986\footnote{National Policy on Education 1986, Government of India (1986)} was the successive effort to bring changes in the education system. Under this policy, the role of education was seen to be manifold and considered to be essential for all with the aim of all- round development including material and spiritual. Further it brings
about changes in the thought patterns and contributes to the national solidarity grounded on
socialistic, secular, and democratic values which are enshrined in our constitution. Education
was considered to be providing manpower, stimulus to research and development in economy
leading to national self-reliance. A unique investment both in the present and future was the
key principle of National Policy on Education.\footnote{Sen, Rahul \& Bhattacharya, D. K. Education in India, 21, \textit{Indian Anthropologist} 68 (1991)} It aims to foster equality and motivation to
the young minds to move towards international cooperation, peaceful co-existence and
fraternity despite their socio- cultural diversities.\footnote{GHOSH, S. C. THE HISTORY OF EDUCATION IN MODERN INDIA, 184 (Orient Longman Limited, New Delhi, 2000)} To rectify the drawbacks of NPE 1986, a
commission was appointed which came up with several important suggestions and they were
incorporated in the NPE, 1992.

\vspace{-.1cm}

\heading{The Concept of Right to Education in India}

\vspace{-.1cm}

\noi
Right to education, of course, does not mean the right to same kind or same degree of
education for all individuals but it means the minimum standard or basis of quality education
to all, be it an elementary, basic or foundational education. Article 26 of the Universal
Declaration on Human Rights mentions, “Everyone has the right to education. Education
shall be free, at least in the elementary and fundamental stages. Elementary education shall be
compulsory. Technical and professional education shall be made generally available and
higher education shall be equally accessible to all on the basis of merit”.\footnote{Article 26 Clause (1) of Universal Declaration of Human Rights, United Nations Organisation,\\ \url{https://www.un.org/en/about-us/universal-declaration-of-human-rights}} Further it says, “Parents have a prior right to choose the kind of education that shall be given to their
children”.\footnote{\textit{Id,} Clause (3)} The Convention on the Rights of the Child demands the States Parties to
recognise the right of the child to education in a progressive manner particularly making
primary education compulsory and available free to all with the measures like encouraging
regular attendance at schools and reduction of the dropout rates.\footnote{Article 28 Clauses (a) and (e) of Convention on the Rights of the Child, Adopted and opened for signature, ratification and accession by General Assembly resolution 44/25 of 20 November 1989, entry into force 2 September 1990, in accordance with article 49, United Nations Human Rights, \url{https://www.ohchr.org/en/professionalinterest/pages/crc.aspx,} It provides, “States Parties recognize the right of the child to education, and with a view to achieving this right progressively and on the basis of equal
opportunity, they shall, in particular: (a) Make primary education compulsory and available free to all…. (e)
Take measures to encourage regular attendance at schools and the reduction of drop-out rates...”} These international
conventions inadvertently focus on the compulsory and free education to all and further the
obligation to impart education is on both the State as well as parents and guardians.

\vspace{-.1cm}

\noi
It was aptly admitted that right to education, in India, is a fundamental right coupled with
right to quality education. The Constitution of India, by inserting article 21A, made it
obligatory on the part of the state, parents and society at large to play a vital role to create
education as a fundamental human right to all, irrespective of their social and economic
background. The Supreme Court and High Courts have upheld the right to free and
compulsory education through their various landmark judgments. The Parliament
subsequently enacted the Right to Education Act in the year 2009 and implemented in 2010.
Following the 1986 education policy, changes have been taken place to a great extent.
Several mission mode programmes like “District Primary Education Programme (DPEP)
followed by the Sarva Shiksha Abhiyan (SSA), Rashtriya Madhyamik Shiksha Abhiyan
(RMSA), Rashtriya Uchchatar Shiksha Abhiyan (RUSA), the Constitution (Eighty- sixth)
Amendment Act (2002) and the Right of Children to Free and Compulsory Education (2009),
acceptance of external aid for education programmes in the early 1990s and the mobilisation
of additional domestic finances through the levy of an education cess, restructuring of teacher
education programmes, and expansion of both government and private education institutions
at all levels of education”.\footnote{Dewan, Hridaykant, and Archana Mehendale. Towards a New Education Policy: Directions and Considerations. 50, \textit{Economic and Political Weekly,} 15, 16, (2015), \url{www.jstor.org/stable/44002890}}

\vspace{-.1cm}

\heading{Right to Education act}

\vspace{-.1cm}

\noi
It is an axiomatic truth that an individual obtains equal opportunities under right to education
in order to develop his faculties to the fullest to become a complete human being. To
accomplish the goal of right to universal education, article 21A of the constitution obligates
the State to provide education to all. It says, “The State shall provide free and compulsory
education to all children of the age of six to fourteen years in such manner as the State may,
by law, determine”.\footnote{INDIA CONST, art. 21A} In pursuance of this article the Right of Children to Free and
Compulsory Education Act, 2009 (RTE)\footnote{Right of children to Free and Compulsory Education Act, 2009, No. 35, Acts of Parliament, 2009 (India)} was enacted and it came into effect from April 1, 2010.

\vspace{-.1cm}

\noi
The RTE Act, 2009 is a child-centric and proposes the re-orientation of the teacher to
accommodate himself to the situations of the child. It aims to provide the primary education
to all children aged between 6 to 14 years as a fundamental right. The act also mandates 25
per cent reservation for disadvantaged sections of the society which include SCs and STs,
Socially Backward Class, and Differently abled children. It makes provision to include drop
out and other children to the classes appropriate to their age. Sharing of the responsibilities
between central and state governments towards the financial and other educational assistance
to the children, prohibition of deployment of teachers for non-educational work except
specified under the Act, and providing safety and security by eradicating fear, trauma and
anxiety through establishing a child-friendly and child-centred learning are some of the
important features of the Act

\vspace{-.1cm}

\heading{Judicial Response on Right to Education}

\vspace{-.1cm}

\noi
Originally, article referring to the right to education was incorporated under the Directive
Principles of State Policy rather than placing under the Fundamental Rights. Assumptions
could be made that the framers of the Constitution were intended to make the State
responsible in providing quality education to its citizens without differentiating on social or
economic background. When the right to education was denied, the Supreme Court time and
again has interfered to assert the right in a fair way. While upholding the right to education as
a fundamental right under Article 21 of the Constitution, the Supreme Court in \textit{Mohini Jain v. State of Karnataka},\footnote{AIR 1992 SC 1858} observed that a citizen cannot be denied the right to education by
charging exorbitant fee in the form of capitation fee. Proceeding in this line, the Supreme
Court in \textit{Unni Krishnan v. State of TN},\footnote{1993 AIR 217; 1993 SCC (1) 645} reiterated the same and further held that every
citizen has a right to education until the age of 14 years at free of cost and thereafter it is
subjected to the economic and capacity and development of the State.

\noi
Basing on these judgments, the Constitution was amended and inserted three new provisions
to the Constitution; They were Articles 21A, 45 and 51A. Article 21A which says, “The State
shall provide free and compulsory education to all children of the age of six to fourteen years
in such manner as the State may, by law, determine”.\footnote{INDIA CONST. art. 21A} Article 45 provides, “The State shall endeavour to provide early childhood care and education for all children until they complete
the age of six years”.\footnote{INDIA CONST. art.45} Article 51A says, “Who is a parent or guardian to provide
opportunities for education to his child or, as the case may be, ward between the age of six
and fourteen years”.\footnote{INDIA CONST. art.51A} The above-mentioned articles clearly elucidate that the state along with
the parents or guardians of the child have an obligation to provide education at free of cost up
till the age of 14 and early childhood care of children below 6 years. While observing the
responsibilities of state, the Supreme Court in \textit{Avinash Melhrotra v. Union of India \&
Others}\footnote{WRIT PETITION (CIVIL) NO.483 OF 2004} held that there is a fundamental right to get education free from the panic of safety
and donation, and the educational right incorporates the provision of secure schools pursuant
to Articles 21 and 21A of the Constitution of India. No issue where a family seeks to educate
its children (i.e., including private schools), the State must make sure that children suffer no
damage in practicing their fundamental educational right.

\heading{Right to Education in India- a Dream or a\\ Reality?}

\vspace{-.1cm}

\noi
In India, assessing the situation in general, the enrolment figures show an upward trend at all
levels. Statistics of 2018 mention that 1291 lakh students for the primary have been enrolled
in 2015-16 and 676 lakhs for the upper primary, 391 lakhs for the secondary, and 247 lakhs
for the senior secondary have enrolled in 2014-15.\footnote{Children in India, 2018: A Statistical Appraisal, Social Statistical Division, Central Statistical Office Ministry of Statistics and Programme Implementation, Government of India, 41 (2018) \url{http://mospi.nic.in/sites/default/files/publication_reports/Children}\%20in\%20India\%202018\%20\%E2\%80\%93\% 20A\%20Statistical\%20Appraisal\_26oct18.pdf} The status of education in India, however, has acquired a steady but gradual growth since its independence, yet not satisfactory as it ought to be. Inadequate ratio between teacher and pupil, low budget
allocation by the Government for the education, entry of private sector into education due to
investment of FDI, poor implementation of the Education Policies, high taboo system
towards female education in the remote areas, lack of skill and job-oriented education in the
schools, illiteracy and lack of motivation by the parents towards education etc. are some of
the reasons for slow progress of education in India. Largely, on the one hand children belong
to remote and socio-economic backward areas failed to enrol themselves into secondary
education and on the other urban children failed to get into some professional courses due to
unreasonable fee structure. Right to education has become a dream to many rather than a
reality. Overall data of Indian education reveals that the right to education has not been
effectively accomplished despite the efforts of State and non-state entities. The budget
allocations hardly reached to their purpose, and it is observed that public expenditure on
education declined by 4 per cent in 1990 and 3.5 per cent by 2000. Increasing fee hikes,
withdrawal of stipends is the inevitable fallout of the austerity measures of the global
capital.\footnote{Bej, Sourina. How government is planning to put India's education sector on sale, dailyo, (11-12-2015)\\
\url{https://www.dailyo.in/politics/occupy-ugc-smriti-irani-higher-education-modi-government-wto-gats-chennaihrd-ministry-digital-india/story/1/7891.html}}

\vspace{-.1cm}

\noi
The framers of the constitution felt the need of right to education and made provision under
the Directive Principles of State Policy to make the State to shoulder the responsibility of
education. The successive governments, though enacted policies one after the other, failed to
accomplish the intention of the framers of the constitution due to lack of commitment and
widespread corruption in the system. Right to education implies the right to compulsory and
free education at the elementary or foundational level and higher education depending on the
calibre of the students and their economic status. While upholding the right to education
various International Conventions, Agreements, Protocols and Declarations mention that the
essential components of education are availability, accessibility, acceptability and
adaptability.\footnote{General Comment no. 13, para. 6. \url{https://www.ohchr.org/EN/Issues/Education/Training/Compilation/Pages/d)GeneralCommentNo13Therighttoed
ucation(article13)(1999).aspx}} At the national level, the Constitution, the Right to Education Act and Judicial
pronouncements assume that right to education is a fundamental right. The national policies on education have been drafted by taking into consideration the above assumptions. But due
to their poor implementation and moreover the lack of will of the policy makers the universal
right to education remained far from the reality. Among them, ineffectiveness of Right
Education Act and desire towards Privatisation are important one.

\vspace{-.1cm}

\heading{Poor Performance of RTE Act 2009}

\vspace{-.1cm}

\noi
RTE Act 2009 is an attempt to serve the purpose of education to all. Expectations to reap the
fruits of the Act were very high but it failed to render the desired results due to its poor
execution. Though there was an effort to enact a bill in the year 2006, due to lack of funds,
the then Finance Committee and Planning Commission rejected the proposal. In the year
2009, when the bill became an Act and implemented on April 1, 2010, it raised a great deal of
expectations in the minds of general public. It was considered a great step towards education
for all and of national importance. “In principle, the RTE Act 2009, with appropriate
modifications and financial provisioning, offers a great opportunity to correct the anomaly of
poor education outcomes, and can deliver on the long-standing commitment of providing
basic and quality education to the so called ‘demographic dividend’ of the country”.\footnote{Jha, Praveen and Parvati, Pooja. “Right to Education Act 2009: Critical Gaps and Challenges”, XLV, \textit{Economic and Political Weekly,} 20, 23, (2010)}

\vspace{-.1cm}

\noi
The Act had an aspiration to fulfil the aim of Sustainable Development Goal 4 which speaks
about education for all by the end of 2015 along with Millennium Development Goal. But it
faced several challenges in the wake of its implementation. Certain challenges which the Act
faced during its initial stage are mainly the shortage of funds, lack of proper infrastructure,
the accountability of the schools mentioned under the Act, shortage of human resources in the
form of teachers, the implementation of public- private partnerships, etc. Giving a death blow
to the implementation of the Act at the initial stage, various states raised their contention
against the financial burden that they had to bear for the 25 per cent children from
disadvantaged groups who seek education under the Act.\footnote{Deccan Herald, No Funds for RTE, says Maya, (2010) \url{https://www.deccanherald.com/content/62144/nofunds-rte-maya-pm.html,}} A civil society survey at the
nationwide shows that lack of proper infrastructure and human resource in the form of
teachers had equally a great set back to the Act for its effective implementation. While
exhibiting the eligibility of qualified teachers and the availability of infrastructure, the report
observes, “…a shockingly high percentage, 93, of teacher candidates failed in the National 
Teacher Eligibility Test conducted by the Central Board of Secondary Education in 2010-11.
In 2009-10, the failure was 91 per cent in the national examination, meant to test the
candidates' teaching aptitude and a prerequisite for appointment of .95.2 per cent of schools
are not yet compliant with the complete set of RTE infrastructure indicators”.\footnote{The Hindu, Lack of school infrastructure makes a mockery of RTE, (April 05, 2012)   \url{https://www.thehindu.com/news/national/lack-of-school-infrastructure-makes-a-mockery-ofrte/article3281720.ece}} Even the “bureaucratic apathy and weak institutional mechanisms are some factors that have
contributed to this”\footnote{Business Line, Courting justice for the right to education, (Jan. 13, 2018)
 \url{https://www.thehindubusinessline.com/opinion/courting-justice-for-the-right-to-education/article9557598.ece}} While delving on the accountability of the schools are concerned, it is
not clear the reason behind exempting or leaving out the accountability of unaided schools
under the Act.\footnote{RTE \textit{supra note}, 33, Sec. 21} The concept of admission of a child of any age to the appropriate class under section 4 of the Act is absurd and impractical in its very notion of implementation.\footnote{ Id, sec. 4} These above-mentioned issues have made the poor implementation of the Right to Education Act in
its entirety. Adding to that the litigations in the various High Courts regarding fixing the age
limit for admission to a particular class, denial of admission, issues concerning inter-school
transfers, and admission procedures, are some of them.

%~ \vspace{-.1cm}

\heading{Privatization of Education}

%~ \vspace{-.1cm}

\noi
Right to education is not restricted to basic or elementary education but it includes right to
have quality education towards a child’s holistic growth. High fee structure and over
competitiveness is prone to commercialisation of education rather than providing quality
education to all. In the bargain, merit-based education and education to the deserving children
will be at stake. Privatisation of education has hampered the growth of right to education to
all. The governments, since the introduction of Liberalization, Globalisation and Privatization
(LPG) in 1991, has failed to accomplish the goals of education for all by making suitable
provisions of free education under its fiscal policies. An analysis of the National Sample
Survey Organisation (NSSO) data for two time periods-1995-96 and 2014-15, unravel the
fact that the enrolment of the student ratio in both urban and rural areas has remained either
static, or it has reduced in both public and aided institutions, and has increased almost
universally in private unaided institutions.\footnote{Jha, Jyotsna, Education India Private Limited, 42, \textit{India International Centre}, 39, 40 (2015), \textit{JSTOR},\\ \url{www.jstor.org/stable/26316574.}} India being one of the largest education system and highest number of education institutions in terms of enrolment, provides ample
opportunities for the investment and “the higher education sector of India is considered as the
‘sunrise sector’ for investment as it is a market worth US\$15 billion”\footnote{Singh, Vikram. Higher Education of India on the Way to Nairobi for GATS, \textit{People’s Democracy} (2015),\\ \url{https://peoplesdemocracy.in/2015/0816_pd/higher-education-india-way-nairobi-gats.}}

%~ \vspace{2cm}

%~ \vspace{-.1cm}

\heading{Suggestions and Recommendations}

%~ \vspace{-.05cm}

\noi
Taking into consideration above mentioned drawbacks, suggestions for the effective
implementation of the right to education policy would be -

%~ \vspace{-.2cm}

\begin{enumerate}[label=$-$]
\itemsep=0pt
\item Strengthening the provisions of RTE, 2009 and measures to implement it at any
cost. Certain rectifications need to be brought in both willingness on the part of
State to execute the Act in its entirety and to amend some provisions for the
effective implementation of the said Act.

\item Primarily, the State should allocate funds for the same. It should allocate a
substantial amount, not less than 6 per cent of the total budget, for the cause of
education and must spend the whole allocated amount for the purpose without
making any deviations. There must be a strict regulatory body to implement the
same.

\item Well qualified teaching faculty at all levels of education must be promoted by
selecting qualified teachers and further providing avenues for them to have
ongoing formation as years go by. The incentives of promotion must be given
basing on their proficiency in teaching and not merely on the age factor.

\item The amendments to be made to the RTE Act in the spheres of funds allocation to
bear the expenditure of the 25 per cent children from disadvantaged groups who
seek education under the Act. Present system of allocation would certainly bring
dilemma between Centre and the States. Therefore, the responsibility must be
given strictly either to State or Centre to avoid unnecessary confusions unlike the
present system of sharing the responsibility between the Centre and the States.

\item The entire mode of education should be directed towards the holistic growth of a
child rather than rote learning. A comprehensive learning process including extracurricular activities with the academic subjects should be introduced. Technical, vocational and other life-oriented subjects should be introduced to lead a child
towards the mainstream of the society both in the spheres of knowledge and
status.

\item Commercialisation of education should be brought under the strict regulatory
control and all the education institutions must follow the general fee structure
decided by the state authority. All the States must encourage the state-sponsored
schools or aided schools in the remote and socio-economically backward areas
instead of allowing the profiteering private entities.

\item Parameter of discrimination should be merit oriented and not socio-economic
oriented. Elementary and Secondary quality education should be provided to all
without any discrimination and talented children of tribal and backward areas
must be given scholarship on the basis of merit for their higher education.

\item A full-fledged education policy to implement the right to education, in an
effective manner, without discriminating either on socio-economic background or
urban-rural basis, is expected from the state as a welfare state.
\end{enumerate}

\vspace{-.25cm}

\heading{Conclusion}

\vspace{-.1cm}

\noi
Since India is dominated by villages, it is relevant that State imparts education to all without
discrimination on the basis of social and economic backwardness of a child. Private sectors or
non-state entities may provide education at the cost of high fee structure and it could hardly
be accessed by poor and backward categories children. Lack good infrastructure and qualified
dedicated teachers in the remote areas fails to provide quality education to the children, even
at the foundational or elementary level. Government, due to its failure to upgrade public
schools and colleges with the state aid, failed to compete with the private entities with good
infrastructure and dedicated team of experts to raise the standard of education. National
Policy on Education 1986 provides for the Universal Elementary Education to all but the upgradation and privatisation of education sector became a road block for its implementation.
Even the RTE, 2009 failed to accomplish the expected rate of growth. It is also noteworthy
that quality education can only be considered if the holistic growth of a child is taken care.
Literacy alone would never suffice the growth of a child but a comprehensive quality
education which takes care of the development of a child is necessary. Due to lack of
effective implementation of Constitutional provisions on education through National Education Policies and RTE, the education for many has become a dream especially the
socially and economically backward section of the children. Being a fundamental and human
right, the State is obliged to implement the right to education on a war footing to render
justice to all without discriminating on the basis of socio-economic considerations.

\end{multicols}
\label{end2019-art2}
