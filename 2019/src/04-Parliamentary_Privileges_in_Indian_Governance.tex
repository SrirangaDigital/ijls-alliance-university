\setcounter{figure}{0}
\setcounter{table}{0}
\setcounter{footnote}{0}

\articletitle{Parliamentary Privileges in Indian Governance System: Role of Free Speech in Promoting Transparency}
\articleauthor{Dr. Prashna Samaddar\footnote{Assistant Professor, School of Law, Galgotias University} and Victor Nayak\footnote{Assistant Professor, School of Law, Galgotias University}}
\lhead[\textit{\textsf{Dr. Prashna Samaddar and Victor Nayak}}]{}
\rhead[]{\textit{\textsf{Parliamentary Privileges in Indian Governance...}}}

\begin{multicols}{2}

\heading{Introduction}

\vspace{-.1cm}

\noi
Parliamentary privilege as a concept can be defined in various manners and has been done so
by various eminent jurists. May defines the privileges as,

\vspace{-.3cm}

\noi
\begin{quoting}
\textit{“The sum of peculiar rights enjoyed by each House collectively as a constituent part
of the High Court of Parliament, and by members of each House individually, without
which they could not discharge their functions and which exceed those possessed by
other bodies or individuals. Thus, privilege though part of the law of the land, is to a
certain extent an exception from the ordinary law.”}\footnote{Hajare, Shashikant \textit{The law of parliamentary privileges in India: problems and prospects,} Sodhganga (Oct. 30, 2018 10:04 AM)  shodhganga. \url{inflibnet.ac.in/bitstream/10603/52360/12/12_chapter}\%205.pdf.}
\end{quoting}

\vspace{-.3cm}

\noi
The legal concept regarding parliamentary privileges is laid down under Article 105 and
Article 194 of the Indian Constitution. As per these provisions the members of Parliament
have been given protection from being tried or prosecuted under any civil or criminal
proceeding with regards to anything that they might say or approve or disprove by casting
their vote in the parliament or any committee as has been exclusively provided in the
constitution.\footnote{V Razdan, \textit{Parliamentary privileges in India: separation of powers, Sodhganga} (Oct. 30, 2018 10:04 AM)\\ \url{www.shodhganga.inflibnet.ac.in/bitstream/10603/8783/4/04_preface.pdf}} The advantages are sure rights having a place with Parliament aggregately and some others having a place with the Members of Parliament individually, without which it would be inconceivable for the House to keep up with its autonomy of activity or the pride of
position or for the individuals to release their capacities.\footnote{DALIP SINGH, \textit{Parliamentary Privileges In India,} 26 IJPS, No. 1 76, 75-85 (1965).}

\vspace{-.2cm}

\noi
While providing the explanations behind leaving the parliamentary privileges vague in the
Indian Constitution, Dr. B R Ambedkar, while chairing the Constituent Assembly which
drafted the Indian Constituion, brought up that aside from the advantage of the right to speak 
freely of discourse and resistance from capture, the advantages of parliament were a lot more
extensive and amazingly hard to characterize. Thus, according to him it was not workable to
\textit{create a complete code on the above subject matter and incorporate the same as a part of
Indian} Constitution. So, it was thought best to leave it to the Parliament to define and limit its
privileges whereas the Indian Parliament was vested with the same set of privileges that were
enjoyed by the England House of Commons.\footnote{\textit{Id} at 78.}

\noi
The Constitution specifies some of the privileges. They are:

\vspace{-.3cm}

\begin{enumerate}[label=$\bullet$]

\itemsep=0pt

\item the right to speak freely of discourse in Parliament;\footnote{INDIA CONST. art. 105, cl. 1.}

\item not liable to any procedures in any court in regard of anything said or any vote
given by him in Parliament or any board of trustees thereof;\footnote{INDIA CONST. art. 105.}

\item  resistance to an individual from procedures in any court in regard of the
distribution by or under the authority of one or the other Place of Parliament of
any report, paper, votes or procedures.\footnote{\textit{Id.}}

\newpage

\item  Courts are disallowed from questioning the legitimacy of any procedures in
Parliament on the ground of a supposed abnormality of strategy.\footnote{INDIA CONST. art. 105, cl. 1.}

\item  No official or individual from Parliament enabled to control technique or
direct of business or to keep control in Parliament can be dependent upon a
court's locale in regard of the activity by him of those forces.\footnote{INDIA CONST. art. 105, cl. 2.}

\item  No individual can be at risk to any considerate or criminal procedures in any
court for distribution in a paper of a generously obvious report of procedures
of one or the other place of Parliament except if the distribution is
demonstrated to have been made with perniciousness. This invulnerability is
additionally accessible for reports or matters broadcast through remote
telecommunication.\footnote{INDIA CONST. art. 361A.} This immunity, however, is not available to publication
of proceedings of a secret sitting of the House.\footnote{INDIA CONST. art.361A, cl. 1, proviso.}
\end{enumerate}

\vspace{-.3cm}

\noi
Aside from the advantages indicated in the Constitution, the Code of Common System, 1908,
accommodates independence from capture and confinement of individuals under common
interaction during the duration of the gathering of the House or of a board thereof and forty
days before its initiation and forty days after its decision.\footnote{§135A of Code Of Civil Procedure, 1908, No. 5, Acts of Parliament, 1908 (India).}

\noi
Different advantages, as provided under the Rules of Procedure and Conduct of Business in
Lok Sabha\footnote{Rules of Procedure and Conduct of Business in Lok Sabha, 1952 (Aug. 13, 2021, 02:05 PM) \\ \url{http://164.100.47.194 › RULES-2010-P-FINAL_1}} by specific legislative inputs have been discussed hereunder:

\begin{enumerate}[label=$\bullet$]
\itemsep=0pt
\item Exclusion of Individuals from responsibility to fill in as legal hearers.\footnote{SUBHASH C. KASHYAP, OUR PARLIAMENT 234-36 (National Book Trust 1995).}

\item Right of the Parliament to get quick data with regards to capture, confinement,
detainment as well as arrival regarding the Part.\footnote{§§ 229 and 230 of Rules of procedure and conduct of Business in Lok Sabha, 1952}

\item Prohibition with respect to capture and administration of lawful interaction
inside the Parliament premises without acquiring authorization of the House
Speaker.\footnote{Kashyap, \textit{supra} note 15 at 236}

\item Restriction regarding exposure to the procedures or choices of a mysterious
sitting of the House.\footnote{\textit{supra} note 16, §§ 232 and 233.}

\item All legislative Councils can send people, documents, and accounts important
with the end goal of the request by an advisory group.\footnote{\textit{Id} § 252}

\item A Parliamentary Panel might direct promise or insistence to an observer
(witness) analyzed before it.\footnote{\textit{Id} §§ Rules 269 and 270}

\item The proof offered in front of the Panel of Parliamentarians and its statement
and procedures can't be revealed or distributed by anybody unless it comes to
the Parliament.\footnote{\textit{Id} § 272}

\item Option to deny the distribution of its discussions and procedures.\footnote{\textit{Id} § 275}

\item Privilege to leave out outsider from the House.\footnote{\textit{Id} § 249}

\newpage

\item Right to punish individuals in case if any breach of privilege or contempt of the House, whether they are members of the House or not.\footnote{\textit{Id} § 248}
\end{enumerate}

\vspace{-.3cm}

\heading{Contempt and Free Speech: Conflicts}

\noi
As a rule, any demonstration or oversight which deters or blocks either Place of Parliament in
the exhibition of its capacities, or which hinders or obstructs any Part or official of such
House in the release of his obligation, or which has a propensity, straightforwardly or by
implication, to deliver such outcomes might be treated as a hatred despite the fact that there is
no point of reference of the offense.\footnote{ERSKINE MAY, PARLIAMENTARY PRACTICE 115 (21st ed., 1989).}

\noi
At the point when one looks at the connection between the courts also, lawmaking bodies, the
inquiries with respect to the position to choose the presence of an advantage and with regards
to whether the courts could look at the legitimacy of committal by a lawmaking body for its
hatred or break of advantage and so forth have to be tended to.\footnote{M.P. JAIN, INDIAN CONSTITUTIONAL LAW 66 (8$^{\rm th}$ ed. 2018).} Indeed, the circumstances under which the lawmaking bodies guarantee advantages in India get the courts the field regularly. In India, the lawmaking bodies might guarantee the advantages under three
circumstances and where the courts can’t be allowed to have any part to play:

\vspace{-.25cm}

\begin{enumerate}[label=$\bullet$]
\itemsep=0pt
\item at the point where the Constitution provides the same explicitly;

\item the council has made the legislation;

\item a privilege that has been provided to by the House under the Indian Constitution\footnote{\textit{Id} at 68.}
\end{enumerate}

\vspace{-.25cm}

\noi
A portion of the above-expressed issues were analyzed by the Apex Court in Kesava Singh In
re\footnote{AIR 1965 SC 745; This reference was a continuation of the death of a request by a phenomenal Full Seat of
28 Appointed authorities, remaining, under Article 226, the execution of the U.P. Gathering Goal requesting two
Appointed authorities of the Allahabad High Court to be brought into care before the Bar of the House to clarify
why they ought not be rebuffed for the scorn of the House. The two Appointed authorities had conceded the
habeas corpus request and allowed bail to Mr. Kesava Singh who was going through detainment in compatibility
of the Gathering Goal pronouncing him blameworthy of the break of advantage. The goal of the Get together
and the stay request gave by the Full Seat brought about an established impasse. Therefore, the President alluded
the matter under Article 143 to the High Court for its viewpoint.} and the greater part assessment for this judgment is provided regarding a constitutional interpretation as upheld by 6:1 majority:

\vspace{-.2cm}

\begin{enumerate}[label=$\bullet$]
\itemsep=0pt

\item The manner of interpreting Article 194 (as well as Article 105) concerning the nature,
degree and impact of the autonomy of the House rests with the legal executive of the
country.

\item As per powers provided under Article 226, the High Court can investigate the orders
given by the law makers with regards to the articulation “any position” as mentioned
under Article 226 which encompasses the latter too.

\item Article 211 provides clearly that the behaviour of a judicial member in the release of
his obligations cannot be the topic of any activity which can be tabled by the House
while exercising its functions or advantages presented by Article 194(3).

\item Article 212 prevents the Court from regulating the procedure that is to be followed
inside the House but does not prevent it from checking the validity of any action that
has been taken in connection to the same.

\item The first part of Article 194(3) when read with the last part of Article 194(4) ensures
that the future laws define the privileges which should be in conformity to the
fundamental rights and thus such enactment would be “law” as per Article 13 of the
Indian Constitution which thus gives the Court the competency to check the sanctity
of action in light of the existing fundamental rights.
\end{enumerate}

%~ \vspace{-.2cm}

\newpage

\heading{Free Speech and Parliamentary Privileges: Relationship and Issues}

\noi
The ability to speak freely is dependent upon different arrangements of the Constitution and
subject to the principles outlined by the House under its ability to control its own
procedures.\footnote{INDIA CONST. art. 118, cl. 1.} Indian Parliament’s upper and lower house both have outlined certain
guidelines and have approved their directing officials to apply and authorize them. For
instance, the guidelines of the strategy of the lower chamber enforce various impediments
upon the right to speak freely of its individuals and enable the speaker to make a fitting move
guiding the individuals to pull out from the House;\footnote{§ 373, Rules of procedure and conduct of Business in Lok Sabha, 1952.} on requesting his suspension;\footnote{\textit{Id.} § 374.} or ordering the ban of offensive words from the proceedings of the House.\footnote{\textit{Id.} § 380.}

\noi
Besides the Constitution forces another constraint upon the the right to speak freely of
discourse whereby in the Parliament that no conversation will happen concerning the direct
of any appointed authority of the High Court or a High Court in the release of the obligations
besides upon a movement for introducing a location to the President appealing to God for the
evacuation of the adjudicator.\footnote{INDIA CONST. art. 121.}

\vspace{-.1cm}

\noi
The expression “powers, privileges and immunity” as provided in the Constitution have
evoked sharp contention in the country since the initiation of the Constitution. This is perhaps
the most disputable provision in the Constitution which looks to connect and equalise, the
advantages and immunity accessible to the individuals from Parliaments in the two nations.\footnote{Singh, \textit{supra} note 5.}

\vspace{-.1cm}

\noi
The House of the People (Lower House) of the Indian Parliament made history on August 29,
1961, by reprimanding a journalist at the Bar of the House and in that context functioning as
the High Court of Parliament for the first time. It was a unique event and has set in motion
the discussion which actually proceeds unabated. The individual to whom the censure was
controlled was the supervisor of Barrage, a Bombay liberal diary. His offense was the
distribution of an article in the diary entitled “The Kripaloony Impeachment”\footnote{Gunupaty Keshavram Reddy v. Nafisul Hassan AIR 1954 SC 636; MSM Sharma v. Shri Krishna AIR 1959, SC 365.} which, as per
the Privileges Committee of the Lower House of Indian Parliament, “in its tenor and
substance criticized a fair individual from this House (J. B. Kripalani) and cast reflections on
him by virtue of his discourse and direct in the House and alluded to him in a derisive and
offending way.”\footnote{M. V. Pylee, \textit{Free Speech and Parliamentary Privileges in India,} 35 Pacific Affairs, 13, 11-23 (1962).}

\vspace{-.1cm}

\noi
The Parliamentary Proceedings (Protection of Publication) Act, 1956 states that no criminal
or civil proceedings may be initiated before any judicial forum against any individual in
matters related to the distribution of a fundamentally correct report which deals with the
Parliament proceedings unless it can be shown with reasonable conviction that it was
specifically instructed by the Speaker of the House to obliterate the same.\footnote{Dr K. Madhusudhana Rao, \textit{Codification of Parliamentary Privileges in India - Some Suggestions,} (2001) 7 SCC (Jour) 21.} This position was 
strengthened after the insertion of Article 361-A brought about by the 44$^{\rm th}$ Constitutional
Amendment Act of 1978.\footnote{\textit{Id.}}

\vspace{-.1cm}

\noi
In Alagaapuram R. Mohanraj and Ors vs Tamil Nadu Legislative Assembly\footnote{WRIT PETITION (CIVIL) NO. 455 OF 2015.} Justice Chelameswar has given a decision in this regard in recent times whereby the Court made the following issues:

\vspace{-.3cm}

\begin{enumerate}
\itemsep=0pt
\item When an individual from a State Assembly takes part in the House proceedings,
does it come under the purview of freedom of speech and expression as guaranteed
under Article 19(1)(a) of Indian Constitution?

\item Whether any act of anybody or authority empowered by any law hinders any
individual from taking part in the discussions held at any sessions being conducted
by the body or authority in question and that hindrance leads to preventing that
individual from exercising his fundamental right to speech and expression as
provided under Article 19(1)(a) of Indian Constitution?
\end{enumerate}

%~ \newpage

\vspace{-.3cm}

\noi
Upon examination of the above issues in context of the mandate provided under the
constitution, it was decided by the Apex authority that the scope of freedom of speech and
expression as guaranteed under Article 19(1)(a) is quite distinct and holds a higher position as
compared to the same right when provided as a privilege under Article 105 or Article 194 of
Indian Constitution. There are 4 major factors which go on to highlight the mentioned
distinction:

\vspace{-.3cm}

\begin{enumerate}[label=$\bullet$]
\itemsep=0pt
\item The former is wider in scope as its available to every citizen of India, whereas the
latter is applicable and enjoyed only by the legislators;

\item  whereas the former is unassailable, the latter is applicable only while individual
remains as a member of the Parliament;

\item more importantly Article 19(1)(a) has not confinement issues unlike the other set of
provisions i.e. Article 105 and 194 which are restricted to the legislative premises

\item though both the sets of provisions provide for some restrictions upon the enjoyment
of the rights, but the freedom of speech in as a part of parliamentary privileges are
regulated by legislative bodies or as imposed by the Indian constitution as under
Article 121 and 211 of Indian constitution.
\end{enumerate}

\vspace{-.3cm}

\heading{Codification and Parliamentary Privileges: Concluding Remarks}

\vspace{-.2cm}

\noi
There is an indisputable difference in the position regarding the supremacy of all rights and
advantages and the privileges that are present under Indian legal framework. In India the
legislative bodies decide about the privileges in context of its scope, violation and
punishment for any such violation with regards to their speech in the Parliament which is
somewhat unsettling and raises some grave questions regarding the unrestricted power that is
available and somehow its undermines the principles of Indian constitutional and democratic
framework.\footnote{Subodh Asthana, \textit{Parliamentary Privileges in India,} iPleaders (Jul. 23, 2018, 02:05 PM)
 \url{https://blog.ipleaders.in/parliamentary-priviledges/#Misuse_of_Parliamentary_Privileges.}}

\noi
It is the utmost responsibility of the authorities to maintain the balance between the privileges
and the rights so that it does not affect the constitutional framework. Though in many judicial
opinions judiciary has respected the privileges of the members of Parliament and State
Assemblies but later even they also felt that the privileges should be restricted in such a way
that it does not damage the fabric of Indian democracy. It has been seen that on many
occasions the freedom of press have been curtailed in the garb of these privileges. Paying
heed to the Constitution of various countries, the Apex Court in case M.P.V. Sundaramier
and Co. v. Territory of Andhra Pradesh advised:\footnote{AIR 1958 SC 468.}

\vspace{-.4cm}

\noi
\begin{quoting}
\textit{“The strings of our Constitution were no vulnerability taken from other Government
Constitution yet when they were woven into the surface of our Constitution their
compass and their structure experienced changes. Thusly, critical as the American
decisions are as demonstrating how the request is overseen in the Government
Constitution exceptional thought should be taken in applying them in the
comprehension of our Indian Constitution.”}
\end{quoting}

\vspace{-.4cm}

\noi
The National Commission to Review the Working of the Constitution (NCRWC) in its report
has mentioned that:\footnote{NCRWC Report, 2002 (Jul. 23, 2018, 02:05 PM) \url{https://legalaffairs.gov.in/ncrwc-report}}

\vspace{-.4cm}

\noi
\begin{quoting}
\textit{“the advantages of lawmaking bodies should be portrayed and delimited for the free
and independent working of Parliament and State Councils.”}
\end{quoting}

\vspace{-.4cm}

\noi
Thus, it needs to be understood that the process of codification has its advantages and
establishes the principles of Rule of Law. It thus goes without saying that the advantages of
the Parliamentarians need to be systematized by removing penal provisions for any violations
of the privileges by any common man. Parliamentarians and the greater part of the Presiding
Officials have gone against the transition to classify them on the ground that as the legal
understanding of the law is the obligation of none else except for the legal executive. Article
105 clause (3) and Article 194 clause (4) of the Constitution of India, 1950 are provision
which enable for characterizing the forces, advantages and insusceptibilities of Indian
Parliament just as its individuals and committees. The un-codified and characterize corrective
forces of authoritative bodies in India lead to legitimate polemics between assemblies, court
and resident in India. As rightly put by Justice Iyer \textit{“Parliament of India is not and can never
be a court and we have separate judiciary”}.\footnote{Dr. Jyoti Dharm and Mr. Gaurav Deswal, \textit{Parliamentary Privileges In India: A Comprehensive Study,} 3 BLR 177, 172-177 (2016).}
\end{multicols}
