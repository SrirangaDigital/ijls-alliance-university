\setcounter{figure}{0}
\setcounter{table}{0}
\setcounter{footnote}{0}

\articletitle{Right to Refuse Treatment Vis-À-Vis Passive Euthanasia: Judicial Approach}
\articleauthor{Sayan Das\footnote{Research Scholar \& Assistant Professor, School of Law, Galgotias University.}}
\lhead[\textit{\textsf{Sayan Das}}]{}
\rhead[]{\textit{\textsf{Right to Refuse Treatment Vis...}}}

\begin{multicols}{2}

\heading{Introduction}

\noi
The landmark judgment in \textit{Aruna Ramchandra Shanbaug}\footnote{Aruna Ramchandra Shanbaug v Union of India and others, 2011 AIR SC 1290.} modified indefinitely India’s
controversial approach to euthanasia by authorizing perpetual life support, a method of
passive euthanasia, life support system withdrawal only for patients in a permanent
vegetative state (PVS). Passive euthanasia will or can “only be allowed in cases where the
person is in a persistent vegetative state or terminally ill”\footnote{Id.}
, according to the verdict. Under certain circumstances and conditions, the act of withdrawing or removing life-support
medical treatment from a terminally ill or permanent vegetative patient state may be allowed.
Another noteworthy court ruling followed after seven years of the Aruna Shanbaug case on
9$^{\rm th}$ March, 2018 through \textit{Common Cause (a reg. society) v. Union of India}\footnote{Common Cause v Union of India, (2018) SCC Online SC 208.}
 where the Supreme Court of India constituted with five judges has ruled that having the right to die with
dignity is a fundamental and basic right. In seeking the solution or a way out, the issues
which grappled the Hon’ble Court in \textit{Common Cause} were: Is it unconstitutional for a person
to refuse medical care, or is it illegal for them to refuse a certain sort of medical treatment? If
this happens, can the individuals concerned make their own decisions about what steps can be
taken in the future if they lose control of their faculties? To the question of whether an
individual has a right and so imposes a duty on a medical professional who treats the
individual, the answer is that this does lay an obligation on the doctor. To what extent, if any,
does this obligation need qualifications is the next issue; Additionally, whether it is
permissible for a medical practitioner to withhold or refuse medical treatment towards the
end of an individual’s life who has lost control of his or her faculties and this desire was
expressed when he or she was able to make an informed decision and was able to think
clearly. That individual’s capacity to make an informed decision in a clear mind would likely
allow them to make decision and decline medical treatment if there is no reasonable hope for
recovery. The Bench went even further and found that the practice of passive euthanasia and 
advance medical directives are likewise legally acceptable. Also, there should be less
unpleasantness in the process of dying for patients who are terminally sick or for patients
who are in a vegetative state, because those individuals should be able to have an
undisturbed, peaceful passing.

\noi
Consequently, passive euthanasia is now legal in India, despite the fact that granting sanction
for passive euthanasia has triggered some substantial concerns in various legislation and
conceptions of human dignity. According to the Supreme Court's ruling, the Indian
Constitution in its Article 21 not only provides the right to life with dignity but on the other, a
negative right such as the right to die with dignity (passive euthanasia by removal of life
support) that is now allowed for those with a terminal illness and/or those in a persistent
vegetative condition. The aim of the study is to analyze and critically examine the idea of the
right to reject or refuse treatment and the admissibility to that right while considering legal
and ethical concerns in congruence to passive euthanasia. As can be seen, legalization of
passive euthanasia has important and dangerous implications, especially for the right to
health. The approval of passive euthanasia in common law has at times been compared to
active euthanasia; consequently, expanding the concept of passive euthanasia has severe
implications in regard to the right to refuse treatment concept already prevailing, whereas
passive euthanasia is seen by some exponents \& rigorous supporters of active euthanasia as
no form of euthanasia at all.

\noi
Patients hold the most crucial right of right to self-determination, which means they can
decide for themselves whether their bodies will be treated medically or not. It is important to
recognize that the aforementioned fundamental human rights inherent in informed consent
also have a significant corollary: the ability to refuse treatment. To accept therapy, one has
the freedom to object. However, when one "believes" they should be treated, then their selfdetermination has diminished to the point where it can be described as an "obligation" to
follow one's doctor's orders. On the other hand, there are still reports of cases in which
individuals are still being treated despite making intelligent objections or withdrawing
consent, even though courts universally acknowledge patients' rights to refuse treatment, and
although the methods for implementing this right are similar in all countries, there have been
disparities in how these standards have been stated and implemented.\footnote{\textit{The Right to Refuse Treatment: A Model Act,} (May 18, 2018, 06:25 PM), \url{https://www.ncbi.nlm.nih.gov/pmc/articles/PMC1651109/pdf/amjph00643-0086.pdf.}}

\noi
Patients have the legal capacity to consent to medical treatment, or reject permission, if they
are able to grasp and remember information and make a rational choice that considers the
consequences of their treatment choices. In the medical field, all patients who are considered
adults are presumed to be competent. This presumption can however be challenged. A
competent individual refusing treatment can give rise to a lawsuit for battery; the doctor who
takes the patient to the therapy regardless of their will is at fault (unlawful physical contact).
Since it is possible to sue the doctor in civil courts, or even to prosecute him in criminal
courts, he is vulnerable to claims in both courts. Patients who can make decisions regarding
their treatment and well-being have absolute right to reject or to continue with treatment,
where non-treatment leads to certain death. For example, Ms B is a patient being kept alive
by artificial life support system, a ventilator which allows her to move and speak despite her
paralysis. They refused to withdraw the ventilator, so she begged the physicians to do it. The
court found that Ms B was competent, and that thus she had the right to refuse even lifesaving therapy, since she had the right to refuse treatment, her doctors acted unlawfully in
prolonging her ventilation. Although the right to refuse treatment established by these
instances appears to be absolute, as well as extending to requests for rejection that are
manifestly suicidal, there appear to be no limitations. While it appears that people have the
freedom to refuse medical interventions and at the same time, they do not possess the right to
decide or consent to.\footnote{John Keown, \textit{Medical murder by omission? The law and ethics of withholding and withdrawing treatment and tube feeding,} (Jun. 05, 2018, 08:30 PM),\\  \url{https://pdfs.semanticscholar.org/2870/a75cb6a6c85bf46feb2f15d2669b2ddd35ac.pdf.}}

\heading{Right to Refuse-Reject Treatment}

\noi
Even if right of patients to refuse or reject treatment has already been widely recognised or
what has been prevailing with or without any rule book, the acceptance of this even being
represented in such official documents as the Patient's Charter is still in its early stages.
While the ability to perform euthanasia on patients may be much anticipated, it must be
acknowledged that allowing such euthanasia does not, in any way, give people permission to
kill patients. One could justifiably consider it pointless to argue for such a seemingly apparent argument, but because many modern experts in medical law and medical ethics are
appearing to presume the opposite.\footnote{Susan L Lowe, \textit{The right to refuse treatment is not a right to be killed,} Journal of Medical Ethics, 1997 23: 154-158, (Mar. 28, 2018, 05:45 PM),  \url{https://www.ncbi.nlm.nih.gov/pmc/articles/PMC1377341/pdf/jmedeth00308-0028.pdf.}}

\noi
Article 21 of the Constitution of India guarantees that a person's right to life and personal
liberty is guaranteed. Each citizen or individual has the right to live out one’s life and to
retain personal freedom unless those rights are taken away as part of a legal procedure
defined or established by law. The phrase in grammatical form may appear negative but
judicial interpretations have found that it is really a powerful expression of many of the
positives in society. Many fundamental rights can be found under Article 21 of the
Constitution, where they grow, flourish, and gain nutrition. One must see these fundamental
human rights through the Indian constitutional provisions that advocates about preserving
human dignity while choosing whether or not to accept treatment. Although there is a
considerable body of law supporting this proposition, the understanding that any legislation
that comes into conflict with or that does anything to limit the Constitutional Rights of India
citizens is null and void has long been settled.\footnote{S Balakrishnan \& RK Mani, \textit{The constitutional and legal provisions in Indian law for limiting life support,} IJCCM 2005 Volume 9 (2) 108-114, (Apr. 01, 2018, 09:10 PM), \url{http://www.ijccm.org/article.asp?issn=0972-5229;year=2005;volume=9;issue=2;spage=108;epage=114;aulast=Balakrishnan.}}

\noi
A fundamental natural right is described in Article 21: to be given the opportunity to lead a
peaceful and dignified life. For those who choose to end their lives, suicide is not a choice; it
is a process, a cessation of life, an extinction of existence. While the "right to life" includes
not allowing suicide, suicide is incompatible and contradictory with this idea in regard to the
notion of dignity, and right to die concept along with dignity are pertinent principle. As
healthcare advances, it is the responsibility of the state to safeguard the health of its citizens
while also enhancing health care facilities. Physicians, on the other hand, have a duty to give
good medical care, but not to harm or neglect patients. This is relevant in the context, and one
of the pertinent rights of a patient is the right to discontinue or reject or refuse medical
treatment. A right to reject or refuse medical treatment has been prevailing from a long time
and is also well-established in law, especially in instances where life-sustaining or lifeprolonging treatments are in question. An example of someone who refuses treatment is
someone who has blood cancer, and who refuses to undergo chemotherapy or receive feeds
via a nasogastric tube. By granting or allowing patients the right to refuse or reject medical 
treatment, a way to perform passive euthanasia has been ultimately given. Some people argue
that the provision that allows for medical termination of pregnancy before 16 weeks into a
pregnancy is also considered to be a form of active involuntary euthanasia.\footnote{Suresh Bada Math\& Santosh K. Chaturvedi, \textit{Euthanasia: Right to life vs right to die,} IJMR 2012 Dec; 136(6): 899–902, (Mar. 20, 2018, 07:00 PM), \url{https://www.ncbi.nlm.nih.gov/pmc/articles/PMC3612319.}}

\noi
With the patients having the right to refuse-reject treatment, particularly when they are in a
life-threatening scenario, they have the potential to refuse. It is essential to secure the
patient's refusal in the event of a witness. The paper confirming the refusal must be signed by
the witness. Due to the fact that a patient's refusal to consent to a life-saving procedure will
invalidate the surgery or treatment, it is often in the patient's or authorized representative's
best interest to inform the hospital administrator about the non-performance of the procedure
and allow the administrator to take appropriate action. To prevent an adult patient from
leaving a hospital against his will is against the law. If a patient demands that he be
discharged from the hospital even if medical advice indicates otherwise, then this should be
recorded, and his signature obtained.\footnote{Ajay Kumar et. al., \textit{Consent and the Indian medical practitioner,} IJA 2015 Volume 59 (11) 695-700, (Apr. 01, 2018, 09:10 PM), \url{http://www.ijaweb.org/article.asp?issn=0019- 5049;year=2015;volume=59;issue=11;spage=695;epage=700;aulast=Kumar.}}

\noi
This is one of the crucial matters that shapes the delivery of medical treatments today,
especially when it comes to permission. There is no further need except that the patient is
able to make decisions, because his desires are all that is needed to give permission for
medical treatment. He is able to give consent to medical treatment even if the therapy would
save his life, as long as his preferences and his ability to make decisions are there. As the job
description of today's medical lawyer plainly outlines, this includes a considerable deal of
ethical consideration, and is central to modern medical legislation The Nuremberg Code of
1947 was a widely adopted declaration in which it was said that dignity is inherent to all
human beings, and this is to be preserved. The Nuremberg Code was put in place following
World War II to address the crimes that the Nazi administration committed against both
humans and animals in their quest for biological and medical knowledge. A way to assist or
get voluntary and informed consent is to implement a policy that mandates involvement of
human subjects in all research studies. In keeping with the Declaration of Helsinki, which is
the ethical guideline of the World Medical Association, the Declaration of Helsinki which
was adopted in 1964 stressed the importance of properly informing study subjects of the 
aims, methods, anticipated benefits, possible hazards, and any discomfort that may be
associated with the research. Several international treaties and declarations, such as the
declaration on human subjects and protection of human research participants, as well as the
Nuremberg Code, support the requirement to get agreement from patients before conducting
and/or providing medical treatments. This article deals with the entire spectrum of concerns
related to consent in the existing legal context in India. Nowadays, it appears that the circle of
legal development on consent has almost been completed in the relevant jurisdiction, as the
Indian Supreme Court determined that it is not only “consent” or “informed consent” but
prior informed consent must be required by law in every case, except in the limited
circumstances of emergency. To be very frank, this puts the doctor in an awkward position.
So, it is important to examine the notion of "consent and medical treatment" to gain a deeper
grasp of its delicate and fundamental features.\footnote{Omprakash V. Nandimath, \textit{Consent and medical treatment: The legal paradigm in India,} Indian J Urol. 2009 Jul-Sep; 25(3): 343–347, (Mar. 21, 2018, 03:30 PM), \url{https://www.ncbi.nlm.nih.gov/pmc/articles/PMC2779959.}}

\noi
According to the 196$^{\rm th}$ Report on Medical Treatment to Terminally Ill Patients (protection of
patients and medical practitioners) of the Law Commission, the patient (competent) has the
legal and constitutional right to refuse medical treatment that would result in a transient
increase in life expectancy. In the patient's final moments, life hangs in the balance. There
isn't even a glimmer of hope for recovery. When one is in incredible pain and in a state of
mental agony, one wants to live out his life without the use of any artificial methods. She/he
wants to avoid spending money on something that has no effect. One takes care of his or her
well-being over that of suffering. If one must be kept in the critical care unit for a few days or
months prior to dying, then he or she does not want to be treated like a "cabbage". His right
to privacy must be maintained, which includes protection from unwanted interference and
infringement of his bodily integrity. As in Gian Kaur's case, the natural process of his death
has already begun, and he wishes to die peacefully and dignifiedly. No law can prevent him
from taking this path. Leaving aside the argument for decriminalizing attempted suicide, this
is not a situation comparable to suicide. One will not undergo invasive medical treatment,
regardless of how his doctor or relatives try to compel him to do so.\footnote{\textit{The 196$^{\rm th}$ Report on Medical Treatment to Terminally Ill Patients (protection of patients and medical
practitioners),} The Law Commission of India, March 2006, \url{www.lawcommissionofindia.nic.in/.}}

\noi
Furthermore, the best interest principle adopted by Lord Goff in Airedale when he placed the
child's interests ahead of the mother's best interests is relevant here. This question to be
considered in these situations is, therefore, whether it is in the child's best interests that 
treatment that causes him to live longer than his biological age should be continued. In the
instance of Airedale, it was established that it was acceptable for doctors to stop treating
patients who refuse therapy. The doctor's responsibility is to act in the patient's best interests
if the patient is unable to communicate.\footnote{Airedale N.H.S. Trust v. Bland, (1993) 2 WLR 316: (1993) 1 All ER 821, HL.}

\heading{Passive Euthanasia}

\noi
Passive euthanasia is a more common occurrence in the majority of the hospitals in the
county, where patients and their families are no longer willing or able to continue with lifeprolonging treatment because of the price. If euthanasia is allowed, the Indian commercial health sector will make a killing off of the elderly and disabled citizens, many of whom would otherwise die waiting for expensive medical care.\footnote{\textit{Supra} Note 10.}

\noi
Euthanasia cannot be perceived to be passive unless one also has an understanding of active
euthanasia, which was widely studied and discussed after the Aruna Shanbaug case through
the Law Commission's \textit{241$^{\rm st}$ Report on Passive Euthanasia: A Relook.} It was made clear in
the Supreme Court of India's ruling in Aruna Ramachandra Shanbaug v. Union of India, as to
whether or not Aruna Shanbaug's identity had been stolen. Active euthanasia, alternatively,
includes taking steps to end a patient's suffering, such as injecting them with a lethal
substance like Sodium Pentothal, which causes the patient to fall into a deep sleep in a few
seconds, and from there, death is painless and peaceful. Thus, the amount of good
accomplished via ending a person's suffering by a positive deed is the same as murder.
Euthanasia which includes active intervention to stop the agony and suffering of a patient
about to die is an activity that is performed on that patient. A treatment approach that has had
success in the field is the administration of a medication that allows the patient would go into
deep sleep for a few seconds, and subsequently the patient passes away quietly and
painlessly. Because in this case, it is defined as euthanasia to alleviate the sufferings of a
person in the end phase of terminal illness, this qualifies as a sort of euthanasia. Penalizing
patients for any part of their medical treatment, regardless of their expressed desires, is seen
as a felony throughout the world, with the exception of where it is sanctioned by statute, as
the Supreme Court of the United States demonstrated earlier. Legalization of euthanasia in
India is prohibited under both Section 303 \& Section 306 of Indian Penal Code. To commit
suicide is a crime under Section 306 of IPC (abetment to suicide). In other words, negative  
euthanasia is generally known as passive euthanasia. This withholding of medical treatment,
for example, the withdrawing of antibiotics if there is no rational reason to prescribe them,
and the withholding of life support system when a patient is going to die and does not have a
chance of survival unless it is provided, are all methods that are examples of covering
something up. While there is no legal requirement for active euthanasia, passive euthanasia is
permitted nonetheless, as long as the necessary conditions and precautions are in place.
Active euthanasia, as defined by the Supreme Court, is most heavily weighted in terms of
significance. Passive euthanasia is defined as the following: “Nothing is done to induce the
patient's death in passive euthanasia, but in active euthanasia, something may be done to
speed up or assist the patient's passing”. In Aruna's case, Hon’ble Judges used the word above
as a source of additional insight, stating “Passive euthanasia suggests that the doctors aren't
actively killing him; they are simply passively allowing him to die." When asked about how
people react when someone makes a life-saving manoeuvre, the court stated that although we
generally applaud someone who saves another person, we do not commonly blame someone
for failing to do so. The Supreme Court came to a definitive conclusion, stating that while the
debate about whether or not active euthanasia should be allowed can be controversial, there is
no room for uncertainty about passive euthanasia, the position which argues that passive
euthanasia should be permitted no matter whether one does or does not take action. Voluntary
euthanasia is further subdivided into voluntary and non-voluntary passive euthanasia.
Voluntary euthanasia, in which the patient agrees to accept the procedure, is called by that
name. Involuntary euthanasia is commonly referred to as non-voluntary euthanasia because
the patient lacks the capacity to give consent, for example, if he is in a vegetative state. This
is the moment when the Supreme Court issued an additional statement, saying: "In the former
case, there are no issues; nevertheless, in the latter case, we will be discussing various
questions raised by it." This was the first time the Supreme Court stepped in to regulate nonvoluntary passive euthanasia since the patient was in a vegetative condition.\footnote{\textit{The 241st Report on Passive Euthanasia – A Relook,} The Law Commission of India, Report No. 241, August 2012,\\ \url{http://www.lawcommissionofindia.nic.in.}}

\noi
The inability to provide patients with instructions for discontinuing and withdrawing life
support from a patient when their quality of life has deteriorated to an unacceptable level is
seen as the greatest hindrance to competent end-of-life care in India. Additionally, it indicates
that physicians are apprehensive about the possibility of facing civil or criminal culpability
when they're compelled to make medical choices to limit life-sustaining measures. While the  
argument about whether it is acceptable to apply a specific medical treatment to an individual
and whether the individual has the right to refuse treatment rests on a broader question of
which society and nation's interests come first, the dispute concerns which of the two
treatment paths is best for the individual. As when it comes to public health, the claims of the
society prevail. This is a great example, as it can be mandatory vaccination to prevent an
epidemic from breaking out. But where treatment is designed for an individual and his
immediate family member, individuals should be treated on an individual basis, even if that
contradicts the claims of an organization. With regard to the people who have a demonstrated
right to refuse treatment, they have an unquestionable right to say no. The right to freedom
originates from the need of the community to respect, safeguard, and not encroach upon the
individual's ability to have his own views and ideas when it comes to subjects relating to
individual sovereignty, which is a clear indication of a free society.\footnote{\textit{Supra} Note 7.}

\vspace{.1cm}

\noi
Prior to the judgment, in 2006, the Law Commission had laid down an act for \textit{The Medical
Treatment of Terminally Ill Patients (protection of patients, medical practitioners)}.\footnote{\textit{The 196$^{\rm th}$ Report on Medical Treatment to Terminally Ill Patients (protection of patients and medical
practitioners),} The Law Commission of India, March 2006, \url{www.lawcommissionofindia.nic.in/.}} A
'competent' patient who is going through terminal illness has the right to discontinue or refuse
treatment if the patient has been made aware of all of the facts concerning the disease and
treatment, as well as the right to know the results of any medical tests. In that case, the doctor
must respect the decision to withhold or withdraw treatment or life support system. But in
cases where the patient is incompetent which includes those considered to be under the
influence of others and those who are diagnosed with minor personality disorders and are
unable to make decisions for themselves, the doctors will have to make the best decisions. As
with other controversial medical treatments, the parents' and guardians' preferences may
prevail in the event that the subject is allowed to live with medical treatment that will lead to
his inevitable death. On the other hand, the health ministry decided not to pass any legislation
on euthanasia at the time.

\vspace{.1cm}

\noi
In Law Commission’s \textit{196$^{\rm th}$ Report on Medical Treatment to Terminally Ill Patients
(protection of patients and medical practitioners),} to summarize, the Commission explicitly
states that they feel it necessary to say that the use of the terms 'assisted suicide' and
'euthanasia' as crime categories should continue. It has a specific objective, namely, to
research a variety of legal issues and concept related to "withdrawal of life support measures"  
and to make recommendations to the medical profession regarding how to make decisions
about withdrawing life support, which the inquiry will do solely on the matters of the
conditions in which doctors should or should not withdraw support if they believe it is in the
patient's best interests. Furthermore, there is debate on the circumstances in which a patient
may refuse or reject treatment and seek a discontinuation or withholding of treatment or life
support measures, if the patient has intentionally made a decision. Decades ago, when the
science of medicine and health technology had not advanced to include artificial techniques
of prolonging the life of patients who were dying from natural causes, including ventilators
and feeding devices, such people perished of natural causes. Today, everyone agrees on this:
While the general public considers this acceptable, an opposing theory describes it as
dangerous and selfish, a dereliction of a patient's civil duty to reject contemporary medical
treatment and let natural world do its thing since it has achieved in better times. Precise
choices have been reached throughout the globe in regard to people who are both conscious
and capable but who have cancer that has already spread. People who understand they may
make an informed choice to die peacefully and who request that they need not be provided
medical care which could only extend their life, are expected to have their wishes honoured.
Although a substantial percentage of certain patients have reached a stage of their illness
where doctors have determined that there are no realistic medical prospects for recovery, to
date, many of these patients have been treated by renowned doctors, making their condition
far more complicated than it may appear at first glance. While modern medicine and
technology could provide people with the tools to extend their lives without providing any
purpose for doing so, patients may have to go through pain and suffering in their extended
lives. The majority of patients who get palliative care have done so to relieve the symptoms
of pain and suffering, where some of these patients do not want to have any medical
treatments that could extend life or delay death.\footnote{ Id.}

\vspace{-.1cm}

\noi
Even in the \textit{210$^{\rm th}$ Report on Humanization and Decriminalization of Attempt to Suicide}\footnote{\textit{The 210$^{\rm th}$ Report on Humanization and Decriminalization of Attempt to Suicide,} the Law Commission of India, Report No. 210, October 2008, \url{http://www.lawcommissionofindia.nic.in.}}
 given by the Law Commission of India, it was stated that there was an undeniable obligation
to provide every Indian with the option to live to the end of one's natural life with human
dignity. The notion of right to live should include the right to die with dignity and comparing the right to end one's life prematurely through suicide to the right to die in a way that
addresses misery or misery to one's body till the normal life span is attained is improper.

\vspace{-.1cm}

\heading{The Concern}

\vspace{-.1cm}

\noi
Technological advancements in healthcare and health have led to improvements in life
support mechanisms that have allowed for extended periods of time spent on life support,
sometimes years long, for which the concept of the right to refuse treatment has become a
talking point. Patients who are already terminally ill or in a persistent vegetative condition
can pursue the right to refuse or reject treatment in order to extend their comfort, but also the
right to be free from any kind of invasive procedures, including surgery, prescription
medications, or life enhancing system of any form in order to prevent the lingering pain. A
competent individual who is not experiencing medical condition that renders them unable to
give informed consent is allowed to refuse medical treatment. Although the conservative
viewpoint on these three diverse topics differs greatly from suicide, physician-assisted suicide
and any form of euthanasia. The ethics of euthanasia and assisted suicide is put into question
when a terminally sick patient refuses or rejects to undergo artificial life support system or
any treatment. It is clear that this cannot be called either euthanasia or assisted suicide.
Refusing treatment is synonymous with death for those suffering from a terminal illness.
There is just one possible consequence when it comes to either killing oneself or refusing
therapy. They all lead to death. However, refusing to accept therapy cannot be regarded to be
suicide.

\noi
However, it may be argued that this problem in that there is a limitation of the patient’s
freedom to deny treatment, rather than just wanting to die, is less critical. By refusing medical
treatment, patients who suffer from a disease that has not yet developed may die prematurely,
due to their underlying illness. In this way, their deaths are mostly triggered by the illness,
and not due to an action or inaction that they themselves chose to implement in the form of a
self-inflicted passive euthanasia system. An inquiry into prevalent legislation and rules of
terminally ill people would lead to the conclusion that all individuals who can give informed
consent have always had the right to refuse healthcare and have the right to determine the
circumstances of their own death, which is impervious to passive euthanasia, and is permitted
in India.

\noi
While the question of whether a patient is allowed to refuse treatment seems affirmative, the
question raises serious concerns about the refusal of the treatment and the life support system
that involves passive euthanasia. Touched by the more complicated questions: Does the right
to refuse to treat the system even in accordance with medical directives extend to denial of
life-support? Affirmative errors in the law are the questions of the legality of an individual
exercising his or her right with informed decision or through \textit{parens patriae} as regards the
law on suicide.

\noi
In Indian Law, none of these topics has explicitly established its view. It must be first
established that a specific professional body has an opinion on the issue before the legislative
regulation of life-support measures can be developed. As no pertinent case law exists in the
country, this is essential.\footnote{\textit{Supra} Note 7.} Even the landmark opinions in Aruna Shanbaug and Common
Cause cases on the assimilation of Reports of Law Commission of India are not able to take
care of the gaps in problems relating to passive Euthanasia and the right to reject processing
to the present. Since time immemorial the right to refuse treatment is generally considered as
an issue that hinders the state's health and health responsibility and is now a type of passive
euthanasia which is allowed in the face of the right to reject treatment. The ruling appears to
serve the elite strata of the society as a legal punishment to restrict finances, paying little
mind to the society at a stage in which disabilities and the public health sector are being cared
for in healthcare services. The impoverished people have always lived with the choice of
right to refuse treatment and enable passive euthanasia to continue infringing their rights to
improved healthcare in the nation.

\heading{Conclusion}

\noi
Indian law neglects to emphasize the fields of medico-legal concerns even after the two key
decisions towards the end of life. While suicide laws muddle this problem, since the latter
concerns only a determined decision to harm ourselves in absence of any illness, there is no
clarity if a patient is entitled to reject treatment. According to the interpretation of Article 21
and 14 by the Supreme Court, there is no moral distinction between rejecting or removing
life-support. They both serve to stop counterproductive interventions, such as death due to
allowing it to occur by way of treatment or through euthanasia using a new definition for the
law on privacy in the perspective of terminal health and medical interventions. For the use of 
life support, receiving palliative care is absolutely necessary. Palliative care is the physician’s
major responsibility, and the main aim of any medical intervention is to relieve pain and
suffering. It is relevant to add that the notion of the right to refusal and passive euthanasia
authorized in India is the opposing side of a coin, yet it helps terminally ill patients’
emotional and psychological pain as both stand for their right to self-determination.

\noi
The Judicial approach towards this controversial concern has walked through some steps in
contrast to the regulations in developed nations towards a new age of healthcare and medicine
for terminally ill patients, as the Supreme Court has established decisions to the decision by
authorizing passive euthanasia on the difficult topic of euthanasia. In a society in front of a
complicated medical, social and legal challenge in medico-legal ethics, the sentences put
down are aimed at maintaining harmony. Legislation is required to safeguard end-use
patients, as well as the care of physicians, according to the Law Commission
Recommendation. Need for new legislations on these issues cannot be withheld as for the
development and enhancement of the rights of the patients, laws need to be framed in the area
of Right to Refuse Treatment or Informed Refusal of Treatment, Withdrawal or withholding
of Life Sustaining Treatment, Right to Palliative Care and in absence the intervention of
Judiciary to formulate guidelines for upholding the Basic Structure of the Constitution by
allowing basic human right to life.

\end{multicols}
